\refstepcounter{appendixctr}\label{hwansappendix}%
\appendix\chapter{Appendix \ref{hwansappendix}: Hints and Solutions}
	
%==================================================================
%==================================================================
%========================= Self-Checks ============================
%==================================================================
%==================================================================




\noindent\formatlikesection{Answers to Self-Checks}

%----------------------------------------------------------------------------------------

\noindent\formatlikesubsection{Answers to Self-Checks for Chapter \ref{ch:energy}}\\
\scanshdr{conservation} A conservation law in physics says that the total
amount always remains the same. You can't get rid of it even if you want to.

\scanshdr{bacteria-queue}
Exponents have to do with multiplication, not addition. The
first line should be 100 times longer than the second,
not just twice as long.

\scanshdr{mars-and-venus}
Doubling $d$ makes $d^2$ four times bigger, so the gravitational field experienced
by Mars is four times weaker.

%----------------------------------------------------------------------------------------

\noindent\formatlikesubsection{Answers to Self-Checks for Chapter \ref{ch:momentum}}\\
\scanshdr{moonviolatestranslation} No, it doesn't violate symmetry. Space-translation symmetry
only says that space itself has the same properties everywhere. It doesn't say that all regions
of space have the same stuff in them. The experiment on the earth comes out a certain way because
that region of space has a planet in it. The experiment on the moon comes out different because
that region of space has the moon in it. of the apparatus, which you forgot to take with you.

\scanshdr{markcm} The camera is moving at half the speed at which the light ball is initially
moving. After the collision, it keeps on moving at the same speed --- your five x's all line
on a straight line. Since the camera moves in a straight line with constant speed, it is
showing an inertial frame of reference.

\scanshdr{thirdframepcons} The table looks like this:

\velocitytable{$-1$}{0}{$+1$}{0}{$-1$}{$-1$}{../../../cp/ch02/figs/darkball}{../../../cp/ch02/figs/lightball}

\noindent Observers in all three frames agree on the changes in velocity, even though they disagree
on the velocities themselves.

\scanshdr{heavierball} The motion would be the same. The force on the ball would be 20 newtons,
so with each second it would gain 20 units of momentum. But 20 units of momentum for a 2-kilogram
ball is still just 10 m/s of velocity.

%----------------------------------------------------------------------------------------

\noindent\formatlikesubsection{Answers to Self-Checks for Chapter \ref{ch:angular-momentum}}\\
\scanshdr{importance-of-torque-equations}
The definition of torque is important, and so is the equation $F=\pm Fr$. The two equations
in between are just steps in a derivation of  $F=\pm Fr$.

%----------------------------------------------------------------------------------------

\noindent\formatlikesubsection{Answers to Self-Checks for Chapter \ref{ch:relativity}}\\
\scanshdr{unequalcollisioncons} The total momentum is zero before the collision. After the
collision, the two momenta have reversed their directions, but they still cancel.
Neither object has changed its kinetic energy, so the total energy before and after the collision
is also the same.

%----------------------------------------------------------------------------------------

\noindent\formatlikesubsection{Answers to Self-Checks for Chapter \ref{ch:electricity}}\\
\scanshdr{typesofcharge} Either type can be involved in either an attraction or a
repulsion. A positive charge could be involved in either an attraction (with a negative
charge) or a repulsion (with another positive), and a negative could participate
in either an attraction (with a positive) or a repulsion (with a negative).

\scanshdr{inductionneg} It wouldn't make any difference. The roles of the positive and
negative charges in the paper would be reversed, but there would still be a net attraction.

%----------------------------------------------------------------------------------------

\noindent\formatlikesubsection{Answers to Self-Checks for Chapter \ref{ch:fields}}\\
\scanshdr{alternator}
An induced electric field can only be created by a \emph{changing} magnetic
field. Nothing is changing if your car is just sitting there. A point on the
coil won't experience a changing magnetic field unless the coil is already
spinning, i.e., the engine has already turned over.

%----------------------------------------------------------------------------------------

\noindent\formatlikesubsection{Answers to Self-Checks for Chapter \ref{ch:ray-model}}\\
\scanshdr{two-reflections-from-same-point}
Only 1 is correct. If you draw the normal that bisects the solid ray, it
also bisects the dashed ray.

\scanshdr{candlefivetimes}{He's five times farther away than she is, so the light he sees
is 1/25 the brightness.}

\scanshdr{move-object-laterally} You should have found from your ray diagram
that an image is still formed, and it has simply moved down the same distance
as the real face. However, this new image would only be visible from high up,
and the person can no longer see his own image. 

\scanshdr{real-do-and-di}
Increasing the distance from the face to the mirror has decreased the distance
from the image to the mirror. This is the opposite of what happened with the virtual
image.

%----------------------------------------------------------------------------------------

\noindent\formatlikesubsection{Answers to Self-Checks for Chapter \ref{ch:waves}}\\
\scanshdr{ribbon-on-spring}
The leading edge is moving up, the trailing edge is moving down, and the top of the
hump is motionless for one instant.

%==================================================================
%==================================================================
%========================= Solutions ==============================
%==================================================================
%==================================================================

\hwanssection{Solutions to Selected Homework Problems}

\beginsolutions{energy}

\hwsolnhdr{mg-to-kg}

\begin{equation*}
  134\ \zu{mg} \times \frac{10^{-3}\ \gunit}{1\ \zu{mg}} \times \frac{10^{-3}\ \kgunit}{1\ \zu{g}} = 1.34\times10^{-4}\ \kgunit
\end{equation*}

\beginsolutions{angular-momentum}
\hwsolnhdr{pliers}
The pliers are not moving, so their angular momentum
remains constant at zero, and the total torque on them must
be zero. Not only that, but each half of the pliers must
have zero total torque on it. This tells us that the
magnitude of the torque at one end must be the same as that
at the other end. The distance from the axis to the nut is
about 2.5 cm, and the distance from the axis to the centers
of the palm and fingers are about 8 cm. The angles are close
enough to $90\degunit$ that we can pretend they're 90 degrees,
considering the rough nature of the other assumptions and
measurements. The result is $(300\ \nunit)(2.5\ \zu{cm})=(F)(8\ \zu{cm})$,
 or $F=90$ N.
