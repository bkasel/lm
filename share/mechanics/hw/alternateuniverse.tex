Suppose that we inhabited a universe in which, instead
of Newton's law of gravity, we had $F=k\sqrt{m_1m_2}/r^2$, where $k$ is some
constant with different units than $G$. (The force is still
attractive.) However, we assume that $a=F/m$ and the rest of
Newtonian physics remains true, and we use $a=F/m$ to define
our mass scale, so that, e.g., a mass of 2 kg is one which
exhibits half the acceleration when the same force is
applied to it as to a 1 kg mass.\hwendpart
 %
 (a) Is this new law of
gravity consistent with Newton's third law?\hwendpart
 %
 (b) Suppose you
lived in such a universe, and you dropped two unequal masses
side by side. What would happen?\hwendpart
 %
 (c) Numerically, suppose a
1.0-kg object falls with an acceleration of 10 $\munit/\sunit^2$. What
would be the acceleration of a rain drop with a mass of 0.1
g? Would you want to go out in the rain?\hwendpart
 %
 (d) If a falling
object broke into two unequal pieces while it fell, what
would happen?\hwendpart
 %
 (e) Invent a law of gravity that results in
behavior that is the opposite of what you found in part b.
[Based on a problem by Arnold Arons.]
