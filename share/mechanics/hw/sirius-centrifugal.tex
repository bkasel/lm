The bright star Sirius has a mass of $4.02\times10^{30}\ \kgunit$ and lies at
a distance of $8.1\times10^{16}\ \munit$ from our solar system. Suppose you're
standing on a merry-go-round carousel rotating with a period of 10 seconds,
and Sirius is on the horizon.
You adopt a rotating, noninertial frame of reference, in which the carousel
is at rest, and the universe is spinning around it. If you drop a corndog, you see
it accelerate horizontally away from the axis, and you interpret this as the
result of some horizontal force. This force does not actually exist; it only
seems to exist because you're insisting on using a noninertial frame.
Similarly, calculate the force that seems to act on Sirius in this frame of reference.
Comment on the physical plausibility of this force, and on what object could
be exerting it.\answercheck
