% meta {"stars":1}
The figure shows a microscopic view of the innermost tracks of a music CD.
The pits represent the pattern of ones and zeroes that encode the musical waveform.
Because the laser that reads the data has to sweep over a fixed amount of data
per unit time, the disc spins at a decreasing angular velocity as the music is
played from the inside out. The linear velocity $v$, not the angular velocity,
is constant. Each track is separated from its neighbors on either side by a fixed distance $p$, called
the pitch. Although the tracks are actually concentric circles, we will idealize
them in this problem as a type of spiral, called an Archimedean spiral,
whose turns have constant spacing, $p$, along any radial line.
Our goal is to find the angular acceleration of this idealized CD, in terms of the constants
$v$ and $p$, and the radius $r$ at which the laser is positioned.\\
(a) Use geometrical reasoning to constrain the dependence of the result on $p$.\hwendpart
(b) Use units to further constrain the result up to a unitless multiplicative constant.\hwendpart
(c) Find the full result. [Hint: Find a differential equation involving $r$ and its time derivative,
and then solve this equation by separating variables.]\answercheck\hwendpart
(d) Consider the signs of the variables in your answer to part c, and show that
your equation still makes sense when the direction of rotation is reversed.\hwendpart
(e) Similarly, check that your result makes sense regardless of whether we view the
CD player from the front or the back. (Clockwise seen from one side is counterclockwise
from the other.)
