The figure shows the International Space Station (ISS). One of the purposes of the ISS is
supposed to be to carry out experiments in microgravity. However, the following factor
limits this application. The ISS orbits the earth once every
92.6 minutes. It is desirable to keep the same side of the station always oriented toward the
earth, which means that the station has to rotate with the same period. In the photo, the
direction of orbital motion is left or right on the page, so the rotation is about the
axis shown as up and down on the page. The greatest distance of any pressurized compartment
from the axis of rotation is 36.5 meters. Find the acceleration due to the rotation at this point,
and the apparent weight of
a 60 kg astronaut at that location.\answercheck
