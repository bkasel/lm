% meta {"stars":1}
%
An object has underdamped motion as depicted in the
figure, where $T = 2\pi/\omega$, and, as described in the text, $\omega$ differs
from $\omega_0=\sqrt{k/m}$.\\
%
(a) What fraction of the energy was lost during this first cycle?
This fraction is lost in every ensuing cycle.\answercheck\hwendpart
%
(b) Where will the object be (in terms of $A$) after the second full
oscillation?\answercheck\hwendpart
%
(c) By assuming $\omega \approx \omega_0$, what is the
value of $b$? To express your answer, write $b = C\sqrt{km}$, and
solve for the unitless constant $C$.\answercheck\hwendpart
%
(d) Use this value of $b$ to find the percentage increase in the
period of the motion as compared to the undamped case. You should get
an answer much less than 1\%, which means the approximation made in
part c was justified.\answercheck
