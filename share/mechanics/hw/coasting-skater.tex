An ice skater builds up some speed, and then coasts across the ice passively in
a straight line. (a) Analyze the forces, using
a table in the format shown in
m4_ifelse(__problems,1,[:section \ref{sec:newton-2-analysis-of-forces}:],[:__subsection_or_section(analysis-of-forces):]).\hwendpart (b) If his initial speed is $v$, and the
coefficient of kinetic friction is $\mu_k$, find the maximum theoretical distance he can glide before
coming to a stop. Ignore air resistance.\answercheck\hwendpart (c) Show that your answer to part b has the right units.\hwendpart
(d) Show that your answer to part b depends on the variables in a way that makes
sense physically.\hwendpart (e) Evaluate your answer numerically for $\mu_k=0.0046$, and a
world-record speed of $14.58$ m/s. (The coefficient of friction was measured by
De Koning et al., using special skates worn by real speed skaters.)\answercheck\hwendpart (f) Comment on whether
your answer in part e seems realistic. If it doesn't, suggest possible reasons why.\hwendpart
