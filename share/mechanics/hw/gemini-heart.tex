The 1961-66 US Gemini program launched pairs of astronauts into earth
orbit in tiny capsules, on missions lasting up to 14 days. The figure
shows the two seats, in a cross-sectional view from the front, as if looking
into a car through the windshield.  During the Gemini 8 mission, a
malfunctioning thruster in the Orbit Attitude and Maneuvering System
(OAMS) caused the capsule to roll, i.e., to rotate in the plane of the
page.  The rate of rotation got faster and faster, reaching 296
degrees per second before pilot Neil Armstrong shut down the OAMS
system by hand and succeeded in canceling the rotation using a
separate set of re-entry thrusters. At the peak rate of rotation, the
astronauts were approaching the physiological limits under which their
hearts would no longer be able to circulate blood, potentially causing
them to black out or go blind. Superimposing the approximate location
of a human heart on the original NASA diagram, it looks like
Armstrong's heart was about 45 cm away from the axis of rotation. Find
the acceleration experienced by his heart, in units of $g$.\answercheck
