Refrigerators, air conditioners, and heat pumps are heat engines that
work in reverse.
You put in mechanical work, and the effect is to take heat out of
a cooler reservoir and deposit heat in a warmer one: $Q_L+W=Q_H$.
As with the heat engines discussed previously, the efficiency is
defined as the energy transfer you want ($Q_L$ for a refrigerator or
air conditioner, $Q_H$ for a heat pump) divided by the energy transfer
you pay for ($W$).

Efficiencies are supposed to be unitless, but the
efficiency of an air conditioner is normally given in terms of an EER rating
(or a more complex version called an SEER). The EER is defined
as $Q_L/W$, but expressed in the barbaric units of of Btu/watt-hour.
A typical EER rating
for a residential air conditioner is about 10 Btu/watt-hour, corresponding to an
efficiency of about 3.
The standard temperatures used for testing an air conditioner's efficiency
are $80\degunit F$ ($27\degcunit$) inside and
 $95\degunit F$ ($35\degcunit$) outside.

\noindent (a) What would be the EER rating of a reversed Carnot engine used as
an air conditioner? \answercheck\hwendpart
(b) If you ran a 3-kW residential air conditioner, with an efficiency of 3,
 for one hour, what would
be the effect on the total entropy of the universe? Is your answer
consistent with the second law of thermodynamics? \answercheck
