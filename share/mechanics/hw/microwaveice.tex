A microwave oven works by twisting molecules one way and
then the other, counterclockwise and then clockwise about
their own centers, millions of times a second. If you put an
ice cube or a stick of butter in a microwave, you'll observe
that the solid doesn't heat very quickly, although
eventually melting begins in one small spot. Once this
spot forms, it grows rapidly, while the rest of the solid
remains solid; it appears
that a microwave oven heats a liquid much more
rapidly than a solid. Explain why this should happen, based
on the atomic-level description of heat, solids, and liquids.
m4_ifelse(__problems,1,[::],[:%
(See, e.g., figure \figref{random-motion} on page \pageref{fig:random-motion}.):])

Don't repeat the following common mistakes:

\noindent\emph{In a solid, the atoms are packed more tightly and have less
space between them.} Not true. Ice floats because it's \emph{less} dense
than water.

\noindent\emph{In a liquid, the atoms are moving much faster.} No, the difference
in average speed between ice at $-1\degcunit$ and water at $1\degcunit$ is
only 0.4\%.
