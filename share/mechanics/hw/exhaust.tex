% meta {"stars":1}
 A rocket ejects exhaust with an exhaust velocity $u$.
The rate at which the exhaust mass is used (mass per unit
time) is $b$. We assume that the rocket accelerates in a
straight line starting from rest, and that no external
forces act on it. Let the rocket's initial mass (fuel plus
the body and payload) be $m_i$, and $m_f$ be its final mass,
after all the fuel is used up. (a) Find the rocket's final
velocity, $v$, in terms of $u$, $m_i$, and $m_f$. Neglect the effects
of special relativity. (b) A
typical exhaust velocity for chemical rocket engines is 4000
m/s. Estimate the initial mass of a rocket that could
accelerate a one-ton payload to 10\% of the speed of light,
and show that this design won't work. (For the sake of the
estimate, ignore the mass of the fuel tanks. The speed is fairly
small compared to $c$, so it's not an unreasonable approximation
to ignore relativity.)
\answercheck
