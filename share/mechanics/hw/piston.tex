A pneumatic spring consists of a piston riding on top
of the air in a cylinder. The upward force of the air on
the piston is given by $F_{air}=ax^{-1.4}$, where $a$ is a
constant with funny units of $\nunit\unitdot\munit^{1.4}$. For simplicity,
assume the air only supports the weight, $F_W$, of the
piston itself, although in practice this device is used to
support some other object. The equilibrium position, $x_0$,
is where $F_W$ equals $-F_{air}$. (Note that in the main
text I have assumed the equilibrium position to be at $x=0$,
but that is not the natural choice here.) Assume friction
is negligible, and consider a case where the amplitude of
the vibrations is very small. Let $a=1.0\ \nunit\unitdot\munit^{1.4}$, 
$x_0=1.00\ \munit$, and $F_W=-1.00\ \nunit$. The piston is released from
$x=1.01\ \munit$. Draw a neat, accurate graph of the total
force, $F$, as a function of $x$, on graph paper, covering
the range from $x=0.98\ \munit$ to 1.02 $\munit$. Over this small
range, you will find that the force is very nearly
proportional to $x-x_0$. Approximate the curve with a
straight line, find its slope, and derive the approximate
period of oscillation. \answercheck
