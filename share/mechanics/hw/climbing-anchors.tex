For safety, mountain climbers often wear a climbing harness and tie in to other climbers on a rope
team or to anchors such as pitons or snow anchors. When using anchors, the climber usually wants to tie in
to more than one, both for extra strength and for redundancy in case one fails.
The figure shows such an arrangement, with the climber hanging from a pair of anchors forming a ``Y''
at an angle $\theta$. The metal piece at the center is called a carabiner.
The usual advice is to make $\theta<90\degunit$; for large values of $\theta$,
the stress placed on the anchors can be many times greater than the actual load $L$,
so that two anchors are actually \emph{less} safe than one.\hwendpart
(a) Find the force $S$ at each anchor in terms of $L$ and $\theta$.  \answercheck\hwendpart
(b) Verify that your answer makes sense in the case of $\theta=0$.\hwendpart
(c) Interpret your answer in the case of $\theta=180\degunit$.\hwendpart
(d) What is the smallest value of $\theta$ for which $S$ equals or exceeds $L$, so that for larger angles a failure
of at least one anchor is \emph{more} likely than it would have been with a single anchor?\answercheck\hwendpart
