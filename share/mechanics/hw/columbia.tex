%%%%%
%%%%% This problem is used by: 2cl,1,columbia
%%%%%
One theory about the destruction of the space shuttle
Columbia in 2003 is that one of its wings had been damaged
on liftoff by a chunk of foam insulation that fell off of
one of its external fuel tanks. The New York Times reported
on June 5, 2003, that NASA engineers had recreated the
impact to see if it would damage a mock-up of the shuttle's
wing. ``Before last week's test, many engineers at NASA said
they thought lightweight foam could not harm the seemingly
tough composite panels, and privately predicted that the
foam would bounce off harmlessly, like a Nerf ball.'' In
fact, the 1.7-pound piece of foam, moving at 531 miles per
hour, did serious damage. A member of the board investigating
the disaster said this demonstrated that ``people's
intuitive sense of physics is sometimes way off.'' (a)
Compute the kinetic energy of the foam, and (b) compare with
the energy of a 170-pound boulder moving at 5.3 miles per hour (the
speed it would have if you dropped it from about knee-level).\answercheck\hwendpart
(c) The boulder is a hundred times
more massive, but its speed is a hundred times smaller, so
what's counterintuitive about your results?
