All stars, including our sun, show variations in their
light output to some degree. Some stars vary their
brightness by a factor of two or even more, but our sun has
remained relatively steady during the hundred years or so
that accurate data have been collected. Nevertheless, it is
possible that climate variations such as ice ages are
related to long-term irregularities in the sun's light
output. If the sun was to increase its light output even
slightly, it could melt enough Antarctic ice to
flood all the world's coastal cities. The total sunlight
that falls on Antarctica amounts to about $1\times10^{16}$
watts. In the absence of natural or human-caused climate change, this heat input to the poles is balanced
by the loss of heat via winds, ocean currents, and emission
of infrared light, so that there is no net melting or
freezing of ice at the poles from year to year. Suppose that
the sun changes its light output by some small percentage,
but there is no change in the rate of heat loss by the polar
caps. Estimate the percentage by which the sun's light
output would have to increase in order to melt enough ice to
raise the level of the oceans by 10 meters over a period of
10 years. (This would be enough to flood New York, London,
and many other cities.) Melting 1 kg of ice requires $3\times10^3$ J.
