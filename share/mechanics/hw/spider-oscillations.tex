(a) We observe that the amplitude of a certain free oscillation decreases from $A_\zu{o}$ to $A_\zu{o}/Z$ after $n$ oscillations.
Find its $Q$.\answercheck\hwendpart
(b) The figure is from
\emph{Shape memory in Spider draglines}, Emile, Le Floch, and Vollrath, \emph{Nature} 440:621 (2006).
Panel 1 shows an electron microscope's image of a thread of spider silk.
In 2, a spider is hanging from such a thread. From an evolutionary point of
view, it's probably a bad thing for the spider if it twists back and forth while hanging
like this. (We're referring to a back-and-forth rotation about the axis of the thread, not
a swinging motion like a pendulum.) The authors speculate that such a vibration could make the spider easier for predators to see,
and it also seems to me that it would be a bad thing just because the spider wouldn't be able
to control its orientation and do what it was trying to do. Panel 3 shows a graph of such an oscillation, which
the authors measured using a video camera and a computer, with a 0.1 g mass hung from it in place of
a spider. Compared to human-made fibers such as kevlar or copper wire, the spider thread has an
unusual set of properties:
\begin{enumerate}
\item It has a low $Q$, so the vibrations damp out quickly. 
\item It doesn't become brittle with repeated twisting as a copper wire would.
\item When twisted, it tends to settle in to a new equilibrium angle, rather than
        insisting on returning to its original angle. You can see this in 
        panel 2, because although the experimenters initially
        twisted the wire by 35 degrees, the thread only performed oscillations with an amplitude
        much smaller than $\pm35$ degrees, settling down to a new equilibrium at 27 degrees.
\item Over much longer time scales (hours), the thread eventually resets itself to its original
        equilbrium angle (shown as zero degrees on the graph). (The graph reproduced here only
        shows the motion over a much shorter time scale.) Some human-made materials have this
        ``memory'' property as well, but they typically need to be heated in order to make them
        go back to their original shapes.
\end{enumerate}
Focusing on property number 1, estimate the $Q$ of spider silk from the graph.\answercheck
