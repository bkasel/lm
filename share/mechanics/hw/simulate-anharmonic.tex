A mass $m$ on a spring oscillates around an equilibrium at $x=0$. Any function $U(x)$
        with an equilibrium at $x=0$ can be approximated as  $U(x)=(1/2)kx^2$,
        and if the energy
        is symmetric with respect to positive and negative values of $x$, then the next
        level of improvement in such an approximation would be $U(x)=(1/2)kx^2+bx^4$.
        The general idea here is that any smooth function
        can be approximated locally by a polynomial, and if you want a better approximation,
        you can use a polynomial with more terms in it. When you ask your calculator
        to calculate a function like $\sin$ or $e^x$, it's using a polynomial approximation
        with 10 or 12 terms. Physically,
        a spring with a positive value of $b$ gets stiffer when stretched strongly
        than an ``ideal'' spring with $b=0$. A spring with a negative $b$ is like a person
        who cracks under stress --- when you stretch it too much, it becomes more elastic
        than an ideal spring would. We should not expect any spring to give totally ideal
        behavior no matter no matter how much it is stretched. For example, there has to be
        some point at which it breaks.

        Do a numerical simulation of the oscillation of a mass on a spring whose
         energy has a nonvanishing $b$. Is the
        period still independent of amplitude? Is the amplitude-independent equation
        for the period still approximately valid for small enough amplitudes? Does the
        addition of a positive $x^4$ term tend to increase the period, or decrease it?
        Include a printout of your program and its output with your homework paper.
