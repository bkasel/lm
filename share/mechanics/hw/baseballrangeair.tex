In this problem you'll extend the analysis in problem
        \ref{hw:baseballrange} 
        m4_ifelse(__me,1,on p.~\pageref{hw:baseballrange})
        to include air friction by writing a computer program. For a game played at
        sea level, the force due to air friction
        is approximately $(7\times10^{-4}\ \nunit\unitdot\sunit^2/\munit^2)v^2$,
        in the direction opposite to the motion of the ball.%%%
%\footnote{A standard baseball is supposed to have a circumference
%        of $9\frac{1}{8}$ inches. A standard way of paramerizing the force of fluid friction is $F=(1/2)\rho AC_D v^2$, where $\rho$ is
%        the density of the fluid, $A$ is the object's cross-sectional area, and $C_D$ is a unitless constant. E. Meyer and
%        J. Bohn, in a 2008 paper published at arxiv.org, survey existing data on $C_D$ for baseballs and estimate it to be in the range
%        from about 0.13 to 0.5. This leads to a figure something like the one given.} 
The mass of a
        baseball is 0.146 kg.\hwendpart
        (a) For a ball hit at a speed
        of 45.0 m/s from a height of 1.0 m, find the optimal angle and the resulting
        range.         <% hw_answer %>\hwendpart
        (b) How much farther would the ball fly at the Colorado Rockies' stadium,
        where the thinner air gives 18 percent less air friction? 
        <% hw_answer %>
