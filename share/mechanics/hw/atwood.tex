Unequal masses $M$ and $m$ are suspended from a pulley
as shown in the figure.\hwendpart
 %
(a) Analyze the forces in which mass $m$ participates, using
a table in the format shown in m4_ifelse(__problems,1,[:section \ref{sec:newton-2-analysis-of-forces}:],[:__subsection_or_section(analysis-of-forces):]). [The forces in
which the other mass participates will of course be similar,
but not numerically the same.]\hwendpart
 %
(b) Find the magnitude of the accelerations of the two
masses. [Hints: (1) Pick a coordinate system, and use
positive and negative signs consistently to indicate the
directions of the forces and accelerations. (2) The two
accelerations of the two masses have to be equal in
magnitude but of opposite signs, since one side eats up rope
at the same rate at which the other side pays it out. (3)
You need to apply Newton's second law twice, once to each
mass, and then solve the two equations for the unknowns: the
acceleration, $a$, and the tension in the rope, $T$.] \answercheck\hwendpart
 %
(c) Many people expect that in the special case of $M=m$,
the two masses will naturally settle down to an equilibrium
position side by side. Based on your answer from part b,
is this correct? \hwendpart
 %
(d) Find the tension in the rope, $T$.\answercheck\hwendpart
 %
(e) Interpret your equation from part d in the special case where
one of the masses is zero. Here ``interpret'' means to figure out
what happens mathematically, figure out what should happen
physically, and connect the two.\hwendpart
