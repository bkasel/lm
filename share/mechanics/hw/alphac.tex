You take a trip in your spaceship to another star.
Setting off, you increase your speed at a constant
acceleration. Once you get half-way there, you start
decelerating, at the same rate, so that by the time you get
there, you have slowed down to zero speed. You see the
tourist attractions, and then head home by the same method.\hwendpart
 %
(a) Find a formula for the time, $T$, required for the round
trip, in terms of $d$, the distance from our sun to the
star, and $a$, the magnitude of the acceleration. Note that
the acceleration is not constant over the whole trip, but
the trip can be broken up into constant-acceleration parts.\hwendpart
 %
(b) The nearest star to the Earth (other than our own sun)
is Proxima Centauri, at a distance of $d=4\times10^{16}\ \munit$.
Suppose you use an acceleration of $a=10\ \munit/\sunit^2$, just enough
to compensate for the lack of true gravity and make you feel
comfortable. How long does the round trip take, in years?\hwendpart
 %
(c) Using the same numbers for $d$ and $a$, find your
maximum speed. Compare this to the speed of light, which is
$3.0\times10^8$  \ \munit/\sunit. (Later in this course, you will learn
that there are some new things going on in physics when one
gets close to the speed of light, and that it is impossible
to exceed the speed of light. For now, though, just use the
simpler ideas you've learned so far.) \answercheck
