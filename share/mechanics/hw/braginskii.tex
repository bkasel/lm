The apparatus in figure \figref{braginskii} on page \pageref{fig:braginskii}
        had a natural period of oscillation of 5 hours and 20 minutes.
        The authors estimated, based on calculations of internal friction in the tungsten
        wire, that its $Q$ was on the order of $10^6$, but they were unable
        to measure it empirically because it would have taken years for the amplitude
        to die down by any measurable amount. Although each aluminum or platinum
        mass was really moving along an arc of a circle, any actual oscillations caused
        by a violation of the equivalence of gravitational and inertial mass
         would have been measured in millions
        of a degree, so it's a good approximation to say that each mass's motion was
        along a (very short!) straight line segment. We can also treat each mass as if it
        was oscillating separately from the others. If the principle of equivalence
        had been violated at the $10^{-12}$ level, the limit of their experiment's
        sensitivity, the sun's gravitational force on one of the 0.4-gram masses
        would have been about $3\times10^{-19}$ N, oscillating with a period of
        24 hours due to the rotation of the earth. (We ignore the inertia of the arms,
        whose total mass was only about 25\% of the total mass of the rotating assembly.)\hwendpart
        (a) Find the amplitude of the resulting
        oscillations, and determine the angle to which they would have corresponded,
        given that the radius of the balance arms was 10 cm.<% hw_answer %>\hwendpart
        (b) Show that even if their estimate of $Q$ was wildly wrong, it wouldn't have affected this
        result.
