This problem is a numerical example of the imaginary
experiment discussed on p.~\pageref{galilean-energy} regarding the
relationship between energy and relative motion. Let's say
that the pool balls both have masses of 1.00 kg. Suppose
that in the frame of reference of the pool table, the cue
ball moves at a speed of 1.00 m/s toward the eight ball,
which is initially at rest. The collision is head-on, and as
you can verify for yourself the next time you're playing
pool, the result of such a collision is that the incoming
ball stops dead and the ball that was struck takes off with
the same speed originally possessed by the incoming ball.
(This is actually a bit of an idealization. To keep things
simple, we're ignoring the spin of the balls, and we assume
that no energy is liberated by the collision as heat or
sound.) (a) Calculate the total initial kinetic energy and
the total final kinetic energy, and verify that they are
equal. (b) Now carry out the whole calculation again in the
frame of reference that is moving in the same direction that
the cue ball was initially moving, but at a speed of 0.50
m/s. In this frame of reference, both balls have nonzero
initial and final velocities, which are different from what
they were in the table's frame. [See also problem
\ref{hw:energy-frames} on p.~\pageref{hw:energy-frames}.]
