The moon doesn't really just orbit the Earth. By
Newton's third law, the moon's gravitational force on the
earth is the same as the earth's force on the moon, and the
earth must respond to the moon's force by accelerating. If
we consider the earth and moon in isolation and ignore
outside forces, then Newton's first law says their common
center of mass doesn't accelerate, i.e., the earth wobbles
around the center of mass of the earth-moon system once per
month, and the moon also orbits around this point. The
moon's mass is 81 times smaller than the earth's. Compare
the kinetic energies of the earth and moon.
m4_ifelse(__problems,1,[::],[:%
(We know that the center of mass is a kind of balance point,
so it must be closer to the earth than to the moon. In fact,
the distance from the earth to the center of mass is 1/81
of the distance from the moon to the center of mass,
which makes sense intuitively, and can be proved rigorously
using the equation on page \pageref{cm-equation}.)
:])
