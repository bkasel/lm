% meta {"stars":1}
 The following reasoning leads to an apparent paradox;
explain what's wrong with the logic. A baseball player hits
a ball. The ball and the bat spend a fraction of a second in
contact. During that time they're moving together, so their
accelerations must be equal. Newton's third law says that
their forces on each other are also equal. But $a=F/m$, so
how can this be, since their masses are unequal? (Note that
the paradox isn't resolved by considering the force of the
batter's hands on the bat. Not only is this force very small
compared to the ball-bat force, but the batter could have
just thrown the bat at the ball.)
