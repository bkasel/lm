% meta {"stars":1}
In this problem we investigate the notion of division by a vector.\\
(a) Given a nonzero vector $\vc{a}$ and a scalar $b$, suppose we wish to find
a vector $\vc{u}$ that is the solution of $\vc{a}\cdot\vc{u}=b$.
Show that the solution is not unique, and give a geometrical description
of the solution set.\hwendpart
(b) Do the same thing for the equation $\vc{a}\times\vc{u}=\vc{c}$.\hwendpart
(c) Show that the \emph{simultaneous} solution of these two equations exists
and is unique.\hwendpart
\hwremark{This is one motivation for constructing the number system called the
quaternions. For a certain period around 1900, quaternions were more popular
than the system of vectors and scalars more commonly used today. They still
have some important advantages over the scalar-vector system for certain
applications, such as avoiding a phenomenon known as gimbal lock in controlling
the orientation of bodies such as spacecraft.}
