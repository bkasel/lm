(a) A free neutron (as opposed to a neutron bound into
an atomic nucleus) is unstable, and decays radioactively
into a proton, an electron, and a particle called an
antineutrino, which fly off in three different directions.
 The masses are as follows:

\qquad\begin{tabular}{ll}
        neutron                & $1.67495\times10^{-27}$  kg\\
        proton                & $1.67265\times10^{-27}$  kg\\
        electron        & $0.00091\times10^{-27}$  kg\\
        antineutrino        & negligible\\
\end{tabular}

\noindent Find the energy released in the decay of a free neutron.\answercheck\hwendpart
(b) Neutrons and protons make up essentially all of the mass of the ordinary
matter around us. We observe that the universe around us has no free neutrons, but
lots of free protons
(the nuclei of hydrogen, which is the element that 90\% of the universe
is made of). We find neutrons only inside nuclei along with other neutrons and
protons, not on their own.

If there are processes that can convert neutrons into protons, we might imagine
that there could also be proton-to-neutron conversions, and indeed such a process
does occur sometimes in nuclei that contain both neutrons and protons:
a proton can decay into a
neutron, a positron, and a neutrino. A positron is a particle with the same
properties as an electron, except that its electrical charge is positive
(see chapter \ref{ch:electricity}). A neutrino, like an antineutrino, has negligible mass.

Although such a process
can occur within a nucleus, explain why it cannot happen to
a free proton. (If it could, hydrogen would be radioactive, and you
wouldn't exist!)
