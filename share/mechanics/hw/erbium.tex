The nucleus $^{168}\zu{Er}$ (erbium-168) contains 68 protons (which is what
makes it a nucleus of the element erbium) and 100 neutrons. It has an
ellipsoidal shape like an American football, with one long axis and two short axes that are of equal
diameter. Because this is a subatomic system, consisting of only 168 particles,
its behavior shows some clear quantum-mechanical properties. It can only have
certain energy levels, and it makes quantum leaps between these levels. Also,
its angular momentum can only have certain values, which are all multiples of
$2.109\times10^{-34}\ \kgunit\cdot\munit^2/\sunit$. The table shows some of the observed
angular momenta and energies of $^{168}\zu{Er}$, in SI units
($\kgunit\cdot\munit^2/\sunit$ and joules).\\
\hspace{50mm}\begin{tabular}{ll}
        $L\times10^{34}$        & $E\times10^{14}$\\
        0                &          0\\
    2.109        &     1.2786\\
    4.218        &     4.2311\\
    6.327        &     8.7919\\
    8.437        &     14.8731\\
    10.546        &     22.3798\\
    12.655        &     31.135\\
    14.764        &     41.206\\
    16.873        &     52.223\\
\end{tabular}\hwendpart
(a) These data can be described to a good approximation as a rigid end-over-end
rotation. Estimate a single best-fit
value for the moment of inertia from the data, and check how well the
data agree with the assumption of rigid-body rotation.<% hw_hint("erbium") %>\answercheck\hwendpart
(b) Check whether this moment of inertia is on the right order of magnitude.
The moment of inertia depends on both the size and the shape of the nucleus.
For the sake of this rough check, ignore the fact that the nucleus is not quite
spherical. To estimate its size, use the fact that a neutron or proton has a volume
of about $1\ \zu{fm}^3$ (one cubic femtometer, where $1\ \zu{fm}=10^{-15}\ \munit$),
and assume they are closely packed in the nucleus.
