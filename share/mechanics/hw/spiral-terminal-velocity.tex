% meta {"stars":2}
A bead slides down along a piece of wire that is in the shape of a helix. The helix lies on the surface of
a vertical cylinder of radius $r$, and the vertical distance between turns is $d$.\hwendpart
(a) Ordinarily when an object slides downhill under the influence of kinetic friction, the velocity-independence
of kinetic friction implies that the acceleration is constant, and therefore there is no limit to the object's
velocity. Explain the physical reason why this argument fails here, so that the bead will in fact have some
limiting velocity.\hwendpart
(b) Find the limiting velocity.\hwendpart
(c) Show that your result has the correct behavior in the limit of $r \rightarrow \infty$.
[Problem by B. Korsunsky.]\answercheck
