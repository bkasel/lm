Astronomers have recently observed stars orbiting at
very high speeds around an unknown object near the center of
our galaxy. For stars orbiting at distances of about
$10^{14}\ \munit$ from the object, the orbital velocities are about
$10^6$  m/s. Assuming the orbits are circular, estimate the
mass of the object, in units of the mass of the sun,
$2\times10^{30}$  kg. If the object was a tightly packed
cluster of normal stars, it should be a very bright source
of light. Since no visible light is detected coming from it,
it is instead believed to be a supermassive black hole.\answercheck
