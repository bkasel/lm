(a) Suppose a rotating spherical body such as a planet
has a radius $r$ and a uniform density $\rho $, and the time
required for one rotation is $T$. At the surface of the
planet, the apparent acceleration of a falling object is
reduced by the acceleration of the ground out from under it.
Derive an equation for the apparent acceleration of gravity,
$g$, at the equator in terms of $r$, $\rho$, $T$, and $G$.\answercheck\hwendpart
 %
(b) Applying your equation from a, by what fraction is
your apparent weight reduced at the equator compared to the
poles, due to the Earth's rotation?\answercheck\hwendpart
 %
(c) Using your equation from a, derive an equation giving
the value of $T$ for which the apparent acceleration of
gravity becomes zero, i.e., objects can spontaneously drift
off the surface of the planet. Show that $T$ only depends on
$\rho $, and not on $r$.\answercheck\hwendpart
 %
(d) Applying your equation from c, how long would a day
have to be in order to reduce the apparent weight of objects
at the equator of the Earth to zero? [Answer: 1.4 hours]\hwendpart
 %
(e) Astronomers have discovered objects
they called pulsars, which emit bursts of radiation at
regular intervals of less than a second. If a pulsar is to
be interpreted as a rotating sphere beaming out a natural
``searchlight'' that sweeps past the earth with each
rotation, use your equation from c to show that its
density would have to be much greater than that of ordinary matter.\hwendpart
 %
(f) Astrophysicists predicted decades ago that
certain stars that used up their sources of energy could
collapse, forming a ball of neutrons with the fantastic
density of $\sim10^{17}\ \kgunit/\munit^3$. If this is what pulsars
really are, use your equation from c to explain why no
pulsar has ever been observed that flashes with a period of
less than 1 ms or so.
