% meta {"stars":0}
You're in your car (a Honda Accord) on the freeway traveling
behind a Ford F-150 pickup truck. The truck is moving at a steady speed of
$30.0\ \munit/\sunit$, you're speeding at $40.0\ \munit/\sunit$, and you're cruising 45
meters behind the F-150. At $t=0$, the F-150 slams on his/her brakes,
and decelerates at a rate of $5.0\ \munit/\sunit^2$. You don't notice this
until $t=1.0\ \sunit$, where you begin decelerating at $10.0\ \munit/\sunit^2$.\\
%
(a) Let positive $x$ be the direction of motion, and let your position be
$x=0$ at $t=0$. Write an equation for
$x_\text{F}(t)$, the position of the Ford as a function of time.\answercheck\hwendpart
%
(b) Write an equation for $x_\text{H}(t)$ (for $t > 2\ \sunit$), the
position of the Honda as a function of time. Why isn't this
formula valid for $t < 2\ \sunit$?\answercheck\hwendpart
%
(c) By subtracting one from the other, find an expression for the
distance between the two vehicles as a function of time, $d(t)$
(valid for $t > 2\ \sunit$ until the truck stops). Does the equation
$d(t) = 0$ have any solutions? What does this tell you?\answercheck\hwendpart
%
(d) As you should have discovered in the previous part, the two
vehicles do not collide. What is the minimum distance between the two
vehicles?\answercheck\hwendpart
%
(e) At what time does this minimum distance occur?\answercheck
