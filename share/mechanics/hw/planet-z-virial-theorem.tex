A $20.0\ \kgunit$ satellite has a circular orbit with a period of $2.40$
hours and a radius of $8.00 \times 10^6\ \munit$ around planet Z. The
magnitude of the gravitational acceleration on the surface of the
planet is $8.00\ \munit/\sunit^2$.\\
%
(a) What is the mass of planet Z?\answercheck\hwendpart
%
(b) What is the radius of planet Z?\answercheck\hwendpart
%
(c) Find the KE and the PE of the satellite. What is the ratio PE/KE
(including both magnitude and sign)? You should get an integer. This is a special
case of something called the \emph{virial theorem}.\answercheck\hwendpart
%
(d) Someone standing on the surface of the planet sees a moon
orbiting the planet (a circular orbit) with a period of 20.0 days.
What is the distance between planet Z and its moon?\answercheck
