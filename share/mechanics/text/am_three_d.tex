<% begin_sec("Angular momentum in three dimensions",4,'amthreed') %>
Conservation of angular momentum produces some
surprising phenomena when extended to three dimensions.
Try the following experiment, for example. Take off your shoe,
and toss it in to the air, making it spin along its long
(toe-to-heel) axis. You should observe a nice steady pattern
of rotation. The same happens when you spin the shoe about
its shortest (top-to-bottom) axis. But something unexpected
happens when you spin it about its third (left-to-right)
axis, which is intermediate in length between the other two.
Instead of a steady pattern of rotation, you will observe
something more complicated, with the shoe changing its
orientation with respect to the rotation axis.

<% begin_sec("Rigid-body kinematics in three dimensions") %>

How do we generalize rigid-body kinematics to three
dimensions? When we wanted to generalize the kinematics of a
moving particle to three dimensions, we made the numbers 
$r$, $v$, and $a$
into vectors \vc{r}, \vc{v}, and \vc{a}. This worked because these
quantities all obeyed the same laws of vector addition. For
instance, one of the laws of vector addition is that, just
like addition of numbers, vector addition gives the same
result regardless of the order of the two quantities being
added. Thus you can step sideways 1 m to the right and then
step forward 1 m, and the end result is the same as if you
stepped forward first and then to the side. In order words,
it didn't matter whether you took 
$\Delta\vc{r}_1+\Delta\vc{r}_2$ or $\Delta\vc{r}_2+\Delta\vc{r}_1$. In
math this is called the commutative property of addition.

<%
  fig(
    'book',
    %q{%
      Performing the rotations in one order gives
      one result, 3, while reversing the order gives a different result, 5.
    },
    {
      'width'=>'wide',
      'sidecaption'=>true
    }
  )
%>

Angular motion, unfortunately doesn't have this property, as
shown in figure \figref{book}. Doing a rotation about the $x$
axis and then about $y$ gives one result, while doing
them in the opposite order gives a different result.
These operations don't ``commute,'' i.e., it makes a difference
what order you do them in.

<% marg(0) %>
<%
  fig(
    'righthandrule',
    %q{%
      The right-hand rule for associating a vector
      with a direction of rotation.
    }
  )
%>
<% end_marg %>
This means that there is in general no possible way to
construct a $\Delta\btheta$ vector. However, if you try doing the
operations shown in figure \figref{book} using small rotation, say about 10 degrees
instead of 90, you'll find that the result is nearly the
same regardless of what order you use; small rotations are
very nearly commutative. Not only that, but the result of
the two 10-degree rotations is about the same as a single,
somewhat larger, rotation about an axis that lies
symmetrically at between the $x$ and $y$ axes at 45 degree
angles to each one. This is exactly what we would expect if
the two small rotations did act like vectors whose
directions were along the axis of rotation. We therefore
define  a $\der\btheta$ vector whose magnitude is the amount of
rotation in units of radians, and whose direction is along
the axis of rotation. Actually this definition is ambiguous,
because there it could point in either direction along the
axis. We therefore use a right-hand rule as shown in figure \figref{righthandrule}
 to define the direction of the $\der\btheta$ vector, and the $\bomega$
vector, $\bomega=\der\btheta/\der t$, based on it. Aliens on planet
Tammyfaye may decide to define it using their left hands
rather than their right, but as long as they keep their
scientific literature separate from ours, there is no
problem. When entering a physics exam, always be sure to
write a large warning note on your left hand in magic marker
so that you won't be tempted to use it for the right-hand
rule while keeping your pen in your right.

<% self_check('righthanddtheta',<<-'SELF_CHECK'
Use the right-hand rule to determine the directions of the
$\\bomega$
vectors in each rotation shown in figures \\figref{book}/1 through  \\figref{book}/5.
  SELF_CHECK
  ) %>

Because the vector relationships among $\der\btheta$, $\bomega$, and $\balpha$ are
strictly analogous to the ones involving $\der\vc{r}$, $\vc{v}$, and $\vc{a}$ (with
the proviso that we avoid describing large rotations using
$\Delta\btheta$ vectors), any operation in $\vc{r}$-$\vc{v}$-$\vc{a}$ vector kinematics has
an exact analog in $\btheta$-$\bomega$-$\balpha$ kinematics.

\begin{eg}{Result of successive 10-degree rotations}
\egquestion
What is the result of two successive (positive)
10-degree rotations about the $x$ and  $y$ axes? That is, what
single rotation about a single axis would be equivalent to
executing these in succession?

\eganswer
The result is only going to be approximate, since
10 degrees is not an infinitesimally small angle, and we are
not told in what order the rotations occur. To some
approximation, however, we can add the $\Delta\btheta$ vectors in exactly
the same way we would add $\Delta \vc{r}$ vectors, so we
have
\begin{align*}
        \Delta\btheta        &\approx  \Delta\btheta_1 +  \Delta\btheta_2 \\
                &\approx  \zu{(10\ degrees)}\hat{\vc{x}} + \zu{(10\ degrees)}\hat{\vc{y}}\eqquad.
\end{align*}
This is a vector with a magnitude of  
$\sqrt{\text{(10 deg)}^2+\text{(10 deg)}^2}=\text{14\ deg}$, and it
points along an axis midway between the $x$ and $y$ axes.
\end{eg}

 % 
<% end_sec() %>
<% begin_sec("Angular momentum in three dimensions") %>
<% begin_sec("The vector cross product") %>\index{cross product}\index{vector cross product}\index{vector product, cross}
In order to expand our system of three-dimensional
kinematics to include dynamics, we will have to generalize
equations like $v_t=\omega r$,
$\tau=rF \sin\theta_{rF}$, and $L=rp \sin\theta_{rp}$, each of which
involves three quantities that we have either already
defined as vectors or that we want to redefine as vectors.
Although the first one appears to differ from the others in
its form, it could just as well be rewritten as $v_t=\omega r \sin\theta_{\omega r}$, since $\theta_{\omega r}=90\degunit$, and
$\sin\theta_{\omega r}=1$.

It thus appears that we have discovered something general
about the physically useful way to relate three vectors in a
multiplicative way: the magnitude of the result always seems
to be proportional to the product of the magnitudes of the
two vectors being ``multiplied,'' and also to the sine of
the angle between them.

Is this pattern just an accident? Actually the sine factor
has a very important physical property: it goes to zero when
the two vectors are parallel. This is a Good Thing. The
generalization of angular momentum into a three-dimensional
vector, for example, is presumably going to describe not
just the clockwise or counterclockwise nature of the motion
but also from which direction we would have to view the
motion so that it was clockwise or counterclockwise. (A
clock's hands go counterclockwise as seen from behind the
clock, and don't rotate at all as seen from above or to the
side.) Now suppose a particle is moving directly away from
the origin, so that its \vc{r} and \vc{p} vectors are parallel. It is
not going around the origin from any point of view, so its
angular momentum vector had better be zero.

Thinking in a slightly more abstract way, we would expect
the angular momentum vector to point perpendicular to the
plane of motion, just as the angular velocity vector points
perpendicular to the plane of motion. The plane of motion is
the plane containing both \vc{r} and \vc{p}, if we place the two
vectors tail-to-tail. But if \vc{r} and \vc{p} are parallel and are
placed tail-to-tail, then there are infinitely many planes
containing them both. To pick one of these planes in
preference to the others would violate the symmetry of
space, since they should all be equally good. Thus the
zero-if-parallel property is a necessary consequence of the
underlying symmetry of the laws of physics.

The following definition of a kind of vector multiplication
is consistent with everything we've seen so far, and on p.~\pageref{misc:uniquexproof}
 we'll prove that the definition is unique, i.e., if
we believe in the symmetry of space, it is essentially the
only way of defining the multiplication of two vectors to
produce a third vector:

\enlargethispage{-2\baselineskip}

\mythmhdr{Definition of the vector cross product}\\
The cross product $\vc{A}\times\vc{B}$ of two vectors $\vc{A}$ and $\vc{B}$ is defined as
follows:\\
(1) Its magnitude is defined by $|\vc{A}\times\vc{B}| = |\vc{A}| |\vc{B}| \sin\theta_{AB}$,
where $\theta_{AB}$ is the angle between $\vc{A}$ and $\vc{B}$ when they are placed
tail-to-tail.\\
(2) Its direction is along the line perpendicular to both $\vc{A}$ and $\vc{B}$.
Of the two such directions, it is the one that obeys
the right-hand rule shown in  figure \figref{righthandxprod}.\label{vectorcrossproductdef}

<% marg(30) %>
<%
  fig(
    'righthandxprod',
    %q{The right-hand rule for the direction of the vector cross product.}
  )
%>
<% end_marg %>
The name ``cross product'' refers to the  symbol, and
distinguishes it from the dot product, which acts on two
vectors but produces a scalar.

Although the vector cross-product has nearly all the
properties of numerical multiplication, e.g.,
 $\vc{A}\times(\vc{B}+\vc{C}) = \vc{A}\times\vc{B}+\vc{A}\times\vc{C}$,
it lacks the usual property of commutativity. Try applying
the right-hand rule to find the direction of the vector
cross product $\vc{B}\times\vc{A}$ using the two vectors shown in the figure.
This requires starting with a flattened hand with the four
fingers pointing along $\vc{B}$, and then curling the hand so that
the fingers point along $\vc{A}$. The only possible way to do this
is to point your thumb toward the floor, in the opposite
direction. Thus for the vector cross product we have
\begin{equation*}
        \vc{A}\times\vc{B} = -\vc{B}\times\vc{A}\eqquad,
\end{equation*}
a property known as anticommutativity. The vector cross
product is also not associative, i.e.,
 $\vc{A}\times(\vc{B}\times\vc{C})$ is usually not
the same as  $(\vc{A}\times\vc{B})\times\vc{C}$.

<% marg(20) %>
<%
  fig(
    'parallelogram',
    %q{The magnitude of the cross product is the area of the shaded parallelogram.}
  )
%>
\spacebetweenfigs
<%
  fig(
    'cyclicpermutation',
    %q{A cyclic change in the $x$, $y$, and $z$ subscripts.}
  )
%>

<% end_marg %>
A geometric interpretation of the cross product, \figref{parallelogram},  is that
if both $\vc{A}$ and $\vc{B}$ are vectors with units of distance, then the
magnitude of their cross product can be interpreted as the
area of the parallelogram they form when placed
tail-to-tail.

A useful expression for the components of the vector cross
product in terms of the components of the two vectors being
multiplied is as follows:
\begin{align*}
        (\vc{A}\times\vc{B})_x        &=  A_yB_z - B_yA_z\\
        (\vc{A}\times\vc{B})_y        &=  A_zB_x - B_zA_x\\
        (\vc{A}\times\vc{B})_z        &=  A_xB_y - B_xA_y
\end{align*}
I'll prove later that these expressions are
equivalent to the previous definition of the cross product.
Although they may appear formidable, they have a simple
structure: the subscripts on the right are the other two
besides the one on the left, and each equation is related to
the preceding one by a cyclic change in the subscripts, \figref{cyclicpermutation}.
If the subscripts were not treated in some completely
symmetric manner like this, then the definition would
provide some way to distinguish one axis from another, which
would violate the symmetry of space.

<% self_check('anticommutative',<<-'SELF_CHECK'
Show that the component equations are consistent with the
rule $\\vc{A}\\times\\vc{B} = -\\vc{B}\\times\\vc{A}$.
  SELF_CHECK
  ) %>

<% end_sec() %>
<% begin_sec("Angular momentum in three dimensions") %>
In terms of the vector cross product, we have:
\begin{align*}
        \vc{v} &= \bomega \times \vc{r}\\
        \vc{L} &= \vc{r}  \times \vc{p}\\
        \btau &= \vc{r} \times \vc{F}
\end{align*}

But wait, how do we know these equations are even correct?
For instance, how do we know that the quantity defined by $\vc{r}\times\vc{p}$
 is in fact conserved? Well, just as we saw on page \pageref{subsec:dotproduct}
that the dot product is unique (i.e., can only be defined in one way while observing
rotational invariance), the cross product is also unique, as proved
on page \pageref{misc:uniquexproof}. If $\vc{r}\times\vc{p}$ was not conserved, then there
could not be any generally conserved quantity that would
reduce to our old definition of angular momentum in the
special case of plane rotation. This doesn't prove conservation of angular momentum ---
only experiments can prove that --- but it does prove that if angular momentum
is conserved in three dimensions, there is only one possible way to generalize from
two dimensions to three.

<% marg(25) %>
<%
  fig(
    'precessiona',
    %q{The position and momentum vectors of an atom in the spinning top.}
  )
%>
\spacebetweenfigs
<%
  fig(
    'precessionb',
    %q{The right-hand rule for the atom's contribution to the angular momentum.}
  )
%>
<% end_marg %>
\begin{eg}{Angular momentum of a spinning top}
As an
illustration, we consider the angular momentum of a spinning
top. Figures \figref{precessiona} and \figref{precessionb} show 
the use of the vector cross product to
determine the contribution of a representative atom to the
total angular momentum. Since every other atom's angular
momentum vector will be in the same direction, this will
also be the direction of the total angular momentum of the
top. This happens to be rigid-body rotation, and perhaps not
surprisingly, the angular momentum vector is along the same
direction as the angular velocity vector.
\end{eg}

Three
important points are illustrated by this example:
(1)
When we do the full three-dimensional treatment of angular
momentum, the ``axis'' from which we measure the position
vectors is just an arbitrarily chosen point. If this had not
been rigid-body rotation, we would not even have been able
to identify a single line about which every atom
circled.
(2) Starting from figure \figref{precessiona}, we had to
rearrange the vectors to get them tail-to-tail before
applying the right-hand rule. If we had attempted to apply
the right-hand rule to figure \figref{precessiona}, the direction of the
result would have been exactly the opposite of the correct
answer.
(3) The equation $\vc{L}=\vc{r}\times\vc{p}$ cannot be applied all at
once to an entire system of particles. The total momentum of
the top is zero, which would give an erroneous result of
zero angular momentum (never mind the fact that \vc{r} is not
well defined for the top as a whole).

Doing the right-hand rule like this requires some practice.
I urge you to make models like \figref{precessionb} out of rolled
up pieces of paper and to practice with the model in various
orientations until it becomes natural.

\begin{eg}{Precession}\label{eg:precession}
Figure \figref{precessionc} shows a counterintuitive
example of the concepts we've been discussing. One expects
the torque due to gravity to cause the top to flop down.
Instead, the top remains spinning in the horizontal plane,
but its axis of rotation starts moving in the direction
shown by the shaded arrow. This phenomenon is called
precession.
Figure \figref{precessiond} shows that the torque due to
gravity is out of the page. (Actually we should add up all
the torques on all the atoms in the top, but the qualitative
result is the same.)
Since torque is the rate of change
of angular momentum, $\btau=\der\vc{L}/\der t$, 
the $\Delta\vc{L}$ vector must be in
the same direction as the torque (division by a positive
scalar doesn't change the direction of the vector). As shown
in \figref{precessione}, this causes the angular momentum vector to
twist in space without changing its magnitude.
\end{eg}

<% marg(60) %>

<%
  fig(
    'precessionc',
    %q{A top is supported at its tip by a pinhead. (More practical devices to demonstrate this would use a double bearing.)}
  )
%>
\spacebetweenfigs
<%
  fig(
    'precessiond',
    %q{The torque made by gravity is in the horizontal plane.}
  )
%>
\spacebetweenfigs
<%
  fig(
    'precessione',
    %q{The $\Delta\vc{L}$ vector is in the same direction as the torque, out of the page.}
  )
%>
<% end_marg %>
For similar reasons, the Earth's axis precesses once every
26,000 years (although not through a great circle, since the
angle between the axis and the force isn't 90 degrees as
in figure \figref{precessionc}). This precession is due to a
torque exerted by the moon. If the Earth was a perfect
sphere, there could be no precession effect due to symmetry.
However, the Earth's own rotation causes it to be slightly
flattened (oblate) relative to a perfect sphere, giving it
``love handles'' on which the moon's gravity can act. The
moon's gravity on the nearer side of the equatorial bulge 
is stronger, so the torques do not cancel out perfectly. 
Presently the earth's axis very nearly lines up with the
star Polaris, but in 12,000 years, the pole star will be Vega
instead.

\begin{eg}{The frisbee}
The flow of the air over a flying frisbee generates lift, and the lift
at the front and back of the frisbee isn't necessarily balanced.
If you throw a frisbee without rotating it, as if you were
shooting a basketball with two hands, you'll find that it
pitches, i.e., its nose goes either up or down. When I do this
with my frisbee, it goes nose down, which apparently means
that the lift at the back of the disc is greater than the lift at
the front. The two torques are unbalanced, resulting in a total
torque that points to the left.

The way you actually throw a frisbee is with one hand, putting a lot
of spin on it. If you throw backhand, which is how most people first
learn to do it, the angular momentum vector points down (assuming you're right-handed). On my
frisbee, the aerodynamic torque to the left would therefore tend
to make the angular momentum vector precess in the clockwise direction
as seen by the thrower. This would cause the disc to roll to the right,
and therefore follow a curved trajectory. Some specialized discs, used in the
sport of disc golf, are actually designed intentionally to show this behavior;
they're known as ``understable'' discs. However, the typical frisbee that most
people play with is designed to be stable: as the disc rolls to one side, the airflow
around it is altered in way that tends to bring the disc back into level flight. Such
a disc will therefore tend to fly in a straight line, provided that it is thrown
with enough angular momentum.
\end{eg}

\begin{eg}{Finding a cross product by components}\label{eg:xprodcomps}
\egquestion
What is the torque produced by a force given by
$\hat{\vc{x}}+2\hat{\vc{y}}+3\hat{\vc{z}}$
 (in units of Newtons) acting on a point whose radius
vector is $4\hat{\vc{x}}+5\hat{\vc{y}}$ (in meters)?

\eganswer
It's helpful to make a table of the components as
shown in the figure. The results are
\begin{alignat*}{2}
        \tau_x        &=   r_{y} F_{z} -  F_y r_{z} 
                        =&  15\ \nunit\unitdot\munit        \\
        \tau_y        &=   r_{z} F_x -  F_{z} r_{x} 
                        =&  - 12\ \nunit\unitdot\munit        \\
        \tau_z        &=   r_x F_{y} -  F_{x} r_y 
                        =&   3\ \nunit\unitdot\munit
\end{alignat*}
\end{eg}

<% marg(50) %>
<%
  fig(
    'rftable',
    %q{Example \ref{eg:xprodcomps}.}
  )
%>
<% end_marg %>

\begin{eg}{Torque and angular momentum}\label{eg:torqueproof}\index{torque!related to force}
In this example, we prove explicitly the consistency of the
equations involving torque and angular momentum that we
proved above based purely on symmetry.
Starting from the definition of torque, we
have
\begin{align*}
        \btau                &= \frac{\der\vc{L}}{\der t} \\
                        &= \frac{\der}{\der t}\sum_{i} \vc{r}_i\times\vc{p}_i \\
                        &= \sum_{i} \frac{\der}{\der t}(\vc{r}_i\times\vc{p}_{i})\eqquad.
\end{align*}
The derivative of a cross product can be evaluated in the same way
as the derivative of an ordinary scalar product:
\begin{equation*}
                \btau = \sum_i\left[
                                \left(\frac{\der\vc{r}_{i}}{\der t}\times\vc{p}_i\right)
                                +\left(\vc{r}_i\times\frac{\der\vc{p}_{i}}{\der t}\right)
                        \right]
\end{equation*}
The first term is zero for each particle, since the velocity
vector is parallel to the momentum vector. The derivative
appearing in the second term is the force acting on
the particle, so
\begin{equation*}
        \btau                =    \sum_i \vc{r}_{i}\times\vc{F}_i\eqquad,
\end{equation*}
which is the relationship we set out to prove.
\end{eg}

<% end_sec() %>
<% end_sec() %>
<% begin_sec("Rigid-body dynamics in three dimensions") %>
The student who is not madly in love with mathematics may
wish to skip the rest of this section after absorbing the
statement that, for a typical, asymmetric object, the angular
momentum vector and the angular velocity vector need not be
parallel. That is, only for a body that possesses 
symmetry about the rotation axis is it true that $\vc{L}=I\bomega$ (the
rotational equivalent of $\vc{p}=m\vc{v}$) for some scalar $I$.

Let's evaluate the angular momentum of a rigidly rotating
system of particles:
\begin{align*}
        \vc{L}        &= \sum_i \vc{r}_i \times \vc{p}_i \\
                        &= \sum_i m_i \vc{r}_i \times \vc{v}_i \\
                        &= \sum_i m_i \vc{r}_i \times (\bomega \times \vc{r}_i)
\end{align*}
An important mathematical skill is to know when to give up
and back off. This is a complicated expression, and there is
no reason to expect it to simplify and, for example, take
the form of a scalar multiplied by $\omega$. Instead we examine its
general characteristics. If we expanded it using the
equation that gives the components of a vector cross
product, every term would have one of the $\omega$ components
raised to the first power, multiplied by a bunch of other
stuff. The most general possible form for the result
is
\begin{align*}
                L_x        &=  I_{xx}\omega_x + I_{xy}\omega_y + I_{xz}\omega_z\\
                L_y        &=  I_{yx}\omega_x + I_{yy}\omega_y + I_{yz}\omega_z\\
                L_z        &=  I_{zx}\omega_x + I_{zy}\omega_y + I_{zz}\omega_z\eqquad,
\end{align*}
which you may recognize as a case of matrix multiplication.
The moment of inertia is not a scalar, and not a
three-component vector. It is a matrix specified by nine
numbers, called its matrix elements.

The elements of the moment of inertia matrix will depend on
our choice of a coordinate system. In general, there will be
some special coordinate system, in which the matrix has a
simple diagonal form:
\begin{alignat*}{2}
        L_x        &=  I_{xx}\omega_x & & \\
        L_y        &=   & I_{yy}\omega_y & \\
        L_z        &=  & & I_{zz}\omega_z\eqquad.
\end{alignat*}

The three special axes that cause this simplification are
called the principal axes of the object, and the
corresponding coordinate system is the principal axis
system. For symmetric shapes such as a rectangular box or an
ellipsoid, the principal axes lie along the intersections of
the three symmetry planes, but even an asymmetric body has
principal axes.

We can also generalize the plane-rotation equation
$K=(1/2)I\omega^2$ to three dimensions as follows:
\begin{align*}
        K        &= \sum_i \frac{1}{2} m_i v_i^2 \\
                & = \frac{1}{2} \sum_i m_i (\bomega \times \vc{r}_i) \cdot (\bomega \times \vc{r}_i) 
\end{align*}
We want an equation involving the moment of inertia, and
this has some evident similarities to the sum we originally
wrote down for the moment of inertia. To massage it into the
right shape, we need the vector identity 
$(\vc{A}\times\vc{B})\cdot\vc{C}= (\vc{B}\times\vc{C}) \cdot \vc{A}$,
which we state without proof. We then write
\begin{align*}
        K        & = \frac{1}{2} \sum_i m_i 
                        \left[ \vc{r}_i \times (\bomega \times \vc{r}_i)
                        \right] \cdot \bomega \\
                &= \frac{1}{2} \bomega \cdot \sum_i m_i\vc{r}_i \times (\bomega \times \vc{r}_i)  \\
                &= \frac{1}{2} \vc{L} \cdot \bomega
\end{align*}

As a reward for all this hard work, let's analyze the
problem of the spinning shoe that I posed at the beginning
of the chapter. The three rotation axes referred to there
are approximately the principal axes of the shoe. While the
shoe is in the air, no external torques are acting on it, so
its angular momentum vector must be constant in magnitude
and direction. Its kinetic energy is also constant. That's in the room's
frame of reference, however. The principal axis frame is
attached to the shoe, and tumbles madly along with it. In
the principal axis frame, the kinetic energy and the magnitude of the
angular momentum stay constant, but the actual direction of
the angular momentum need not stay fixed (as you saw in the
case of rotation that was initially about the
intermediate-length axis). Constant $|\vc{L}|$ gives
\begin{equation*}
        L_x^2 + L_y^2+ L_z^2   =   \text{constant}\eqquad.
\end{equation*}

In the principal axis frame, it's easy to solve for the components of $\bomega$ in
terms of the components of $\vc{L}$, so we eliminate $\omega$ from the expression $2K=\vc{L}\cdot\bomega$, giving        
\begin{equation*}
   \frac{1}{I_{xx}}L_x^2 +  \frac{1}{I_{yy}}L_y^2 +  \frac{1}{I_{zz}}L_z^2 =   \text{constant \#2}  .
\end{equation*}

The first equation is the equation of a sphere in the three
dimensional space occupied by the angular momentum vector,
while the second one is the equation of an ellipsoid. The
top figure corresponds to the case of rotation about the
shortest axis, which has the greatest moment of inertia
element. The intersection of the two surfaces consists only
of the two points at the front and back of the sphere. The
angular momentum is confined to one of these points, and
can't change its direction, i.e., its orientation with
respect to the principal axis system, which is another way
of saying that the shoe can't change its orientation with
respect to the angular momentum vector. In the bottom
figure, the shoe is rotating about the longest axis. Now the
angular momentum vector is trapped at one of the two points
on the right or left. In the case of rotation about the axis
with the intermediate moment of inertia element, however,
the intersection of the sphere and the ellipsoid is not just
a pair of isolated points but the curve shown with the
dashed line. The relative orientation of the shoe and the
angular momentum vector can and will change.
<% marg(104) %>
<%
  fig(
    'shoe',
    %q{%
      Visualizing surfaces of constant energy and angular
      momentum in $L_x$-$L_y$-$L_z$ space.
    }
  )
%>
<% end_marg %>

One application of the moment of inertia tensor is to video
games that simulate car racing or flying airplanes.

One more exotic example has to do with
nuclear physics. Although you have probably visualized
atomic nuclei as nothing more than featureless points, or
perhaps tiny spheres, they are often ellipsoids with one
long axis and two shorter, equal ones. Although a spinning
nucleus normally gets rid of its angular momentum via gamma
ray emission within a period of time on the order of
picoseconds, it may happen that a deformed nucleus gets into
a state in which has a large angular momentum is along its
long axis, which is a very stable mode of rotation. Such
states can live for seconds or even years! (There is more to
the story --- this is the topic on which I wrote my Ph.D.
thesis --- but the basic insight applies even though the full
treatment requires fancy quantum mechanics.)
<% marg(0) %>
<%
  fig(
    'explorer-one',
    'The Explorer I satellite.'
  )
%>
<% end_marg %>

Our analysis has so far assumed that the kinetic energy
of rotation energy can't be converted into other forms
of energy such as heat, sound, or vibration. When this
assumption fails, then rotation about the axis of least
moment of inertia becomes unstable, and will eventually
convert itself into rotation about the axis whose moment
of inertia is greatest. This happened to the U.S.'s first
artificial satellite, Explorer I, launched in 1958. 
Note the long, floppy antennas, which tended to dissipate kinetic energy into vibration.
It had been designed to spin about its minimimum-moment-of-inertia
axis, but almost immediately, as soon as it was in space, it began spinning end over
end. It was nevertheless able to carry out its science mission, which didn't
depend on being able to maintain a stable orientation, and
it discovered the Van Allen radiation belts.
<% end_sec %>
<% end_sec('amthreed') %>
