\begin{summary}

\begin{vocab}

\vocabitem{center of mass}{the balance point of an object}

\vocabitem{velocity}{the rate of change of position; the slope of the
tangent line on an $x-t$ graph.}

\end{vocab}

\begin{notation}

\notationitem{$x$}{a point in space}
\notationitem{$t$}{a point in time, a clock reading}
\notationitem{$\Delta$}{``change in;'' the value of a variable afterwards minus its value before}
\notationitem{$\Delta x$}{a distance, or more precisely a change in $x$, which
may be less than the distance traveled; its plus or minus
sign indicates direction}
\notationitem{$\Delta t$}{a duration of time}
\notationitem{$v$}{velocity}
\notationitem{$v_{AB}$}{the velocity of object A relative to object B}
\end{notation}

\begin{othernotation}

\notationitem{displacement}{a name for the symbol $\Delta x$}
\notationitem{speed}{the absolute value of the velocity, i.e., the velocity
stripped of any information about its direction}
\end{othernotation}

\begin{summarytext}

An object's center of mass is the point at which it can be
balanced. For the time being, we are studying the mathematical
description only of the motion of an object's center of mass
in cases restricted to one dimension. The motion of an
object's center of mass is usually far simpler than the
motion of any of its other parts.

It is important to distinguish location, $x$, from distance,
$\Delta x$, and clock reading, $t$, from time interval
$\Delta t$. When an object's $x-t$ graph is linear, we
define its velocity as the slope of the line, 
$\Delta x/\Delta t$. When the graph is curved, we generalize the
definition so that the velocity is the
m4_ifelse(__me,1,[:%
derivative $\der x/\der t$.
:],[:%
slope of the tangent
line at a given point on the graph.
:])

Galileo's principle of inertia states that no force is
required to maintain motion with constant velocity in a
straight line, and absolute motion does not cause any
observable physical effects. Things typically tend to reduce
their velocity relative to the surface of our planet only
because they are physically rubbing against the planet (or
something attached to the planet), not because there is
anything special about being at rest with respect to the
earth's surface. When it seems, for instance, that a force
is required to keep a book sliding across a table, in fact
the force is only serving to cancel the contrary force of friction.

Absolute motion is not a well-defined concept, and if two
observers are not at rest relative to one another they will
disagree about the absolute velocities of objects. They
will, however, agree about relative velocities. If object A
is in motion relative to object $B$, and $B$ is in motion
relative to $C$, then A's velocity relative to $C$ is given
by $v_{AC}=v_{AB}+v_{BC}$. Positive and negative signs are
used to indicate the direction of an object's motion.

\end{summarytext}

\end{summary}
