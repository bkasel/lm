\index{angular momentum!introduction to}

``Sure, and maybe the sun won't come up tomorrow.'' Of
course, the sun only appears to go up and down because the
earth spins, so the cliche should really refer to the
unlikelihood of the earth's stopping its rotation abruptly
during the night. Why can't it stop? It wouldn't violate
conservation of momentum, because the earth's rotation
doesn't add anything to its momentum. While California spins
in one direction, some equally massive part of India goes
the opposite way, canceling its momentum. A halt to Earth's
rotation would entail a drop in kinetic energy, but that
energy could simply be converted into some other form, such as heat.

Other examples along these lines are not hard to find. A
hydrogen atom spins at the same rate for billions of years.
A high-diver who is rotating when he comes off the board
does not need to make any physical effort to continue
rotating, and indeed would be unable to stop rotating
before he hit the water.

These observations have the hallmarks of a conservation law:

\oneofaseriesofpoints{A closed system is involved.}{Nothing is making an effort to
twist the earth, the hydrogen atom, or the high-diver. They
are isolated from rotation-changing influences, i.e.,
they are closed systems.}

\oneofaseriesofpoints{Something remains unchanged.}{There appears to be a numerical
quantity for measuring rotational motion such that the total
amount of that quantity remains constant in a closed system.}

\oneofaseriesofpoints{Something can be transferred back and forth without changing
the total amount.}{In figure \figref{longjump},
the jumper wants to get his feet out in front of
him so he can keep from doing a ``face plant'' when he
lands. Bringing his feet forward would involve a certain
quantity of counterclockwise rotation, but he didn't start
out with any rotation when he left the ground. Suppose we
consider counterclockwise as positive and clockwise as
negative. The only way his legs can acquire some positive
rotation is if some other part of his body picks up an equal
amount of negative rotation. This is why he swings his arms
up behind him, clockwise.}

<%
  fig(
    'longjump',
    %q{An early photograph of an old-fashioned long-jump.},
    {
      'width'=>'fullpage'
    }
  )
%>

What numerical measure of rotational motion is conserved?
Car engines and old-fashioned LP records have speeds of
rotation measured in rotations per minute (r.p.m.), but the
number of rotations per minute (or per second) is not a
conserved quantity. A twirling figure skater, for instance,
can pull her arms in to increase her r.p.m.'s. The first
section of this chapter deals with the numerical definition
of the quantity of rotation that results in a valid conservation law.

<% begin_sec("Conservation of angular momentum",0) %>

When most people think of rotation, they think of a solid
object like a wheel rotating in a circle around a fixed
point. Examples of this type of rotation, called \index{rigid
rotation!defined}rigid rotation or rigid-body rotation,
include a spinning top, a seated child's swinging leg, and a
helicopter's spinning propeller. Rotation, however, is a
much more general phenomenon, and includes noncircular
examples such as a comet in an elliptical orbit around the
sun, or a cyclone, in which the core completes a circle more
quickly than the outer parts.

If there is a numerical measure of rotational motion that is
a conserved quantity, then it must include nonrigid cases
like these, since nonrigid rotation can be traded back and
forth with rigid rotation. For instance, there is a trick
for finding out if an egg is raw or hardboiled. If you spin
a hardboiled egg and then stop it briefly with your finger,
it stops dead. But if you do the same with a raw egg, it
springs back into rotation because the soft interior was
still swirling around within the momentarily motionless
shell. The pattern of flow of the liquid part is presumably
very complex and nonuniform due to the asymmetric shape of
the egg and the different consistencies of the yolk and the
white, but there is apparently some way to describe the
liquid's total amount of rotation with a single number, of
which some percentage is given back to the shell when you release it.

The best strategy is to devise a way of defining the amount
of rotation of a single small part of a system. The amount
of rotation of a system such as a cyclone will then be
defined as the total of all the contributions from
its many small parts.

<% marg(70) %>
<%
  fig(
    'putty-door',
    %q{%
      An overhead view of a piece of putty being
      thrown at a door. Even though the putty is neither spinning
      nor traveling along a curve, we must define it as having some
      kind of ``rotation'' because it is able to make the door rotate.
    }
  )
%>
\spacebetweenfigs
<%
  fig(
    'putty-angular',
    %q{%
      As seen by someone standing at the axis, the
      putty changes its angular position. We therefore define it as having
      angular momentum.
    }
  )
%>

<% end_marg %>
The quest for a conserved quantity of rotation even requires
us to broaden the rotation concept to include cases where
the motion doesn't repeat or even curve around. If you throw
a piece of putty at a door, the door will recoil and start
rotating. The putty was traveling straight, not in a circle,
but if there is to be a general conservation law that can
cover this situation, it appears that we must describe the
putty as having had some ``rotation,'' which it then gave up
to the door. The best way of thinking about it is to
attribute rotation to any moving object or part of an object
that changes its angle in relation to the axis of rotation.
In the putty-and-door example, the hinge of the door is the
natural point to think of as an axis, and the putty changes
its angle as seen by someone standing at the hinge. For this
reason, the conserved quantity we are investigating is
called \index{angular momentum!defined}\emph{angular\/} momentum.
The symbol for angular momentum can't be $a$ or $m$,
since those are used for acceleration and mass, so the
symbol $L$ is arbitrarily chosen instead.

Imagine a 1-kg blob of putty, thrown at the door at a speed
of 1 m/s, which hits the door at a distance of 1 m from
the hinge. We define this blob to have 1 unit of angular
momentum. When it hits the door, the door
will recoil and start rotating.
 We can use the speed at which the door recoils as a measure
of the angular momentum the blob brought in.\footnote{We assume that the door is much
 more massive than the blob. Under this assumption, the speed at which the
 door recoils is much less than the original speed of the blob, so the blob
 has lost essentially all its angular momentum, and given it to the door.}

Experiments show, not surprisingly, that a 2-kg blob thrown
in the same way makes the door rotate twice as fast, so the
angular momentum of the putty blob must be proportional to mass,
\begin{equation*}
 L\propto m\eqquad.
\end{equation*}

Similarly, experiments show that doubling the velocity of
the blob will have a doubling effect on the result, so its
angular momentum must be proportional to its velocity as well,

\begin{equation*}
 L\propto mv\eqquad.
\end{equation*}

You have undoubtedly had the experience of approaching a
closed door with one of those bar-shaped handles on it and
pushing on the wrong side, the side close to the hinges. You
feel like an idiot, because you have so little leverage that
you can hardly budge the door. The same would be true with
the putty blob. Experiments would show that the amount of
rotation the blob can give to the door is proportional to
the distance, $r$, from the axis of rotation, so angular
momentum must also be proportional to $r$,
\begin{equation*}
 L\propto mvr\eqquad.
\end{equation*}

<% marg(100) %>
<%
  fig(
    'putty-radial',
    %q{%
      A putty blob thrown directly at the axis has no
      angular motion, and therefore no angular momentum. It will not cause the
      door to rotate.
    }
  )
%>
\spacebetweenfigs
<%
  fig(
    'putty-vperp',
    %q{%
      Only the component of the velocity vector
      perpendicular to the dashed line
      should be counted into the definition of angular momentum.
    }
  )
%>

<% end_marg %>
We are almost done, but there is one missing ingredient. We
know on grounds of symmetry that a putty ball thrown
directly inward toward the hinge will have no angular
momentum to give to the door. After all, there would not
even be any way to decide whether the ball's rotation was
clockwise or counterclockwise in this situation. It is
therefore only the component of the blob's velocity vector
perpendicular to the door that should be counted in
its angular momentum,
                \index{angular momentum!definition}

\begin{equation*}
L = m v_{\perp}r\eqquad.
\end{equation*}
More generally, $v_{\perp}$ should be thought of as the component of
the object's velocity vector that is perpendicular to the
line joining the object to the axis of rotation.

We find that this equation agrees with the definition of the
original putty blob as having one unit of angular momentum,
and we can now see that the units of angular momentum are
$(\kgunit\unitdot\munit/\sunit)\unitdot\munit$, i.e., $\kgunit\unitdot\munit^2/\sunit$.
This gives us a way of
calculating the angular momentum of any material object or
any system consisting of material objects:

\begin{important}[angular momentum of a material object]
The angular momentum of a moving particle is
\begin{equation*}
L = mv_{\perp}r\eqquad,
\end{equation*}
where $m$ is its mass, $v_{\perp}$ is the component of its velocity
vector perpendicular to the line joining it to the axis of
rotation, and $r$ is its distance from the axis.
Positive and negative signs are used to describe opposite directions
of rotation.

\quad The angular momentum of a finite-sized object or a system of
many objects is found by dividing it up into many small parts,
applying the equation to each part, and adding to find the total
amount of angular momentum.
\end{important}
Note that $r$ is not necessarily the radius of a circle. (As
implied by the qualifiers, matter isn't the only thing that
can have angular momentum. Light can also have angular
momentum, and the above equation would not apply to light.)

<% marg(0) %>
<%
  fig(
    'figure-skater',
    %q{%
      A figure skater pulls in her arms so
      that she can execute a spin more rapidly.
    }
  )
%>
<% end_marg %>
Conservation of angular momentum has been verified over and
over again by experiment, and is now believed to be one of
the three most fundamental principles of physics, along with
conservation of energy and momentum.

\begin{eg}{A figure skater pulls her arms in}
When a figure skater is twirling, there is very little
friction between her and the ice, so she is essentially a
closed system, and her angular momentum is conserved. If she
pulls her arms in, she is decreasing $r$ for all the atoms
in her arms. It would violate conservation of angular
momentum if she then continued rotating at the same speed,
i.e., taking the same amount of time for each revolution, because her
arms' contributions to her angular momentum would have
decreased, and no other part of her would have increased its
angular momentum. This is impossible because it would
violate conservation of angular momentum. If her total
angular momentum is to remain constant, the decrease in $r$
for her arms must be compensated for by an overall increase
in her rate of rotation. That is, by pulling her arms in,
she substantially reduces the time for each rotation.
\end{eg}

\vfill\pagebreak[4]

<%
  fig(
    'viola-at-frog',
    %q{Example \ref{eg:viola-at-frog}.},
    {
      'width'=>'wide'
    }
  )
%>

\begin{eg}{Changing the axis}\label{eg:viola-at-frog}
An object's angular momentum can be different depending on
the axis about which it rotates. Figure \ref{fig:viola-at-frog}
shows two double-exposure photographs
a viola player tipping the bow in order to cross from one
string to another. Much more angular momentum is required
when playing near the bow's handle, called the frog, as
in the panel on the right; not only are most
of the atoms in the bow at greater distances, $r$,
from the axis of rotation, but the ones in the tip also
have more momentum, $p$. It is difficult for the player to
quickly transfer a large angular momentum into the bow, and then
transfer it back out just as quickly. (In the language of section
\ref{sec:torque}, large torques are required.) This is one of the reasons
that string players tend to stay near the middle of the bow
as much as possible.
\end{eg}

<% marg(57) %>
<%
  fig(
    'tidal-slowdown',
    %q{%
      Example \ref{eg:tidal-slowdown}. A view of the earth-moon system from
      above the north pole. All distances
      have been highly distorted for legibility. The earth's rotation is
      counterclockwise from this point of view (arrow).
      The moon's gravity creates a bulge on
      the side near it, because its gravitational pull is stronger there, and an
      ``anti-bulge'' on the far side, since its
      gravity there is weaker. For simplicity,
      let's focus on the tidal bulge closer to
      the moon. Its frictional force is trying
      to slow down the earth's rotation, so
      its force on the earth's solid crust is
      toward the bottom of the figure. By
      Newton's third law, the crust must thus
      make a force on the bulge which is
      toward the top of the figure. This
      causes the bulge to be pulled forward
      at a slight angle, and the bulge's gravity therefore pulls the moon forward,
      accelerating its orbital motion about
      the earth and flinging it outward.
    }
  )
%>
<% end_marg %>

\begin{eg}{Earth's slowing rotation and the receding moon}\label{eg:tidal-slowdown}
As noted in chapter 1, the earth's rotation is actually
slowing down very gradually, with the kinetic energy being
dissipated as heat by friction between the land and the
tidal bulges raised in the seas by the earth's gravity. Does
this mean that angular momentum is not really perfectly
conserved? No, it just means that the earth is not quite a
closed system by itself. If we consider the earth and moon
as a system, then the angular momentum lost by the earth
must be gained by the moon somehow. In fact very precise
measurements of the distance between the earth and the moon
have been carried out by bouncing laser beams off of a
mirror left there by astronauts, and these measurements show
that the moon is receding from the earth at a rate of 4
centimeters per year! The moon's greater value of $r$ means
that it has a greater angular momentum, and the increase
turns out to be exactly the amount lost by the earth. In the
days of the dinosaurs, the days were significantly shorter,
and the moon was closer and appeared bigger in the sky.

But what force is causing the moon to speed up, drawing it
out into a larger orbit? It is the gravitational forces of
the earth's tidal bulges. The effect is described qualitatively
in the caption of the figure. The result would obviously be
extremely difficult to calculate directly, and this is one
of those situations where a conservation law allows us to
make precise quantitative statements about the outcome of a
process when the calculation of the process itself would be
prohibitively complex.
\end{eg}

<% begin_sec("Restriction to rotation in a plane") %>

Is angular momentum a vector, or is it a scalar?  On
p.~\pageref{define-vector}, we defined the distinction between a
vector and a scalar in terms of the quantity's behavior when rotated.
If rotation doesn't change it, it's a scalar. If rotation affects it
in the same way that it would affect an arrow, then it's a vector.
Using these definitions, figure \figref{man-on-bike-flipping} shows
that angular momentum cannot be a scalar. 

<%
  fig(
    'man-on-bike-flipping',
    %q{Angular momentum is not a scalar. If we turn the picture around, the angular momentum does change:
       the counterclockwise motion of the wheels becomes clockwise from our new point of view.},
    {
      'width'=>'wide',
      'sidecaption'=>true
    }
  )
%>

It turns out that there is a way of
defining angular momentum as a vector, but 
m4_ifelse(__lm_series,1,[:in this book:],[:until section \ref{sec:amthreed}:])
the examples will be confined to a single plane of rotation, i.e.,
effectively two-dimensional situations. In this special case, we can
choose to visualize the plane of rotation from one side or the other,
and to define clockwise and counterclockwise rotation as having
opposite signs of angular momentum. 

Figure \figref{can-rolling-with-axis} shows a can rolling down a board.
Although the can is three-dimensional, we can view it from the side and project out
the third dimension, reducing the motion to rotation in a plane.
This means that the axis is a \emph{point}, even though the word ``axis''
often connotes a line in students' minds, as in an $x$ or $y$ axis.

<%
  fig(
    'can-rolling-with-axis',
    %q{We reduce the motion to rotation in a plane, and the axis is then a point.},
    {
      'width'=>'wide',
      'sidecaption'=>true
    }
  )
%>

\startdq

\begin{dq}
Conservation of plain old momentum, $p$, can be thought of
as the greatly expanded and modified descendant of Galileo's
original principle of inertia, that no force is required to
keep an object in motion. The principle of inertia is
counterintuitive, and there are many situations in which it
appears superficially that a force \emph{is} needed to
maintain motion, as maintained by Aristotle. Think of a
situation in which conservation of angular momentum, $L$,
also seems to be violated, making it seem incorrectly that
something external must act on a closed system to keep its
angular momentum from ``running down.''
\end{dq}

<% end_sec() %>
<% end_sec() %>
<% begin_sec("Angular momentum in planetary motion",3) %>
\index{angular momentum!related to area swept out}

We now discuss the application of conservation of angular
momentum to planetary motion, both because of its intrinsic
importance and because it is a good way to develop a visual
intuition for angular momentum.

\index{Kepler!law of equal areas}Kepler's law of equal areas
states that the area swept out by a planet in a certain
length of time is always the same. Angular momentum had not
been invented in Kepler's time, and he did not even know the
most basic physical facts about the forces at work. He
thought of this law as an entirely empirical and unexpectedly
simple way of summarizing his data, a rule that succeeded in
describing and predicting how the planets sped up and slowed
down in their elliptical paths. It is now fairly simple,
however, to show that the equal area law amounts to a
statement that the planet's angular momentum stays constant.

<% marg(50) %>
<%
  fig(
    'area-swept-out',
    %q{%
      The planet's angular momentum is related to the
      rate at which it sweeps out area.
    }
  )
%>
<% end_marg %>
There is no simple geometrical rule for the area of a pie
wedge cut out of an ellipse, but if we consider a very short
time interval, as shown in figure \figref{area-swept-out}, the shaded shape
swept out by the planet is very nearly a triangle. We do
know how to compute the area of a triangle. It is one half
the product of the base and the height:
\begin{equation*}
                \text{area}         =    \frac{1}{2}bh\eqquad.
\end{equation*}

We wish to relate this to angular momentum, which contains
the variables $r$ and $v_{\perp}$ . If we consider the sun to be the
axis of rotation, then the variable $r$ is identical to the
base of the triangle, $r=b$. Referring to the magnified
portion of the figure, $v_{\perp}$ can be related to $h$, because
the two right triangles are similar:
\begin{equation*}
  \frac{h}{\text{distance traveled}} = \frac{v_\perp}{|\vc{v}|}
\end{equation*}
The area can thus be rewritten as
\begin{equation*}
                \text{area}         =    \frac{1}{2}r\frac{v_\perp(\text{distance traveled})}{|\vc{v}|}\eqquad.
\end{equation*}
The distance traveled equals $|\vc{v}|\Delta t$, so this simplifies to
\begin{equation*}
                \text{area}         =   \frac{1}{2}rv_\perp \Delta t\eqquad.
\end{equation*}
We have found the following relationship between angular
momentum and the rate at which area is swept out:
\begin{equation*}
                L         =     2m\frac{\text{area}}{\Delta t}\eqquad.
\end{equation*}
The factor of 2 in front is simply a matter of convention,
since any conserved quantity would be an equally valid
conserved quantity if you multiplied it by a constant. The
factor of $m$ was not relevant to Kepler, who did not know
the planets' masses, and who was only describing the motion
of one planet at a time.

We thus find that Kepler's equal-area law is equivalent to a
statement that the planet's angular momentum remains
constant. But wait, why should it remain constant? --- the
planet is not a closed system, since it is being acted on by
the sun's gravitational force. There are two valid answers.
The first is that it is actually the total angular momentum
of the sun plus the planet that is conserved. The sun,
however, is millions of times more massive than the typical
planet, so it accelerates very little in response to the
planet's gravitational force. It is thus a good approximation
to say that the sun doesn't move at all, so that no angular
momentum is transferred between it and the planet.

The second answer is that to change the planet's angular
momentum requires not just a force but a force applied in a
certain way. In section \ref{sec:torque} we discuss the transfer of
angular momentum by a force, but the basic idea here is that
a force directly in toward the axis does not change
the angular momentum.

\startdqs

<% marg(0) %>
<%
  fig(
    'dq-sweep-out-triangles',
    %q{Discussion question \ref{dq:sweep-out-triangles}.},
    {
      'anonymous'=>true
    }
  )
%>
<% end_marg %>
\begin{dq}\label{dq:sweep-out-triangles}
Suppose an object is simply traveling in a straight line
at constant speed. If we pick some point not on the line and
call it the axis of rotation, is area swept out by the
object at a constant rate? Would it matter if we chose a different axis?
\end{dq}

\begin{dq}\label{dq:conical-pendulum}
The figure is a strobe photo of a pendulum bob,
taken from underneath the pendulum looking straight up. The
black string can't be seen in the photograph. The bob was
given a slight sideways push when it was released, so it did
not swing in a plane. The bright spot marks the center, i.e.,
the position the bob would have if it hung straight down at
us. Does the bob's angular momentum appear to remain
constant if we consider the center to be the axis of
rotation? What if we choose a different axis?
\end{dq}

<%
  fig(
    'conical-pendulum',
    %q{Discussion question \ref{dq:conical-pendulum}.},
    {
      'width'=>'wide',
      'anonymous'=>true
    }
  )
%>

<% end_sec() %>
<% begin_sec("Two theorems about angular momentum",3,'am-theorems-stated') %>

With plain old momentum, $p$, we had the freedom to work in
any inertial frame of reference we liked. The same object
could have different values of momentum in two different
frames, if the frames were not at rest with respect to each
other. Conservation of momentum, however, would be true in
either frame. As long as we employed a single frame
consistently throughout a calculation, everything would work.

The same is true for angular momentum, and in addition there
is an ambiguity that arises from the definition of an axis
of rotation. For a wheel, the natural choice of an axis of
rotation is obviously the axle, but what about an egg
rotating on its side? The egg has an asymmetric shape, and
thus no clearly defined geometric center. A similar issue
arises for a cyclone, which does not even have a sharply
defined shape, or for a complicated machine with many gears.
The following theorem, the first of two presented in this
section without proof, explains how to deal with this issue.
Although I have put descriptive titles above both theorems,
they have no generally accepted names.\index{choice of axis
theorem}\index{angular momentum!choice of axis theorem}

\begin{lessimportant}[the choice of axis theorem]
It is entirely arbitrary what point one defines as the axis for
purposes of calculating angular momentum. If a closed system's
angular momentum is conserved when calculated with one choice of
axis, then it will also be conserved for any other choice.
Likewise, any inertial frame of reference may be used.
\end{lessimportant}

\pagebreak

<% marg(20) %>
<%
  fig(
    'asteroids-colliding',
    %q{Example \ref{eg:asteroids-colliding}.}
  )
%>
\spacebetweenfigs
<%
  fig(
    'diver',
    %q{%
      Everyone has a strong tendency to
      think of the diver as rotating about his
      own center of mass. However, he is
      flying in an arc, and he also has angular momentum because of this motion.
      
    }
  )
%>
\spacebetweenfigs
<%
  fig(
    'spin-theorem',
    %q{%
      This rigid object has angular momentum both because it is spinning about
      its center of mass and because it is
      moving through space.
    }
  )
%>

<% end_marg %>

\begin{eg}{Colliding asteroids described with different axes}\label{eg:asteroids-colliding}
Observers on planets A and $B$ both see the two asteroids
colliding. The asteroids are of equal mass and their impact
speeds are the same. Astronomers on each planet decide to
define their own planet as the axis of rotation. Planet A is
twice as far from the collision as planet $B$. The asteroids
collide and stick. For simplicity, assume planets A and
$B$ are both at rest.

With planet A as the axis, the two asteroids have the same
amount of angular momentum, but one has positive angular
momentum and the other has negative. Before the collision,
the total angular momentum is therefore zero. After the
collision, the two asteroids will have stopped moving, and
again the total angular momentum is zero. The total angular
momentum both before and after the collision is zero, so
angular momentum is conserved if you choose planet A as the axis.

The only difference with planet B as axis is that $r$ is
smaller by a factor of two, so all the angular momenta are
halved. Even though the angular momenta are different than
the ones calculated by planet A, angular momentum is still conserved.
\end{eg}

The earth spins on its own axis once a day, but simultaneously
travels in its circular one-year orbit around the sun, so
any given part of it traces out a complicated loopy path. It
would seem difficult to calculate the earth's angular
momentum, but it turns out that there is an intuitively
appealing shortcut: we can simply add up the angular
momentum due to its spin plus that arising from its center
of mass's circular motion around the sun. This is a special
case of the following general theorem:

\index{spin theorem}\index{angular momentum!spin theorem}\index{spin theorem}
\begin{lessimportant}[the spin theorem]
An object's angular momentum with respect to some outside
axis A can be found by adding up two parts:\\
(1) The first part is the object's angular momentum
found by using its own center of mass as the axis, i.e., the
angular momentum the object has because it is spinning.\\
(2) The other part equals the angular momentum that the object
would have with respect to the axis A if it had all its mass
concentrated at and moving with its center of mass.
\end{lessimportant}


\begin{eg}{A system with its center of mass at rest}
In the special case of an object whose center of mass is at
rest, the spin theorem implies that the object's angular
momentum is the same regardless of what axis we choose.
(This is an even stronger statement than the choice of axis
theorem, which only guarantees that angular momentum is
conserved for any given choice of axis, without specifying
that it is the same for all such choices.)
\end{eg}



m4_ifelse(__lm_series,1,[:
\pagebreak
\begin{eg}{Angular momentum of a rigid object}
\egquestion A motorcycle wheel has almost all its mass
concentrated at the outside. If the wheel has mass $m$ and
radius $r$, and the time required for one revolution is $T$,
what is the spin part of its angular momentum?

\eganswer This is an example of the commonly encountered
special case of rigid motion, as opposed to the rotation of
a system like a hurricane in which the different parts take
different amounts of time to go around. We don't really have
to go through a laborious process of adding up contributions
from all the many parts of a wheel, because they are all at
about the same distance from the axis, and are all moving
around the axis at about the same speed. The velocity is all
perpendicular to the spokes,
\begin{align*}
 v_\perp &= v \\
 &= (\text{circumference})/T \\
 &= 2\pi r/T\eqquad,
\end{align*}
and the angular momentum of the wheel about its center is
\begin{align*}
 L &= mv_{\perp}r \\
 &= m(2\pi r/T)r \\
 &= 2\pi mr^2/T\eqquad.
\end{align*}
\end{eg}

Note that although the factors of $2\pi $ in this expression
is peculiar to a wheel with its mass concentrated on the
rim, the proportionality to $m/T$ would have been the same
for any other rigidly rotating object. Although an object
with a noncircular shape does not have a radius, it is also
true in general that angular momentum is proportional to the
square of the object's size for fixed values of $m$ and $T$.
For instance doubling an object's size doubles both the $v_{\perp}$
and $r$ factors in the contribution of each of its parts to
the total angular momentum, resulting in an overall
factor of four increase.

The figure shows some examples of angular momenta of
various shapes rotating about their centers of mass. The
equations for their angular momenta were derived using
calculus, as described in my calculus-based book Simple
Nature. Do not memorize these equations!

<%
  fig(
    'angular-momenta-of-objects',
    '',
    {'width'=>'fullpage'}
  )
%>

%--- In Mechanics, these two examples are in the section on rigid-body rotation:
__incl(eg/hammer-throw)

__incl(eg/toppling-rod)
:])

\startdq

\begin{dq}
In the example of the colliding asteroids, suppose planet A
was moving toward the top of the page, at the same speed as
the bottom asteroid. How would planet A's astronomers
describe the angular momenta of the asteroids? Would angular
momentum still be conserved?
\end{dq}

<% end_sec() %>
<% begin_sec("Torque: the rate of transfer of angular momentum",nil,'torque') %>

Force can be interpreted as the rate of transfer of
momentum. The equivalent in the case of angular momentum is
called \emph{torque}\index{torque!defined} (rhymes with
``fork''). Where force tells us how hard we are pushing or
pulling on something, torque indicates how hard we are
twisting on it. Torque is represented by the Greek letter
tau, $\tau $, and the rate of change of an object's angular
momentum equals the total torque acting on it,
m4_ifelse(__calc,1,[:%
\begin{equation*}
                \tau_{total}  =  \frac{\der L}{\der t}\eqquad.
\end{equation*}
:],[:%
\begin{equation*}
                \tau_{total}  =  \frac{\Delta L}{\Delta t}\eqquad.
\end{equation*}
(If the angular momentum does not change at a constant rate,
the total torque equals the slope of the tangent line on a
graph of $L$ versus $t$.)
:])

<% marg(10) %>
<%
  fig(
    'transfer-e-p-l',
    %q{%
      Energy, momentum, and angular momentum can be transferred. The rates
      of transfer are called power, force, and torque.
    }
  )
%>
<% end_marg %>
As with force and momentum, it often happens that angular
momentum recedes into the background and we focus our
interest on the torques. The torque-focused point of view is
exemplified by the fact that many scientifically untrained
but mechanically apt people know all about torque, but none
of them have heard of angular momentum. Car enthusiasts
eagerly compare engines' torques, and there is a tool called
a torque wrench which allows one to apply a desired amount
of torque to a screw and avoid overtightening it.

<% begin_sec("Torque distinguished from force") %>

Of course a force is necessary in order to create a torque
--- you can't twist a screw without pushing on the wrench
--- but force and torque are two different things. One
distinction between them is direction. We use positive and
negative signs to represent forces in the two possible
directions along a line. The direction of a torque, however,
is clockwise or counterclockwise, not a linear direction.

The other difference between torque and force is a matter of
leverage. A given force applied at a door's knob will change
the door's angular momentum twice as rapidly as the same
force applied halfway between the knob and the hinge. The
same amount of force produces different amounts of
torque in these two cases.

It is possible to have a zero total torque with a nonzero
total force. An airplane with four jet engines, \figref{airplane}, would be
designed so that their forces are balanced on the left and
right. Their forces are all in the same direction, but the
clockwise torques of two of the engines are canceled by the
counterclockwise torques of the other two, giving zero total torque.
<% marg(m4_ifelse(__me,1,-90,10)) %>
<%
  fig(
    'airplane',
    %q{%
      The plane's four engines produce zero
      total torque but not zero total force.
    }
  )
%>
<% end_marg %>


Conversely we can have zero total force and nonzero total
torque. A merry-go-round's engine needs to supply a nonzero
torque on it to bring it up to speed, but there is zero
total force on it. If there was not zero total force on it,
its center of mass would accelerate!

<% end_sec() %>

m4_ifelse(__lm_series,1,[:\pagebreak:],[::])

<% begin_sec("Relationship between force and torque") %>\index{torque!relationship to force}

How do we calculate the amount of torque produced by a given
force? Since it depends on leverage, we should expect it to
depend on the distance between the axis and the point of
application of the force. We'll derive an equation relating
torque to force for a particular very simple situation, and
state without proof that the equation actually applies to all situations.

<% marg(0) %>
<%
  fig(
    'tetherball',
    %q{%
      The boy makes a torque on the tetherball.
    }
  )
%>
<% end_marg %>
To try to pin down this relationship
more precisely, let's imagine hitting a tetherball, figure \figref{tetherball}.
The boy applies a force $F$ to the ball for a short time $\Delta t$, accelerating
the ball from rest to a velocity $v$. Since force is the rate of transfer of
momentum, we have
\begin{align*}
  F & = \frac{m\Delta v}{\Delta t}\eqquad. \\
\intertext{Since the initial velocity is zero, $\Delta v$ is the same as the final
velocity $v$. Multiplying both sides by $r$ gives}
  Fr & = \frac{mvr}{\Delta t}\eqquad. \\
\end{align*}
But $mvr$ is simply the amount of angular momentum he's given
the ball, so $mvr/\Delta t$ also equals the amount of torque he applied. The result
of this example is
\begin{equation*}
  \tau = Fr\eqquad.
\end{equation*}

Figure \figref{tetherball} was drawn so that the force $F$ was in the direction
tangent to the circle, i.e., perpendicular to the radius $r$.
If the boy had applied a force \emph{parallel} to the radius line, either directly
inward or outward, then the ball would not have picked up any clockwise or counterclockwise
angular momentum.

If a force acts at an angle other than 0 or 90\degunit with
respect to the line joining the object and the axis, it
would be only the component of the force perpendicular to
the line that would produce a torque,
\begin{equation*}
                \tau   =  F_{\perp}r\eqquad.
\end{equation*}
Although this result was proved under a simplified set of
circumstances, it is more generally valid:

<% marg(0) %>
<%
  fig(
    'wrench-geometry',
    %q{%
      The geometric relationships referred to
      in the relationship between force and torque.
    }
  )
%>
<% end_marg %>

\begin{lessimportant}[relationship between force and torque]
The rate at which a force transfers angular momentum to an object,
i.e., the torque produced by the force, is given by
\begin{equation*}
|\btau| = r|F_\perp|\eqquad,
\end{equation*}
where $r$ is the distance from the axis to the point of
application of the force, and $F_\perp$ is the component of
the force that is perpendicular to the line joining the axis
to the point of application.
\end{lessimportant}

The equation is stated with absolute value signs because the
positive and negative signs of force and torque indicate
different things, so there is no useful relationship between
them. The sign of the torque must be found by physical
inspection of the case at hand.

From the equation, we see that the units of torque can be
written as newtons multiplied by meters. Metric torque
wrenches are calibrated in $\nunit\unitdot\munit$, but American ones use
foot-pounds, which is also a unit of distance multiplied by
a unit of force. We know from our study of mechanical work
that newtons multiplied by meters equal joules, but torque
is a completely different quantity from work, and nobody
writes torques with units of joules, even though it would be
technically correct.

<% self_check('compare-torques',<<-'SELF_CHECK'
Compare the magnitudes and signs of the four torques shown
in the figure.
  SELF_CHECK
  ) %>

<%
  fig(
    'wrench-sc',
    '',
    {
      'width'=>'wide',
      'anonymous'=>true,
      'float'=>false
    }
  )
%>

\begin{eg}{How torque depends on the direction of the force}
\egquestion How can the torque applied to the wrench in the
figure be expressed in terms of $r$, $|F|$, and the angle $\theta $
between these two vectors?

\eganswer The force vector and its $F_{\perp}$ component form the
hypotenuse and one leg of a right triangle,

<%
  fig(
    'wrench-sin-theta',
    '',
    {
      'width'=>'wide',
      'anonymous'=>true,
      'float'=>false
    }
  )
%>

\noindent and the interior angle opposite to $F_{\perp}$ equals $\theta $.
The absolute value of $F_{\perp}$ can thus be expressed as
\begin{equation*}
                F_{\perp}    =    |\vc{F}| \sin  \theta\eqquad,
\end{equation*}
leading to
\begin{equation*}
                |\btau|    =    r |\vc{F}| \sin   \theta\eqquad.
\end{equation*}
\end{eg}

<% marg(0) %>
<%
  fig(
    'wrench-rperp',
    %q{The quantity $r_\perp$.}
  )
%>
<% end_marg %>
Sometimes torque can be more neatly visualized in terms of
the quantity $r_{\perp}$ shown in figure \figref{wrench-rperp}, which
gives us a third way of expressing the relationship
between torque and force:
\begin{equation*}
                |\btau |    =    r_{\perp} |\vc{F}|\eqquad.
\end{equation*}

Of course you would not want to go and memorize all three
equations for torque. Starting from any one of them you
could easily derive the other two using trigonometry.
Familiarizing yourself with them can however clue you in to
easier avenues of attack on certain problems.

<% end_sec() %>
<% begin_sec("The torque due to gravity") %>\index{torque!due to gravity}

Up until now we've been thinking in terms of a force that
acts at a single point on an object, such as the force of
your hand on the wrench. This is of course an approximation,
and for an extremely realistic calculation of your hand's
torque on the wrench you might need to add up the torques
exerted by each square millimeter where your skin touches
the wrench. This is seldom necessary. But in the case of a
gravitational force, there is never any single point at
which the force is applied. Our planet is exerting a
separate tug on every brick in the Leaning Tower of Pisa,
and the total gravitational torque on the tower is the sum
of the torques contributed by all the little forces. Luckily
there is a trick that allows us to avoid such a massive
calculation. It turns out that for purposes of computing the
total gravitational torque on an object, you can get the
right answer by just pretending that the whole gravitational
force acts at the object's center of mass.

\begin{eg}{Gravitational torque on an outstretched arm}\label{eg:arm-outstretched}
\egquestion Your arm has a mass of 3.0 kg, and its center of
mass is 30 cm from your shoulder. What is the gravitational
torque on your arm when it is stretched out horizontally to
one side, taking the shoulder to be the axis?
<% marg(40) %>
<%
  fig(
    'arm-outstretched',
    %q{Example \ref{eg:arm-outstretched}.}
  )
%>
<% end_marg %>

\eganswer The total gravitational force acting on your arm is
\begin{equation*} 
|F|= (3.0\ \kgunit)(9.8\ \munit/\sunit^2)= 29\ \nunit\eqquad.
\end{equation*}
For the purpose of calculating the gravitational torque, we
can treat the force as if it acted at the arm's center of
mass. The force is straight down, which is perpendicular to
the line connecting the shoulder to the center of mass, so
\begin{equation*}
 F_{\perp} =|F|= 29\ \nunit\eqquad.
\end{equation*}
Continuing to pretend that the force acts at the center of
the arm, $r$ equals 30 cm = 0.30 m, so the torque is
\begin{equation*}
 \tau =r F_{\perp} = 9\ \nunit\unitdot\munit\eqquad.
\end{equation*}
\end{eg}

\begin{eg}{Cow tipping}\label{eg:cow-tipping}
In 2005, Dr. Margo Lillie and her graduate student Tracy Boechler
published a study claiming to debunk cow tipping. Their claim
was based on an analysis of the torques that would be required
to tip a cow, which showed that one person wouldn't be able to
make enough torque to do it. A lively discussion ensued on the
popular web site slashdot.org (``news for nerds, stuff that
matters'') concerning the validity of the study. Personally, I
had always assumed that cow-tipping was a group sport anyway,
but as a physicist, I also had some quibbles with their calculation.
Here's my own analysis.

<% marg(0) %>
<%
  fig(
    'cow-tipping',
    %q{Example \ref{eg:cow-tipping}.}
  )
%>
<% end_marg %>

There are three forces on the cow: the force of gravity $\vc{F}_W$, 
the ground's normal force $\vc{F}_N$, and the tippers' force $\vc{F}_A$.

As soon as the cow's left hooves (on the right from
our point of view) break contact with
the ground, the ground's force is being applied only to hooves on the other side.
We don't know the ground's force, and we don't want to find it. Therefore
we take the axis to be at its point of application, so that its torque
is zero.

For the purpose of computing torques, we can pretend that gravity acts at
the cow's center of mass, which I've placed a little lower than the center of
its torso, since its legs and head also have some mass, and the legs are more
massive than the head and stick out farther, so they lower the c.m. more than
the head raises it. The angle $\theta_W$ between the vertical gravitational force and
the line $r_{W}$ is about $14\degunit$. (An estimate by Matt Semke at 
the University of Nebraska-Lincoln gives $20\degunit$, which is in the same ballpark.)

To generate the maximum possible torque with the least possible force,
the tippers want to push at a point as far as possible from the axis, which will
be the shoulder on the other side, and they want to push at a 90 degree angle
with respect to the radius line $r_{A}$.

When the tippers are just barely applying enough force to raise the cow's hooves
on one side, the total torque has to be
just slightly more than zero. (In reality, they want to push a lot harder than this ---
hard enough to impart a lot of angular momentum to the cow fair in a short time,
before it gets mad and hurts them. We're just trying to calculate the bare
minimum force they can possibly use, which is the question that science can answer.)
Setting the total torque equal to zero,
\begin{gather*}
  \tau_{N}+\tau_{W}+\tau_{A} = 0\eqquad, \\
\intertext{and letting counterclockwise torques be positive, we have}
  0-mgr_W\sin\theta_W+F_Ar_A\sin 90\degunit = 0 \\
\end{gather*}
\begin{align*}
  F_A &= \frac{r_W}{r_A} mg \sin\theta_W \\
      &\approx \frac{1}{1.5} (680\ \kgunit)(9.8\ \munit/\sunit^2) \sin 14\degunit \\
      &=1100\ \nunit\eqquad.
\end{align*}
The 680 kg figure for the typical mass of a cow is due to Lillie and Boechler, who
are veterinarians, so I assume it's fairly accurate. My estimate of 1100 N comes out
significantly lower than their 1400 N figure, mainly because their incorrect
placement of the center of mass gives $\theta_W=24\degunit$. I don't think 1100 N is an impossible amount of force
to require of one big, strong person (it's equivalent to lifting about 110 kg, or 240 pounds),
but given that the tippers need to impart a large angular momentum fairly quickly,
it's probably true that several people would be required.

The main practical issue with cow tipping is that cows generally sleep lying down.
Falling on its side can also seriously injure a cow.
\end{eg}

\startdqs

\begin{dq}
This series of discussion questions deals with past
students' incorrect reasoning about the following problem.

\begin{indentedblock}
Suppose a comet is at the point in its orbit shown in the
figure. The only force on the comet is the sun's gravitational force.

\noindent\anonymousinlinefig{../../../share/mechanics/figs/dq-comet}

\noindent Throughout the question, define all torques and angular
momenta using the sun as the axis.

\noindent (1) Is the sun producing a nonzero torque on the comet? Explain.\\
(2) Is the comet's angular momentum increasing, decreasing,
or staying the same? Explain.
\end{indentedblock}

\noindent Explain what is wrong with the following answers. In some
cases, the answer is correct, but the reasoning leading up to it is wrong.
(a) Incorrect answer to part (1): ``Yes, because the sun is
exerting a force on the comet, and the comet is a certain
distance from the sun.''\\
(b) Incorrect answer to part (1): ``No, because the
torques cancel out.''\\
(c) Incorrect answer to part (2): ``Increasing, because the
comet is speeding up.''
\end{dq}

<% marg(0) %>
<%
  fig(
    'dq-claw-hammer',
    %q{Discussion question \ref{dq:claw-hammer}.},
    {
      'anonymous'=>true
    }
  )
%>
<% end_marg %>

\begin{dq}\label{dq:claw-hammer}
Which claw hammer would make it easier to get the nail
out of the wood if the same force was applied in the same direction?
\end{dq}

\begin{dq}
You whirl a rock over your head on the end of a string,
and gradually pull in the string, eventually cutting the
radius in half. What happens to the rock's angular momentum?
What changes occur in its speed, the time required for one
revolution, and its acceleration? Why might the string break?
\end{dq}


\begin{dq}
A helicopter has, in addition to the huge fan blades on
top, a smaller propeller mounted on the tail that rotates in
a vertical plane. Why?
\end{dq}

\begin{dq}\label{dq:twoarm-ride}
The photo shows an amusement park ride whose two cars
rotate in opposite directions. Why is this a good design?
\end{dq}

<% marg(300) %>
<%
  fig(
    'dq-twoarm-ride',
    %q{Discussion question \ref{dq:twoarm-ride}.},
    {
      'anonymous'=>true
    }
  )
%>
<% end_marg %>

<% end_sec() %>
<% end_sec() %>
\vfill
<% begin_sec("Statics",4) %>

<% begin_sec("Equilibrium") %>

There are many cases where a system is not closed but
maintains constant angular momentum. When a merry-go-round
is running at constant angular momentum, the engine's torque
is being canceled by the torque due to friction.

<% marg(80) %>
<%
  fig(
    'windmills',
    %q{%
      The windmills are not closed systems,
      but angular momentum is being transferred out of them at the
      same rate it is transferred in, resulting in constant angular momentum.
      To get an idea of the huge scale of
      the modern windmill farm, note the
      sizes of the trucks and trailers.
    }
  )
%>
<% end_marg %>
When an object has constant momentum and constant angular
momentum, we say that it is in \index{equilibrium!defined}equilibrium.
m4_ifelse(__calc,1,[:In symbols,
\begin{equation*}
  \frac{\der\vc{p}}{\der t} = 0 \ \text{and}\ \frac{\der L}{\der t} = 0 \qquad \text{[conditions for equilibrium]}
\end{equation*}
(or equivalently, zero total force and zero total torque).
%:],
[:%:])
This is a scientific redefinition of the common English
word, since in ordinary speech nobody would describe a car
spinning out on an icy road as being in equilibrium.

Very commonly, however, we are interested in cases where an
object is not only in equilibrium but also at rest, and this
corresponds more closely to the usual meaning of the word.
Trees and bridges have been designed by evolution and
engineers to stay at rest, and to do so they must have not
just zero total force acting on them but zero total torque.
It is not enough that they don't fall down, they also must
not tip over. \index{statics}Statics is the branch of
physics concerned with problems such as these.

Solving statics problems is now simply a matter of applying
and combining some things you already know:

\begin{itemize}

\item You know the behaviors of the various types of forces, for
example that a frictional force is always parallel to
the surface of contact.

\item You know about vector addition of forces. It is the vector
sum of the forces that must equal zero to produce equilibrium.

\item You know about torque. The total torque acting on an
object must be zero if it is to be in equilibrium.

\item You know that the choice of axis is arbitrary, so you can
make a choice of axis that makes the problem easy to solve.

\end{itemize}

\noindent In general, this type of problem could involve four
equations in four unknowns: three equations that say the
force components add up to zero, and one equation that says
the total torque is zero. Most cases you'll encounter will
not be this complicated. In the following example, only the
equation for zero total torque is required in order to get an answer.

\vfill\pagebreak[4]

<% marg(0) %>
<%
  fig(
    'modern-art',
    %q{Example \ref{eg:modern-art}.}
  )
%>
<% end_marg %>

__incl(eg/modern-art)

<% marg(0) %>
<%
  fig(
    'eg-flagpole',
    %q{Example \ref{eg:flagpole}.}
  )
%>
<% end_marg %>
\begin{eg}{A flagpole}\label{eg:flagpole}
\egquestion A 10-kg flagpole is being held up by a lightweight
horizontal cable, and is propped against the foot of a wall
as shown in the figure. If the cable is only capable of
supporting a tension of 70 N, how great can the angle
$\alpha $ be without breaking the cable?

\eganswer All three objects in the figure are supposed to be
in equilibrium: the pole, the cable, and the wall. Whichever
of the three objects we pick to investigate, all the forces
and torques on it have to cancel out. It is not particularly
helpful to analyze the forces and torques on the wall, since
it has forces on it from the ground that are not given and
that we don't want to find. We could study the forces and
torques on the cable, but that doesn't let us use the given
information about the pole. The object we need to analyze is the pole.

The pole has three forces on it, each of which may also
result in a torque: (1) the gravitational force, (2) the
cable's force, and (3) the wall's force.

We are free to define an axis of rotation at any point we
wish, and it is helpful to define it to lie at the bottom
end of the pole, since by that definition the wall's force
on the pole is applied at $r=0$ and thus makes no torque on
the pole. This is good, because we don't know what the
wall's force on the pole is, and we are not trying to find it.

With this choice of axis, there are two nonzero torques on
the pole, a counterclockwise torque from the cable and a
clockwise torque from gravity. Choosing to represent
counterclockwise torques as positive numbers, and using the equation
$|\btau| =r|F| \sin \theta$, we have
\begin{equation*}
                r_{cable} |F_{cable}| \sin   \theta_{cable} - r_{grav}|F_{grav}|\sin  \theta_{grav}  =  0\eqquad.
\end{equation*}
A little geometry gives $\theta_{cable}=90\degunit-\alpha$ and $\theta_{grav}=\alpha$, so
\begin{equation*}
                r_{cable} |F_{cable}| \sin   (90\degunit-\alpha) - r_{grav}|F_{grav}| \sin   \alpha  =  0\eqquad.
\end{equation*}
The gravitational force can be considered as acting at the
pole's center of mass, i.e., at its geometrical center, so
$r_{cable}$ is twice $r_{grav}$, and we can simplify
the equation to read
\begin{equation*}
                2 |F_{cable}| \sin   (90\degunit-\alpha) - |F_{grav}| \sin   \alpha =  0\eqquad.
\end{equation*}
These are all quantities we were given, except for $\alpha$,
which is the angle we want to find. To solve for $\alpha$
we need to use the trig identity
$\sin (90\degunit-x)= \cos x$,
\begin{equation*}
                2 |F_{cable}| \cos  \alpha - |F_{grav}| \sin   \alpha  =  0\eqquad,
\end{equation*}
which allows us to find
\begin{align*}
 \tan\alpha  &= 2\frac{|\vc{F}_{cable}|}{|\vc{F}_{grav}|}\\
 \alpha &= \tan^{-1}\left(2\frac{|\vc{F}_{cable}|}{|\vc{F}_{grav}|}\right)\\
        &= \tan^{-1}\left(2\times\frac{70\ \nunit}{98\ \nunit}\right)\\
        &= 55\degunit\eqquad.
\end{align*}
\end{eg}

<% end_sec() %>
<% begin_sec("Stable and unstable equilibria",4) %>

A pencil balanced upright on its tip could theoretically be
in equilibrium, but even if it was initially perfectly
balanced, it would topple in response to the first air
current or vibration from a passing truck. The pencil can be
put in equilibrium, but not in stable equilibrium. The
things around us that we really do see staying still are all
in stable equilibrium.

Why is one equilibrium stable and another unstable? Try
pushing your own nose to the left or the right. If you push
it a millimeter to the left, your head responds with a gentle force
to the right, which keeps your nose from flying off of your face.
If you push your nose a centimeter to the left, your face's force
on your nose becomes much stronger. The defining
characteristic of a stable equilibrium is that the farther
the object is moved away from equilibrium, the stronger the
force is that tries to bring it back.

<% marg(100) %>
<%
  fig(
    'stable-and-unstable',
    %q{Stable and unstable equilibria.}
  )
%>
\spacebetweenfigs
<%
  fig(
    'swan-lake',
    %q{%
      The dancer's equilibrium is unstable. If she
      didn't constantly make tiny adjustments, she would tip over.
    }
  )
%>
\spacebetweenfigs
<%
  fig(
    'nancy-neutron',
    %q{Example \ref{eg:nancy-neutron}.}
  )
%>
<% end_marg %>
The opposite is true for an unstable equilibrium. In the top
figure, the ball resting on the round hill theoretically has
zero total force on it when it is exactly at the top. But in
reality the total force will not be exactly zero, and the
ball will begin to move off to one side. Once it has moved,
the net force on the ball is greater than it was, and it
accelerates more rapidly. In an unstable equilibrium, the
farther the object gets from equilibrium, the stronger the
force that pushes it farther from equilibrium.

This idea can be rephrased in terms of energy. The difference
between the stable and unstable equilibria shown in figure
\ref{fig:stable-and-unstable} is that in the stable
equilibrium, the potential energy is at a minimum, and moving
to either side of equilibrium will increase it, whereas
the unstable equilibrium represents a maximum.

Note that we are using the term ``stable'' in a weaker sense
than in ordinary speech. A domino standing upright is stable
in the sense we are using, since it will not spontaneously
fall over in response to a sneeze from across the room or
the vibration from a passing truck. We would only call it
unstable in the technical sense if it could be toppled by
\emph{any} force, no matter how small. In everyday usage, of
course, it would be considered unstable, since the force
required to topple it is so small.

\begin{eg}{An application of calculus}\label{eg:nancy-neutron}
\egquestion Nancy Neutron is living in a uranium nucleus that
is undergoing fission. Nancy's potential energy as a function of
position can be approximated by $PE=x^4-x^2$, where all the
units and numerical constants have been suppressed for simplicity.
Use calculus to
locate the equilibrium points, and determine whether they are
stable or unstable.

\eganswer The equilibrium points occur where the PE is at a minimum
or maximum, and minima and maxima occur where the derivative (which equals
minus the force on Nancy) is zero.
This derivative is $\der PE/\der x=4x^3-2x$, and setting it equal to
zero, we have $x=0, \pm1/\sqrt{2}$. Minima occur where the second derivative
is positive, and maxima where it is negative. The second derivative is
$12x^2-2$, which is negative at $x=0$ (unstable)
and positive at $x=\pm1/\sqrt{2}$ (stable).
Interpretation: the graph of the PE is shaped like a rounded letter `W,'
with the two troughs representing the two halves of the splitting nucleus.
Nancy is going to have to decide which half she wants to go with.
\end{eg}

<% end_sec() %>
<% end_sec() %>
<% begin_sec("Simple Machines: the lever",3) %>\index{lever}

Although we have discussed some simple machines such as the
pulley, without the concept of torque we were not yet ready
to address the lever, which is the machine nature used in
designing living things, almost to the exclusion of all
others. (We can speculate what life on our planet might have
been like if living things had evolved wheels, gears,
pulleys, and screws.) The figures show two examples of
levers within your arm. Different muscles are used to flex
and extend the arm, because muscles work only by contraction.

<% marg(0) %>
<%
  fig(
    'biceps',
    %q{The biceps muscle flexes the arm.}
  )
%>
\spacebetweenfigs
<%
  fig(
    'triceps',
    %q{The triceps extends the arm.}
  )
%>

<% end_marg %>
Analyzing example \figref{biceps} physically, there are two forces that
do work. When we lift a load with our biceps muscle, the
muscle does positive work, because it brings the bone in the
forearm in the direction it is moving. The load's force on
the arm does negative work, because the arm moves in the
direction opposite to the load's force. This makes sense,
because we expect our arm to do positive work on the load,
so the load must do an equal amount of negative work on the
arm. (If the biceps was lowering a load, the signs of the
works would be reversed. Any muscle is capable of doing
either positive or negative work.)

There is also a third force on the forearm: the force of the
upper arm's bone exerted on the forearm at the elbow joint
(not shown with an arrow in the figure). This force does no
work, because the elbow joint is not moving.

Because the elbow joint is motionless, it is natural to
define our torques using the joint as the axis. The
situation now becomes quite simple, because the upper arm
bone's force exerted at the elbow neither does work nor
creates a torque. We can ignore it completely. In any lever
there is such a point, called the \index{fulcrum}fulcrum.

If we restrict ourselves to the case in which the forearm
rotates with constant angular momentum, then we know that
the total torque on the forearm is zero,
\begin{equation*}
                \tau_\text{muscle} + \tau_\text{load}  =  0\eqquad.
\end{equation*}
 If we choose to represent counterclockwise torques as
positive, then the muscle's torque is positive, and the
load's is negative. In terms of their absolute values,
\begin{equation*}
                |\tau_\text{muscle}|  =  |\tau_\text{load}|\eqquad.
\end{equation*}
Assuming for simplicity that both forces act at angles of
90\degunit with respect to the lines connecting the axis to the
points at which they act, the absolute values of the torques are
\begin{equation*}
                r_\text{muscle} F_\text{muscle}  =  r_\text{load} F_\text{arm}\eqquad,
\end{equation*}
where $r_\text{muscle}$, the distance from the elbow joint to the
biceps' point of insertion on the forearm, is only a few cm,
while $r_\text{load}$ might be 30 cm or so. The force exerted by
the muscle must therefore be about ten times the force
exerted by the load. We thus see that this lever is a force
reducer. In general, a lever may be used either to increase
or to reduce a force.

Why did our arms evolve so as to reduce force? In general,
your body is built for compactness and maximum speed of
motion rather than maximum force. This is the main
anatomical difference between us and the \index{Neanderthals}Neanderthals
(their brains covered the same range of sizes as those of
modern humans), and it seems to have worked for us.

As with all machines, the lever is incapable of changing the
amount of mechanical work we can do. A lever that increases
force will always reduce motion, and vice versa, leaving the
amount of work unchanged.

It is worth noting how simple and yet how powerful this
analysis was. It was simple because we were well prepared
with the concepts of torque and mechanical work. In anatomy
textbooks, whose readers are assumed not to know physics,
there is usually a long and complicated discussion of the
different types of levers. For example, the biceps lever,
\figref{biceps}, would be classified as a class III lever, since it has
the fulcrum and load on the ends and the muscle's force
acting in the middle. The triceps, \figref{triceps}, is called a class I
lever, because the load and muscle's force are on the ends
and the fulcrum is in the middle. How tiresome! With a firm
grasp of the concept of torque, we realize that all such
examples can be analyzed in much the same way. Physics is at
its best when it lets us understand many apparently
complicated phenomena in terms of a few simple yet powerful concepts.

\vfill

<% end_sec() %>
