Dandelion. Cello. Read those two words, and your brain
instantly conjures a stream of associations, the most
prominent of which have to do with vibrations. Our mental
category of ``dandelion-ness'' is strongly linked to the
color of light waves that vibrate about half a million
billion times a second: yellow. The velvety throb of a cello
has as its most obvious characteristic a relatively low
musical pitch --- the note you are spontaneously imagining
right now might be one whose sound vibrations repeat at a
rate of a hundred times a second.

Evolution has designed our two most important senses around
the assumption that not only will our environment be
drenched with information-bearing vibrations, but in
addition those vibrations will often be repetitive, so that
we can judge colors and pitches by the rate of repetition.
Granting that we do sometimes encounter non-repeating waves
such as the consonant ``sh,'' which has no recognizable
\index{pitch}pitch, why was Nature's assumption of
repetition nevertheless so right in general?

Repeating phenomena occur throughout nature, from the orbits
of electrons in atoms to the reappearance of Halley's
\index{comet}\index{Halley's Comet}Comet every 75 years.
Ancient cultures tended to attribute repetitious phenomena
like the seasons to the cyclical nature of time itself, but
we now have a less mystical explanation. Suppose that
instead of Halley's Comet's true, repeating elliptical orbit
that closes seamlessly upon itself with each revolution, we
decide to take a pen and draw a whimsical alternative path
that never repeats. We will not be able to draw for very
long without having the path cross itself. But at such a
crossing point, the comet has returned to a place it visited
once before, and since its potential energy is the same as
it was on the last visit, conservation of energy proves that
it must again have the same kinetic energy and therefore the
same speed. Not only that, but the comet's direction of
motion cannot be randomly chosen, because angular momentum
must be conserved as well. Although this falls short of
being an ironclad proof that the comet's orbit must repeat,
it no longer seems surprising that it does.

<% marg(70) %>
<%
  fig(
    'comet-goofy-orbit',
    %q{If we try to draw a non-repeating orbit for Halley's Comet, it will inevitably end up crossing itself.}
  )
%>
<% end_marg %>
Conservation laws, then, provide us with a good reason why
repetitive motion is so prevalent in the universe. But it
goes deeper than that. Up to this point in your study of
physics, I have been indoctrinating you with a mechanistic
vision of the universe as a giant piece of clockwork.
Breaking the clockwork down into smaller and smaller bits,
we end up at the atomic level, where the electrons circling
the nucleus resemble --- well, little clocks! From this
point of view, particles of matter are the fundamental
building blocks of everything, and vibrations and waves are
just a couple of the tricks that groups of particles can do.
But at the beginning of the 20th century, the tables were
turned. A chain of discoveries initiated by Albert
\index{Einstein, Albert}Einstein led to the realization that
the so-called subatomic ``particles'' were in fact waves. In
this new world-view, it is vibrations and waves that are
fundamental, and the formation of matter is just one of the
tricks that waves can do.

<% begin_sec("Period, frequency, and amplitude",3) %>\index{work!done by a varying force}

<% marg(0) %>
<%
  fig(
    'mass-on-spring',
    %q{%
      A spring has an equilibrium length, 1, and
      can be stretched, 2, or compressed, 3. A mass attached to the spring can be
      set into motion initially, 4, and will then vibrate, 4-13.
    }
  )
%>
<% end_marg %>
Figure \figref{mass-on-spring} shows our most basic example of a vibration. With
no forces on it, the spring assumes its equilibrium length,
\figref{mass-on-spring}/1. It can be stretched, 2, or compressed, 3. We attach
the spring to a wall on the left and to a mass on the right.
If we now hit the mass with a hammer, 4, it oscillates as
shown in the series of snapshots, 4-13. If we assume that
the mass slides back and forth without friction and that the
motion is one-dimensional, then conservation of energy
proves that the motion must be repetitive. When the block
comes back to its initial position again, 7, its potential
energy is the same again, so it must have the same kinetic
energy again. The motion is in the opposite direction,
however. Finally, at 10, it returns to its initial position
with the same kinetic energy and the same direction of
motion. The motion has gone through one complete cycle, and
will now repeat forever in the absence of friction.

The usual physics terminology for motion that repeats itself
over and over is \index{motion!periodic}periodic motion, and
the time required for one repetition is called the
\index{period!defined}period, $T$. (The symbol $P$ is not
used because of the possible confusion with momentum.) One
complete repetition of the motion is called a cycle.
m4_ifelse(__me,0,[:<% marg(50) %>
<%
  fig(
    'full-period',
    %q{%
      Position-versus-time graphs for half a period and a full period.
    }
  )
%>
<% end_marg %>
:])

We are used to referring to short-period sound vibrations as
``high'' in pitch, and it sounds odd to have to say that
high pitches have low periods. It is therefore more common
to discuss the rapidity of a vibration in terms of the
number of vibrations per second, a quantity called the
\index{frequency!defined}frequency, $f$. Since the period is
the number of seconds per cycle and the frequency is the
number of cycles per second, they are reciprocals of each other,
\begin{equation*}
                f  =  1/T\eqquad.
\end{equation*}
m4_ifelse(__me,1,[:

The forms of various equations turn out to be simpler
when they are expressed not in terms of $f$ but in terms of $\omega=2\pi f$.
It's not a coincidence that this relationship looks the same as the one relating
angular velocity and frequency in circular motion. In machines, mechanical linkages
are used to convert back and forth between vibrational motion and circular motion.
For example, a car engine's pistons oscillate in their cylinders at a frequency $f$,
driving the crankshaft at the same frequency $f$. Either of these motions can be
described using $\omega$ instead of $f$, even though only in the case of the crankshaft's
rotational motion does it make sense to interpret $\omega$ as the number of radians per second.
When the motion is not rotational, we usually refer to $\omega$ as the angular frequency,\index{angular frequency}\index{frequency!angular}
and we often use the word ``frequency'' to mean either
$f$ or $\omega$, relying on context to make the meaning clear.
<% marg(160) %>
<%
  fig(
    'full-period',
    %q{%
      Position-versus-time graphs for half a period and a full period.
    }
  )
%>
\spacebetweenfigs
<%
  fig(
    'locomotive-linkages',
    %q{The locomotive's wheels spin at a frequency of $f$ cycles per second, which can also be described
       as $\omega$ radians per second. The mechanical linkages allow the linear vibration of the steam engine's pistons, at frequency $f$, to drive
       the wheels.}
  )
%>
\spacebetweenfigs
<%
  fig(
    'carnival-game',
    %q{Example \ref{eg:carnival-game}.}
  )
%>
<% end_marg %>
:],[:
<% marg(0) %>
<%
  fig(
    'carnival-game',
    %q{Example \ref{eg:carnival-game}.}
  )
%>
<% end_marg %>
:])
\begin{eg}{A carnival game}\label{eg:carnival-game}
In the carnival game shown in figure \figref{carnival-game}, the rube is
supposed to push the bowling ball on the track just hard
enough so that it goes over the hump and into the valley,
but does not come back out again. If the only types of
energy involved are kinetic and potential, this is
impossible. Suppose you expect the ball to come back to a
point such as the one shown with the dashed outline, then
stop and turn around. It would already have passed through
this point once before, going to the left on its way into
the valley. It was moving then, so conservation of energy
tells us that it cannot be at rest when it comes back to the
same point. The motion that the customer hopes for is
physically impossible. There is a physically possible
periodic motion in which the ball rolls back and forth,
staying confined within the valley, but there is no way to
get the ball into that motion beginning from the place where
we start. There is a way to beat the game, though. If you
put enough spin on the ball, you can create enough kinetic
friction so that a significant amount of heat is generated.
Conservation of energy then allows the ball to be at rest
when it comes back to a point like the outlined one, because
kinetic energy has been converted into heat.
\end{eg}

m4_ifelse(__lm_series,1,\vspace{10mm})


\begin{eg}{Period and frequency of a fly's wing-beats}
A Victorian parlor trick was to listen to the pitch of a
fly's buzz, reproduce the musical note on the piano, and
announce how many times the fly's wings had flapped in one
second. If the fly's wings flap, say, 200 times in one
second, then the frequency of their motion is $f=200/1\ \sunit=200\ \sunit^{-1}$.
The period is one 200th of a second, $T=1/f=(1/200)\ \sunit=0.005\ \sunit$.
\end{eg}

m4_ifelse(__lm_series,1,\pagebreak[4])

Units of inverse second, $\sunit^{-1}$, are awkward in speech, so an
abbreviation has been created. One Hertz, named in honor of
a pioneer of radio technology, is one cycle per second. In
abbreviated form, $1\ \zu{Hz}=1\ \sunit^{-1}$. This is the familiar unit
used for the frequencies on the radio dial.

\begin{eg}{Frequency of a radio station}
\egquestion KKJZ's frequency is 88.1 MHz. What does this mean,
and what period does this correspond to?

\eganswer The metric prefix M- is mega-, i.e., millions. The
radio waves emitted by KKJZ's transmitting antenna vibrate
88.1 million times per second. This corresponds to a period of
\begin{equation*}
 T = 1/f= 1.14\times10^{-8}\ \sunit\eqquad.
\end{equation*}

This example shows a second reason why we normally speak in
terms of frequency rather than period: it would be painful
to have to refer to such small time intervals routinely. I
could abbreviate by telling people that KKJZ's period was
11.4 nanoseconds, but most people are more familiar with the
big metric prefixes than with the small ones.
\end{eg}

Units of frequency are also commonly used to specify the
speeds of computers. The idea is that all the little
circuits on a computer chip are synchronized by the very
fast ticks of an electronic clock, so that the circuits can
all cooperate on a task without getting ahead or behind.
Adding two numbers might require, say, 30 clock cycles.
Microcomputers these days operate at clock frequencies
of about a gigahertz.

<% marg(18) %>
<%
  fig(
    'amplitude-examples',
    %q{%
      1. The amplitude of the vibrations of the mass on a spring could be defined in two different ways. It would
      have units of distance. 2. The amplitude of a swinging pendulum would more naturally be defined as an angle.
    }
  )
%>
<% end_marg %>
We have discussed how to measure how fast something
vibrates, but not how big the vibrations are. The general
term for this is \index{amplitude!defined}amplitude, $A$. The
definition of amplitude depends on the system being
discussed, and two people discussing the same system may not
even use the same definition. In the example of the block on
the end of the spring, \figref{amplitude-examples}/1, the amplitude will be measured in
distance units such as cm. One could work in terms of the
distance traveled by the block from the extreme left to the
extreme right, but it would be somewhat more common in
physics to use the distance from the center to one extreme.
The former is usually referred to as the \index{amplitude!peak-to-peak}peak-to-peak
amplitude, since the extremes of the motion look like
mountain peaks or upside-down mountain peaks on a graph of
position versus time.

In other situations we would not even use the same units for
amplitude. The amplitude of a child on a swing, or a pendulum, \figref{amplitude-examples}/2, would most
conveniently be measured as an angle, not a distance, since
her feet will move a greater distance than her head. The
electrical vibrations in a radio receiver would be measured
in electrical units such as volts or amperes.

<% end_sec() %>
m4_ifelse(__lm_series,1,[:\vfill:])
<% begin_sec("Simple harmonic motion"m4_ifelse(__lm_series,1,[:,4:],[:,nil:]),'shm-k-m') %>\index{work!done by a varying force}

<% begin_sec("Why are sine-wave vibrations so common?") %>

<% marg(0) %>
<%
  fig(
    'sinusoidal-x-t',
    %q{Sinusoidal and non-sinusoidal vibrations.}
  )
%>
<% end_marg %>
If we actually construct the mass-on-a-spring system
discussed in the previous section and measure its motion
accurately, we will find that its $x-t$ graph is nearly a
perfect sine-wave shape, as shown in figure \figref{sinusoidal-x-t}/1. (We call it
a ``sine wave'' or ``sinusoidal'' even if it is a cosine, or
a sine or cosine shifted by some arbitrary horizontal
amount.) It may not be surprising that it is a wiggle of
this general sort, but why is it a specific mathematically
perfect shape? Why is it not a sawtooth shape like 2 or
some other shape like 3? The mystery deepens as we find
that a vast number of apparently unrelated vibrating systems
show the same mathematical feature. A \index{tuning fork}tuning fork,
a sapling pulled to one side and released,
a car bouncing on its shock absorbers, all these systems
will exhibit sine-wave motion under one condition: the
amplitude of the motion must be small.

It is not hard to see intuitively why extremes of amplitude
would act differently. For example, a car that is bouncing
lightly on its shock absorbers may behave smoothly, but if
we try to double the amplitude of the vibrations the bottom
of the car may begin hitting the ground, \figref{sinusoidal-x-t}/4. (Although we
are assuming for simplicity in this chapter that energy is
never dissipated, this is clearly not a very realistic
assumption in this example. Each time the car hits the
ground it will convert quite a bit of its potential and
kinetic energy into heat and sound, so the vibrations would
actually die out quite quickly, rather than repeating for
many cycles as shown in the figure.)

The key to understanding how an object vibrates is to know
how the force on the object depends on the object's
position. If an object is vibrating to the right and left,
then it must have a leftward force on it when it is on the
right side, and a rightward force when it is on the left
side. In one dimension, we can represent the direction of
the force using a positive or negative sign, and since the
force changes from positive to negative there must be a
point in the middle where the force is zero. This is the
equilibrium point, where the object would stay at rest if it
was released at rest. For convenience of notation throughout
this chapter, we will define the origin of our coordinate
system so that $x$ equals zero at equilibrium.

<% marg(0) %>
<%
  fig(
    'force-curve-1',
    %q{The force exerted by an ideal spring, which behaves exactly according to Hooke's law.}
  )
%>
<% end_marg %>
The simplest example is the mass on a spring, for which the
force on the mass is given by \index{Hooke's law}Hooke's law,
\begin{equation*}
                F    =    -kx\eqquad.
\end{equation*}
We can visualize the behavior of this force using a graph of
$F$ versus $x$, as shown in figure \figref{force-curve-1}. The graph is a line, and the
spring constant, $k$, is equal to minus its slope.\index{spring constant}
A stiffer
spring has a larger value of $k$ and a steeper slope.
Hooke's law is only an approximation, but it works very well
for most springs in real life, as long as the spring isn't
compressed or stretched so much that it is permanently bent or damaged.

m4_ifelse(__me,1,[:The following important theorem relates the motion graph to the force graph.

\begin{lessimportant}{Theorem: A linear
force graph makes a sinusoidal motion graph.}

\noindent If the total force on a vibrating object depends only on the
object's position, and is related to the objects displacement
from equilibrium by an equation of the form $F=-kx$,
then the object's motion displays a sinusoidal graph with frequency $\omega=\sqrt{k/m}$.
\end{lessimportant}

\noindent Proof: By Newton's second law, $-kx=ma$, so we need a function $x(t)$ that satisfies the equation
$\der^2 x/\der t^2=-cx$, where for convenience we write $c$ for $k/m$. This type of equation is called
a differential equation, because it relates a function to its own derivative (in this case the second derivative).

Just to make
things easier to think about, suppose that we happen to have an oscillator with $c=1$.
Then our goal is to find a function whose second derivative is equal to minus the
original function. We know of two such functions, the sine and the cosine. These two
solutions can be combined to make anything of the form $P\sin t+Q\cos t$, where $P$ and
$Q$ are constants, and the result will still be a solution. Using trig identities, such
an expression can always be rewritten as $A\cos(t+\delta)$.
<%
  fig(
    'jovian-moons',
    %q{Because simple harmonic motion involves sinusoidal functions, it is equivalent to circular motion that has been projected into one dimension.
       This figure shows a simulated view of Jupiter and its four largest moons at intervals of three hours. Seen from the side from within the plane
        of the solar system, the circular orbits appear linear. In coordinates with the origin at Jupiter, a moon has coordinates $x=r\cos\theta$ and $y=r\sin\theta$,
        where $\theta=\omega t$.
        If the view is along the $y$ axis, then we see only the $x$ motion, which is of the form $A\cos(\omega t)$.},
    {
      'width'=>'wide',
      'sidecaption'=>true
    }
  )
%>

Now what about the more general case where $c$ need not equal 1? The role of $c$ in
$\der^2 x/\der t^2=-cx$ is to set the time scale. For example, suppose we produce a fake video
of an object oscillating according to $A\cos(t+\delta)$, which violates Newton's second law because $c$ doesn't
equal 1, so the acceleration is too small. We can always make the video physically accurate by speeding it
up. This suggests generalizing the solution to $A\cos(\omega t+\delta)$. Plugging in to the
differential equation, we find that $\omega=\sqrt{k/m}$, and $T=2\pi/\omega$ brings us to the
claimed result.

We've proved that anything of this form is a solution, but we haven't shown that any solution must
be of this form. Physically, this must be true because the motion is fully determined by the
oscillator's initial position and initial velocity, which can always be matched by choosing
$A$ and $\delta$ appropriately. Mathematically, the uniqueness result is a standard one about
second-order differential equations.
:],[:The following important theorem, whose proof is given in
optional section \ref{sec:proofs}, relates the motion graph to the force graph.

\index{simple harmonic motion!period of}
\begin{lessimportant}{Theorem: A linear
force graph makes a sinusoidal motion graph.}

\noindent If the total force on a vibrating object depends only on the
object's position, and is related to the objects displacement
from equilibrium by an equation of the form $F=-kx$,
then the object's motion displays a sinusoidal graph with period $T=2\pi\sqrt{m/k}$.
\end{lessimportant}

Even if you do not read the proof, it is not too hard to
understand why the equation for the period makes sense. A
greater mass causes a greater period, since the force will
not be able to whip a massive object back and forth very
rapidly. A larger value of $k$ causes a shorter period,
because a stronger force can whip the object back and forth more rapidly.:])
<% marg(23) %>
<%
  fig(
    'force-curve-2',
    %q{Seen from close up, any $F-x$ curve looks like a line.}
  )
%>
<% end_marg %>

This may seem like only an obscure theorem about the
mass-on-a-spring system, but figure \figref{force-curve-2} shows it to
be far more general than that. Figure \figref{force-curve-2}/1 depicts a force
curve that is not a straight line. A system with this $F-x$
curve would have large-amplitude vibrations that were
complex and not sinusoidal. But the same system would
exhibit sinusoidal small-amplitude vibrations. This is
because any curve looks linear from very close up. If we
magnify the $F-x$ graph as shown in figure \figref{force-curve-2}/2, it becomes very
difficult to tell that the graph is not a straight line. If
the vibrations were confined to the region shown in \figref{force-curve-2}/2,
they would be very nearly sinusoidal. This is the reason why
sinusoidal vibrations are a universal feature of all
vibrating systems, if we restrict ourselves to small
amplitudes. The theorem is therefore of great general
significance. It applies throughout the universe, to objects
ranging from vibrating stars to vibrating nuclei. A
sinusoidal vibration is known as \index{simple harmonic
motion!defined}simple harmonic motion.

m4_ifelse(__me,1,[:This relates to the fundamental idea behind differential calculus, which is
that up close, any smooth function looks linear. To characterize small oscillations about
the equilibrium at $x=0$ in figure \figref{force-curve-1}, all we need to know is 
the derivative $\left.\der F/\der x\right|_{0}$, which equals $-k$.
That is, a force function $F(x)$ has no ``individuality'' except as defined by $k$.

\begin{eg}{Spring constant related to potential energy}
        The same idea about lack of individuality can be expressed in terms of energy.

        On a graph of $PE$ versus $x$, an equilibrium is a local minimum. We can imagine
        an oscillation about this equilibrium point as if a marble was rolling back and forth in the depression of the graph.
        Let's choose a coordinate system in which $x=0$ is the equilibrium, and since the potential energy
        is only well defined up to an additive constant, we'll simply
        define it to be zero at equilibrium:
        \begin{equation*}
                PE(0) = 0
        \end{equation*}
        Since $x=0$ is a local minimum,
        \begin{equation*}
                \frac{\der{}PE}{\der{}x}(0) = 0\eqquad.
        \end{equation*}
        There are still infinitely many functions that could satisfy these criteria,
        including the three shown in figure \figref{curvature}, which are
        $x^2/2$, $x^2/2(1+x^2)$, and $(e^{3x}+e^{-3x}-2)/18$. Note,
        however, how all three functions are virtually identical right near the
        minimum. That's because they all have the same curvature. More specifically,
        each function has its second derivative equal to 1 at $x=0$, and the second
        derivative is a measure of curvature. Since the $F=-\der PE/\der x$ and $k=-\der PE/\der x$, $k$ equals the second derivative
        of the PE,
        \begin{equation*}
                \frac{\der{}^2 PE}{\der{}x^2}(0) = k\eqquad.
        \end{equation*}
        As shown in figure
        \figref{curvature}, any two functions that have $PE(0)=0$, $\der{}PE/\der{}x=0$, and
        $\der{}^2 PE/\der{}x^2=k$, with the same value of $k$, are virtually indistinguishable
        for small values of $x$, so if we want to analyze small oscillations, it doesn't even
        matter which function we assume. For simplicity, we can always use $PE(x)=(1/2)kx^2$,
        which is the form that gives a constant second derivative.
\end{eg}
<% marg(200) %>
<%
  fig(
    'curvature',
    %q{Three functions with the same curvature at $x=0$.}
  )
%>
<% end_marg %>
<% marg(m4_ifelse(__me,1,-20,-100)) %>
<%
  fig(
    'leverspring',
    %q{Example \ref{eg:leverspring}. The rod pivots on the hinge at the bottom.}
  )
%>
<% end_marg %>

        \begin{eg}{A spring and a lever}\label{eg:leverspring}
        \egquestion
        What is the period of small oscillations of the system shown in the figure?
        Neglect the mass of the lever and the spring. Assume that the spring is
        so stiff that gravity is not an important effect. The spring is relaxed
        when the lever is vertical.
        
        \eganswer
        This is a little tricky, because the spring constant $k$, although it
        is relevant, is \emph{not} the $k$ we should be putting into the equation
        $\omega=\sqrt{k/m}$.
        I find this easier to understand by working with energy rather than force.
        (Another method would be to use torque, as in problem \ref{hw:shmanalogy}.)
        The $k$ that goes into $\sqrt{k/m}$ has to be the second
        derivative of $PE$ with respect to the position, $x$, of the mass that's
        moving. The energy $PE$ stored in the spring depends on how far the
        \emph{tip} of the lever is from the center. This distance equals
        $(L/b)x$, so the energy in the spring is
        \begin{align*}
                PE        &= \frac{1}{2}k\left(\frac{L}{b}x\right)^2 \\
                        &= \frac{kL^2}{2b^2}x^2\eqquad, \\
        \end{align*}
        and the $k$ we have to put in $T=2\pi\sqrt{m/k}$ is
        \begin{equation*}
                \frac{\der^2 PE}{\der x^2}        =  \frac{kL^2}{b^2}\eqquad. \\
        \end{equation*}
        The result is
        \begin{align*}
                \omega  &=  \sqrt{\frac{kL^2}{mb^2}} \\
                        &=  \frac{L}{b}\sqrt{\frac{k}{m}} \\
        \end{align*}
        The leverage of the lever makes it as if the spring was stronger,
        decreasing the period of the oscillations by a factor of $b/L$.
        \end{eg}
<% marg(-9) %>
<%
  fig(
    'utube',
    %q{Example \ref{eg:utube}.}
  )
%>
<% end_marg %>

        \begin{eg}{Water in a U-shaped tube}\label{eg:utube}
        \egquestion
        The U-shaped tube in figure \figref{utube} has cross-sectional area
         $A$, and the density of the water inside
        is $\rho$. Find the gravitational potential energy as a function of the quantity $y$ shown in the
        figure, show that there is an equilibrium at $y$=0, and find
        the frequency of oscillation of the water.

        \eganswer
        Potential energy is only well defined
        up to an additive constant. To fix this constant, let's define $PE$ to be zero when $y$=0.
        The difference between $PE( y)$ and $PE(0)$ is the energy that would
        be required to lift a water column of height $y$ out of the right side, and place it
        above the dashed line, on the left side, raising it through a height $y$.
        This water column has height $y$ and
        cross-sectional area $A$, so its volume is $Ay$, its mass is $\rho Ay$, and
        the energy required is $mgy$=$(\rho Ay) gy$=$\rho gAy^2$.
        We then have $PE( y)= PE(0)+\rho gAy^2=\rho gAy^2$.

        The ``spring constant'' is
        \begin{align*}
                 k        &= \frac{\zu{d}^2 PE}{\zu{d} y^2} \\
                                &= 2\rho gA\eqquad.
        \end{align*}
        This is an interesting example, because $k$ can be calculated without any
        approximations, but the kinetic energy requires an approximation, because we
        don't know the details of the pattern of flow of the water. It could be very complicated.
        There will be a tendency for the water near the walls to flow more slowly due to
        friction, and there may also be swirling, turbulent motion. However, if we make
        the approximation that all the water moves with the same velocity as the surface,
        $\der y/\der t$, then the mass-on-a-spring analysis applies. Letting
        $L$ be the total length of the filled part of the tube, the mass is $\rho LA$, and
        we have
        \begin{align*}
                 \omega        &= \sqrt{ k/ m} \\
                               &= \sqrt{\frac{2\rho gA}{\rho LA}} \\
                               &= \sqrt{\frac{2g}{L}}\eqquad.
        \end{align*}
        \end{eg}

        
:])

<% end_sec() %>
<% begin_sec("Period is approximately independent of amplitude, if the amplitude is small.") %>

Until now we have not even mentioned the most counterintuitive
aspect of the equation  m4_ifelse(__me,1,[:$\omega=\sqrt{k/m}$:],[:$T=2\pi\sqrt{m/k}$:]): it does not depend on amplitude at
all. Intuitively, most people would expect the mass-on-a-spring
system to take longer to complete a cycle if the amplitude
was larger. (We are comparing amplitudes that are different
from each other, but both small enough that the theorem
applies.) In fact the larger-amplitude vibrations take the
same amount of time as the small-amplitude ones. This is because
at large amplitudes, the force is greater, and therefore accelerates
the object to higher speeds.

Legend has it that this fact was first noticed by
\index{Galileo}Galileo during what was apparently a less
than enthralling church service. A gust of wind would now
and then start one of the chandeliers in the cathedral
swaying back and forth, and he noticed that regardless of
the amplitude of the vibrations, the period of oscillation
seemed to be the same. Up until that time, he had been
carrying out his physics experiments with such crude
time-measuring techniques as feeling his own pulse or
singing a tune to keep a musical beat. But after going home
and testing a pendulum, he convinced himself that he had
found a superior method of measuring time. Even without a
fancy system of pulleys to keep the pendulum's vibrations
from dying down, he could get very accurate time measurements,
because the gradual decrease in amplitude due to friction
would have no effect on the pendulum's period. (Galileo
never produced a modern-style pendulum clock with pulleys, a
minute hand, and a second hand, but within a generation the
device had taken on the form that persisted for hundreds of years after.)

m4_ifelse(__lm_series,1,\vfill,\pagebreak)

\begin{eg}{The pendulum}
\egquestion Compare the  m4_ifelse(__me,1,frequencies,periods) of pendula having bobs
with different masses.

\eganswer From the equation  m4_ifelse(__me,1,[:$\omega=\sqrt{k/m}$:],[:$T=2\pi\sqrt{m/k}$:]), we might expect that a larger
mass would lead to a m4_ifelse(__me,1,lower frequency,longer period). However, increasing the
mass also increases the forces that act on the pendulum:
gravity and the tension in the string. This increases $k$ as
well as $m$, so the m4_ifelse(__me,1,frequency,period) of a pendulum is independent of $m$.
\end{eg}


<% end_sec() %>

m4_ifelse(__lm_series,1,\vfill)

\startdqs

m4_ifelse(__lm_series,1,\vfill)

\begin{dq}\label{dq:upright-pendulum}
Suppose that a pendulum has a rigid arm mounted on a bearing, rather than a string tied at its top
with a knot. The bob can then oscillate with center-to-side amplitudes greater than $90\degunit$. For the
maximum amplitude of
$180\degunit$, what can you say about the period?
\end{dq}

m4_ifelse(__lm_series,1,\vfill)

\begin{dq}
In the language of calculus, Newton's second law for a simple harmonic oscillator can be written in the
form $\der^2 x/\der t^2=-(\ldots)x$, where ``\ldots'' refers to a constant, and the minus sign
says that if we pull the object away from equilibrium,
a restoring force tries to bring it back to equlibrium, which is the opposite direction.
This is why we get motion that looks like a sine or cosine function: these are functions that, when
differentiated twice, give back the original function but with an opposite sign.
Now consider the example described in discussion question \ref{dq:upright-pendulum}, where a pendulum is
upright or nearly upright. How does the analysis play out differently?
\end{dq}

<% end_sec() %>
m4_ifelse(__me,0,[:
<% begin_sec("Proofs",4,'proofs',{'optional'=>true}) %>\index{work!done by a varying force}

<% marg(30) %>
<%
  fig(
    'shm-proof',
    %q{%
      The object moves along the circle at constant speed, but even though its overall speed is constant,
      the $x$ and $y$ components of its velocity are continuously changing, as shown by the unequal spacing of the points when projected
      onto the line below. Projected onto the line, its motion is the same as that of an object experiencing a force $F=-kx$.
    }
  )
%>
<% end_marg %>
In this section we prove (1) that a linear $F-x$ graph gives
sinusoidal motion, (2) that the period of the motion is $2\pi\sqrt{m/k}$,
and (3) that the period is independent of the amplitude. You
may omit this section without losing the continuity of the chapter.

The basic idea of the proof can be understood by imagining
that you are watching a child on a merry-go-round from far
away. Because you are in the same horizontal plane as her
motion, she appears to be moving from side to side along a
line. Circular motion viewed edge-on doesn't just look like
any kind of back-and-forth motion, it looks like motion with
a sinusoidal $x-t$ graph, because the sine and cosine
functions can be defined as the $x$ and $y$ coordinates of a
point at angle $\theta $ on the unit circle. The idea of the
proof, then, is to show that an object acted on by a force
that varies as $F=-kx$ has motion that is identical
to circular motion projected down to one dimension. The
$v^2/r$ expression will also fall out at the end.

\begin{eg}{The moons of Jupiter}\label{eg:jovian-moons}
Before moving on to the proof, we illustrate the concept
using the moons
of Jupiter. Their discovery by Galileo was an epochal event
in astronomy, because it proved that not everything in the
universe had to revolve around the earth as had been
believed. Galileo's telescope was of poor quality by modern
standards, but figure \figref{jovian-moons} shows a simulation of how
Jupiter and its moons might appear at intervals of three
hours through a large present-day instrument. Because we see
the moons' circular orbits edge-on, they appear to perform
sinusoidal vibrations. Over this time period, the innermost
moon, Io, completes half a cycle.
\end{eg}

<%
  fig(
    'jovian-moons',
    %q{Example \ref{eg:jovian-moons}.},
    {
      'width'=>'wide',
      'sidecaption'=>true
    }
  )
%>


For an object performing uniform circular motion, we have
\begin{equation*}
                |\vc{a}|         =    \frac{v^2}{r}\eqquad.
\end{equation*}
The $x$ component of the acceleration is therefore
\begin{equation*}
                        a_x         =    \frac{v^2}{r}\cos\theta\eqquad,
\end{equation*}
where $\theta $ is the angle measured counterclockwise from
the $x$ axis. Applying Newton's second law,
\begin{align*}
 \frac{F_x}{m} &= -\frac{v^2}{r}\cos\theta\eqquad, \qquad \text{so} \\
 F_x &= -m\frac{v^2}{r}\cos\theta\eqquad.
\end{align*}
Since our goal is an equation involving the period, it is
natural to eliminate the variable $v = \text{circumference}/T=2\pi r/T$, giving
\begin{equation*}
                        F_x         =    -\frac{4\pi^2 mr}{T^2}\cos\theta\eqquad.
\end{equation*}
The quantity $r \cos \theta $ is the same as $x$, so we have
\begin{equation*}
                        F_x         =    -\frac{4\pi^2 m}{T^2}\: x\eqquad.
\end{equation*}
Since everything is constant in this equation except for
$x$, we have proved that motion with force proportional to
$x$ is the same as circular motion projected onto a line,
and therefore that a force proportional to $x$ gives
sinusoidal motion. Finally, we identify the constant factor
of $4\pi^2m/T^2$ with $k$, and solving for $T$ gives the desired
equation for the period,
\begin{equation*}
                        T         =     2\pi\sqrt{\frac{m}{k}}\eqquad.
\end{equation*}
Since this equation is independent of $r$, $T$ is independent of the amplitude,
subject to the initial assumption of perfect $F=-kx$ behavior, which in reality
will only hold approximately for small $x$.

<% end_sec() %>:])
\begin{summary}

\begin{vocab}

\vocabitem{periodic motion}{motion that repeats itself over and over}

\vocabitem{period}{the time required for one cycle of a periodic motion}

\vocabitem{frequency}{the number of cycles per second, the inverse of the period}

\vocabitem{amplitude}{the amount of vibration, often measured from the
center to one side; may have different units depending on
the nature of the vibration}

\vocabitem{simple harmonic motion}{motion whose $x-t$ graph is a sine wave}

\end{vocab}

\begin{notation}

\notationitem{$T$}{period}

\notationitem{$f$}{frequency}

\notationitem{$A$}{amplitude}

\notationitem{$k$}{the slope of the graph of $F$ versus $x$, where $F$ is the
total force acting on an object and $x$ is the object's
position; for a spring, this is known as the spring constant.}

m4_ifelse(__me,1,[:\notationitem{$\omega$ (Greek letter ``omega'')}{$2\pi f$}:])

\end{notation}


\begin{othernotation}

\notationitem{$\nu$}{The Greek letter $\nu $, nu, is used in many books for frequency.}

m4_ifelse(__lm_series,1,[:\notationitem{$\omega$}{The Greek letter $\omega $, omega, is often used as an
abbreviation for $2\pi f$.}:])

\end{othernotation}

\begin{summarytext}

Periodic motion is common in the world around us because of
conservation laws. An important example is one-dimensional
motion in which the only two forms of energy involved are
potential and kinetic; in such a situation, conservation of
energy requires that an object repeat its motion, because
otherwise when it came back to the same point, it would have
to have a different kinetic energy and therefore a
different total energy.

Not only are periodic vibrations very common, but small-amplitude
vibrations are always sinusoidal as well. That is, the $x-t$
graph is a sine wave. This is because the graph of force
versus position will always look like a straight line on a
sufficiently small scale. This type of vibration is called
simple harmonic motion. In simple harmonic motion, the
m4_ifelse(__me,1,frequency,period) is independent of the amplitude, and is given by
\begin{equation*}
m4_ifelse(__me,1,[:\omega=\sqrt{k/m}:],[:T=2\pi\sqrt{m/k}:])\eqquad.
\end{equation*}

\end{summarytext}

\end{summary}
