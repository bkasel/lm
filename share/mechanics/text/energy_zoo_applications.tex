<% begin_sec("Applications",m4_ifelse(__lm_series,1,4,3),'energy-zoo-applications') %>

<% begin_sec("Heat transfer",nil,'heat-transfer') %>

<% begin_sec("Conduction",nil,'heat-conduction') %>

When you hold a hot potato in your hand, energy is transferred from the hot object to
the cooler one. Our microscopic picture of this process (figure \figref{random-motion}, p.~\pageref{fig:random-motion})
tells us that the heat transfer can only occur at the surface of contact, where one layer of atoms in the
potato skin make contact with one such layer in the hand. This type of heat transfer is called
\emph{conduction},\index{conduction of heat}\index{heat transfer!by conduction}
and its rate is proportional to both the surface area and the temperature difference.
<% end_sec('heat-conduction') %>

<% begin_sec("Convection",nil,'convection') %>
In a gas or a liquid, a faster method of heat transfer can occur, because hotter or colder parts
of the fluid can flow, physically transporting their heat energy from one place to another. This
mechanism of heat transfer, \emph{convection},\index{convection}\index{heat transfer!by convection}
is at work in Los Angeles when hot Santa Ana winds blow in from the Mojave Desert. On a cold day,
the reason you feel warmer when there is no wind is that your skin warms a thin layer of air near
it by conduction. If a gust of wind comes along, convection robs you of this layer.
A thermos bottle has inner and outer walls separated by a layer of vacuum, which prevents heat transport
by conduction or convection, except for a tiny amount of conduction through the thin connection between the walls, near the neck, which has
a small cross-sectional area.
<% end_sec('convection') %>

<% begin_sec("Radiation",nil,'heat-radiation') %>
The glow of the sun or a candle flame is an example of heat transfer
by \emph{radiation}.\index{radiation of heat}\index{heat transfer!by
radiation} In this context, ``radiation'' just means anything that
radiates outward from a source, including, in these examples, ordinary
visible light. The power is proportional to the surface area of the radiating object. It also
depends very dramatically on the radiator's absolute temperature, $P\propto T^4$. 

We can easily understand the reason for radiation based on the
picture of heat as random kinetic energy at the atomic scale. Atoms are made out of subatomic particles,
such as electrons and nuclei, that carry electric charge. When a charged particle vibrates, it creates
wave disturbances in the electric and magnetic fields, and the waves have a frequency (number of vibrations
per second) that matches the frequency of the particle's motion. If this frequency is in the right range,
they constitute visible light%
m4_ifelse(__lm_series,1,[:%
 (see section \ref{subsec:em-spectrum}, p.~\pageref{subsec:em-spectrum}).
:],[:%
.
:])%
m4_ifelse(__lm_series,1,[:%
In figure \figref{plutonium-glowing}, the nuclear and electrical potential energy
in the plutonium pellet cause the pellet to heat up, and an equilibrium is reached, in which the heat is radiated
away just as quickly as it is produced. :],[::])%
When an object is closer to room temperature, it glows in the invisible
infrared part of the m4_ifelse(__lm_series,1,[:spectrum (figure \figref{irface}).:],[:spectrum.:])
m4_ifelse(__lm_series,1,[:
<% marg(15) %>
<%
  fig(
    'irface',
    %q{%
      A portrait of a man's face made with infrared light, a color of light that lies beyond the red end of
              the visible rainbow.
              His warm skin emits quite a bit of infrared light energy, while his
              hair, at a lower temperature, emits less.
    }
  )
%>
<% end_marg %>
:])

<% end_sec('heat-radiation') %>

<% end_sec('heat-transfer') %>

<% begin_sec("Earth's energy equilibrium",nil,'earth-energy-equilibrium') %>
Our planet receives a nearly constant amount of energy from the sun (about $1.8\times10^{17}\ \zu{W}$).
If it hadn't had any mechanism for getting rid of that energy, the result would have been some kind of catastrophic explosion
soon after its formation. Even a 10\% imbalance between energy input and output, if maintained steadily from the time
of the Roman Empire until the present, would have been enough to raise the oceans to a boil.
So evidently the earth does dump this energy somehow. How does it do it?
Our planet is surrounded by the vacuum of outer space, like the ultimate thermos bottle.
Therefore it can't expel heat by conduction or convection, but
it does radiate in the infrared, and this is the \emph{only} available mechanism for cooling.

% calc -xe "m=1.4 10^21 kg; c=4 10^3 J/kg; T=100; P=1.8 10^17 W; t=(mcT)/P; t/[365*24*3600]"
% 99 years
<% end_sec('earth-energy-equilibrium') %>

<% begin_sec("Global warming",nil,'global-warming') %>\index{global warming}\index{climate change}

It was realized starting around 1930 that this created a dangerous vulnerability in our biosphere.
Our atmosphere is only about 0.04\% carbon dioxide, but carbon dioxide is an extraordinarily efficient
absorber of infrared light. It is, however, transparent to visible light. Therefore any increase
in the concentration of carbon dioxide would decrease the efficiency of cooling by radiation, while
allowing in just as much heat input from visible light. When we burn fossil fuels such as gasoline
or coal, we release into the atmosphere carbon that had previously been locked away underground.
This results in a shift to a new energy balance. The average temperature $T$ of the land and oceans increases
until the $T^4$ dependence of radiation compensates for the additional absorption of infrared light.

<% marg(-300) %>
<%
  fig(
    'greenhouse-effect',
    %q{%
      The ``greenhouse effect.'' Carbon dioxide in the atmosphere allows visible light in, but partially
      blocks the reemitted infrared light.
    }
  )
%>
\spacebetweenfigs
<%
  fig(
    'global-warming',
    %q{%
      Global average temperatures over the last 2000 years. The black line is from thermometer measurements.
      The colored lines are from various indirect indicators such as tree rings, ice cores, buried pollen, and corals.
    }
  )
%>
<% end_marg %>

By about 1980, a clear scientific consensus had emerged that this effect was real, that it was
caused by human activity, and that it had resulted in an abrupt increase in the earth's
average temperature. We know, for example,  from radioisotope studies that the effect has not been caused by the release of
carbon dioxide in volcanic eruptions. The temperature increase has been verified by multiple independent
methods, including studies of tree rings and coral reefs. Detailed computer models have correctly
predicted a number of effects that were later verified empirically, including a rise in sea levels,
and day-night and pole-equator variations. There is no longer any controversy among climate scientists
about the existence or cause of the effect.

One solution to the problem is to replace fossil fuels with renewable sources of energy such as
solar power and wind. However, these cannot be brought online fast enough to prevent severe
warming in the next few decades, so nuclear power is also a critical piece of the
m4_ifelse(__lm_series,1,[:%
puzzle (see
section \ref{subsec:biological-effects-radiation}, p.~\pageref{subsec:biological-effects-radiation}).
:],[:%
puzzle.
:])

<% end_sec('global-warming') %>

<% end_sec('energy-zoo-applications') %>
