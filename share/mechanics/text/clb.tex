Variety is the spice of life, not of science. The figure
shows a few examples from the bewildering array of forms of
energy that surrounds us. The physicist's psyche rebels
against the prospect of a long laundry list of types of
energy, each of which would require its own equations,
concepts, notation, and terminology. The point at which
we've arrived in the study of energy is analogous to the
period in the 1960's when a half a dozen new subatomic
particles were being discovered every year in particle
accelerators. It was an embarrassment. Physicists began to
speak of the ``\index{particle zoo}particle zoo,'' and it
seemed that the subatomic world was distressingly complex.
The particle zoo was simplified by the realization that most
of the new particles being whipped up were simply clusters
of a previously unsuspected set of more fundamental
particles (which were whimsically dubbed \index{quarks}quarks,
a made-up word from a line of poetry by \index{Joyce,
James}James Joyce, ``Three quarks for Master Mark.'') The
energy zoo can also be simplified, and it is the purpose of
this chapter to demonstrate the hidden similarities between
forms of energy as seemingly different as heat and motion.


<%
  fig(
    'can-imploding',
    %q{%
      A vivid demonstration that heat is a form of
      motion. A small amount of boiling water is poured
      into the empty can, which rapidly fills up with
      hot steam. The can is then sealed tightly, and
      soon crumples. This can be explained as
      follows. The high temperature of the steam is
      interpreted as a high average speed of random
      motions of its molecules. Before the lid was put
      on the can, the rapidly moving steam molecules
      pushed their way out of the can, forcing the
      slower air molecules out of the way. As the steam
      inside the can thinned out, a stable situation was
      soon achieved, in which the force from the less
      dense steam molecules moving at high speed
      balanced against the force from the more dense
      but slower air molecules outside. The cap was
      put on, and after a while the steam inside the
      can reached the same temperature as the air outside.
      The force from the cool,
      thin steam no longer matched the force from the
      cool, dense air outside, and the imbalance of
      forces crushed the can.
    },
    {
      'width'=>'wide',
      'sidecaption'=>true
    }
  )
%>

<% begin_sec("Heat is kinetic energy",0) %>

What is \index{heat!as a form of kinetic energy}heat really?
Is it an invisible fluid that your bare feet soak up from a
hot sidewalk? Can one ever remove all the heat from an
object? Is there a maximum to the temperature scale?

The theory of heat as a \index{heat!as a fluid}fluid seemed
to explain why colder objects absorbed heat from hotter
ones, but once it became clear that heat was a form of
energy, it began to seem unlikely that a material substance
could transform itself into and out of all those other forms
of energy like motion or light. For instance, a compost pile
gets hot, and we describe this as a case where, through the
action of bacteria, chemical energy stored in the plant
cuttings is transformed into heat energy. The heating occurs
even if there is no nearby warmer object that could have
been leaking ``heat fluid'' into the pile.


An alternative interpretation of heat was suggested by the
theory that matter is made of atoms. Since gases are
thousands of times less dense than solids or liquids, the
atoms (or clusters of atoms called molecules) in a gas must
be far apart. In that case, what is keeping all the air
molecules from settling into a thin film on the floor of the
room in which you are reading this book? The simplest
explanation is that they are moving very rapidly, continually
ricocheting off of the floor, walls, and ceiling. Though
bizarre, the cloud-of-bullets image of a gas did give a
natural explanation for the surprising ability of something
as tenuous as a gas to exert huge forces. Your car's tires
can hold it up because you have pumped extra molecules into
them. The inside of the tire gets hit by molecules more
often than the outside, forcing it to stretch and stiffen.

The outward forces of the air in your car's tires increase
even further when you drive on the freeway for a while,
heating up the rubber and the air inside. This type of
observation leads naturally to the conclusion that hotter
matter differs from colder in that its atoms' random motion
is more rapid. In a liquid, the motion could be visualized
as people in a milling crowd shoving past each other more
quickly. In a solid, where the atoms are packed together,
the motion is a random vibration of each atom as it knocks
against its neighbors.


<% marg(0) %>
<%
  fig(
    'random-motion',
    %q{%
      Random motion of atoms in a gas, a
      liquid, and a solid.
    }
  )
%>
<% end_marg %>
We thus achieve a great simplification in the theory of
heat. Heat is simply a form of kinetic energy, the total
kinetic energy of random motion of all the atoms in an
object. With this new understanding, it becomes possible to
answer at one stroke the questions posed at the beginning of
the section. Yes, it is at least theoretically possible to
remove all the heat from an object. The coldest possible
temperature, known as absolute zero, is that at which all
the atoms have zero velocity, so that their kinetic
energies, $(1/2)mv^2$, are all zero. No, there is no maximum amount of
heat that a certain quantity of matter can have, and no
maximum to the temperature scale, since arbitrarily large
values of $v$ can create arbitrarily large amounts of
kinetic energy per atom.

The kinetic theory of heat also provides a simple explanation
of the true nature of temperature. \index{temperature!as a
measure of energy per atom}Temperature is a measure of the
amount of energy per molecule, whereas heat is the total
amount of energy possessed by all the molecules in an object.

There is an entire branch of physics, called \index{thermodynamics}thermodynamics,
that deals with heat and temperature and forms the basis for
technologies such as refrigeration. 
m4_ifelse(__me,1,[:
Thermodynamics is not covered in this book,
:],[:Thermodynamics is
discussed in more detail in optional chapter \ref{ch:thermo},
:])%
and I have provided here only a brief overview of
the thermodynamic concepts that relate directly to energy,
glossing over at least one point that would be dealt with
more carefully in a thermodynamics course: it is really only
true for a gas that all the heat is in the form of kinetic
energy. In solids and liquids, the atoms are close enough to
each other to exert intense electrical forces on each other,
and there is therefore another type of energy involved, the
energy associated with the atoms' distances from each other.
Strictly speaking, heat energy is defined not as energy
associated with random motion of molecules but as any form
of energy that can be conducted between objects in
contact, without any force.

<% end_sec() %>
<% begin_sec("Potential energy: energy of distance or closeness",3) %>%
\index{energy!potential}

We have already seen many examples of energy related to the
distance between interacting objects. When two objects
participate in an attractive noncontact force, energy is
required to bring them farther apart. In both of the
perpetual motion machines that started off the previous
chapter, one of the types of energy involved was the energy
associated with the distance between the balls and the
earth, which attract each other gravitationally. In the
perpetual motion machine with the magnet on the pedestal,
there was also energy associated with the distance between
the magnet and the iron ball, which were attracting each other.

The opposite happens with repulsive forces: two socks with
the same type of static electric charge will repel each
other, and cannot be pushed closer together without supplying energy.

In general, the term \emph{potential energy,} with algebra
symbol\emph{ PE,} is used for the energy associated with the
distance between two objects that attract or repel each
other via a force that depends on the distance between them.
Forces that are not determined by distance do not have
potential energy associated with them. For instance, the
normal force acts only between objects that have zero
distance between them, and depends on other factors besides
the fact that the distance is zero. There is no potential
energy associated with the normal force.

<% marg(0) %>
<%
  fig(
    'pool-skater',
    %q{%
      The skater has converted all his kinetic
      energy into potential energy on the
      way up the side of the pool.
    }
  )
%>
<% end_marg %>
The following are some commonplace examples of potential energy:

\begin{description}
\item[gravitational potential energy:] The skateboarder in the
photo has risen from the bottom of the pool, converting
kinetic energy into gravitational potential energy. After
being at rest for an instant, he will go back down,
converting PE back into KE.

\item[magnetic potential energy:] When a magnetic compass needle is
allowed to rotate, the poles of the compass change their
distances from the earth's north and south magnetic poles,
converting magnetic potential energy into kinetic energy.
(Eventually the kinetic energy is all changed into heat by
friction, and the needle settles down in the position that
minimizes its potential energy.)

\item[electrical potential energy:] Socks coming out of the dryer
cling together because of attractive electrical forces.
Energy is required in order to separate them.

\item[potential energy of bending or stretching:] The force between
the two ends of a spring depends on the distance between
them, i.e., on the length of the spring. If a car is pressed
down on its shock absorbers and then released, the potential
energy stored in the spring is transformed into kinetic and
gravitational potential energy as the car bounces back up.
\end{description}

 I have deliberately avoided introducing the term potential
energy up until this point, because it tends to produce
unfortunate connotations in the minds of students who have
not yet been inoculated with a careful description of the
construction of a numerical energy scale. Specifically,
there is a tendency to generalize the term inappropriately
to apply to any situation where there is the ``potential''
for something to happen: ``I took a break from digging, but
I had potential energy because I knew I'd be ready to work
hard again in a few minutes.''

<% begin_sec("An equation for gravitational potential energy") %>

All the vital points about potential energy can be made by
focusing on the example of gravitational potential energy.
For simplicity, we treat only vertical motion, and motion
close to the surface of the earth, where the gravitational
force is nearly constant. (The generalization to the three
dimensions and varying forces is more easily accomplished
using the concept of work, which is the subject of the next chapter.)

<% marg(20) %>
<%
  fig(
    'pool-skater-line-art',
    %q{%
      As the skater free-falls, his PE is
      converted into KE. (The numbers would
      be equally valid as a description of his
      motion on the way up.)
    }
  )
%>
<% end_marg %>
\begin{numberedequations}\label{pe-derivation}
To find an equation for gravitational PE, we examine the
case of free fall, in which energy is transformed between
kinetic energy and gravitational PE. Whatever energy is lost
in one form is gained in an equal amount in the other form,
so using the notation $\Delta KE$ to stand for
$KE_f-KE_i$ and a similar notation for PE, we have
\begin{equation}
   \Delta KE  =  -\Delta PE_{grav}\eqquad.        
\end{equation}
 It will be convenient to refer to the object as falling, so
that PE is being changed into KE, but the math applies
equally well to an object slowing down on its way up. We
know an equation for kinetic energy,
\begin{equation}
    KE=\frac{1}{2}mv^2\eqquad, 
\end{equation}
so if we can relate $v$ to height, $y$, we will be able to
relate $\Delta PE$ to $y$, which would tell us what
we want to know about potential energy. The $y$ component of
the velocity can be connected to the height via the constant
acceleration equation
\begin{equation}
    v_f^2 = v_i^2 + 2 a \Delta y\eqquad, 
\end{equation}
and Newton's second law provides the acceleration,
\begin{equation}
                a  =  F/m\eqquad,
\end{equation}
in terms of the gravitational force.
\end{numberedequations}

The algebra is simple because both equation [2] and
equation [3] have velocity to the second power. Equation [2]
can be solved for $v^2$ to give $v^2 = 2KE/m$, and
substituting this into equation [3], we find
\begin{equation*}
                2\frac{KE_f}{m}  =  2\frac{KE_i}{m} + 2a\Delta y\eqquad.
\end{equation*}
Making use of equations [1] and [4] gives the simple result
\begin{multline*}
                \Delta PE_{grav}  =  -F\Delta y\eqquad. \hfill
\shoveright{\text{[change in  gravitational PE}}\\
\shoveright{\text{resulting from a change in height $\Delta y$;}}\\
\shoveright{\text{$F$ is the gravitational force on the object,}}\\
\shoveright{\text{i.e., its weight; valid only near the surface}}\\
\shoveright{\text{of the earth, where $F$ is constant]}}\\
\end{multline*}
\index{potential energy!gravitational}\index{energy!gravitational potential energy}

\vfill

\begin{eg}{Dropping a rock}
\egquestion If you drop a 1-kg rock from a height of 1 m,
how many joules of KE does it have on impact with the
ground? (Assume that any energy transformed into heat by air
friction is negligible.)

\eganswer If we choose the $y$ axis to point up, then $F_y$
is negative, and equals $-(1\ \kgunit)(g)=-9.8\ \nunit$. A decrease
in $y$ is represented by a negative value of $\Delta y$,
$\Delta y=-1\ \munit$, so the change in potential energy is
$-(-9.8\ \zu{N})(-1\ \munit)\approx-10\ \zu{J}$. (The proof that newtons
multiplied by meters give units of joules is left as a
homework problem.) Conservation of energy says that the loss
of this amount of PE must be accompanied by a corresponding
increase in KE of 10 J.
\end{eg}

\vfill

It may be dismaying to note how many minus signs had to be
handled correctly even in this relatively simple example: a
total of four. Rather than depending on yourself to avoid
any mistakes with signs, it is better to check whether the
final result make sense physically. If it doesn't,
just reverse the sign.

Although the equation for gravitational potential energy was
derived by imagining a situation where it was transformed
into kinetic energy, the equation can be used in any
context, because all the types of energy are freely
convertible into each other.

\pagebreak

\begin{eg}{Gravitational PE converted directly into heat}
\egquestion A 50-kg firefighter slides down a 5-m pole at
constant velocity. How much heat is produced?

\eganswer Since she slides down at constant velocity, there
is no change in KE. Heat and gravitational PE are the only
forms of energy that change. Ignoring plus and minus signs,
the gravitational force on her body equals $mg$, and
the amount of energy transformed is
\begin{equation*}
                (mg)(5\ \munit)  =  2500\ \junit\eqquad.
\end{equation*}
On physical grounds, we know that there must have been an
increase (positive change) in the heat energy in her hands
and in the flagpole.
\end{eg}

Here are some questions and answers about the interpretation
of the equation $\Delta PE_{grav} =-F\Delta y$ for
gravitational potential energy.

\newcommand{\qandaquestion}{\noindent\textbf{Question:}\ }
\newcommand{\qandaanswer}{\\{}\noindent\textbf{Answer:}\ }

\qandaquestion In a nutshell, why is there a minus sign in the equation?
 %
\qandaanswer It is because we increase the PE by moving the
object in the \emph{opposite} direction compared to the
gravitational force.

\qandaquestion Why do we only get an equation for the \emph{change}
in potential energy? Don't I really want an equation for the
potential energy itself?
 %
\qandaanswer No, you really don't. This relates to a basic fact
about potential energy, which is that it is not a well
defined quantity in the absolute sense. Only changes in
potential energy are unambiguously defined. If you and I
both observe a rock falling, and agree that it deposits 10
J of energy in the dirt when it hits, then we will be
forced to agree that the 10 J of KE must have come from a
loss of 10 joules of PE. But I might claim that it started
with 37 J of PE and ended with 27, while you might swear
just as truthfully that it had 109 J initially and 99 at
the end. It is possible to pick some specific height as a
reference level and say that the PE is zero there, but it's
easier and safer just to work with changes in PE and avoid
absolute PE altogether.

\qandaquestion You referred to potential energy as the energy
that \emph{two} objects have because of their distance from
each other. If a rock falls, the object is the rock.
Where's the other object?
 %
\qandaanswer Newton's third law guarantees that there will always
be two objects. The other object is the planet earth.

\qandaquestion If the other object is the earth, are we talking
about the distance from the rock to the center of the earth
or the distance from the rock to the surface of the earth?
 %
\qandaanswer It doesn't matter. All that matters is the change in
distance, $\Delta y$, not $y$. Measuring from the earth's
center or its surface are just two equally valid choices of
a reference point for defining absolute PE.

\qandaquestion Which object contains the PE, the rock or the earth?
 %
\qandaanswer We may refer casually to the PE of the rock, but
technically the PE is a relationship between the earth and
the rock, and we should refer to the earth and the rock
together as possessing the PE.

\qandaquestion How would this be any different for a force other than gravity?
 %
\qandaanswer It wouldn't. The result was derived under the
assumption of constant force, but the result would be valid
for any other situation where two objects interacted through
a constant force. Gravity is unusual, however, in that the
gravitational force on an object is so nearly constant under
ordinary conditions. The magnetic force between a magnet and
a refrigerator, on the other hand, changes drastically with
distance. The math is a little more complex for a varying
force, but the concepts are the same.

\qandaquestion Suppose a pencil is balanced on its tip and then
falls over. The pencil is simultaneously changing its height
and rotating, so the height change is different for
different parts of the object. The bottom of the pencil
doesn't lose any height at all. What do you do in this situation?
 %
\qandaanswer The general philosophy of energy is that an object's
energy is found by adding up the energy of every little part
of it. You could thus add up the changes in potential energy
of all the little parts of the pencil to find the total
change in potential energy. Luckily there's an easier way!
The derivation of the equation for gravitational potential
energy used Newton's second law, which deals with the
acceleration of the object's center of mass (i.e., its
balance point). If you just define $\Delta y$ as the height
change of the center of mass, everything works out. A huge
Ferris wheel can be rotated without putting in or taking out
any PE, because its center of mass is staying at the same height.

<% self_check('throw-up-or-down',<<-'SELF_CHECK'
A ball thrown straight up will have the same speed on impact
with the ground as a ball thrown straight down at the same
speed. How can this be explained using potential energy?
  SELF_CHECK
  ) %>

\startdq

\begin{dq}
You throw a steel ball up in the air. How can you prove
based on conservation of energy that it has the same speed
when it falls back into your hand? What if you throw a
feather up --- is energy not conserved in this case?
\end{dq}

<% end_sec() %>

<% marg(28.5) %>
<%
  fig(
    'pe-at-the-atomic-level',
    %q{%
      All these energy transformations turn
      out at the atomic level to be changes
      in potential energy resulting from
      changes in the distances between
      atoms.
    }
  )
%>
<% end_marg %>

<% end_sec() %>
<% begin_sec("All energy is potential or kinetic",4) %>

In the same way that we found that a change in temperature
is really only a change in kinetic energy at the atomic
level, we now find that every other form of energy turns out
to be a form of \index{potential energy!electrical}potential
energy. Boiling, for instance, means knocking some of the
atoms (or molecules) out of the liquid and into the space
above, where they constitute a gas. There is a net
attractive force between essentially any two atoms that are
next to each other, which is why matter always prefers to be
packed tightly in the solid or liquid state unless we supply
enough potential energy to pull it apart into a gas. This
explains why water stops getting hotter when it reaches the
boiling point: the power being pumped into the water by your
stove begins going into potential energy rather than kinetic energy.

As shown in figure \figref{pe-at-the-atomic-level}, every stored form of
energy that we encounter in everyday life turns out to be a
form of potential energy at the atomic level. The forces
between atoms are electrical and magnetic in nature, so
these are actually electrical and magnetic potential energies.

m4_ifelse(__lm_series,1,[:%
Although light is a topic of the second half of this course,
it is useful to have a preview of how it fits in here. Light is a
wave composed of oscillating electric and magnetic fields, so
we can include it under the category of electrical and magnetic potential energy.%
:])

Even if we wish to include nuclear reactions in the picture,
there still turn out to be only four fundamental types of energy:
\index{potential energy!nuclear}

\begin{indentedblock}
\textbf{kinetic energy} (including heat)
\end{indentedblock}
\begin{indentedblock}
\textbf{gravitational PE}
\end{indentedblock}
\begin{indentedblock}
\textbf{electrical and magnetic PE} (including light)
\end{indentedblock}
\begin{indentedblock}
\textbf{nuclear PE}
\end{indentedblock}

m4_ifelse(__me,1,[:%
How does light fit into this picture? Optional section \ref{sec:ultrarelativistic}
discussed the idea of modeling a ray of light as a stream of massless particles.
But the way in which we described the energy of such particles was completely
different from the use of $KE=(1/2)mv^2$ for objects made of atoms. Since the
purpose of this chapter has been to bring every form of energy under the same
roof, this inconsistency feels unsatisfying. Section \ref{sec:mass-energy-equivalence}
eliminates this inconsistency.
:])

<% marg(120) %>
<%
  fig(
    'fission',
    %q{%
      %
      This figure looks similar to the previous
       ones, but the scale is a million
      times smaller. The little balls are the
      neutrons and protons that make up the
      tiny nucleus at the center of the
      uranium atom. When the nucleus splits
      (fissions), the potential energy change
      is partly electrical and partly a change
      in the potential energy derived from the
      force that holds atomic nuclei together
      (known as the strong nuclear force).
      
    }
  )
%>
\spacebetweenfigs
<%
  fig(
    'plutonium-glowing',
    %q{%
      A pellet of plutonium-238 glows with its own heat. Its nuclear potential energy
      is being converted into heat, a form of kinetic energy. Pellets of this type are
      used as power supplies on some space probes.
    }
  )
%>
<% end_marg %>

\startdq

\begin{dq}
Referring back to the pictures at the beginning of the
chapter, how do all these forms of energy fit into the
shortened list of categories given above?
\end{dq}
<% end_sec() %>

__incl(text/energy_zoo_applications)
