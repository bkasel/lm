If you drop your shoe and a coin side by side, they hit the
ground at the same time. Why doesn't the shoe get there
first, since gravity is pulling harder on it? How does the
lens of your eye work, and why do your eye's muscles need to
squash its lens into different shapes in order to focus on
objects nearby or far away? These are the kinds of questions
that physics tries to answer about the behavior of light and
matter, the two things that the universe is made of.

<% begin_sec("The scientific method",0) %>

Until very recently in history, no progress was made in
answering questions like these. Worse than that, the
\emph{wrong} answers written by thinkers like the ancient
Greek physicist Aristotle were accepted without question for
thousands of years. Why is it that scientific knowledge has
progressed more since the \index{Renaissance}Renaissance
than it had in all the preceding millennia since the
beginning of recorded history? Undoubtedly the industrial
revolution is part of the answer. Building its centerpiece,
the steam engine, required improved techniques for precise
construction and measurement. (Early on, it was considered a
major advance when English machine shops learned to build
pistons and cylinders that fit together with a gap narrower
than the thickness of a penny.) But even before the
industrial revolution, the pace of discovery had picked up,
mainly because of the introduction of the modern scientific
method. Although it evolved over time, most scientists today
would agree on something like the following list of the
basic principles of the \index{scientific method}scientific method:

<% marg(0) %>
<%
  fig(
    'thy-and-expt',
    %q{Science is a cycle of theory and experiment.}
  )
%>
<% end_marg %>

(1) \emph{Science is a cycle of theory and experiment.}
Scientific theories %
% Footnote is to comply with Ca. state standards. Not needed in SN.
m4_ifelse(__sn,1,[::],[:\footnote{The term ``theory'' in science does not just mean ``what someone thinks,'' or even ``what a lot of scientists think.'' It means
an interrelated set of statements that have predictive value, and that have survived a broad set of empirical tests. Thus, both Newton's law of gravity and
Darwinian evolution are scientific \emph{theories}. A ``hypothesis,'' in contrast to a theory, is any statement of interest that can be empirically tested.
That the moon is made of cheese is a hypothesis, which was empirically tested, for example, by the Apollo
astronauts.\index{theory}\index{hypothesis}}:]) %
are created to explain the results of
experiments that were created under certain conditions. A
successful theory will also make new predictions about new
experiments under new conditions. Eventually, though, it
always seems to happen that a new experiment comes along,
showing that under certain conditions the theory is not a
good approximation or is not valid at all. The ball is then
back in the theorists' court. If an experiment disagrees
with the current theory, the theory has to be changed, not the experiment.

(2) \emph{Theories should both predict and explain.} The
requirement of predictive power means that a theory is only
meaningful if it predicts something that can be checked
against experimental measurements that the theorist did not
already have at hand. That is, a theory should be testable.
Explanatory value means that many phenomena should be
accounted for with few basic principles. If you answer every
``why'' question with ``because that's the way it is,'' then
your theory has no explanatory value. Collecting lots of
data without being able to find any basic underlying
principles is not science.

(3) \emph{Experiments should be reproducible.} An experiment
should be treated with suspicion if it only works for one
person, or only in one part of the world. Anyone with the
necessary skills and equipment should be able to get the
same results from the same experiment. This implies that
science transcends national and ethnic boundaries; you can
be sure that nobody is doing actual science who claims that
their work is ``Aryan, not Jewish,'' ``Marxist, not
bourgeois,'' or ``Christian, not atheistic.'' An experiment
cannot be reproduced if it is secret, so science is
necessarily a public enterprise.
m4_ifelse(__sn,1,[:\enlargethispage{-\baselineskip}:],[::])

<% marg(15) %>
<%
  fig(
    'alchemy',
    %q{%
      A satirical drawing of an alchemist's
      laboratory. H. Cock, after a drawing
      by Peter Brueghel the Elder (16th
      century).
    }
  )
%>
<% end_marg %>
As an example of the cycle of theory and experiment, a vital
step toward modern chemistry was the experimental observation
that the chemical elements could not be transformed into
each other, e.g., lead could not be turned into gold. This
led to the theory that chemical reactions consisted of
rearrangements of the elements in different combinations,
without any change in the identities of the elements
themselves. The theory worked for hundreds of years, and was
confirmed experimentally over a wide range of pressures and
temperatures and with many combinations of elements. Only in
the twentieth century did we learn that one element could be
trans-formed into one another under the conditions of
extremely high pressure and temperature existing in a
nuclear bomb or inside a star. That observation didn't
completely invalidate the original theory of the immutability
of the elements, but it showed that it was only an
approximation, valid at ordinary temperatures and pressures.

<% self_check('psychic',<<-'SELF_CHECK'
A psychic conducts seances in which the spirits of the dead
speak to the participants. He says he has special psychic
powers not possessed by other people, which allow him to
``channel'' the communications with the spirits.  What part
of the scientific method is being violated here?
  SELF_CHECK
  ) %>

 The scientific method as described here is an idealization,
and should not be understood as a set procedure for doing
science. Scientists have as many weaknesses and character
flaws as any other group, and it is very common for
scientists to try to discredit other people's experiments
when the results run contrary to their own favored point of
view. Successful science also has more to do with luck,
intuition, and creativity than most people realize, and the
restrictions of the scientific method do not stifle
individuality and self-expression any more than the fugue
and sonata forms stifled Bach and Haydn. There is a recent
tendency among social scientists to go even further and to
deny that the scientific method even exists, claiming that
science is no more than an arbitrary social system that
determines what ideas to accept based on an in-group's
criteria. I think that's going too far. If science is an
arbitrary social ritual, it would seem difficult to explain
its effectiveness in building such useful items as
airplanes, CD players, and sewers. If \index{alchemy}alchemy
and \index{astrology}astrology were no less scientific in
their methods than chemistry and astronomy, what was it that
kept them from producing anything useful?
m4_ifelse(__sn,1,[:\enlargethispage{-\baselineskip}:],[::])

\startdqswithintro{%
Consider whether or not the scientific method is being
applied in the following examples. If the scientific method
is not being applied, are the people whose actions are being
described performing a useful human activity, albeit
an unscientific one?
}

\begin{dq}
Acupuncture is a traditional medical technique of Asian
origin in which small needles are inserted in the patient's
body to relieve pain. Many doctors trained in the west
consider acupuncture unworthy of experimental study because
if it had therapeutic effects, such effects could not be
explained by their theories of the nervous system. Who is
being more scientific, the western or eastern practitioners?
\end{dq}

\pagebreak[4]

\begin{dq}
Goethe, a German poet, is less well known for his
theory of color. He published a book on the subject, in
which he argued that scientific apparatus for measuring and
quantifying color, such as prisms, lenses and colored
filters, could not give us full insight into the ultimate
meaning of color, for instance the cold feeling evoked by
blue and green or the heroic sentiments inspired by red. Was
his work scientific?
\end{dq}

\begin{dq}
A child asks why things fall down, and an adult answers
``because of gravity.'' The ancient Greek philosopher
Aristotle explained that rocks fell because it was their
nature to seek out their natural place, in contact with the
earth. Are these explanations scientific?
\end{dq}

\begin{dq}
Buddhism is partly a psychological explanation of human
suffering, and psychology is of course a science.  The
Buddha could be said to have engaged in a cycle of theory
and experiment, since he worked by trial and error, and even
late in his life he asked his followers to challenge his
ideas.  Buddhism could also be considered reproducible,
since the Buddha told his followers they could find
enlightenment for themselves if they followed a certain
course of study and discipline.  Is Buddhism a scientific pursuit?
\end{dq}

<% end_sec() %>
<% begin_sec("What is physics?",0) %>

\epigraphlong{Given for one instant an intelligence which could comprehend
all the forces by which nature is animated and the
respective positions of the things which compose it...nothing
would be uncertain, and the future as the past would be laid
out before its eyes.}{Pierre Simon de Laplace}\index{Laplace}\label{laplace-quote}

\index{physics}Physics is the use of the scientific method
to find out the basic principles governing light and matter,
and to discover the implications of those laws. Part of what
distinguishes the modern outlook from the ancient mind-set
is the assumption that there are rules by which the universe
functions, and that those laws can be at least partially
understood by humans. From the Age of Reason through the
nineteenth century, many scientists began to be convinced
that the laws of nature not only could be known but, as
claimed by Laplace, those laws could in principle be used to
predict everything about the universe's future if complete
information was available about the present state of all
light and matter. In subsequent sections, I'll describe two
general types of limitations on prediction using the laws of
physics, which were only recognized in the twentieth century.

\index{matter}Matter can be defined as anything that is
affected by gravity, i.e., that has weight or would have
weight if it was near the Earth or another star or planet
massive enough to produce measurable gravity. \index{light}Light
can be defined as anything that can travel from one place to
another through empty space and can influence matter, but
has no weight. For example, sunlight can influence your body
by heating it or by damaging your DNA and giving you skin
cancer. The physicist's definition of light includes a
variety of phenomena that are not visible to the eye,
including \index{radio waves}radio waves, \index{microwaves}microwaves,
\index{x-rays}x-rays, and gamma rays.
These are the ``colors'' of light that do not happen to fall
within the narrow violet-to-red range of the rainbow that we can see.

<% self_check('cathode-rays',<<-'SELF_CHECK'
At the turn of the 20th century, a strange new phenomenon
was discovered in vacuum tubes: mysterious rays of unknown
origin and nature.  These rays are the same as the ones that
shoot from the back of your TV's picture tube and hit the
front to make the picture.  Physicists in 1895 didn't have
the faintest idea what the rays were, so they simply named
them ``\\index{cathode rays}cathode rays,'' after the name
for the electrical contact from which they sprang.  A fierce
debate raged, complete with nationalistic overtones, over
whether the rays were a form of light or of matter.  What
would they have had to do in order to settle the issue?
  SELF_CHECK
  ) %>

Many physical phenomena are not themselves light or matter,
but are properties of light or matter or interactions
between light and matter. For instance, motion is a property
of all light and some matter, but it is not itself light or
matter. The pressure that keeps a bicycle tire blown up is
an interaction between the air and the tire. Pressure is not
a form of matter in and of itself. It is as much a property
of the tire as of the air. Analogously, sisterhood and
employment are relationships among people but are not people themselves.

Some things that appear weightless actually do have weight,
and so qualify as matter. Air has weight, and is thus a form
of matter even though a cubic inch of air weighs less than a
grain of sand. A helium balloon has weight, but is kept from
falling by the force of the surrounding more dense air,
which pushes up on it. Astronauts in orbit around the Earth
have weight, and are falling along a curved arc, but they
are moving so fast that the curved arc of their fall is
broad enough to carry them all the way around the Earth in a
circle. They perceive themselves as being weightless because
their space capsule is falling along with them, and the
floor therefore does not push up on their feet.

<% marg(60) %>
<%
  fig(
    'einsteins-ring',
    %q{%
      This telescope picture shows two
      images of the same distant object, an
      exotic, very luminous object called a
      quasar. This is interpreted as evidence
      that a massive, dark object, possibly
      a black hole, happens to be between
      us and it. Light rays that would
      otherwise have missed the earth on
      either side have been bent by the dark
      object's gravity so that they reach us.
      The actual direction to the quasar is
      presumably in the center of the image,
      but the light along that central line
      doesn't get to us because it is
      absorbed by the dark object. The
      quasar is known by its catalog number,
      MG1131+0456, or more informally as
      Einstein's Ring.
    }
  )
%>
<% end_marg %>
\begin{optionaltopic}{Optional Topic: Modern Changes in the Definition of Light and Matter}
Einstein predicted as a consequence of his theory of
relativity that light would after all be affected by
gravity, although the effect would be extremely weak under
normal conditions. His prediction was borne out by
observations of the bending of light rays from stars as they
passed close to the sun on their way to the Earth. Einstein's theory
also implied the existence of black holes, stars so
massive and compact that their intense gravity would not
even allow light to escape. (These days there is strong
evidence that black holes exist.)

Einstein's interpretation was that light doesn't really have
mass, but that energy is affected by gravity just like mass
is. The energy in a light beam is equivalent to a certain
amount of mass, given by the famous equation $E=mc^2$, where
$c$ is the speed of light. Because the speed of light is
such a big number, a large amount of energy is equivalent to
only a very small amount of mass, so the gravitational force
on a light ray can be ignored for most practical purposes.

There is however a more satisfactory and fundamental
distinction between light and matter, which should be
understandable to you if you have had a chemistry course. In
chemistry, one learns that electrons obey the \index{Pauli
exclusion principle}Pauli exclusion principle, which forbids
more than one electron from occupying the same orbital if
they have the same spin. The Pauli exclusion principle is
obeyed by the subatomic particles of which matter is
composed, but disobeyed by the particles, called photons, of
which a beam of light is made.

Einstein's theory of relativity is discussed more fully in
book 6 of this series.
\end{optionaltopic}

The boundary between physics and the other sciences is not
always clear. For instance, chemists study atoms and
molecules, which are what matter is built from, and there
are some scientists who would be equally willing to call
themselves physical chemists or chemical physicists. It
might seem that the distinction between physics and biology
would be clearer, since physics seems to deal with inanimate
objects. In fact, almost all physicists would agree that the
basic laws of physics that apply to molecules in a test tube
work equally well for the combination of molecules that
constitutes a bacterium. (Some might believe that something
more happens in the minds of humans, or even those of cats
and dogs.) What differentiates physics from biology is that
many of the scientific theories that describe living things,
while ultimately resulting from the fundamental laws of
physics, cannot be rigorously derived from physical principles.

<% marg(120) %>
<%
  fig(
    'reductionism',
    %q{Reductionism.}
  )
%>
<% end_marg %>
<% begin_sec("Isolated systems and reductionism",nil,'reductionism') %>
To avoid having to study everything at once, scientists
isolate the things they are trying to study. For instance, a
physicist who wants to study the motion of a rotating
gyroscope would probably prefer that it be isolated from
vibrations and air currents. Even in biology, where field
work is indispensable for understanding how living things
relate to their entire environment, it is interesting to
note the vital historical role played by \index{Darwin}Darwin's
study of the Gal\'{a}pagos Islands, which were conveniently
isolated from the rest of the world. Any part of the
universe that is considered apart from the rest can be
called a ``system.''

m4_ifelse(__sn,1,[::],[:\enlargethispage{-2\baselineskip}:])

Physics has had some of its greatest successes by carrying
this process of isolation to extremes, subdividing the
universe into smaller and smaller parts. Matter can be
divided into atoms, and the behavior of individual atoms can
be studied. Atoms can be split apart into their constituent
neutrons, protons and electrons. Protons and neutrons appear
to be made out of even smaller particles called quarks, and
there have even been some claims of experimental evidence
that quarks have smaller parts inside them. This method of
splitting things into smaller and smaller parts and studying
how those parts influence each other is called \index{reductionism}reductionism.
The hope is that the seemingly complex rules governing the
larger units can be better understood in terms of simpler
rules governing the smaller units. To appreciate what
reductionism has done for science, it is only necessary to
examine a 19th-century chemistry textbook. At that time, the
existence of atoms was still doubted by some, electrons were
not even suspected to exist, and almost nothing was
understood of what basic rules governed the way atoms
interacted with each other in chemical reactions. Students
had to memorize long lists of chemicals and their reactions,
and there was no way to understand any of it systematically.
Today, the student only needs to remember a small set of
rules about how atoms interact, for instance that atoms of
one element cannot be converted into another via chemical
reactions, or that atoms from the right side of the periodic
table tend to form strong bonds with atoms from the left side.

m4_ifelse(__sn,1,[::],[:\enlargethispage{-2\baselineskip}:])

\startdqs

\begin{dq}
I've suggested replacing the ordinary dictionary
definition of light with a more technical, more precise one
that involves weightlessness. It's still possible, though,
that the stuff a lightbulb makes, ordinarily called
``light,'' does have some small amount of weight. Suggest an
experiment to attempt to measure whether it does.
\end{dq}

\begin{dq}
Heat is weightless (i.e., an object becomes no heavier
when heated), and can travel across an empty room from the
fireplace to your skin, where it influences you by heating
you. Should heat therefore be considered a form of light by
our definition? Why or why not?
\end{dq}

\begin{dq}
Similarly, should sound be considered a form of light?
\end{dq}

<% end_sec('reductionism') %>
<% end_sec() %>
<% begin_sec("How to learn physics",0) %>

\epigraphlong{For as knowledges are now delivered, there is a kind of
contract of error between the deliverer and the receiver;
for he that delivereth knowledge desireth to deliver it in
such a form as may be best believed, and not as may be best
examined; and he that receiveth knowledge desireth rather
present satisfaction than expectant inquiry.}{Francis Bacon}\index{Bacon, Francis}

Many students approach a science course with the idea that
they can succeed by memorizing the formulas, so that when a
problem is assigned on the homework or an exam, they will be
able to plug numbers in to the formula and get a numerical
result on their calculator. Wrong! That's not what learning
science is about! There is a big difference between
memorizing formulas and understanding concepts. To start
with, different formulas may apply in different situations.
One equation might represent a definition, which is always
true. Another might be a very specific equation for the
speed of an object sliding down an inclined plane, which
would not be true if the object was a rock drifting down to
the bottom of the ocean. If you don't work to understand
physics on a conceptual level, you won't know which
formulas can be used when.

Most students taking college science courses for the first
time also have very little experience with interpreting the
meaning of an equation. Consider the equation $w=A/h$
relating the width of a rectangle to its height and area. A
student who has not developed skill at interpretation might
view this as yet another equation to memorize and plug in to
when needed. A slightly more savvy student might realize
that it is simply the familiar formula $A=wh$ in a
different form. When asked whether a rectangle would have a
greater or smaller width than another with the same area but
a smaller height, the unsophisticated student might be at a
loss, not having any numbers to plug in on a calculator. The
more experienced student would know how to reason about an
equation involving division --- if $h$ is smaller, and $A$
stays the same, then $w$ must be bigger. Often, students
fail to recognize a sequence of equations as a derivation
leading to a final result, so they think all the intermediate
steps are equally important formulas that they should memorize.

When learning any subject at all, it is important to become
as actively involved as possible, rather than trying to read
through all the information quickly without thinking about
it. It is a good idea to read and think about the questions
posed at the end of each section of these notes as you
encounter them, so that you know you have understood
what you were reading.

Many students' difficulties in physics boil down mainly to
difficulties with math. Suppose you feel confident that you
have enough mathematical preparation to succeed in this
course, but you are having trouble with a few specific
things. In some areas, the brief review given in this
chapter may be sufficient, but in other areas it probably
will not. Once you identify the areas of math in which you
are having problems, get help in those areas. Don't limp
along through the whole course with a vague feeling of dread
about something like scientific notation. The problem will
not go away if you ignore it. The same applies to essential
mathematical skills that you are learning in this course for
the first time, such as vector addition.

Sometimes students tell me they keep trying to understand a
certain topic in the book, and it just doesn't make sense.
The worst thing you can possibly do in that situation is to
keep on staring at the same page. Every textbook explains
certain things badly --- even mine! --- so the best thing to
do in this situation is to look at a different book. Instead
of college textbooks aimed at the same mathematical level as
the course you're taking, you may in some cases find that
high school books or books at a lower math level give
clearer explanations.

 Finally, when reviewing for an exam, don't simply read back
over the text and your lecture notes. Instead, try to use an
active method of reviewing, for instance by discussing some
of the discussion questions with another student, or doing
homework problems you hadn't done the first time.

<% end_sec() %>
m4_ifelse(__sn,1,[:
<% begin_sec("Velocity and acceleration") %>
Calculus was invented by a physicist, Isaac Newton, because he needed it as
a tool for calculating velocity and acceleration; in your introductory calculus
course, velocity and acceleration were probably presented as some of the first
applications.

If an object's position as a function of time is given by the function $x(t)$,
then its velocity and acceleration are given by the first and second derivatives
with respect to time,
\begin{align*}
  v &= \frac{\der x}{\der t} \\
\intertext{and}
  a &= \frac{\der^2 x}{\der t^2}\eqquad.
\end{align*}
The notation relates in a logical way to the units of the quantities. Velocity
has units of m/s, and that makes sense because $\der x$ is interpreted as an
infinitesimally small distance, with units of meters, and $\der t$ as an infinitesimally
small time, with units of seconds. The seemingly weird and inconsistent placement of the superscripted twos
in the notation for the acceleration is likewise meant to suggest the units: something on
top with units of meters, and something on the bottom with units of seconds squared.

Velocity and acceleration have completely different physical interpretations.
Velocity is a matter of opinion. Right now as you sit in a chair and read this book,
you could say that your velocity was zero, but an observer watching the Earth rotate
would say that you had a velocity of hundreds of miles an hour. Acceleration represents
a \emph{change} in velocity, and it's not a matter of opinion. Accelerations produce
physical effects, and don't occur unless there's a force to cause them. For example,
gravitational forces on Earth cause falling objects to have an acceleration of $9.8\ \munit/\sunit^2$.

\vspace{0mm plus 15mm}

\begin{eg}{Constant acceleration}\label{eg:diving-board}
\egquestion
How high does a diving board have to be above the water if the diver is to have as much as 1.0 s
in the air?

\eganswer
The diver starts at rest, and has an acceleration of $9.8\ \munit/\sunit^2$. 
We need to find a connection between the distance she travels and time it takes. In other words,
we're looking for information about the function $x(t)$, given information about the acceleration.
To go from acceleration to position, we need to integrate twice:
\begin{align*}
  x &= \int \int a \der t \der t \\
    &= \int \left(at+v_\zu{o}\right) \der t \qquad \text{[$v_\zu{o}$ is a constant of integration.]} \\
    &= \int at \der t \qquad \text{[$v_\zu{o}$ is zero because she's dropping from rest.]} \\
    &= \frac{1}{2}at^2+x_\zu{o} \qquad \text{[$x_\zu{o}$ is a constant of integration.]} \\
    &= \frac{1}{2}at^2 \qquad \text{[$x_\zu{o}$ can be zero if we define it that way.]}
\end{align*}
Note some of the good problem-solving habits demonstrated here. We solve the problem symbolically, and only
plug in numbers at the very end, once all the algebra and calculus are done. One should also make
a habit, after finding a symbolic result, of checking whether the dependence on the variables
make sense. A greater value of $t$ in this expression would lead to a greater value for $x$; that makes
sense, because if you want more time in the air, you're going to have to jump from higher up. A greater
acceleration also leads to a greater height; this also makes sense, because the stronger gravity is,
the more height you'll need in order to stay in the air for a given amount of time.
Now we plug in numbers.
\begin{align*}
  x  &= \frac{1}{2}\left(9.8\ \munit/\sunit^2\right)(1.0\ \sunit)^2  \\
    &= 4.9\ \munit
\end{align*}
Note that when we put in the numbers,
we check that the units work out correctly, $\left(\munit/\sunit^2\right)(\sunit)^2=\munit$. We should
also check that the result makes sense: 4.9 meters is pretty high, but not unreasonable.
\end{eg}

The notation $\der q$ in calculus represents an infinitesimally small change in the variable $q$.
The corresponding notation for a finite change in a variable is $\Delta q$. For example,
if $q$ represents the value of a certain stock on the stock market, and the value falls from
$q_\zu{o}=5$ dollars initially to $q_f=3$ dollars finally, then $\Delta q=-2$ dollars. When we
study linear functions, whose slopes are constant, the derivative is synonymous with the slope
of the line, and $\der y/\der x$ is the same thing as $\Delta y/\Delta x$,
the rise over the run.

Under conditions of constant acceleration,
we can relate velocity and time,
\begin{equation*}
  a = \frac{\Delta v}{\Delta t}\eqquad,
\end{equation*}
or, as in the example \ref{eg:diving-board}, position and time,
\begin{equation*}
  x =  \frac{1}{2}at^2 + v_\zu{o} t + x_\zu{o}\eqquad.
\end{equation*}
It can also be handy to have a relation involving velocity and position, eliminating time.
Straightforward algebra gives
\begin{equation*}
  v_f^2 = v_\zu{o}^2 + 2 a \Delta x\eqquad,
\end{equation*}
where $v_f$ is the final velocity, $v_\zu{o}$ the initial velocity, and $\Delta x$ the
distance traveled.

\vspace{0mm plus 5mm}

\worked{mars-drop-time}{Dropping a rock on Mars}

\vspace{0mm plus 5mm}

\worked{dodge-viper}{The Dodge Viper}

\vspace{0mm plus 5mm}

<% end_sec %>
:],[::])% end if SN
<% begin_sec("Self-evaluation",0) %>

The introductory part of a book like this is hard to write,
because every student arrives at this starting point with a
different preparation. One student may have grown up
outside the U.S. and so may be completely comfortable with
the metric system, but may have had an algebra course in
which the instructor passed too quickly over scientific
notation. Another student may have already taken m4_ifelse(__sn,1,[:vector:],[::]) calculus,
but may have never learned the metric system. The following
self-evaluation is a checklist to help you figure out what
you need to study to be prepared for the rest of the course.

\noindent\begin{tabular}{|p{60mm}|p{60mm}|}
\hline
\textbf{If you disagree with this statement\ldots} & \textbf{you should study this section:}\\
\hline
I am familiar with the basic metric units of meters, kilograms, and seconds, and the most
common metric prefixes: milli- (m), kilo- (k), and centi- (c). & __subsection_or_section(metric-basics) Basic of the Metric System \\
\hline
m4_ifelse(__sn,1,,[:I know about the newton, a unit of force & __subsection_or_section(newton-unit) The newton, the Metric Unit of Force\\
\hline
:])I am familiar with these less common metric prefixes: mega- (M), micro- ($\mu$), and nano- (n). &
__subsection_or_section(exotic-prefixes) Less Common Metric Prefixes \\
\hline
I am comfortable with scientific notation. & __subsection_or_section(scientific-notation) Scientific Notation \\
\hline
I can confidently do metric conversions. & __subsection_or_section(conversions) Conversions \\
\hline
I understand the purpose and use of significant figures. & __subsection_or_section(sig-figs) Significant Figures \\
\hline
\end{tabular}

\noindent It wouldn't hurt you to skim the sections you think you
already know about, and to do the self-checks in those sections.

<% end_sec() %>
<% begin_sec("Basics of the metric system",m4_ifelse(__sn,1,[:4:],[:0:]),'metric-basics') %>\index{metric system}

<% begin_sec("The metric system") %>\index{metric system}

Every country in the world besides the U.S.~uses a
system of units known in English as the ``metric system.\footnote{Liberia and Myanmar have not legally adopted metric units, but use them
in everyday life.}''
This system is entirely decimal, thanks to the same
eminently logical people who brought about the \index{French
Revolution}French Revolution. In deference to France, the
system's official name is the Syst\`{e}me International, or SI,
meaning International System.
The system uses a
single, consistent set of Greek and Latin \index{metric system!prefixes}\index{mega-
(metric prefix)}\index{kilo- (metric prefix)}\index{centi-
(metric prefix)}\index{milli- (metric prefix)}\index{micro-
(metric prefix)}\index{nano- (metric prefix)}prefixes
that modify the basic units. Each
prefix stands for a power of ten, and has an abbreviation
that can be combined with the symbol for the unit. For
instance, the meter is a unit of distance. The prefix kilo-
stands for $10^3$, so a kilometer, 1 km, is a thousand meters.

The basic units of the SI are the meter for
distance, the second for time, and the kilogram (not the gram) for mass.

The following are the most common metric prefixes. You
should memorize them.

\begin{tabular}{llllp{60mm}}
    \multicolumn{2}{c}{prefix} & meaning & \multicolumn{2}{c}{example} \\
    kilo-  &   k   & $10^3$     &   60 kg     & =  a person's mass  \\
    centi- &   c   & $10^{-2}$  &   28 cm     & =  height of a piece of paper  \\
    milli- &   m   & $10^{-3}$  &   1 ms     & =  time for one vibration of a guitar string playing the note D
\end{tabular}


The prefix centi-, meaning $10^{-2}$, is only used in the
centimeter; a hundredth of a gram would not be written as 1
cg but as 10 mg. The centi- prefix can be easily remembered
because a cent is $10^{-2}$  dollars. The official SI
abbreviation for seconds is ``s'' (not ``sec'') and grams
are ``g'' (not ``gm'').

<% end_sec() %>
<% begin_sec("The second") %>

When I stated briefly above that the second was a unit of
time, it may not have occurred to you that this was not
much of a definition. We can make a dictionary-style
definition of a term like ``time,'' or give a general description
like Isaac Newton's: ``Absolute, true, and mathematical time, of itself, and from
its own nature, flows equably without relation to anything external\ldots''
Newton's characterization sounds impressive, but 
physicists today would consider it useless as a definition
of time. Today, the physical sciences are based on
\index{operational definitions}\label{operational-definitions}operational definitions,
which means definitions that spell out the actual steps
(operations) required to measure something numerically.

In an era when our toasters, pens, and coffee pots tell
us the time, it is far from obvious to most people what is
the fundamental operational definition of time. Until
recently, the hour, minute, and second were defined
operationally in terms of the time required for the earth to
rotate about its axis. Unfortunately, the Earth's rotation
is slowing down slightly, and by 1967 this was becoming an
issue in scientific experiments requiring precise time
measurements. The \index{second (unit)}second was therefore
redefined as the time required for a certain number of
vibrations of the light waves emitted by a cesium atoms in a
lamp constructed like a familiar neon sign but with the neon
replaced by cesium. The new definition not only promises to
stay constant indefinitely, but for scientists is a more
convenient way of calibrating a clock than having to carry
out astronomical measurements.

<% self_check('operational-definition',<<-'SELF_CHECK'
What is a possible operational definition of how strong a person is?
  SELF_CHECK
  ) %>

<% end_sec() %>
<% begin_sec("The meter") %>

The French originally defined the meter as $10^{-7}$  times
the distance from the equator to the north pole, as measured
through Paris (of course). Even if the definition was
operational, the operation of traveling to the north pole
and laying a surveying chain behind you was not one that
most working scientists wanted to carry out. Fairly soon, a
standard was created in the form of a metal bar with two
scratches on it. This was replaced by an atomic standard in
1960, and finally in 1983 by the
current definition, which is that the speed of light has a defined value
in units of m/s.
<% marg(m4_ifelse(__sn,1,[:0:],[:47:])) %>
<%
  fig(
    'france',
    %q{The original definition of the meter.}
  )
%>
\spacebetweenfigs
<%
  fig(
    'standard-kilogram',
    %q{A duplicate of the Paris kilogram, maintained at the Danish National Metrology Institute.
       As of 2019, the kilogram is no longer defined in terms of a physical standard.}
  )
%>
<% end_marg %>

<% end_sec() %>
<% begin_sec("The kilogram") %>

The third base unit of the SI is the \index{kilogram}kilogram,
a unit of mass.  Mass is intended to be a measure of the
amount of a substance, but that is not an operational
definition. Bathroom scales work by measuring our planet's
gravitational attraction for the object being weighed, but
using that type of scale to define mass operationally would
be undesirable because gravity varies in strength from place
to place on the earth. The kilogram was for a long time defined
by a physical artifact (figure \figref{standard-kilogram}),
but in 2019 it was redefined by giving a defined value
to Planck's 
m4_ifelse(__me,1,constant,constant (p.~\pageref{planck-constant})), which plays a fundamental
role in the description of the atomic world.

<% end_sec() %>
m4_ifelse(__sn,1,,[:\vfill:])
<% begin_sec("Combinations of metric units") %>

Just about anything you want to measure can be measured with
some combination of meters, kilograms, and seconds.  Speed
can be measured in m/s, volume in $\munit^3$, and density in
$\kgunit/\munit^3$. Part of what makes the SI great is this basic
simplicity. No more funny units like a cord of wood, a bolt
of cloth, or a jigger of whiskey. No more liquid and dry
measure. Just a simple, consistent set of units. The SI\index{SI units}
measures put together from meters, kilograms, and seconds
make up the mks system.\index{mks units} For example, the mks unit of
speed is m/s, not km/hr.

<% end_sec() %>
m4_ifelse(__sn,1,,[:\vfill:])
<% begin_sec("Checking units") %>

A useful technique for finding mistakes in one's algebra is to analyze
the units associated with the variables.

\begin{eg}{Checking units}\label{eg:checking-units}
\egquestion
Jae starts from the formula $V=\frac{1}{3}Ah$ for the volume of a
cone, where $A$ is the area of its base, and $h$ is its height.
He wants to find an equation that will tell him how tall a conical
tent has to be in order to have a certain volume, given its radius.
His algebra goes like this:
\begin{align}
  V &= \frac{1}{3}Ah \\
  A &= \pi r^2 \\
  V &= \frac{1}{3}\pi r^2 h\\
  h &= \frac{\pi r^2}{3V}
\end{align}
Is his algebra correct? If not, find the mistake.

\eganswer
Line 4 is supposed to be an equation for the height, so the units of the
expression on the right-hand side had better equal meters.
The pi and the 3 are unitless, so we can ignore them.
In terms of units, line 4 becomes
\begin{equation*}
  \munit = \frac{\munit^2}{\munit^3} = \frac{1}{\munit}\eqquad.
\end{equation*}
This is false, so there must be a mistake in the algebra. The units
of lines 1, 2, and 3 check out, so the mistake must be in the step
from line 3 to line 4. In fact the result should have been
\begin{equation*}
  h = \frac{3V}{\pi r^2}\eqquad.
\end{equation*}
Now the units check: $\munit = \munit^3/\munit^2$.
\end{eg}

\startdq

\begin{dq}
\index{Newton, Isaac!definition of time}Isaac Newton wrote,
``\ldots the natural days are truly unequal, though they are
commonly considered as equal, and used for a measure of
time\ldots It may be that there is no such thing as an equable
motion, whereby time may be accurately measured. All motions
may be accelerated or retarded\ldots'' Newton was right. Even
the modern definition of the second in terms of light
emitted by cesium atoms is subject to variation. For
instance, magnetic fields could cause the cesium atoms to
emit light with a slightly different rate of vibration. What
makes us think, though, that a pendulum clock is more
accurate than a sundial, or that a cesium atom is a more
accurate timekeeper than a pendulum clock? That is, how can
one test experimentally how the accuracies of different
time standards compare?
\end{dq}

<% end_sec() %>
<% end_sec() %>
m4_ifelse(__sn,1,,[:<% begin_sec("The Newton, the metric unit of force",4,'newton-unit') %>

A force is a push or a pull, or more generally anything that
can change an object's speed or direction of motion. A force
is required to start a car moving, to slow down a baseball
player sliding in to home base, or to make an airplane turn.
(Forces may fail to change an object's motion if they are
canceled by other forces, e.g., the force of gravity pulling
you down right now is being canceled by the force of the
chair pushing up on you.) The metric unit of force is the
Newton, defined as the force which, if applied for one
second, will cause a 1-kilogram object starting from rest to
reach a speed of 1 m/s. Later chapters will discuss the
force concept in more detail. In fact, this entire book is
about the relationship between force and motion.


In __subsection_or_section(metric-basics), I gave a gravitational definition
of mass, but by defining a numerical scale of force, we can
also turn around and define a scale of mass without
reference to gravity. For instance, if a force of two
Newtons is required to accelerate a certain object from rest
to 1 m/s in 1 s, then that object must have a mass of 2
kg. From this point of view, mass characterizes an object's
resistance to a change in its motion, which we call inertia
or inertial mass. Although there is no fundamental reason
why an object's resistance to a change in its motion must be
related to how strongly gravity affects it, careful and
precise experiments have shown that the inertial definition
and the gravitational definition of mass are highly
consistent for a variety of objects. It therefore doesn't
really matter for any practical purpose which definition one adopts.

\startdq

\begin{dq}
Spending a long time in weightlessness is unhealthy. One of
the most important negative effects experienced by
astronauts is a loss of muscle and bone mass. Since an
ordinary scale won't work for an astronaut in orbit, what is
a possible way of monitoring this change in mass? (Measuring
the astronaut's waist or biceps with a measuring tape is not
good enough, because it doesn't tell anything about bone
mass, or about the replacement of muscle with fat.)
\end{dq}

<% end_sec() %>:])<% begin_sec("Less common metric prefixes",m4_ifelse(__sn,1,[:4:],[:0:]),'exotic-prefixes') %>

<% marg(m4_ifelse(__sn,1,[:0:],[:10:])) %>
<%
  fig(
    'metric-mnemonic',
    %q{%
      This is a mnemonic to help you remember the most important metric
      prefixes. The word ``little'' is to remind you that the list starts with the prefixes used for
      small quantities and builds upward. The exponent changes by 3, except that of course that we
      do not need a special prefix for $10^0$, which equals one.
    }
  )
%>
<% end_marg %>
The following are three metric prefixes which, while less
common than the ones discussed previously, are well worth memorizing.

\noindent\begin{tabular}{llllp{48mm}}
    \multicolumn{2}{c}{prefix} & meaning & \multicolumn{2}{c}{example} \\
    mega-   &  M      &$10^6$     & 6.4 Mm             &=  radius of the earth  \\
    micro-  &  $\mu$  &$10^{-6}$  & 10 $\mu\munit$     &=  size of a white blood cell  \\
    nano-   &   n     &$10^{-9}$  & 0.154 nm           &=  distance between carbon nuclei in an ethane molecule
\end{tabular}

Note that the abbreviation for micro is the Greek letter mu,
$\mu$ --- a common mistake is to confuse it with m
(milli) or M (mega).

There are other prefixes even less common, used for
extremely large and small quantities.  For instance,
$1\ \text{femtometer}=10^{-15}\ \munit$ is a convenient unit of distance in
nuclear physics, and $1\ \text{gigabyte}=10^9$  bytes is used for
computers' hard disks.  The international committee that
makes decisions about the SI has recently even added some
new prefixes that sound like jokes, e.g., 
$1\ \text{yoctogram}=10^{-24}\ \gunit$ is about half the mass of a proton.  In the
immediate future, however, you're unlikely to see prefixes
like ``yocto-'' and ``zepto-'' used except perhaps in trivia
contests at science-fiction conventions or other geekfests.

<% self_check('slow-time',<<-'SELF_CHECK'
Suppose you could slow down time so that according to your
perception, a beam of light would move across a room at the
speed of a slow walk.  If you perceived a nanosecond as if
it was a second, how would you perceive a microsecond?
  SELF_CHECK
  ) %>

<% end_sec() %>
<% begin_sec("Scientific notation",0,'scientific-notation') %>

Most of the interesting phenomena in our universe
are not on the human scale. It would take about 1,000,000,000,000,000,000,000
bacteria to equal the mass of a human body. When the
physicist Thomas Young discovered that light was a wave, it
was back in the bad old days before scientific notation, and
he was obliged to write that the time required for one
vibration of the wave was 1/500 of a millionth of a
millionth of a second. Scientific notation is a less awkward
way to write very large and very small numbers such as
these.  Here's a quick review.

Scientific notation means writing a number in terms of a
product of something from 1 to 10 and something else that is
a power of ten. For instance,
\begin{align*}
& 32 = 3.2 \times 10^1\\
& 320 =  3.2 \times 10^2\\
& 3200 = 3.2 \times 10^3  \quad\ldots
\end{align*}
Each number is ten times bigger than the previous one.

Since $10^1$  is ten times smaller than $10^2$ , it makes
sense to use the notation $10^0$  to stand for one, the
number that is in turn ten times smaller than $10^1$ .
Continuing on, we can write $10^{-1}$  to stand for 0.1, the
number ten times smaller than $10^0$ . Negative exponents
are used for small numbers:

\begin{align*}
&3.2 =  3.2 \times 10^0\\
&0.32 = 3.2 \times 10^{-1}\\
&0.032 = 3.2 \times 10^{-2} \quad\ldots
\end{align*}

A common source of confusion is the notation used on the
displays of many calculators. Examples:

\vspace{8mm}

\hspace{10mm}\begin{tabular}{ll}
$3.2 \times 10^6$  &     (written notation)\\
3.2E+6             &     (notation on some calculators)\\
$3.2^6$            &     (notation on some other calculators)
\end{tabular}

\vspace{8mm}

\noindent The last example is particularly unfortunate, because
$3.2^6$ really stands for the number 
$3.2 \times 3.2 \times 3.2 \times 3.2 \times 3.2 \times 3.2= 1074$, a totally different number from $3.2 \times 10^6=3200000$.
The calculator notation should never be used in writing.
It's just a way for the manufacturer to save money by
making a simpler display.

<% self_check('bacteria-queue',<<-'SELF_CHECK'
A student learns that $10^4$  bacteria, standing in line to
register for classes at Paramecium Community College, would
form a queue of this size:

\\includegraphics{../share/intro/figs/sc-bacteria-queue-1}

\\noindent The student concludes that $10^2$  bacteria would form a
line of this length:

\\includegraphics{../share/intro/figs/sc-bacteria-queue-2}

\\noindent Why is the student incorrect?
  SELF_CHECK
  ) %>

<% end_sec() %>
<% begin_sec("Conversions",0,'conversions') %>\index{conversions of units}\index{units, conversion of}

Conversions are one of the three essential mathematical skills, summarized on pp.\pageref{begin-skills}-\pageref{end-skills},
that you need for success in this course.

I suggest you avoid memorizing lots of conversion factors
between SI units and U.S. units, but two that do come in handy are:

\begin{indentedblock}
\noindent{}1 inch =  2.54 cm

\noindent An object with a weight on Earth of 2.2 pounds-force has a mass of 1 kg.
\end{indentedblock}

\noindent The first one is the present definition of the inch, so it's
exact. The second one is not exact, but is good enough for
most purposes. (U.S. units of force and mass are confusing,
so it's a good thing they're not used in science. In U.S.
units, the unit of force is the pound-force, and the best
unit to use for mass is the slug, which is about 14.6 kg.)

More important than memorizing conversion factors is
understanding the right method for doing conversions. Even
within the SI, you may need to convert, say, from grams to
kilograms. Different people have different ways of thinking
about conversions, but the method I'll describe here is
systematic and easy to understand. The idea is that if 1 kg
and 1000 g represent the same mass, then we can consider a fraction like
\begin{equation*}
  \frac{10^3\ \gunit}{1\ \kgunit}
\end{equation*}
to be a way of expressing the number one. This may bother
you. For instance, if you type 1000/1 into your calculator,
you will get 1000, not one. Again, different people have
different ways of thinking about it, but the justification
is that it helps us to do conversions, and it works! Now if
we want to convert 0.7 kg to units of grams, we can multiply
kg by the number one:
\begin{equation*}
  0.7\ \kgunit \times \frac{10^3\ \gunit}{1\ \kgunit}
\end{equation*}
If you're willing to treat symbols such as ``kg'' as if they
were variables as used in algebra (which they're really
not), you can then cancel the kg on top with the kg on the
bottom, resulting in
\begin{equation*}
  0.7\ \cancel{\kgunit} \times \frac{10^3\ \gunit}{1\ \cancel{\kgunit}}  = 700\ \gunit\eqquad.
\end{equation*}
To convert grams to kilograms, you would simply flip the
fraction upside down.

One advantage of this method is that it can easily be
applied to a series of conversions. For instance, to convert
one year to units of seconds,

\begin{multline*}
1\ \cancel{\text{year}} \times
\frac{365\ \cancel{\text{days}}}{1\ \cancel{\text{year}}} \times
\frac{24\ \cancel{\text{hours}}}{1\ \cancel{\text{day}}} \times
\frac{60\ \cancel{\text{min}}}{1\ \cancel{\text{hour}}} \times
\frac{60\ \sunit}{1\ \cancel{\text{min}}} = \\
= 3.15 \times 10^7\ \sunit\eqquad.
\end{multline*}

<% begin_sec("Should that exponent be positive, or negative?") %>

A common mistake is to write the conversion fraction
incorrectly. For instance the fraction
\begin{equation*}
  \frac{10^3\ \kgunit}{1\ \gunit} \qquad \text{(incorrect)}
\end{equation*}
does not equal one, because $10^3$  kg is the mass of a car,
and 1 g is the mass of a raisin. One correct way of
setting up the conversion factor would be
\begin{equation*}
  \frac{10^{-3}\ \kgunit}{1\ \gunit} \qquad \text{(correct)}\eqquad.
\end{equation*}
You can usually detect such a mistake if you take the time
to check your answer and see if it is reasonable.

If common sense doesn't rule out either a positive or a
negative exponent, here's another way to make sure you get
it right. There are big prefixes and small prefixes:

\begin{tabular}{ll}
    big prefixes:    & k \quad    M \\
    small prefixes:  &  m  \quad  $\mu$  \quad   n \\
\end{tabular}

\noindent (It's not hard to keep straight which are which, since
``mega'' and ``micro'' are evocative, and it's easy to
remember that a kilometer is bigger than a meter and a
millimeter is smaller.) In the example above, we want the
top of the fraction to be the same as the bottom. Since $k$
is a big prefix, we need to \emph{compensate} by putting a
small number like $10^{-3}$  in front of it, not a big
number like $10^3$.

\worked{mg-to-kg}{a simple conversion}

\worked{geometric-mean}{the geometric mean}

\startdq

\begin{dq}
Each of the following conversions contains an error.  In
each case, explain what the error is.

(a) $1000\ \kgunit \times \frac{1\ \kgunit}{1000\ \gunit}  = 1\ \gunit$

(b) $50\ \munit \times \frac{1\ \zu{cm}}{100\ \munit}    =  0.5\ \zu{cm}$

(c) ``Nano'' is $10^{-9}$, so there are $10^{-9}$  nm in a meter.

(d) ``Micro'' is $10^{-6}$, so 1 kg is $10^6\ \mu\zu{g}$.

\end{dq}

<% end_sec() %>
<% end_sec() %>
<% begin_sec("Significant figures",0,'sig-figs') %>\index{significant figures}

The international governing body for football (``soccer'' in the US) says
the ball should have a circumference of 68 to 70 cm. Taking the middle of
this range and dividing by $\pi$ gives a diameter of approximately
$21.96338214668155633610595934540698196\ \zu{cm}$.
The digits after the first few are completely meaningless. Since the
circumference could have varied by about a centimeter in either direction,
the diameter is fuzzy by something like a third of a centimeter. We say that
the additional, random digits are not \emph{significant figures}.\index{significant figures}
If you write down a number with a lot of gratuitous insignificant figures,
it shows a lack of scientific literacy and imples to other people a greater
precision than you really have.

As a rule of thumb, the result of a calculation has as many significant
figures, or ``sig figs,'' as the least accurate piece of data that went
in. In the example with the soccer ball, it didn't do us any good to know
$\pi$ to dozens of digits, because the bottleneck in the precision of the
result was the figure for the circumference, which was two sig figs.
The result is $22\ \zu{cm}$.
The rule of thumb works best for multiplication and division.

\enlargethispage{-2\baselineskip}

For calculations involving multiplication and division,
a given fractional or ``percent'' error in one of the inputs
causes the same fractional error in the output. The number of
digits in a number provides a rough measure of its possible
fractional error. These are called significant figures
or ``sig figs.'' Examples:

\begin{tabular}{|l|p{80mm}|}
\hline
3.14 & 3 sig figs \\
\hline
3.1  & 2 sig figs \\
\hline
0.03 & 1 sig fig, because the zeroes are just placeholders \\
\hline
$3.0\times10^1$ & 2 sig figs \\
\hline
30   & could be 1 or 2 sig figs, since we can't tell if the 0 is a placeholder or a real sig fig \\
\hline
\end{tabular}

\noindent In such calculations, your result should not
have more than the number of sig figs in the least accurate
piece of data you started with.

\enlargethispage{-2\baselineskip}

\begin{eg}{Sig figs in the area of a triangle}
\egquestion A triangle has an area of $6.45\ \munit^2$ and a base
with a width of $4.0138\ \munit$. Find its height.

\eganswer The area is related to the base and height by
$A=bh/2$.
\begin{align*}
  h &= \frac{2A}{b} \\
    &= 3.21391200358762\ \munit \quad \text{(calculator output)} \\
    &= 3.21\ \munit
\end{align*}
The given data were 3 sig figs and 5 sig figs. We're limited by the
less accurate piece of data, so the final result is 3 sig figs.
The additional digits on the calculator don't mean anything, and
if we communicated them to another person, we would create the false
impression of having determined $h$ with more precision than we really obtained.
\end{eg}

<% self_check('nigeria',<<-'SELF_CHECK'
The following quote is taken from an editorial by
Norimitsu Onishi in the New York Times, August 18, 2002.

\\noindent \\begin{indentedblock}
\\noindent Consider Nigeria. Everyone agrees it is Africa's most
populous nation. But what is its population? The United
Nations says 114 million; the State Department, 120 million.
The World Bank says 126.9 million, while the Central
Intelligence Agency puts it at 126,635,626.
\\end{indentedblock}

\noindent What should bother you about this?
  SELF_CHECK
  ) %>

Dealing correctly with significant figures can save you
time! Often, students copy down numbers from their
calculators with eight significant figures of precision,
then type them back in for a later calculation. That's a
waste of time, unless your original data had that kind of
incredible precision.

<% self_check('count-sig-figs',<<-'SELF_CHECK'
How many significant figures are there in each of the
following measurements?

(1) 9.937 m

(2) 4.0 s

(3) 0.0000000000000037 kg
  SELF_CHECK
  ) %>

The rules about significant figures are only rules of thumb,
and are not a substitute for careful thinking. For instance,
\$20.00 + \$0.05 is \$20.05. It need not and should not be
rounded off to \$20. In general, the sig fig rules work best
for multiplication and division, and we sometimes also apply them when
doing a complicated calculation that involves many types of
operations. For simple addition and subtraction, it makes
more sense to maintain a fixed number of digits after the decimal point.

When in doubt, don't use the sig fig rules at all. Instead,
intentionally change one piece of your initial data by the
maximum amount by which you think it could have been off,
and recalculate the final result. The digits on the end that
are completely reshuffled are the ones that are meaningless,
and should be omitted.

\begin{eg}{A nonlinear function}
\egquestion How many sig figs are there in $\sin 88.7\degunit$?

\eganswer
We're using a sine function, which isn't addition, subtraction, multiplication, or division.
It would be reasonable to guess that since the input angle had 3 sig figs,
so would the output. But if this was an important calculation and we really needed to know,
we would do the following:
\begin{align*}
  \sin 88.7\degunit &= 0.999742609322698 \\
  \sin 88.8\degunit &= 0.999780683474846
\end{align*}
Surprisingly, the result appears to have as many as 5 sig figs, not just 3:
\begin{equation*}
  \sin 88.7\degunit = 0.99974\eqquad,
\end{equation*}
where the final 4 is uncertain but may have some significance.
The unexpectedly high precision of the result is because the sine function is nearing its
maximum at 90 degrees, where the graph flattens out and becomes insensitive to the input angle.
\end{eg}

<% end_sec() %>

<% begin_sec("A note about diagrams",0,'diagrams') %>

<% marg(0) %>
<%
  fig(
    'tomato',
    %q{A diagram of a tomato.}
  )
%>
<% end_marg %>

A quick note about diagrams. Often when you solve a problem, the best way to get started
and organize your thoughts is by drawing a diagram. For an artist, it's desirable to be
able to draw a recognizable, realistic, perspective picture of a tomato, like the one at the top of
figure \figref{tomato}. But in science and engineering, we usually don't draw solid figures in perspective,
because that would make it difficult to label distances and angles. Usually we want views or cross-sections that project
the object into its planes of symmetry, as in the line drawings in the figure.
<% end_sec() %>
