A hydrogen atom consists of an electron and a proton. For our
        present purposes, we'll think of the electron as orbiting in
        a circle around the proton.
        
        The subatomic particles called muons behave exactly like
        electrons, except that a muon's mass is greater by a factor
        of 206.77.  Muons are continually bombarding the Earth as
        part of the stream of particles from space known as cosmic
        rays.  When a muon strikes an atom, it can displace one of
        its electrons.  If the atom happens to be a hydrogen atom,
        then the muon takes up an orbit that is on the average
        206.77 times closer to the proton than the orbit of the
        ejected electron.  How many times greater is the electric
        force experienced by the muon than that previously
        felt by the electron?
