The figure shows a simplified diagram of a device called a tandem
accelerator, used for accelerating beams of ions up to speeds on the
order of 1-10\% of the speed of light. (Since these velocities are not
too big compared to $c$, you can use nonrelativistic physics
throughout this problem.) The nuclei of these ions collide with the
nuclei of atoms in a target, producing nuclear reactions for
experiments studying the structure of nuclei. The outer shell of the
accelerator is a conductor at zero voltage (i.e., the same voltage as
the Earth). The electrode at the center, known as the ``terminal,'' is
at a high positive voltage, perhaps millions of volts.  Negative ions
with a charge of $-1$ unit (i.e., atoms with one extra electron) are
produced offstage on the right, typically by chemical reactions with
cesium, which is a chemical element that has a strong tendency to give
away electrons. Relatively weak electric and magnetic forces are used
to transport these $-1$ ions into the accelerator, where they are
attracted to the terminal. Although the center of the terminal has a
hole in it to let the ions pass through, there is a very thin carbon
foil there that they must physically penetrate. Passing through the
foil strips off some number of electrons, changing the atom into a
positive ion, with a charge of $+n$ times the fundamental charge. Now
that the atom is positive, it is repelled by the terminal, and
accelerates some more on its way out of the accelerator. 

(a) Find the velocity, $v$, of the emerging beam of positive ions, in
terms of $n$, their mass $m$, the terminal voltage $V$, and
fundamental constants. Neglect the small change in mass caused by the
loss of electrons in the stripper foil.\answercheck\hwendpart

(b) To fuse protons with protons, a minimum beam velocity of
about 11\% of the speed of light is required. What terminal
voltage would be needed in this case?\answercheck\hwendpart

(c) In the setup described in part b, we need a target
containing atoms whose nuclei are single protons, i.e., a target
made of hydrogen. Since hydrogen is a gas, and we want a foil
for our target, we have to use a hydrogen compound, such as
a plastic. Discuss what effect this would have on the experiment.
