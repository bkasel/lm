Brass and wind instruments go up in pitch as the musician
warms up. As a typical empirical example, a trumpet's frequency might go
up by about 1\%. Let's consider possible physical reasons for the
change in pitch. (a) Solids generally expand with increasing
temperature, because the stronger random motion of the atoms
tends to bump them apart. Brass expands by $1.88\times10^{-5}$
 per degree C. Would this tend to raise the pitch, or
lower it? Estimate the size of the effect in comparison with
the observed change in frequency. (b) The speed of sound in
a gas is proportional to the square root of the absolute
temperature, where zero absolute temperature is -273 degrees
C. As in part a, analyze the size and direction of the
effect.
