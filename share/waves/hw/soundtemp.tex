In section \ref{sec:waves-on-a-string}, we saw that the speed of waves on a
string depends on the ratio of $T/\mu $, i.e., the speed of
the wave is greater if the string is under more tension, and
less if it has more inertia. This is true in general: the
speed of a mechanical wave always depends on the medium's
inertia in relation to the restoring force (tension,
stiffness, resistance to compression,...). Based on these
ideas, explain why the speed of sound in air is significantly greater on a hot day,
while the speed of sound in
liquids and solids shows almost no variation with temperature.
