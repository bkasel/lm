Refraction occurs only at the boundary between two substances,
which in this case
means the surface of the lens. Light doesn't get bent at all
inside the lens, so the thickness of the lens isn't really
what's important. What matters is the angles of the lens'
surfaces at various points.

Ray 1 makes an angle of zero with respect to the normal as
it enters the lens, so it doesn't get bent at all, and likewise
at the back.

At the edge of the lens, 2, the front and
back are not parallel, so a ray that traverses the lens at
the edge ends up being bent quite a bit.

Although I drew both ray 1 and ray 2 coming in along the axis
of the lens, it really doesn't matter. For instance, ray 3
bends on the way in, but bends an equal amount on the way
out, so it still emerges from the lens moving in the same
direction as the direction it originally had.

Summarizing and systematizing these observations, we can say that
for a ray that enters the lens at the center, where the surfaces
are parallel, the sum of the two deflection angles is zero. Since
the total deflection is zero at the center, it must be larger away
from the center.

\anonymousinlinefig{explainlens}
