
(a)

\begin{tabular}{l}
   top spring's rightward force on connector\\
    ...connector's leftward force on top spring\\
   bottom spring's rightward force on connector\\
    ...connector's leftward force on bottom spring\\
   hand's leftward force on connector\\
    ...connector's rightward force on hand
\end{tabular}

Looking at the three forces on the connector, we see that
the hand's force must be double the force of either spring.
The value of $x-x_\zu{o}$ is the same for both springs and for
the arrangement as a whole, so the spring constant must be
$2k$. This corresponds to a stiffer spring (more force to
produce the same extension).

(b) Forces in which the left spring participates:

\begin{tabular}{l}
   hand's leftward force on left spring\\
    ...left spring's rightward force on hand\\
   right spring's rightward force on left spring\\
    ...left spring's leftward force on right spring
\end{tabular}

Forces in which the right spring participates:

\begin{tabular}{l}
   left spring's leftward force on right spring\\
    ...right spring's rightward force on left spring\\
   wall's rightward force on right spring\\
    ...right spring's leftward force on wall
\end{tabular}

Since the left spring isn't accelerating, the total force on
it must be zero, so the two forces acting on it must be
equal in magnitude. The same applies to the two forces
acting on the right spring. The forces between the two
springs are connected by Newton's third law, so all eight of
these forces must be equal in magnitude. Since the value of
$x-x_\zu{o}$ for the whole setup is double what it is for either
spring individually, the spring constant of the whole setup
must be $k/2$, which corresponds to a less stiff spring.



