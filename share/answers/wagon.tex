(a) Since the wagon has no acceleration, the total
forces in both the $x$ and $y$ directions must be zero. There
are three forces acting on the wagon: $T$, $\zb{F}_g$, and the normal
force from the ground, $\zb{F}_n$. If we pick a coordinate system
with $x$ being horizontal and $y$ vertical, then the angles of
these forces measured counterclockwise from the $x$ axis are
$90\degunit-\phi$, $270\degunit$, and $90\degunit+\theta$, respectively.
We have
\begin{align*}
	F_{x,total} &= T\cos(90\degunit-\phi) + F_{g}\cos(270\degunit) +
				F_ncos(90\degunit+\theta) \\
	F_{y,total} &= T\sin(90\degunit-\phi) + F_{g}\sin(270\degunit) +
				F_nsin(90\degunit+\theta)\eqquad,
\end{align*}
which simplifies to
\begin{align*}
	0 &= T \sin \phi - F_n \sin \theta\\
	0 &= T \cos \phi  - F_{g} + F_n \cos \theta  .
\end{align*}
The normal force is a quantity that we are not given and do
not wish to find, so we should choose it to eliminate.
Solving the first equation for $F_n=(\sin \phi/\sin \theta)T$, we
eliminate $F_n$ from the second equation,
\begin{equation*}
	0 = T \cos \phi  - F_{g} + T \sin \phi \cos \theta/\sin \theta
\end{equation*}
and solve for $T$, finding
\begin{equation*}
	T = \frac{F_g}{\cos\phi+\sin\phi\cos\theta/\sin\theta}
\end{equation*}
Multiplying both the top and the bottom of the fraction by
$\sin \theta$, and using the trig identity for $\sin(\theta+\phi)$ gives the
desired result,
\begin{equation*}
	T = \frac{\sin\theta}{\sin(\theta+\phi)}F_gs
\end{equation*}
(b) The case of $\phi=0$, i.e. pulling straight up on the wagon,
results in $T=F_{g}$: we simply support the wagon and it glides
up the slope like a chair-lift on a ski slope. In the case
of $\phi=180\degunit-\theta$, $T$ becomes infinite. Physically this is
because we are pulling directly into the ground, so no
amount of force will suffice.
