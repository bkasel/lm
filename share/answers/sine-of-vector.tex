We'll use the same approach as in the example
in section \ref{sec:rotational-invariance}, which is to find an example
such that when the calculation is carried out in a rotated frame of reference,
the result is clearly not the same vector expressed in the new frame.
Let $\vc{A}=\pi\vc{\hat{x}}$ in the original coordinate system. Then
in this coordinate system $\vc{B}=0$. 

But now suppose we choose a new coordinate
system, rotated by 10 degrees relative to the first one. In this new coordinate
system, $A_x$ is a little less than $\pi$. Since $A_x$ is no longer a multiple of
$\pi$, $B_x$ is no longer zero, and $\vc{B}$ is no longer zero. The nonzero $\vc{B}$
computed in the new coordinate system is clearly not the same as the old $\vc{B}$
expressed in a new way, since rotating our coordinate system should not change the
magnitudes of vectors.



