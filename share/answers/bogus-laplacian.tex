(a) The quantity $x-y$ vanishes along the line $y=x$ lying in the
first quadrant at a 45-degree angle between the axes. Squaring produces
a trough parallel to this line, with a parabolic cross-section. Geometrically,
the Laplacian can be interpreted as a measure of how much the value of $f$
at a point differs from its average value on a small circle centered on
that point. The trough is concave up, so we can predict that the Laplacian
will be positive everywhere.\\
(b) The zero result is clearly wrong because it disagrees with our conclusion
from part a that the Laplacian is positive. A correct calculation gives
$\partial^2(x-y)^2/\partial x^2+\partial^2(x-y)^2/\partial y^2=4$.\\
(c) If we rotate our coordinate axes counterclockwise by 45 degrees, then we have a parabolic
trough oriented along the $x$ axis. In terms of these new coordinates,
$\partial f/\partial x=0$, while $\partial f/\partial y$ is nonzero almost
everywhere.\\
Remark: The mistake described in the question is a common one, and is apparently
based on the idea that the notation $\nabla^2$ must mean applying an operator
$\nabla$ twice. For those with some exposure to vector calculus, it may be
of interest to note that the Laplacian \emph{is} equivalent to the divergence
of the gradient, which can be notated either $\operatorname{div}(\operatorname{grad} f)$ or
$\nabla\cdot(\nabla f)$. The important thing to recognize is that the gradient,
notated $\operatorname{grad} f$ or $\nabla f$, outputs a \emph{vector}, not a scalar
like the quantity $Q$ defined in this problem.
