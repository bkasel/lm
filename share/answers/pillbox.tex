(a) Let the Gaussian pillbox of example \ref{eg:pillbox}, p.~\pageref{eg:pillbox}, have
area $A$ on each flat face. Gauss's law is
\begin{align*}
        4\pi k q_{in}        &= \sum \vc{E}_j\cdot\vc{A}_j \\
        4\pi k \sigma A        &= \sum |\vc{E}_j||\vc{A}_j|\eqquad.
\end{align*}
By symmetry, we would expect the field to be perpendicular to the plane and
the same on both sides of the plane. This
is a rather subtle point, which we'll come back to at the end. Assuming this to be true,
there is no flux through the sides of the pillbox, only through the flat faces, and
the magnitude of the field is the same everywhere on the faces, so we can take it outside the sum.
\begin{align*}
        &4\pi k \sigma A        = |\vc{E}|\cdot 2 A \\
        &E = 2\pi k \sigma.
\end{align*}

Returning to the symmetry issue, the tricky point is that we could add on to our solution
any other solution that was valid in a vacuum. For instance, we could add a uniform electric
field that would cancel out the sheet's field on one side, while doubling it on the other.

(b) The most efficient way to do this is to make use of a similar equation whose units
we already know to be correct, such as the equation $E=kq/r^2$ for a point charge.
Since $q/r^2$ has the same units as $\sigma$, the units do check out.
