We can think of this as a polynomial in $x$ or a polynomial in $y$ --- 
their roles are symmetric. Let's call $x$ the variable.  By the
fundamental theorem of algebra, it must be possible to factor it into
a product of three linear factors, if the coefficients are allowed to
be complex. Each of these factors causes the product to be zero for a
certain value of $x$. But the condition for the expression to be zero
is $x^3=y^3$, which basically means that the ratio of $x$ to $y$ must
be a third root of 1.  The problem, then, boils down to finding the
three third roots of 1, as in problem \ref{hw:cube-roots-of-unity}.
Using the result of that problem, we find that there are zeroes when
$x/y$ equals $1$, $e^{2\pi i/3}$, and $e^{4\pi i/3}$. This tells us
that the factorization is $(x-y)(x-e^{2\pi i/3}y)(x-e^{4\pi i/3}y)$.

The second part of the problem asks us to factorize as much as possible using real coefficients.
Our only hope of doing this is to multiply out the two factors that involve complex coefficients,
and see if they produce something real. In fact, we can anticipate that it will work, because
the coefficients are complex conjugates of one another, and when a quadratic has two complex
roots, they are conjugates. The result is $(x-y)(x^2+xy+y^2)$.
