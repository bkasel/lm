(a) Roughly speaking, the thermal energy is $\sim k_BT$ (where $k_B$ is the Boltzmann constant),
and we need this to be on the same order of magnitude as $ke^2/r$ (where $k$ is the Coulomb constant).
For this type of rough estimate it's
not especially crucial to get all the factors of two right, but let's do so anyway. Each proton's average
kinetic energy due to motion along a particular axis is $(1/2)k_BT$. If two protons are colliding along a certain
line in the center-of-mass frame, then their average combined kinetic energy due to motion along that axis
is $2(1/2)k_BT=k_BT$. So in fact the factors of 2 cancel. We have $T=ke^2/k_Br$.\\
(b) The units are $\zu{K}=(\junit\unitdot\munit/\zu{C}^2)(\zu{C}^2)/((\junit/\zu{K})\unitdot\munit)$, which does
work out.\\
(c) The numerical result is $\sim 10^{10}\ \text{K}$, which as suggested is much higher than the temperature at the core
of the sun.
