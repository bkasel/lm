(a) Plugging in, we find
\begin{equation*}
  \sqrt{\frac{1-w}{1+w}} =   \sqrt{\frac{1-u}{1+u}}   \sqrt{\frac{1-v}{1+v}}\eqquad.
\end{equation*}
(b) First let's simplify by squaring both sides.
\begin{equation*}
  \frac{1-w}{1+w} =   \frac{1-u}{1+u}  \cdot \frac{1-v}{1+v}\eqquad.
\end{equation*}
For convenience, let's write $A$ for the right-hand side of this equation. We then have
\begin{gather*}
  \frac{1-w}{1+w} = A \\
  1-w = A+Aw\eqquad.
\end{gather*}
Solving for $w$,
\begin{align*}
  w &= \frac{1-A}{1+A} \\
    &= \frac{(1+u)(1+v)-(1-u)(1-v)}{(1+u)(1+v)+(1-u)(1-v)} \\
    &= \frac{2(u+v)}{2(1+uv)} \\
    &= \frac{u+v}{1+uv}
\end{align*}
(c) This is all in units where $c=1$. The correspondence principle says that we should get $w\approx u+v$ when
both $u$ and $v$ are small compared to 1. Under those circumstances, $uv$ is the product of two very small
numbers, which makes it very, very small. Neglecting this term in the denominator, we recover the nonrelativistic result.


