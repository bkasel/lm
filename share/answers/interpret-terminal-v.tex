(1) The case of 
$\rho=0$ represents an object falling in a vacuum, i.e.,
 there is no density of air. The terminal velocity
would be infinite. Physically, we know that an object falling
 in a vacuum would never stop speeding up, since there
would be no force of air friction to cancel the force of gravity. 
(2) The 4-cm ball would have a mass that was greater
by a factor of $4\times4\times4$, but its cross-sectional area would
 be greater by a factor of $4\times4$. Its terminal velocity would be
greater by a factor of $\sqrt{4^3/4^2}=2$.
(3) It isn't of any general importance. It's just an example of one
physical situation. You should not memorize it.



