(a) In problem \#2 we found that the equation relating
the object and image distances was of the form
$1/f=-1/d_i+1/d_o$. Let's make $f=1.00$ m. To get a virtual
image we need $d_o<f$, so let $d_o=0.50$ m. Solving for
$d_i$, we find $d_i=1/(1/d_o-1/f)=1.00$ m. The magnification
is $M=d_i/d_o=2.00$. If we change $d_o$ to 0.55 m, the
magnification becomes 2.22. The magnification changes
somewhat with distance, so the store's ad must be assuming
you'll use the mirror at a certain distance. It can't have a
magnification of 5 at all distances.\\
(b) Theoretically yes, but in practical terms no.
If you go through a calculation similar to the one in part a,
you'll find that
the images of both planets are formed at almost exactly the
same $d_i$, $d_i=f$, since $1/d_o$ is
pretty close to zero for any astronomical object.
The more distant planet has an image half as big
($M=d_i/d_o$, and $d_o$ is doubled), but
we're talking about \emph{angular} magnification here,
so what we care about is the angular size of the image
compared to the angular size of the object. The more
distant planet has half the angular size, but its image
has half the angular size as well, so the angular magnification
is the same. If you think about it, it wouldn't make much sense
for the angular magnification to depend on the planet's distance
--- if it did, then determining astronomical distances would
be much easier than it actually is!
