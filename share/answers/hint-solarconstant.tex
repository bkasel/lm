We're ignoring the fact that the light consists of little
wavepackets, and imagining it as a simple sine wave. But wait, there's
more good news! The energy density depends on the squares of the fields,
which means the squares of some sine waves. Well, when you square a sine
wave that varies from $-1$ to $+1$, you get a sine wave that goes from
0 to $+1$, and the average value of that sine wave is 1/2. That means
you don't have to do an integral like $U=\int (\der U/\der V)\der V$.
All you have to do is throw in the appropriate factor of 1/2, and you
can pretend that the fields have their constant values
$\tilde{\vc{E}}$ and $\tilde{\vc{B}}$ everywhere.
