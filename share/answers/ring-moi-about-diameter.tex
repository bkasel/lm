The moment of inertia is $I=\int r^2\der m$.
Let the ring have total mass $M$ and radius $b$.
The proportionality
\begin{equation*}
  \frac{M}{2\pi} = \frac{\der m}{\der\theta}
\end{equation*}
gives a change of variable that results in
\begin{equation*}
  I = \frac{M}{2\pi}\int_0^{2\pi} r^2\der\theta .
\end{equation*}
If we measure $\theta$ from the axis of rotation, then
$r=b\sin\theta$, so this becomes
\begin{equation*}
  I = \frac{Mb^2}{2\pi}\int_0^{2\pi} \sin^2\theta \der\theta .
\end{equation*}
The integrand averages to 1/2 over the $2\pi$ range of integration,
so the integral equals $\pi$. We therefore have
$I=\frac{1}{2}Mb^2$. This is, as claimed, half the value for rotation
about the symmetry axis.
