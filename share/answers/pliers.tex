The pliers are not moving, so their angular momentum
remains constant at zero, and the total torque on them must
be zero. Not only that, but each half of the pliers must
have zero total torque on it. This tells us that the
magnitude of the torque at one end must be the same as that
at the other end. The distance from the axis to the nut is
about 2.5 cm, and the distance from the axis to the centers
of the palm and fingers are about 8 cm. The angles are close
enough to $90\degunit$ that we can pretend they're 90 degrees,
considering the rough nature of the other assumptions and
measurements. The result is $(300\ \nunit)(2.5\ \zu{cm})=(F)(8\ \zu{cm})$,
 or $F=90$ N.



