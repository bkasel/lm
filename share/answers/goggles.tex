Normally, in air, your eyes do most of their focusing at
the air-eye boundary. When you swim without goggles,
there is almost no difference in speed at the
water-eye interface, so light is not strongly refracted there
(see figure), and the image is far behind the retina.

Goggles fix this problem for the following reason.
The light rays cross a water-air boundary as they enter
the goggles, but they're coming in along the normal, so
they don't get bent. At the air-eye boundary, they get
bent the same amount they normally would when you
weren't swimming.

\anonymousinlinefig{goggles}
