(a) The force of gravity on an object can't just be $g$, both because
$g$ doesn't have units of force and because the force of gravity is
different for different objects.\\
(b) The force of gravity on an object can't just be $m$ either.
This again has the wrong units, and it also can't be right because
it should depend on how strong gravity is in the region of space
where the object is.\\
(c) If the object happened to be free-falling, then the only force
acting on it would be gravity, so by Newton's second law,
$a=F/m$, where $F$ is the force that we're trying to find.
Solving for $F$, we have $F=ma$. But the acceleration of a free-falling
object has magnitude $g$, so the magnitude of the force is $mg$.
The force of gravity on an object doesn't depend on what else is
happening to the object, so the force of gravity must also be equal to $mg$
if the object doesn't happen to be free-falling.
