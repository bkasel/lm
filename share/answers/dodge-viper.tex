(a) Solving for $\Delta x=\frac{1}{2}at^2$ for $a$, 
we find $a=2\Delta x/t^2=5.51\ \munit/\sunit^2$.
(b) $v=\sqrt{2a\Delta x}=66.6$ m/s. (c) The actual car's final
velocity is less than that of the idealized constant-acceleration
car. If the real car and the idealized car covered the
quarter mile in the same time but the real car was moving
more slowly at the end than the idealized one, the real car
must have been going faster than the idealized car at the
beginning of the race. The real car apparently has a greater
acceleration at the beginning, and less acceleration at the
end. This make sense, because every car has some maximum
speed, which is the speed beyond which it cannot accelerate.



