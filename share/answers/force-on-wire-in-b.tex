(a) Current means how much charge passes by a given
point per unit time. During a time interval $\Delta $t, all
the charge carriers in a certain region behind the point
will pass by. This region has length $v\Delta t$ and
cross-sectional area $A$, so its volume is \emph{Av}$\Delta
$t, and the amount of charge in it is \emph{Avnq}$\Delta $t.
To find the current, we divide this amount of charge by
$\Delta $t, giving $I=\emph{Avnq}$. (b) A segment of the
wire of length $L$ has a force \emph{QvB} acting on it,
where $Q=\emph{ALnq}$ is the total charge of the moving
charge carriers in that part of the wire. The force per unit
length is $\emph{ALnqvB}/L=\emph{AnqvB}$. (c) Dividing the
two results gives $F/L=\emph{IB}$.



