For instance, imagine a small sphere around the negative charge,
which we would sketch on the two-dimensional paper as a circle. The field points
inward at every point on the sphere, so all the contributions to the flux are negative.
There is no cancellation, and the total flux is negative, which is consistent with
Gauss' law, since the sphere encloses a negative charge. Copying the same surface onto
the field of the bar magnet, however, we find that there is inward flux on the top
and outward flux on the bottom, where the surface is inside the magnet. According to
Gauss' law for magnetism, these cancel exactly, which is plausible based on the figure.
