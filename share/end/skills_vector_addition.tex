__subsection_or_section(vector-addition), p.~__pageref_subsection_or_section(vector-addition)

Example:
The $\Delta\vc{r}$ vector from San Diego to Los Angeles has magnitude 190 km and direction 129\degunit counterclockwise from east.
The one from LA to Las Vegas is 370 km at 38\degunit counterclockwise from east. Find the distance and direction from
San Diego to Las Vegas.

\eganswer
Graphical addition is discussed on p.~\pageref{subsec:vector-addition-graphical}. Here we concentrate on analytic addition,
which involves adding the $x$ components to find the total $x$ component, and similarly for $y$. The trig needed in order
to find the components of the second leg (LA to Vegas) is laid out in figure \figref{eg-la-vegas} on p.~\pageref{fig:eg-la-vegas} and explained in detail
in example \ref{eg:la-vegas-components} on p.~\pageref{eg:la-vegas-components}:
\begin{align*}
  \Delta x_2 &= (370\ \zu{km})\cos 38\degunit = 292\ \zu{km} \\
  \Delta y_2 &= (370\ \zu{km})\sin 38\degunit = 228\ \zu{km} 
\end{align*}
(Since these are intermediate results, we keep an extra sig fig to avoid accumulating too much rounding error.)
Once we understand the trig for one example, we don't need to reinvent the wheel every time. The pattern is completely universal,
provided that we first make sure to get the angle expressed according to the usual trig convention, counterclockwise from the $x$ axis.
Applying the pattern to the first leg, we have:
\begin{align*}
  \Delta x_1 &= (190\ \zu{km})\cos 129\degunit = -120\ \zu{km} \\
  \Delta y_1 &= (190\ \zu{km})\sin 129\degunit = 148\ \zu{km} 
\end{align*}
For the vector directly from San Diego to Las Vegas, we have
\begin{align*}
  \Delta x &= \Delta x_1 +  \Delta x_2 = 172\ \zu{km}\\
  \Delta y &= \Delta y_1 +  \Delta y_2 = 376\ \zu{km}\eqquad.
\end{align*}
The distance from San Diego to Las Vegas is found using the Pythagorean theorem,
\begin{equation*}
  \sqrt{(172\ \zu{km})^2+(376\ \zu{km})^2} = 410\ \zu{km}
\end{equation*}
(rounded to two sig figs because it's one of our final results). The direction
is one of the two possible values of the inverse tangent
\begin{equation*}
  \tan^{-1} (\Delta y/\Delta x) = \{65\degunit,245\degunit\}\eqquad.
\end{equation*}
Consulting a sketch shows that the first of these values is the correct one.\label{end-skills}
