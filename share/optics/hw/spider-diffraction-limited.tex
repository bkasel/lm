Figure \figref{jumping-spider} on p.~\pageref{fig:jumping-spider} shows the anatomy of a
jumping spider's principal eye. The smallest feature the spider can distinguish is limited
by the size of the receptor cells in its retina. (a) By making measurements on the diagram,
estimate this limiting angular size in units of minutes of arc (60 minutes = 1 degree).\answercheck
(b) Show that this is greater than, but roughly in the same ballpark as, the limit imposed
by diffraction for visible light.\answercheck
\hwremark{Evolution is a scientific theory that makes testable predictions, and if observations
contradict its predictions, the theory can be disproved. It would be maladaptive for the spider
to have retinal receptor cells with sizes much less than the limit imposed by diffraction, since it
would increase complexity without giving any improvement in visual acuity. The results of this
problem confirm that, as predicted by Darwinian evolution, this is not the case. Work by M.F.~Land
in 1969 shows that in this spider's eye, aberration is a somewhat bigger effect than diffraction,
so that the size of the receptors is very nearly at an evolutionary optimum.}
