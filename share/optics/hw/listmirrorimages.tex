As discussed in question \ref{hw:convexraytracing}, there are two types of curved
mirrors, concave and convex. Make a list of all the possible
combinations of types of images (virtual or real) with types
of mirrors (concave and convex). (Not all of the four
combinations are physically possible.) Now for each one, use
ray diagrams to determine whether increasing the distance of
the object from the mirror leads to an increase or a
decrease in the distance of the image from the mirror.

Draw BIG ray diagrams! Each diagram should use up about
half a page of paper.

Some tips: To draw a ray diagram, you need two rays. For one
of these, pick the ray that comes straight along the
mirror's axis, since its reflection is easy to draw. After
you draw the two rays and locate the image for the original
object position, pick a new object position that results in
the same type of image, and start a new ray diagram, in a
different color of pen, right on top of the first one. For
the two new rays, pick the ones that just happen to hit the
mirror at the same two places; this makes it much easier to
get the result right without depending on extreme accuracy
in your ability to draw the reflected rays.
