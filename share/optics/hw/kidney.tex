Ultrasound, i.e., sound waves with frequencies too high
to be audible, can be used for imaging fetuses in the womb
or for breaking up kidney stones so that they can be
eliminated by the body. Consider the latter application.
Lenses can be built to focus sound waves, but because the
wavelength of the sound is not all that small compared to
the diameter of the lens, the sound will not be concentrated
exactly at the geometrical focal point. Instead, a
diffraction pattern will be created with an intense central
spot surrounded by fainter rings. About 85\% of the power
is concentrated within the central spot. The angle of the
first minimum (surrounding the central spot) is given by
$\sin \theta =\lambda/b$, where $b$ is the diameter of the
lens. This is similar to the corresponding equation for a
single slit, but with a factor of 1.22 in front which arises
from the circular shape of the aperture. Let the distance
from the lens to the patient's kidney stone be $L=20$ cm.
You will want $f>20$ kHz, so that the sound is inaudible.
Find values of $b$ and $f$ that would result in a usable
design, where the central spot is small enough to lie within
a kidney stone 1 cm in diameter.
