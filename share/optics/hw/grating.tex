\answercheck The beam of a laser passes through a diffraction
grating, fans out, and illuminates a wall that is perpendicular
to the original beam, lying at a distance of 2.0 m from
the grating. The beam is produced by a helium-neon laser,
and has a wavelength of 694.3 nm. The grating has 2000 lines
per centimeter. (a) What is the distance on the wall between
the central maximum and the maxima immediately to its right
and left? (b) How much does your answer change when you
use the small-angle approximations $\theta\approx\sin\theta\approx\tan\theta$?
