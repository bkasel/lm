Electron microscopes can make images of individual atoms,
but why will a visible-light microscope never be able to?
Stereo speakers create the illusion of music that comes from
a band arranged in your living room, but why doesn't the
stereo illusion work with bass notes? Why are computer chip
manufacturers investing billions of dollars in equipment to
etch chips with x-rays instead of visible light?

The answers to all of these questions have to do with the
subject of wave optics. So far this book has discussed the
interaction of light waves with matter, and its practical
applications to optical devices like mirrors, but we have
used the ray model of light almost exclusively. Hardly ever
have we explicitly made use of the fact that light is an
electromagnetic wave. We were able to get away with the
simple ray model because the chunks of matter we were
discussing, such as lenses and mirrors, were thousands of
times larger than a wavelength of light. We now turn to
phenomena and devices that can only be understood using the
wave model of light.

<% begin_sec("Diffraction",0) %>

<% marg(m4_ifelse(__sn,1,[:40:],[:0:])) %>
<%
  fig(
    'double-slit-water-waves',
    %q{%
      In this view from overhead, a straight, sinusoidal
      water wave encounters a barrier with two gaps in it. Strong wave vibration occurs at
      angles X and Z, but there is none at all at angle Y. (The figure has been retouched
      from a real photo of water waves. In reality, the waves beyond the barrier would be
      much weaker than the ones before it, and they would therefore be difficult to see.)
    }
  )
%>
\spacebetweenfigs
<%
  fig(
    'double-slit-no-diffraction',
    %q{This doesn't happen.}
  )
%>

<% end_marg %>
Figure \figref{double-slit-water-waves} shows a typical problem in wave optics, enacted
with water waves. It may seem surprising that we don't get a
simple pattern like figure \figref{double-slit-no-diffraction}, but the pattern would only
be that simple if the wavelength was hundreds of times
shorter than the distance between the gaps in the barrier
and the widths of the gaps.

Wave optics is a broad subject, but this example will help
us to pick out a reasonable set of restrictions to make
things more manageable:

(1) We restrict ourselves to cases in which a wave travels
through a uniform medium, encounters a certain area in which
the medium has different properties, and then emerges on the
other side into a second uniform region.

(2) We assume that the incoming wave is a nice tidy
sine-wave pattern with wavefronts that are lines (or, in
three dimensions, planes).

(3) In figure \figref{double-slit-water-waves} we can see that the wave pattern
immediately beyond the barrier is rather complex, but
farther on it sorts itself out into a set of wedges
separated by gaps in which the water is still. We will
restrict ourselves to studying the simpler wave patterns
that occur farther away, so that the main question of
interest is how intense the outgoing wave is at a given angle.

The kind of phenomenon described by restriction (1) is
called \index{diffraction!defined}\emph{diffraction}. Diffraction
can be defined as the behavior of a wave when it encounters
an obstacle or a nonuniformity in its medium. In general,
diffraction causes a wave to bend around obstacles and make
patterns of strong and weak waves radiating out beyond the
obstacle. Understanding diffraction is the central problem
of wave optics. If you understand diffraction, even the
subset of diffraction problems that fall within restrictions
(2) and (3), the rest of wave optics is icing on the cake.

Diffraction can be used to find the structure of an unknown
diffracting object: even if the object is too small to study
with ordinary imaging, it may be possible to work backward
from the diffraction pattern to learn about the object. The
structure of a crystal, for example, can be determined from
its x-ray diffraction pattern.

Diffraction can also be a bad thing. In a telescope, for
example, light waves are diffracted by all the parts of the
instrument. This will cause the image of a star to appear
fuzzy even when the focus has been adjusted correctly. By
understanding diffraction, one can learn how a telescope
must be designed in order to reduce this problem ---
essentially, it should have the biggest possible diameter.

There are two ways in which restriction (2) might commonly
be violated. First, the light might be a mixture of
wavelengths. If we simply want to observe a diffraction
pattern or to use diffraction as a technique for studying
the object doing the diffracting (e.g., if the object is too
small to see with a microscope), then we can pass the light
through a colored filter before diffracting it.
<% marg(m4_ifelse(__sn,1,[:50:],[:25:])) %>
<%
  fig(
    'double-slit-setup',
    %q{%
      A practical, low-tech setup for observing
      diffraction of light.
    }
  )
%>
\vspace{15mm}
<%
  fig(
    'scaling',
    %q{%
      The bottom figure is simply a copy of the
      middle portion of the top one, scaled up by a factor of two. All the angles
      are the same. Physically, the angular pattern of the diffraction fringes
      can't be any different if we scale both $\lambda$ and $d$ by the same
      factor, leaving $\lambda/d$ unchanged.
    }
  )
%>
<% end_marg %>

A second issue is that light from sources such as the sun or
a lightbulb does not consist of a nice neat plane wave,
except over very small regions of space. Different parts of
the wave are out of step with each other, and the wave is
referred to as \index{incoherent light}\emph{incoherent}. One way
of dealing with this is shown in figure \figref{double-slit-setup}. After filtering
to select a certain wavelength of red light, we pass the
light through a small pinhole. The region of the light that
is intercepted by the pinhole is so small that one part of
it is not out of step with another. Beyond the pinhole,
light spreads out in a spherical wave; this is analogous to
what happens when you speak into one end of a paper towel
roll and the sound waves spread out in all directions from
the other end. By the time the spherical wave gets to the
double slit it has spread out and reduced its curvature, so
that we can now think of it as a simple plane wave.

If this seems laborious, you may be relieved to know that
modern technology gives us an easier way to produce a
single-wavelength, coherent beam of light: the laser.

The parts of the final image on the screen in \figref{double-slit-setup} are called
diffraction \index{diffraction!fringe}\index{fringe!diffraction}fringes.
The center of each fringe is a point of maximum brightness,
and halfway between two fringes is a minimum.

\startdq

\begin{dq}
Why would x-rays rather than visible light be used to find
the structure of a crystal? Sound waves are used to make
images of fetuses in the womb. What would influence the
choice of wavelength?
\end{dq}

<% end_sec() %>
<% begin_sec("Scaling of Diffraction",0) %>\index{diffraction!scaling of}

This chapter has ``optics'' in its title, so it is nominally
about light, but we started out with an example involving
water waves. Water waves are certainly easier to visualize,
but is this a legitimate comparison? In fact the analogy
works quite well, despite the fact that a light wave has a
wavelength about a million times shorter. This is because
diffraction effects scale uniformly. That is, if we enlarge
or reduce the whole diffraction situation by the same
factor, including both the wavelengths and the sizes of the
obstacles the wave encounters, the result is still a valid solution.

This is unusually simple behavior! In __subsection_or_section(scaling)
we saw many examples of more complex scaling, such as
the impossibility of bacteria the size of dogs, or the need
for an elephant to eliminate heat through its ears because
of its small surface-to-volume ratio, whereas a tiny shrew's
life-style centers around conserving its body heat.

Of course water waves and light waves differ in many ways,
not just in scale, but the general facts you will learn
about diffraction are applicable to all waves. In some ways
it might have been more appropriate to insert this chapter
after __section_or_chapter(bounded-waves) on bounded waves, but many of the
important applications are to light waves, and you would
probably have found these much more difficult without any
background in optics.

Another way of stating the simple scaling behavior of
diffraction is that the diffraction angles we get depend
only on the unitless ratio $\lambda $/d, where $\lambda$ is
the wavelength of the wave and $d$ is some dimension of the
diffracting objects, e.g., the center-to-center spacing
between the slits in figure \figref{double-slit-water-waves}. If, for instance, we scale
up both $\lambda $ and $d$ by a factor of 37, the ratio
$\lambda /d$ will be unchanged.
m4_ifelse(__sn,1,[:
%%%%%%%%%%%% figure needs to be on a different page for LM -- see below
<% marg() %>
<%
  fig(
    'huygens',
    %q{Christiaan Huygens (1629-1695).}
  )
%>
<% end_marg %>
:],[::])

<% end_sec() %>
<% begin_sec("The Correspondence Principle",0) %>

The only reason we don't usually notice diffraction of light
in everyday life is that we don't normally deal with objects
that are comparable in size to a wavelength of visible
light, which is about a millionth of a meter. Does this mean
that wave optics contradicts ray optics, or that wave optics
sometimes gives wrong results? No. If you hold three fingers
out in the sunlight and cast a shadow with them, \emph{either}
wave optics or ray optics can be used to predict the
straightforward result: a shadow pattern with two bright
lines where the light has gone through the gaps between your
fingers. Wave optics is a more general theory than ray
optics, so in any case where ray optics is valid, the two
theories will agree. This is an example of a general idea
enunciated by the physicist \index{Bohr!Niels}Niels Bohr,
called the \index{correspondence principle}\emph{correspondence
principle:\/} when flaws in a physical theory lead to the
creation of a new and more general theory, the new theory
must still agree with the old theory within its more
restricted area of applicability. After all, a theory is
only created as a way of describing experimental observations.
If the original theory had not worked in any cases at all,
it would never have become accepted.

m4_ifelse(__sn,1,[::],[:
%%%%%%%%%%%% figure needs to be on a different page for SN -- see above
<% marg(70) %>
<%
  fig(
    'huygens',
    %q{Christiaan Huygens (1629-1695).}
  )
%>
<% end_marg %>
:])

In the case of optics, the correspondence principle tells us
that when $\lambda /d$ is small, both the ray and the wave
model of light must give approximately the same result.
Suppose you spread your fingers and cast a shadow with them
using a coherent light source. The quantity $\lambda /d$ is
about $10^{-4}$, so the two models will agree very closely. (To
be specific, the shadows of your fingers will be outlined by
a series of light and dark fringes, but the angle subtended
by a fringe will be on the order of $10^{-4}$ radians, so
they will be too tiny to be visible.

\pagebreak

<% self_check('diffract-around-body',<<-'SELF_CHECK'
What kind of wavelength would an electromagnetic wave have
to have in order to diffract dramatically around your body?
Does this contradict the correspondence principle?
  SELF_CHECK
  ) %>

<% end_sec() %>
<% begin_sec("Huygens' Principle",0) %>

<% marg(m4_ifelse(__sn,1,[:70:],[:10:])) %>
<%
  fig(
    'double-slit-water-waves-photo',
    %q{Double-slit diffraction.}
  )
%>
\spacebetweenfigs
<%
  fig(
    'huygens-1',
    %q{%
      A wavefront can be analyzed by the principle of superposition,
      breaking it down into many small parts.
    }
  )
%>
\spacebetweenfigs
<%
  fig(
    'huygens-2',
    %q{%
      If it was by itself, each of the parts would spread out as a
      circular ripple.
    }
  )
%>
\spacebetweenfigs
<%
  fig(
    'huygens-3',
    %q{Adding up the ripples produces a new wavefront.}
  )
%>

<% end_marg %>
Returning to the example of double-slit diffraction, \figref{double-slit-water-waves-photo},
note the strong visual impression of two overlapping sets of
concentric semicircles. This is an example of \index{Huygens'
principle}\emph{Huygens' principle}, named after a Dutch physicist
and astronomer. (The first syllable rhymes with ``boy.'')
Huygens' principle states that any wavefront can be broken
down into many small side-by-side wave peaks, \figref{huygens-1}, which
then spread out as circular ripples, \figref{huygens-2}, and by the
principle of superposition, the result of adding up these
sets of ripples must give the same result as allowing the
wave to propagate forward, \figref{huygens-3}. In the case of sound or
light waves, which propagate in three dimensions, the
``ripples'' are actually spherical rather than circular, but
we can often imagine things in two dimensions for simplicity.

In double-slit diffraction the application of Huygens'
principle is visually convincing: it is as though all the
sets of ripples have been blocked except for two. It is a
rather surprising mathematical fact, however, that Huygens'
principle gives the right result in the case of an
unobstructed linear wave, \figref{huygens-2} and \figref{huygens-3}. A theoretically
infinite number of circular wave patterns somehow conspire
to add together and produce the simple linear wave motion
with which we are familiar.

Since Huygens' principle is equivalent to the principle of
superposition, and superposition is a property of waves,
what Huygens had created was essentially the first wave
theory of light. However, he imagined light as a series of
pulses, like hand claps, rather than as a sinusoidal wave.

The history is interesting. Isaac \index{Newton, Isaac!particle theory of light}Newton
loved the atomic theory of matter so much that he searched
enthusiastically for evidence that light was also made of
tiny particles. The paths of his light particles would
correspond to rays in our description; the only significant
difference between a ray model and a particle model of light
would occur if one could isolate individual particles and
show that light had a ``graininess'' to it. Newton never did
this, so although he thought of his model as a particle
model, it is more accurate to say he was one of the builders
of the ray \index{ray model of light}model.

Almost all that was known about reflection and refraction of
light could be interpreted equally well in terms of a
\index{particle model of light}particle model or a wave
model, but Newton had one reason for strongly opposing
Huygens' \index{wave model of light}wave theory. Newton knew
that waves exhibited diffraction, but diffraction of light
is difficult to observe, so Newton believed that light did
not exhibit diffraction, and therefore must not be a wave.
Although Newton's criticisms were fair enough, the debate
also took on the overtones of a nationalistic dispute
between England and continental Europe, fueled by English
resentment over Leibniz's supposed plagiarism of Newton's
calculus. Newton wrote a book on optics, and his prestige
and political prominence tended to discourage questioning of his model.

<% marg(m4_ifelse(__sn,1,[:70:],[:53.5:])) %>
<%
  fig(
    'young',
    %q{Thomas Young}
  )
%>
\spacebetweenfigs
<%
  fig(
    'double-slit-water-waves',
    %q{Double-slit diffraction.},
    {'suffix'=>'2'}
  )
%>
\spacebetweenfigs
<%
  fig(
    'double-slit-overlapping',
    %q{Use of Huygens' principle.}
  )
%>
\spacebetweenfigs
<%
  fig(
    'double-slit-path-length',
    %q{Constructive interference along the center-line.}
  )
%>

<% end_marg %>
\index{Young, Thomas}Thomas Young (1773-1829) was the person
who finally, a hundred years later, did a careful search for
wave interference effects with light and analyzed the
results correctly. He observed double-slit diffraction of
light as well as a variety of other diffraction effects, all
of which showed that light exhibited wave interference
effects, and that the wavelengths of visible light waves
were extremely short. The crowning achievement was the
demonstration by the experimentalist Heinrich \index{Hertz,
Heinrich!Heinrich}Hertz and the theorist James Clerk
\index{Maxwell, James Clerk}Maxwell that light was an
\emph{electromagnetic} wave. Maxwell is said to have related
his discovery to his wife one starry evening and told her
that she was the only other person in the world who knew what starlight was.

<% end_sec() %>
<% begin_sec("Double-Slit Diffraction",0) %>\index{double-slit diffraction}\index{diffraction!double-slit}

Let's now analyze double-slit diffraction, \figref{double-slit-water-waves2}, using
Huygens' principle. The most interesting question is how to
compute the angles such as X and Z where the wave
intensity is at a maximum, and the in-between angles like
Y where it is minimized. Let's measure all our angles
with respect to the vertical center line of the figure,
which was the original direction of propagation of the wave.

If we assume that the width of the slits is small (on the
order of the wavelength of the wave or less), then we can
imagine only a single set of Huygens ripples spreading out
from each one, \figref{double-slit-overlapping}.  White lines represent peaks, black
ones troughs. The only dimension of the diffracting
slits that has any effect on the geometric pattern of the
overlapping ripples is then the center-to-center distance,
$d$, between the slits.

We know from our discussion of the scaling of diffraction
that there must be some equation that relates an angle like
$\theta_Z$ to the ratio $\lambda /d$,
\begin{equation*}
        \frac{\lambda}{d} \leftrightarrow \theta_Z\eqquad.
\end{equation*}
If the equation for $\theta_Z$ depended on some other
expression such as $\lambda +d$ or $\lambda^2/d$, then it
would change when we scaled $\lambda $ and $d$ by the same
factor, which would violate what we know about the
scaling of diffraction.

Along the central maximum line, X, we always have positive
waves coinciding with positive ones and negative waves
coinciding with negative ones. (I have arbitrarily chosen to
take a snapshot of the pattern at a moment when the waves
emerging from the slit are experiencing a positive peak.)
The superposition of the two sets of ripples therefore
results in a doubling of the wave amplitude along this line.
There is constructive interference. This is easy to explain,
because by symmetry, each wave has had to travel an equal
number of wavelengths to get from its slit to the center line, \figref{double-slit-path-length}:
Because both sets of ripples have ten wavelengths to cover in
order to reach the point along direction X, they will be in step when they get there.

<% marg(55) %>
<%
  fig(
    'double-slit-derivation-1',
    %q{%
      The waves travel distances $L$ and $L'$ from the two
      slits to get to the same point in space, at an angle $\theta$ from the center line.
    }
  )
%>
\spacebetweenfigs
<%
  fig(
    'double-slit-derivation-2',
    %q{%
      A close-up view of figure \figref{double-slit-derivation-1}, showing
      how the path length difference $L-L'$ is related to $d$ and to the angle $\theta$.
    }
  )
%>

<% end_marg %>
At the point along direction Y shown in the same figure,
one wave has traveled ten wavelengths, and is therefore at a
positive extreme, but the other has traveled only nine and a
half wavelengths, so it at a negative extreme. There is
perfect cancellation, so points along this line experience no wave motion.

But the distance traveled does not have to be equal in order
to get constructive interference. At the point along
direction Z, one wave has gone nine wavelengths and the
other ten. They are both at a positive extreme.

<% self_check('trough-trough',<<-'SELF_CHECK'
At a point half a wavelength below the point marked along
direction X, carry out a similar analysis.
  SELF_CHECK
  ) %>

To summarize, we will have perfect constructive interference
at any point where the distance to one slit differs from the
distance to the other slit by an integer number of
wavelengths. Perfect destructive interference will occur
when the number of wavelengths of path length difference
equals an integer plus a half.

Now we are ready to find the equation that predicts the
angles of the maxima and minima. The waves travel different
distances to get to the same point in space, \figref{double-slit-derivation-1}. We need to
find whether the waves are in phase (in step) or out of
phase at this point in order to predict whether there will
be constructive interference, destructive interference, or
something in between.

One of our basic assumptions in this chapter is that we will
only be dealing with the diffracted wave in regions very far
away from the object that diffracts it, so the triangle is
long and skinny. Most real-world examples with diffraction
of light, in fact, would have triangles with even skinner
proportions than this one. The two long sides are therefore
very nearly parallel, and we are justified in drawing the
right triangle shown in figure \figref{double-slit-derivation-2}, labeling one leg of the
right triangle as the difference in path length , $L-L'$,
and labeling the acute angle as $\theta $. (In reality this
angle is a tiny bit greater than the one labeled $\theta $
in figure \figref{double-slit-derivation-1}.)

The difference in path length is related to $d$ and
$\theta $ by the equation
\begin{equation*}
        \frac{L-L'}{d}          =  \sin  \theta\eqquad.
\end{equation*}
Constructive interference will result in a maximum at angles
for which $L-L'$ is an integer number of wavelengths,
\begin{multline*}
        L-L'  =  m\lambda\eqquad. \hfill
                  \shoveright{\text{[condition for a maximum;}}\\
                  \text{$m$ is an integer]}
\end{multline*}
Here $m$ equals 0 for the central maximum, $-1$ for the first
maximum to its left, $+2$ for the second maximum on the right,
etc. Putting all the ingredients together, we find 
$m\lambda/d=\sin \theta $, or
\begin{multline*}
        \frac{\lambda}{d} = \frac{\sin\theta}{m}\eqquad. \hfill
                  \shoveright{\text{[condition for a maximum;}}\\
                  \text{$m$ is an integer]}
\end{multline*}
Similarly, the condition for a minimum is
\begin{multline*}
        \frac{\lambda}{d} = \frac{\sin\theta}{m}\eqquad. \hfill
                  \shoveright{\text{[condition for a minimum;}}\\
                  \text{$m$ is an integer plus $1/2$]}
\end{multline*}
That is, the minima are about halfway between the maxima.

\vspace{2mm plus 3mm}

As expected based on scaling, this equation relates angles
to the unitless ratio $\lambda /d$. Alternatively, we could
say that we have proven the scaling property in the special
case of double-slit diffraction. It was inevitable that the
result would have these scaling properties, since the whole
proof was geometric, and would have been equally valid when
enlarged or reduced on a photocopying machine!

<% marg(m4_ifelse(__sn,1,[:120:],[:81:])) %>
<%
  fig(
    'double-slit-d',
    %q{Cutting $d$ in half doubles the angles of the diffraction fringes.}
  )
%>
\spacebetweenfigs
<%
  fig(
    'double-slit-wavelength',
    %q{%
      Double-slit diffraction patterns of long-wavelength red light (top)
      and short-wavelength blue light (bottom).
    }
  )
%>

<% end_marg %>

\vspace{2mm plus 3mm}

Counterintuitively, this means that a diffracting object
with smaller dimensions produces a bigger diffraction pattern, \figref{double-slit-d}.

\vspace{4mm plus 3mm}

\enlargethispage{-4\baselineskip}

\begin{eg}{Double-slit diffraction of blue and red light}
Blue light has a shorter wavelength than red. For a given
double-slit spacing $d$, the smaller value of $\lambda /d$
for leads to smaller values of $\sin \theta $, and
therefore to a more closely spaced set of diffraction fringes, (g)
\end{eg}

\vspace{2mm plus 3mm}

\begin{eg}{The correspondence principle}
Let's also consider how the equations for double-slit
diffraction relate to the correspondence principle. When the
ratio $\lambda /d$ is very small, we should recover the case
of simple ray optics. Now if $\lambda /d$ is small, 
$\sin\theta $ must be small as well, and the spacing between the
diffraction fringes will be small as well. Although we have
not proven it, the central fringe is always the brightest,
and the fringes get dimmer and dimmer as we go farther from
it. For small values of $\lambda /d$, the part of the
diffraction pattern that is bright enough to be detectable
covers only a small range of angles. This is exactly what we
would expect from ray optics: the rays passing through the
two slits would remain parallel, and would continue moving
in the $\theta =0$ direction. (In fact there would be images
of the two separate slits on the screen, but our analysis
was all in terms of angles, so we should not expect it to
address the issue of whether there is structure within a set
of rays that are all traveling in the $\theta =0$ direction.)
\end{eg}
<% marg(-20) %>
<%
  fig(
    'diffraction-graph',
    %q{Interpretation of the angular spacing $\Delta\theta$ in example \ref{eg:diffraction-delta-theta}.
       It can be defined
       either from maximum to maximum or from
       minimum to minimum. Either way, the result is the same. It does not make sense to try to interpret
       $\Delta\theta$ as the width of a fringe; one can see from the graph and from the
       image below that it is not obvious either
       that such a thing is well defined or that it would be the same for all fringes.}
  )
%>
<% end_marg %>

\enlargethispage{\baselineskip}

\begin{eg}{Spacing of the fringes at small angles}\label{eg:diffraction-delta-theta}
At small angles, we can use the approximation $\sin\theta\approx\theta$, which is
valid if $\theta $ is measured in radians. The equation for
double-slit diffraction becomes simply
\begin{equation*}
 \frac{\lambda}{d} = \frac{\theta}{m}\eqquad,
\end{equation*}
which can be solved for $\theta $ to give
\begin{equation*}
  \theta = \frac{m\lambda}{d}\eqquad.
\end{equation*}
The difference in angle between successive fringes is the
change in $\theta $ that results from changing $m$
by plus or minus one,
\begin{equation*}
                \Delta\theta  = \frac{\lambda}{d}\eqquad.
\end{equation*}
For example, if we write $\theta_7$ for the angle of the
seventh bright fringe on one side of the central maximum and
$\theta_8$ for the neighboring one, we have
\begin{align*}
                \theta_8-\theta_7 &= \frac{8\lambda}{d}-\frac{7\lambda}{d}\\
                         &= \frac{\lambda}{d}\eqquad,
\end{align*}
and similarly for any other neighboring pair of fringes.
\end{eg}

Although the equation $\lambda /d=\sin \theta /m$ is only
valid for a double slit, it is can still be a guide to our
thinking even if we are observing diffraction of light by a
virus or a flea's leg: it is always true that

(1) large values of $\lambda /d$ lead to a broad diffraction pattern, and

(2) diffraction patterns are repetitive.

In many cases the equation looks just like 
$\lambda /d =\sin \theta /m$ but with an extra numerical factor thrown
in, and with $d$ interpreted as some other dimension of the
object, e.g., the diameter of a piece of wire.

\enlargethispage{-4\baselineskip}

<% end_sec() %>
<% begin_sec("Repetition",3) %>

Suppose we replace a double slit with a triple slit, \figref{triple-slit}. We
can think of this as a third \index{repetition of
diffracting objects}repetition of the structures that were
present in the double slit. Will this device be an
improvement over the double slit for any practical reasons?
<% marg(0) %>
<%
  fig(
    'triple-slit',
    %q{A triple slit.}
  )
%>
\spacebetweenfigs
<%
  fig(
    'two-slits-and-five',
    %q{A double-slit diffraction pattern (top), and a pattern made by five slits (bottom).}
  )
%>
<% end_marg %>

The answer is yes, as can be shown using figure \figref{triple-slit-numerical}.
 For ease of visualization, I have violated our usual
rule of only considering points very far from the diffracting
object. The scale of the drawing is such that a wavelengths
is one cm. In \subfigref{triple-slit-numerical}{1}, all three waves travel an integer number
of wavelengths to reach the same point, so there is a bright
central spot, as we would expect from our experience with
the double slit. In figure \subfigref{triple-slit-numerical}{2}, we show the path lengths to
a new point. This point is farther from slit A by a quarter
of a wavelength, and correspondingly closer to slit C. The
distance from slit B has hardly changed at all. Because
the paths lengths traveled from slits A and C differ by
half a wavelength, there will be perfect destructive
interference between these two waves. There is still some
uncanceled wave intensity because of slit B, but the
amplitude will be three times less than in figure \subfigref{triple-slit-numerical}{1},
resulting in a factor of 9 decrease in brightness. Thus, by
moving off to the right a little, we have gone from the
bright central maximum to a point that is quite dark.
<%
  fig(
    'triple-slit-numerical',
    %q{%
      1. There is a bright central maximum. 2. At this point
      just off the central maximum, the path lengths traveled by the three waves have changed.
    },
    {
      'width'=>'fullpage'
    }
  )
%>

Now let's compare with what would have happened if slit C
had been covered, creating a plain old double slit. The
waves coming from slits A and B would have been out of
phase by 0.23 wavelengths, but this would not have caused
very severe interference. The point in figure \subfigref{triple-slit-numerical}{2} would have
been quite brightly lit up.

\enlargethispage{-\baselineskip}

To summarize, we have found that adding a third slit narrows
down the central fringe dramatically. The same is true for
all the other fringes as well, and since the same amount of
energy is concentrated in narrower diffraction fringes, each
fringe is brighter and easier to see, \figref{two-slits-and-five}.

This is an example of a more general fact about diffraction:
if some feature of the diffracting object is repeated, the
locations of the maxima and minima are unchanged, but
they become narrower.

Taking this reasoning to its logical conclusion, a
diffracting object with thousands of slits would produce
extremely narrow fringes. Such an object is called a
\index{diffraction grating}diffraction grating.

<% end_sec() %>
<% begin_sec("Single-Slit Diffraction",0) %>\index{single-slit!diffraction}\index{diffraction!single-slit}

<% marg(m4_ifelse(__sn,1,[:0:],[:0:])) %>
<%
  fig(
    'single-slit-water-waves',
    %q{Single-slit diffraction of water waves.}
  )
%>
\spacebetweenfigs
<%
  fig(
    'single-slit',
    %q{Single-slit diffraction of red light. Note the double width of the central maximum.}
  )
%>
\spacebetweenfigs
<%
  fig(
    'single-slit-simulated-with-three-sources',
    %q{%
      A pretty good simulation of the single-slit pattern of figure \figref{single-slit-water-waves}, made
      by using three motors to produce overlapping ripples from three neighboring points in the water.
    }
  )
%>

<% end_marg %>
If we use only a single slit, is there diffraction? If the
slit is not wide compared to a wavelength of light, then we
can approximate its behavior by using only a single set of
Huygens ripples. There are no other sets of ripples to add
to it, so there are no constructive or destructive
interference effects, and no maxima or minima. The result
will be a uniform spherical wave of light spreading out in
all directions, like what we would expect from a tiny
lightbulb. We could call this a diffraction pattern, but it
is a completely featureless one, and it could not be used,
for instance, to determine the wavelength of the light, as
other diffraction patterns could.

All of this, however, assumes that the slit is narrow
compared to a wavelength of light. If, on the other hand,
the slit is broader, there will indeed be interference among
the sets of ripples spreading out from various points along
the opening. Figure \figref{single-slit-water-waves} shows an example with water waves,
and figure \figref{single-slit} with light.

<% self_check('single-slit-wavelength',<<-'SELF_CHECK'
How does the wavelength of the waves compare with the width
of the slit in figure \figref{single-slit-water-waves}?
  SELF_CHECK
  ) %>

We will not go into the details of the analysis of
single-slit diffraction, but let us see how its properties
can be related to the general things we've learned about
diffraction. We know based on scaling arguments that the
angular sizes of features in the diffraction pattern must be
related to the wavelength and the width, $a$, of the slit by
some relationship of the form
\begin{equation*}
        \frac{\lambda}{a} \leftrightarrow \theta\eqquad.
\end{equation*}
This is indeed true, and for instance the angle between the
maximum of the central fringe and the maximum of the next
fringe on one side equals $1.5\lambda/a$. Scaling arguments will
never produce factors such as the 1.5, but they tell us that
the answer must involve $\lambda /a$, so all the familiar
qualitative facts are true. For instance, shorter-wavelength
light will produce a more closely spaced diffraction pattern.

\enlargethispage{-\baselineskip}

An important scientific example of single-slit diffraction
is in telescopes. Images of individual stars, as in
figure \figref{pleiades-closeup}, are a good way to examine diffraction effects,
because all stars except the sun are so far away that no
telescope, even at the highest magnification, can image
their disks or surface features. Thus any features of a
star's image must be due purely to optical effects such as
diffraction. A prominent cross appears around the brightest
star, and dimmer ones surround the dimmer stars. Something
like this is seen in most telescope photos, and indicates
that inside the tube of the telescope there were two
perpendicular struts or supports. Light diffracted around
these struts. You might think that diffraction could be
eliminated entirely by getting rid of all obstructions in
the tube, but the circles around the stars are diffraction
effects arising from single-slit diffraction at the mouth of
the \index{telescope}telescope's tube! (Actually we have not
even talked about diffraction through a circular opening, but
the idea is the same.) Since the angular sizes of the
diffracted images depend on $\lambda $/a, the only way to
improve the resolution of the images is to increase the
diameter, $a$, of the tube. This is one of the main reasons
(in addition to light-gathering power) why the best
telescopes must be very large in diameter.

<% marg(131) %>
<%
  fig(
    'pleiades-closeup',
    %q{%
      An image of the Pleiades star cluster.
      The circular rings around the bright stars are due to single-slit
      diffraction at the mouth of the telescope's tube.
    }
  )
%>

<% end_marg %>
<% self_check('radio-telescopes',<<-'SELF_CHECK'
What would this imply about radio telescopes as compared
with visible-light telescopes?
  SELF_CHECK
  ) %>

<% marg(30) %>
<%
  fig(
    'very-large-array',
    %q{A radio telescope.}
  )
%>

<% end_marg %>

m4_ifelse(__sn,1,[:\enlargethispage{\baselineskip}:],[::])
Double-slit diffraction is easier to understand conceptually
than single-slit diffraction, but if you do a double-slit
diffraction experiment in real life, you are likely to
encounter a complicated pattern like figure \subfigref{double-slit-fringes-realistic}{1},
rather than the simpler one, 2, you were expecting.
This is because the slits are fairly big compared to the
wavelength of the light being used. We really have two
different distances in our pair of slits: $d$, the distance
between the slits, and $w$, the width of each slit. Remember
that smaller distances on the object the light diffracts
around correspond to larger features of the diffraction
pattern. The pattern 1 thus has two spacings in it: a short
spacing corresponding to the large distance $d$, and a long
spacing that relates to the small dimension $w$.

<%
  fig(
    'double-slit-fringes-realistic',
    %q{%
      1. A diffraction pattern formed by a real double slit. The width of each slit is fairly big
      compared to the wavelength of the light. This is a real photo. 2. This idealized pattern is not likely to occur in real life. To get it,
      you would need each slit to be so narrow that its width was comparable to the wavelength of the light, but that's not usually possible. This is not a real
      photo. 3. A real photo of a single-slit diffraction pattern caused by a slit whose width is the same as the widths of the slits used to make the
      top pattern.
    },
    {
      'width'=>'fullpage'
    }
  )
%>


\startdq

\begin{dq}
Why is it optically impossible for bacteria to evolve eyes
that use visible light to form images?
\end{dq}

<% end_sec() %>
<% begin_sec("The Principle of Least Time",nil,'',{'optional'=>true,'calc'=>true}) %>

In m4_ifelse(__sn,1,[:subsection:],[:section:]) 
__bare_subsection_or_section(least-time-reflection) 
and
__bare_subsection_or_section(least-time-refraction),
we saw how in the ray model of
light, both refraction and reflection can be described in an
elegant and beautiful way by a single principle, the
principle of least time. We can now justify the principle of
least time based on the wave model of light. Consider an
example involving reflection, \figref{reflection-least-time}. Starting at point A,
Huygens' principle for waves tells us that we can think of
the wave as spreading out in all directions. Suppose we
imagine all the possible ways that a ray could travel from A
to B. We show this by drawing 25 possible paths, of which
the central one is the shortest. Since the principle of
least time connects the wave model to the ray model, we
should expect to get the most accurate results when the
wavelength is much shorter than the distances involved ---
for the sake of this numerical example, let's say that a
wavelength is 1/10 of the shortest reflected path from A to
B. The table, 2, shows the distances traveled by the 25 rays.
m4_ifelse(__sn,0,[:
<% marg(0) %>
<%
  fig(
    'reflection-least-time',
    %q{Light could take many different paths from A to B.}
  )
%>
<% end_marg %>
:])


Note how similar are the distances traveled by the group of
7 rays, indicated with a bracket, that come closest to
obeying the principle of least time. If we think of each one
as a wave, then all 7 are again nearly in phase at point
B. However, the rays that are farther from satisfying the
principle of least time show more rapidly changing
distances; on reuniting at point B, their phases are a
random jumble, and they will very nearly cancel each other
out. Thus, almost none of the wave energy delivered to point
B goes by these longer paths. Physically we find, for
instance, that a wave pulse emitted at A is observed at B
after a time interval corresponding very nearly to the
shortest possible path, and the pulse is not very ``smeared
out'' when it gets there. The shorter the wavelength
compared to the dimensions of the figure, the more accurate
these approximate statements become.\index{least time, principle of}

Instead of drawing a finite number of rays, such 25, what
happens if we think of the angle, $\theta $, of emission of
the ray as a continuously varying variable? Minimizing the
distance $L$ requires
\begin{equation*}
        \frac{\der L}{\der\theta} = 0\eqquad.
\end{equation*}

Because $L$ is changing slowly in the vicinity of the angle
that satisfies the principle of least time, all the rays
that come out close to this angle have very nearly the same
$L$, and remain very nearly in phase when they reach B.
This is the basic reason why the discrete table, \subfigref{reflection-least-time}{2}, turned
out to have a group of rays that all traveled nearly the same distance.
m4_ifelse(__sn,1,[:
<% marg(150) %>
<%
  fig(
    'reflection-least-time',
    %q{Light could take many different paths from A to B.}
  )
%>
<% end_marg %>
:])

As discussed in __subsection_or_section(least-time-reflection), the principle of least time is
really a principle of least \emph{or greatest} time. This
makes perfect sense, since $\der L/\der \theta =0$ can in general
describe either a minimum or a maximum

The principle of least time is very general. It does not
apply just to refraction and reflection --- it can even be
used to prove that light rays travel in a straight line
through empty space, without taking detours! This general
approach to wave motion was used by Richard Feynman, one of
the pioneers who in the 1950's reconciled quantum mechanics
with relativity. A very readable explanation is
given in a book Feynman wrote for laypeople, QED: The
Strange Theory of Light and Matter.

<% end_sec() %>
