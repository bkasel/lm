Infants are always fascinated by the antics of the Baby in
the Mirror. Now if you want to know something about mirror
images that most people don't understand, try this. First
bring this page closer and closer to your eyes, until you
can no longer focus on it without straining. Then go in the
bathroom and see how close you can get your face to the
surface of the mirror before you can no longer easily focus
on the image of your own eyes. You will find that the
shortest comfortable eye-mirror distance is much less than
the shortest comfortable eye-paper distance. This demonstrates
that the image of your face in the mirror acts as if it had
depth and existed in the space \emph{behind} the mirror. If
the image was like a flat picture in a book, then you
wouldn't be able to focus on it from such a short distance.

In this chapter we will study the images formed by flat and
curved mirrors on a qualitative, conceptual basis. Although
this type of image is not as commonly encountered in
everyday life as images formed by lenses, images formed by
reflection are simpler to understand, so we discuss them
first. In __section_or_chapter(images-2) we will turn to a more mathematical
treatment of images made by reflection. Surprisingly, the
same equations can also be applied to lenses, which are
the topic of __section_or_chapter(refraction).

<% begin_sec("A virtual image",0) %>\index{images!formed by plane mirrors}\index{images!virtual}

<% marg(0) %>
<%
  fig(
    'flat-mirror',
    %q{An image formed by a mirror.}
  )
%>
<% end_marg %>
We can understand a mirror image using a ray diagram. Figure \figref{flat-mirror}
 shows several light rays, 1, that originated by
diffuse reflection at the person's nose. They bounce off the
mirror, producing new rays, 2. To anyone whose eye is in
the right position to get one of these rays, they appear to
have come from a behind the mirror, 3, where they would
have originated from a single point. This point is where the
tip of the image-person's nose appears to be. A similar
analysis applies to every other point on the person's face,
so it looks as though there was an entire face behind the
mirror. The customary way of describing the situation
requires some explanation:

\begin{description}

\item[Customary description in physics:] There is an image of the
face behind the mirror.

\item[Translation:] The pattern of rays coming from the mirror is
exactly the same as it would be if there were a face behind
the mirror. Nothing is really behind the mirror.

\end{description}

This is referred to as a \emph{virtual} image, because the
rays do not actually cross at the point behind the mirror.
They only appear to have originated there.

<% self_check('move-object-laterally',<<-'SELF_CHECK'

Imagine that the person in figure \figref{flat-mirror} moves his face down
quite a bit --- a couple of feet in real life, or a few
inches on this scale drawing. The mirror stays where it is. Draw a new ray diagram. Will
there still be an image? If so, where is it visible from?

  SELF_CHECK
  ) %>

The geometry of specular reflection tells us that rays 1 and
2 are at equal angles to the normal (the imaginary
perpendicular line piercing the mirror at the point of
reflection). This means that ray 2's imaginary continuation,
3, forms the same angle with the mirror as ray 1. Since
each ray of type 3 forms the same angles with the mirror
as its partner of type 1, we see that the distance of the
image from the mirror is the same as that of the actual face from
the mirror, and it lies directly across from it. The image
therefore appears to be the same size as the actual face.

<%
  fig(
    'eg-eye-chart',
    'Example \ref{eg:eye-chart}.',
    {'width'=>'wide','sidecaption'=>true}
  )
%>

\begin{eg}{An eye exam}\label{eg:eye-chart}
Figure \figref{eg-eye-chart} shows a typical setup in an optometrist's examination room.
The patient's vision is supposed to be tested at a distance of 6 meters (20 feet in the U.S.),
but this distance is larger than the amount of space available in the room. Therefore a mirror is
used to create an image of the eye chart behind the wall.
\end{eg}

\begin{eg}{The Praxinoscope}\index{praxinoscope}
Figure \figref{praxinoscope} shows an old-fashioned device called
a praxinoscope, which displays an animated picture when spun.
The removable strip of paper with the pictures printed on it has twice
the radius of the inner circle made of flat mirrors, so each picture's
virtual image is at the center. As the wheel spins, each picture's image
is replaced by the next.
\end{eg}

<% marg(30) %>
<%
  fig(
    'praxinoscope',
    'The praxinoscope.'
  )
%>
<% end_marg %>

\vfill\pagebreak[4]

\startdq

\begin{dq}
The figure shows an object that is off to one side of a
mirror. Draw a ray diagram. Is an image formed? If so, where
is it, and from which directions would it be visible?

\end{dq}

<%
  fig(
    'dq-image-off-on-side',
    '',
    {'anonymous'=>true,'width'=>'fullpage','float'=>false}
  )
%>

<% end_sec() %>
<% begin_sec("Curved mirrors",4) %>\index{images!formed by curved mirrors}

<% marg(0) %>
<%
  fig(
    'virtual',
    %q{An image formed by a curved mirror.}
  )
%>
\spacebetweenfigs
<%
  fig(
    'curved-mirror-mag',
    %q{%
      The image is magnified by the same factor in
      depth and in its other dimensions.
    }
  )
%>
\spacebetweenfigs
<%
  fig(
    'lion',
    %q{%
      Increased magnification always comes at the expense of decreased field of view.
    }
  )
%>
<% end_marg %>
An image in a flat mirror is a pretechnological example:
even animals can look at their reflections in a calm pond.
We now pass to our first nontrivial example of the
manipulation of an image by technology: an image in a
curved mirror. Before we dive in, let's consider why this is
an important example. If it was just a question of
memorizing a bunch of facts about curved mirrors, then you
would rightly rebel against an effort to spoil the beauty of
your liberally educated brain by force-feeding you
technological trivia. The reason this is an important
example is not that curved mirrors are so important in and
of themselves, but that the results we derive for curved
bowl-shaped mirrors turn out to be true for a large class of
other optical devices, including mirrors that bulge outward
rather than inward, and lenses as well. A microscope or a
telescope is simply a combination of lenses or mirrors or
both. What you're really learning about here is the basic
building block of all optical devices from movie projectors to octopus eyes.

Because the mirror in figure \figref{virtual} is curved, it bends the
rays back closer together than a flat mirror would: we
describe it as \index{converging}\emph{converging}. Note that the
term refers to what it does to the light rays, not to the
physical shape of the mirror's surface . (The surface itself
would be described as \emph{concave}. The term is not all
that hard to remember, because the hollowed-out interior of
the mirror is like a cave.) It is surprising but true that
all the rays like 3 really do converge on a point, forming
a good image. We will not prove this fact, but it is true
for any mirror whose curvature is gentle enough and that is
symmetric with respect to rotation about the perpendicular
line passing through its center (not asymmetric like a
potato chip). The old-fashioned method of making mirrors and
lenses is by grinding them in grit by hand, and this
automatically tends to produce an almost perfect spherical surface.

Bending a ray like 2 inward implies bending its imaginary
continuation 3 outward, in the same way that raising one
end of a seesaw causes the other end to go down. The image
therefore forms deeper behind the mirror. This doesn't just
show that there is extra distance between the image-nose and
the mirror; it also implies that the image itself is bigger
from front to back. It has been \index{magnification!by a converging mirror}\emph{magnified} in the front-to-back direction.

It is easy to prove that the same magnification also applies
to the image's other dimensions. Consider a point like E
in figure \figref{curved-mirror-mag}. The trick is that out of all the rays
diffusely reflected by E, we pick the one that happens to
head for the mirror's center, C. The equal-angle property
of specular reflection plus a little straightforward
geometry easily leads us to the conclusion that triangles
ABC and CDE are the same shape, with ABC being simply a
scaled-up version of CDE. The magnification of depth equals
the ratio BC/CD, and the up-down magnification is AB/DE. A
repetition of the same proof shows that the magnification in
the third dimension (out of the page) is also the same. This
means that the image-head is simply a larger version of the
real one, without any distortion. The scaling factor is
called the magnification, $M$. The image in the figure is
magnified by a factor $M=1.9$.

Note that we did not explicitly specify whether the mirror
was a sphere, a paraboloid, or some other shape. However, we
assumed that a focused image would be formed, which would
not necessarily be true, for instance, for a mirror that was
asymmetric or very deeply curved.

<% end_sec() %>
<% begin_sec("A real image",0,'real-image') %>\index{images!real}

If we start by placing an object very close to the mirror,
\subfigref{real-and-virtual}{1}, and then move it farther and farther away, the image at
first behaves as we would expect from our everyday
experience with flat mirrors, receding deeper and deeper
behind the mirror. At a certain point, however, a dramatic
change occurs. When the object is more than a certain
distance from the mirror, \subfigref{real-and-virtual}{2}, the image appears upside-down
and in \emph{front} of the mirror.

<%
  fig(
    'real-and-virtual',
    %q{%
      1. A virtual image. 2. A real image. 
      As you'll verify in homework problem \ref{hw:invertedforehead}, the image is upside-down
    },
    {
      'width'=>'wide',
      'sidecaption'=>true
    }
  )
%>

Here's what's happened. The mirror bends light rays inward,
but when the object is very close to it, as in \subfigref{real-and-virtual}{1}, the rays
coming from a given point on the object are too strongly
diverging (spreading) for the mirror to bring them back
together. On reflection, the rays are still diverging, just
not as strongly diverging. But when the object is sufficiently
far away, \subfigref{real-and-virtual}{2}, the mirror is only intercepting the rays that
came out in a narrow cone, and it is able to bend these
enough so that they will reconverge.

Note that the rays shown in the figure, which both
originated at the same point on the object, reunite when
they cross. The point where they cross is the image of the
point on the original object. This type of image is called a
\emph{real image}, in contradistinction to the virtual
images we've studied before. 

\begin{important}
Definition: A real image is one where rays actually cross.
A virtual image is a point from which rays only appear to have come.
\end{important}

The use of the word ``real'' is
perhaps unfortunate. It sounds as though we are saying the
image was an actual material object, which of course it is not.

The distinction between a real image and a virtual image is
an important one, because a real image can be projected onto a
screen or photographic film. If a piece of paper is inserted
in figure \subfigref{real-and-virtual}{2} at the location of the image, the image will
be visible on the paper (provided the object is bright and
the room is dark). Your eye uses a lens to make a real image
on the \index{retina}retina.

<% self_check('real-do-and-di',<<-'SELF_CHECK'
Sketch another copy of the face in figure \subfigref{real-and-virtual}{1}, even farther
from the mirror, and draw a ray diagram. What has happened
to the location of the image?
  SELF_CHECK
  ) %>

<% end_sec() %>
<% begin_sec("Images of images",0) %>\index{images!of images}

If you are wearing glasses right now, then the light rays
from the page are being manipulated first by your glasses
and then by the lens of your eye. You might think that it
would be extremely difficult to analyze        this, but in fact it
is quite easy. In any series of optical elements (mirrors or
lenses or both), each element works on the rays furnished by
the previous element in exactly the same manner as if the
image formed by the previous element was an actual object.
<% marg(0) %>
<%
  fig(
    'newtonian-telescope',
    %q{A Newtonian telescope being used with a camera. }
  )
%>
<% end_marg %>

Figure \figref{newtonian-telescope} shows an example involving only mirrors. The
Newtonian telescope,\index{telescope} invented by Isaac \index{Newton, Isaac!Newtonian telescope}
Newton, consists of a large curved mirror, plus a
second, flat mirror that brings the light out of the tube.
(In very large telescopes, there may be enough room to put a
camera or even a person inside the tube, in which case the
second mirror is not needed.) The tube of the telescope is
not vital; it is mainly a structural element, although it
can also be helpful for blocking out stray light. The lens
has been removed from the front of the camera body, and is
not needed for this setup. Note that the two sample rays
have been drawn parallel, because an astronomical telescope
is used for viewing objects that are extremely far away.
These two ``parallel'' lines actually meet at a certain
point, say a crater on the moon, so they can't actually be
perfectly parallel, but they are parallel for all practical
purposes since we would have to follow them upward for a
quarter of a million miles to get to the point where they intersect.
<% marg(35) %>
<%
  fig(
    'newtonian-telescope-eye',
    %q{%
      A Newtonian telescope being used for visual rather than
      photographic observing. In real life, an eyepiece lens is normally used for
      additional magnification, but this simpler setup will also work.
    }
  )
%>
<% end_marg %>

The large curved mirror by itself would form an image $\zu{I}$, but
the small flat mirror creates an image of the image, $\zu{I}'$. The
relationship between $\zu{I}$ and $\zu{I}'$ is exactly the same as it
would be if $\zu{I}$ was an actual object rather than an image: $\zu{I}$
and $\zu{I}'$ are at equal distances from the plane of the mirror,
and the line between them is perpendicular to the plane of the mirror.

One surprising wrinkle is that whereas a flat mirror used by
itself forms a virtual image of an object that is real, here
the mirror is forming a real image of virtual image $\zu{I}$. This
shows how pointless it would be to try to memorize lists of
facts about what kinds of images are formed by various
optical elements under various circumstances. You are better
off simply drawing a ray diagram.


<%
  fig(
    'angular-size',
    %q{%
      The angular size of the flower depends on its
      distance from the eye.
    },
    {
      'width'=>'wide',
      'sidecaption'=>true,
      'sidepos'=>'b'
    }
  )
%>

Although the main point here was to give an example of an
image of an image, figure \figref{newtonian-telescope-eye} also shows an interesting case where we
need to make the distinction between \emph{magnification}
and \emph{angular magnification}\index{angular magnification}\index{magnification!angular}.
$\zu{I}$f you are looking at the moon through this telescope, then
the images $\zu{I}$ and $\zu{I}'$ are much \emph{smaller} than the actual
moon. Otherwise, for example, image $\zu{I}$ would not fit inside
the telescope! However, these images are very close to your
eye compared to the actual moon. The small size of the image
has been more than compensated for by the shorter distance.
The important thing here is the amount of \emph{angle}
within your field of view that the image covers, and it is
this angle that has been increased. The factor by which it
is increased is called the \emph{angular magnification}, $M_a$.

\pagebreak[4]

<%
  fig(
    'ladybug',
    %q{%
      The person uses a mirror to get a view of both sides of the ladybug. Although the flat mirror
      has $M=1$, it doesn't give an angular magnification of 1.
      The image is farther from the eye than the object, so the angular magnification
      $M_a=\alpha_i/\alpha_o$ is less than one.
    },
    {
      'width'=>'wide',
      'sidecaption'=>true
    }
  )
%>

\startdqs

\begin{dq}\label{dq:parallel-mirrors}
Locate the images of you that will be formed if you
stand between two parallel mirrors.
\end{dq}

  % couldn't get this to lay out properly using eruby-style figures
\fullpagewidthfignocaption{dq-parallel-mirrors}

\pagebreak


\begin{dq}
Locate the images formed by two perpendicular mirrors, as
in the figure. What happens if the mirrors are not
perfectly perpendicular?
\end{dq}

  % couldn't get this to lay out properly using eruby-style figures
\fullpagewidthfignocaption{dq-perpendicular-mirrors}

\pagebreak[4]

\begin{dq}
Locate the images formed by the periscope.
\end{dq}

  % couldn't get this to lay out properly using eruby-style figures
\fullpagewidthfignocaption{dq-periscope}

\pagebreak

<% end_sec() %>
