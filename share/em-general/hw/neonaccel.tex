A neon light consists of a long glass tube full of neon,
with metal caps on the ends.  Positive charge is placed on
one end of the tube, negative on the other.  The
electric forces generated can be strong enough to strip
electrons off of a certain number of neon atoms.  Assume for
simplicity that only one electron is ever stripped off of
any neon atom.  When an electron is stripped off of an atom,
both the electron and the neon atom (now an ion) have
electric charge, and they are accelerated by the forces
exerted by the charged ends of the tube.  (They do not feel
any significant forces from the other ions and electrons
within the tube, because only a tiny minority of neon atoms
ever gets ionized.)  Light is finally produced when ions are
reunited with electrons.  Give a numerical comparison of the magnitudes and
directions of the accelerations of the electrons and ions.
[You may need some data from
m4_ifelse(__lm_series,1,[:page \pageref{datatable}:],[:appendix \ref{dataappendix}, p.~\pageref{dataappendix}:])%
.]%
\answercheck
