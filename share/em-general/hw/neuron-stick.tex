The figure shows a neuron, which is the type of cell your
        nerves are made of.  Neurons serve to transmit sensory
        information to the brain, and commands from the brain to the
        muscles.  All this data is transmitted electrically, but
        even when the cell is resting and not transmitting any
        information, there is a layer of negative electrical charge
        on the inside of the cell membrane, and a layer of positive
        charge just outside it.  This charge is in the form of
        various ions dissolved in the interior and exterior fluids. 
        Why would the negative charge remain plastered against the
        inside surface of the membrane, and likewise why doesn't the
        positive charge wander away from the outside surface?
