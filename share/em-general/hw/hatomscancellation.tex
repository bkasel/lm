If you put two hydrogen atoms near each other, they will
feel an attractive force, and they will pull together to
form a molecule. (Molecules consisting of two hydrogen
atoms are the normal form of hydrogen gas.) How is this
possible, since each is
electrically neutral? Shouldn't the attractive and
repulsive forces all cancel out exactly? Use the raisin
cookie model. (Students who have taken chemistry often try
to use fancier models to explain this, but if you can't
explain it using a simple model, you probably don't
understand the fancy model as well as you thought you did!)
It's not so easy to prove that the force should actually be
attractive rather than repulsive, so just concentrate on
explaining why it doesn't necessarily have to vanish
completely.
