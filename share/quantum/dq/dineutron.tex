\begin{dq}
Neutrons attract each other via the strong nuclear force,
so according to classical physics it should be possible to
form nuclei out of clusters of two or more neutrons, with no
protons at all. Experimental searches, however, have failed
to turn up evidence of a stable two-neutron system
(dineutron) or larger stable clusters. These systems are apparently
not just unstable in the sense of being able to beta decay but
unstable in the sense that they don't hold together at all.
Explain based on
quantum physics why a dineutron might spontaneously fly apart.
\end{dq}
