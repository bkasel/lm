(a) Consider the set of vectors in two dimensions. This set P is a
vector space, and can be visualized as a plane, with each vector being
like an arrow that extends from the origin to a particular point. Now
consider the line $\ell$ defined by the equation $y=x$ in Cartesian
coordinates, and the ray $r$ defined by $y=x$ with $x\ge0$. Sketch
$\ell$ and $r$.  If we consider $\ell$ and $r$ as subsets of the
arrows in P, is $\ell$ a vector space? Is $r$?\hwendpart
%
(b) Consider the set C of angles $0\le\theta<2\pi$. Define addition on
C by adding the angles and then, if necessary, bringing the result
back into the required range. For example, if $x=\pi$ and $y=3\pi/2$,
then $x+y=\pi/2$.  Thus if we visualize C as a circle, every point on
the circle has a single number to represent it, not multiple
representations such as $\pi/2$ \emph{and} $5\pi/2$.  Suppose we want
to make C into a vector space over the real numbers, so that elements
of C are the vectors, while a scalar $\alpha$ can be \emph{any} real
number, not just a number from $0$ to $2\pi$. Then for example if
$\alpha=2$ is a scalar and $v=\pi$ is a vector, then $\alpha v=0$.
Find an example to prove that C is not a vector space, because it
violates the distributive property $\alpha(v+w)=\alpha v+\alpha
w$.\hwendpart
<% hw_solution %>
