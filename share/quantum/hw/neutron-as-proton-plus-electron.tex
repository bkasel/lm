The electron, proton, and neutron were discovered, respectively, in 1897, 1919, and 1932.
The neutron was late to the party, and 
some physicists felt that it was unnecessary to consider it as fundamental.
Maybe it could be explained
as simply a proton with an electron trapped inside it. The charges would cancel out,
giving the composite particle the correct neutral charge, and the masses at least
approximately made sense (a neutron is heavier than a proton). (a) Given that the
diameter of a proton is on the order of $10^{-15}\ \munit$, use the Heisenberg
uncertainty principle to estimate the trapped electron's minimum momentum.\answercheck\hwendpart
(b) Find the electron's minimum kinetic energy.\answercheck\hwendpart
(c) Show via $E=mc^2$ that the proposed explanation fails, because the
contribution to the neutron's mass from the electron's kinetic energy would be
many orders of magnitude too large.
