 The wavefunction of the electron in the ground state
of a hydrogen atom, shown in the top left of figure \figref{hydrogen-three-states} on p.~\pageref{fig:hydrogen-three-states}, is
\begin{equation*}
            \Psi  =  \pi^{-1/2} a^{-3/2} e^{-r/a}\eqquad,
\end{equation*}
where $r$ is the distance from the proton, and $a=\hbar^2/kme^2=5.3\times10^{-11}\ \munit$
is a constant that sets the size of the wave. The figure doesn't show the proton; let's take the proton to be
a sphere with a radius of $b=0.5$ fm.\\
(a) Reproduce figure \figref{hydrogen-three-states} in a rough sketch, and indicate, relative to the size of your sketch,
some idea of how big $a$ and $b$ are.\hwendpart
(b) Calculate symbolically, without plugging in numbers, the
probability that at any moment, the electron is inside the
proton.  [Hint: Does it matter if you plug in $r=0$ or
$r=b$ in the equation for the wavefunction?]\answercheck\hwendpart
(c) Calculate the probability numerically.\answercheck\hwendpart
(d) Based on the equation for the wavefunction, is it valid
to think of a hydrogen atom as having a finite size? Can $a$
be interpreted as the size of the atom, beyond which there
is nothing? Or is there any limit on how far the electron
can be from the proton?\hwendpart
