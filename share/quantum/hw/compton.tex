An electron is initially at rest.
A photon collides with the electron and rebounds from the
collision at 180 degrees, i.e., going back along the path on
which it came. The rebounding photon has a different energy,
and therefore a different frequency and wavelength. Show
that, based on conservation of energy and momentum, the
difference between the photon's initial and final wavelengths
must be $2h/mc$, where $m$ is the mass of the
electron. The experimental verification of this type of
``pool-ball'' behavior by Arthur Compton in 1923 was taken
as definitive proof of the particle nature of light. Note that
we're not making any nonrelativistic approximations. To keep the
algebra simple, you should use natural units --- in fact, it's a good
idea to use even-more-natural-than-natural units, in which we have
not just $c=1$ but also $h=1$, and $m=1$ for the mass of the electron.
You'll also probably want to use the relativistic relationship $E^2-p^2=m^2$,
which becomes $E^2-p^2=1$ for the energy and momentum of the electron in
these units.
