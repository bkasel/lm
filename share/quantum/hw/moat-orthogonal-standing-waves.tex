In example \ref{eg:moat-orthogonal-traveling-waves} on p.~\pageref{eg:moat-orthogonal-traveling-waves},
we defined wavefunctions for the traveling-wave solutions to the Schr\"odinger equation
in a ``quantum moat,'' and calculated the inner product
$\langle \text{ccw} | \text{cw} \rangle=0$ to verify that the counterclockwise and clockwise
traveling waves were orthogonal, as must be the case for distinguishable states.
Let's now define standing-wave versions $|\text{c}\rangle = \cos\theta$
and $|\text{s}\rangle = \sin\theta$. Verify by direct calculation that 
$\langle \text{c} | \text{s} \rangle=0$.
\hwremark{Since phases are undetectable in quantum mechanics, it may seem strange
that we can tell the difference between $|\text{c}\rangle$ and $|\text{s}\rangle$,
which differ only in their phase. But it is only absolute phases that are unobservable,
not relative ones. In the double-slit experiment, for example, the interference pattern
is an observable effect of the relative phases of the parts of the waves coming through
the two slits. In the present example, $|\text{c}\rangle$ and $|\text{s}\rangle$ would
also be distinguishable based on their probability distributions, e.g., $|\text{s}\rangle$
has zero probability of being detected at $\theta=0$.
}
