  Microscopic circuits are etched on the surface of a silicon chip. The equivalent of a wire
  in such an integrated circuit is called a ``trace.'' We consider the case
  where the trace is narrow enough to make quantum effects relevant, and we treat
  an electron inside the trace using the two-dimensional Schr\"odinger equation.
  We describe the trace as an infinite strip running parallel to the $x$ axis, extending
  from $y=0$ to $y=b$. The potential is
  \begin{equation*}
    U = \begin{cases}
      0, & 0<y< b \\
      +\infty, & y\le 0 \text{\ or\ } y\ge b \\
    \end{cases}
  \end{equation*}
 For convenience of notation, let $a=\pi/b$. Consider the following wavefunctions:
  \begin{align*}
    \Psi_1 &= e^{i(-kx-\omega t)}\sin ay \\
    \Psi_2 &= e^{-i\omega t}\sin kx\:e^{ay} \\
    \Psi_3 &= e^{-i\omega t}e^{kx}\:\sin ay \\
    \Psi_4 &= e^{-i\omega t}(\sin 2ax\:\sin ay+\sin ax\:\sin 2ay)
  \end{align*}
  The symbols $\omega$ and $k$ stand for real constants.
  Identify the wavefunctions that have the following properties. Exactly one of the wavefunctions
  has each property. Explain all answers.\\
  (a) cannot be a solution of the Schr\"odinger equation for this potential\\
  (b) is a traveling wave solution\\
  (c) is a solution that could represent the case where $0<y<b$ is classically forbidden\\
  (d) is not separable

