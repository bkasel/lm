As you might have guessed from the equations given in problem \ref{hw:ho-orthogonal},
the $m$th excited state of the one-dimensional quantum harmonic oscillator has
a wavefunction of the form
\begin{equation*}
  \Psi_m(x)=H_m(x)e^{-x^2/2}.
\end{equation*}
Here $H_m$ is a polynomial of order $m$, and $H_m$ is an even function if $m$ is
even, odd if $m$ is odd. Given these assumptions, it is possible to find $\Psi_2$
simply from the requirement that it be orthogonal to $\Psi_0$ and $\Psi_1$, without
having to solve the Schr\"{o}dinger equation. Find $H_2$ by this method. (Don't worry
about normalization or phase.)
Hint: First use integration by parts to relate the integral
$\int_{-\infty}^{\infty}x^m e^{-x^2}\der x$ to the corresponding integral
for $m-2$.\answercheck
