In the photoelectric effect, electrons are observed with
virtually no time delay ($\sim10$ ns), even when the light
source is very weak. (A weak light source does however only
produce a small number of ejected electrons.) The purpose of
this problem is to show that the lack of a significant time
delay contradicted the classical wave theory of light, so
throughout this problem you should put yourself in the shoes
of a classical physicist and pretend you don't know about
photons at all. At that time, it was thought that the
electron might have a radius on the order of $10^{-15}$ m. 
(Recent experiments have shown that if the electron has any
finite size at all, it is far smaller.)\hwendpart
(a) Estimate the power that would be soaked up by a single
electron in a beam of light with an intensity of 1 $\zu{mW}/\zu{m}^2$.\answercheck\hwendpart
(b) The energy, $E_s$, required for the electron to escape
through the surface of the cathode is on the order of 
$10^{-19}$ J. Find how long it would take the electron to
absorb this amount of energy, and explain why your result
constitutes strong evidence that there is something wrong
with the classical theory.\answercheck
