(a) A nuclear physicist is studying a nuclear reaction
caused in an accelerator experiment, with a beam of ions
from the accelerator striking a thin metal foil and causing
nuclear reactions when a nucleus from one of the beam ions
happens to hit one of the nuclei in the target.  After the
experiment has been running for a few hours, a few billion
radioactive atoms have been produced, embedded in the
target.  She does not know what nuclei are being produced,
but she suspects they are an isotope of some heavy element
such as Pb, Bi, Fr or U. Following one such experiment,
she takes the target foil out of the accelerator, sticks it
in front of a detector, measures the activity every 5 min,
and makes a graph (figure).  The isotopes she thinks may
have been produced are:

\begin{tabular}{ll}
isotope         & half-life (minutes)\\
$^{211}\zu{Pb}$ & 36.1\\
$^{214}\zu{Pb}$ & 26.8\\
$^{214}\zu{Bi}$ & 19.7\\
$^{223}\zu{Fr}$ & 21.8\\
$^{239}\zu{U}$ & 23.5\\
\end{tabular}

\noindent Which one is it?\hwendpart
(b) Having decided that the original experimental conditions
produced one specific isotope, she now tries using beams of
ions traveling at several different speeds, which may cause
different reactions.  The following table gives the activity
of the target 10, 20 and 30 minutes after the end of the
experiment, for three different ion speeds.

\begin{tabular}{llll}
& \multicolumn{3}{l}{activity (millions of decays/s) after\ldots}\\
&        10 min        & 20 min & 30 min\\
first ion speed                & 1.933        & 0.832        & 0.382 \\
second ion speed        & 1.200        & 0.545        & 0.248\\
third ion speed                & 7.211        & 1.296        & 0.248
\end{tabular}

\noindent Since such a large number of decays is being counted, assume
that the data are only inaccurate due to rounding off when
writing down the table.  Which are consistent with the
production of a single isotope, and which imply that more
than one isotope was being created?
