All helium on earth is from the decay of naturally occurring heavy radioactive elements
 such as uranium. Each alpha particle that is emitted ends up claiming two electrons,
 which makes it a helium atom. If the original $^{238}\zu{U}$ atom is in solid
 rock (as opposed to the earth's molten regions), the He atoms are unable to
 diffuse out of the rock. This problem involves dating a rock using the known
 decay properties of uranium 238. Suppose a geologist finds a sample of hardened lava,
 melts it in a furnace, and finds that it contains 1230 mg of uranium
 and 2.3 mg of helium.  $^{238}\zu{U}$ decays by alpha emission, with a half-life
 of $4.5\times10^9$ years.  The subsequent chain of alpha and electron (beta) decays involves
 much shorter half-lives, and terminates in the 
stable nucleus $^{206}\zu{Pb}$.  Almost all natural
 uranium is $^{238}\zu{U}$, and the chemical composition
 of this rock indicates that there were
 no decay chains involved other than that of $^{238}\zu{U}$.\hwendpart
(a) How many alphas are emitted per decay chain? [Hint: Use conservation of mass.]\hwendpart
(b) How many electrons are emitted per decay chain?
  [Hint: Use conservation of charge.]\hwendpart
(c) How long has it been since the lava originally hardened?\answercheck
