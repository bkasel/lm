\index{duality!wave-particle}\index{wave-particle duality}

How can light be both a particle and a wave? We are now
ready to resolve this seeming contradiction. Often in
science when something seems paradoxical, it's because we
(1) don't define our terms carefully, or (2) don't test our
ideas against any specific real-world situation. Let's
define particles and waves as follows:

\begin{itemize}
\item \index{wave!definition of}Waves exhibit superposition, and
specifically interference phenomena.
\item \index{particle!definition of}Particles can only exist in
whole numbers, not fractions.
\end{itemize}

As a real-world check on our philosophizing, there is one
particular experiment that works perfectly. We set up a
double-slit interference experiment that we know will
produce a diffraction pattern if light is an honest-to-goodness
wave, but we detect the light with a detector that is
capable of sensing individual photons, e.g., a digital
camera. To make it possible to pick out individual dots due to
individual photons, we must use filters to cut down the
intensity of the light to a very low level, just as in the
photos by Prof. Page on p.~\pageref{lymanpage}. The whole thing is
sealed inside a light-tight box. The results are shown in
figure \figref{ccd-diffraction}. (In fact, the similar
figures in on page \pageref{lymanpage}
 are simply cutouts from these figures.)

<%
  fig(
    'ccd-diffraction',
    %q{%
      Wave interference patterns photographed by Prof. Lyman
       Page with a digital camera. Laser light with a single well-defined wavelength
       passed through a series of absorbers to cut down its intensity, then through a
       set of slits to produce interference, and finally into
       a digital camera chip. (A triple slit was actually
       used, but for conceptual simplicity we discuss the results
       in the main text as if it was a
       double slit.) In panel 2 the intensity has been
       reduced relative to 1, and even more so for panel 3.
    },
    {
      'width'=>'wide',
      'sidecaption'=>true
    }
  )
%>

Neither the pure wave theory nor the pure particle theory
can explain the results. If light was only a particle and
not a wave, there would be no interference effect. The
result of the experiment would be like firing a hail of
bullets through a double slit, \figref{double-slit-bullets}. Only two spots directly
behind the slits would be hit.

If, on the other hand, light was only a wave and not a
particle, we would get the same kind of diffraction pattern
that would happen with a water wave, \figref{interference}. There would be no
discrete dots in the photo, only a  diffraction pattern that
shaded smoothly between light and dark.

Applying the definitions to this experiment, light must be
both a particle and a wave. It is a wave because it exhibits
interference effects. At the same time, the fact that the
photographs contain discrete dots is a direct demonstration
that light refuses to be split into units of less than a
single photon. There can only be whole numbers of photons:
four photons in figure \figref{ccd-diffraction}/3, for example.

<% marg(100) %>
<%
  fig(
    'double-slit-bullets',
    %q{Bullets pass through a double slit. }
  )
%>
\spacebetweenfigs
<%
  fig(
    'interference',
    %q{A water wave passes through a double slit.}
  )
%>
\spacebetweenfigs
<%
  fig(
    'skier',
    %q{A single photon can go through both slits.}
  )
%>
<% end_marg %>

<% begin_sec("A wrong interpretation: photons interfering with each other") %>
One possible interpretation of wave-particle duality that
occurred to physicists early in the game was that perhaps
the interference effects came from photons interacting with
each other. By analogy, a water wave consists of moving
water molecules, and interference of water waves results
ultimately from all the mutual pushes and pulls of the
molecules. This interpretation has been conclusively disproved by
forming interference patterns with light so dim that no more than one photon
is in flight at a time. In figure \figref{ccd-diffraction}/3, for example, the
intensity of the light has been cut down so much by the
absorbers that if it was in the open, the average separation
between photons would be on the order of a kilometer! Although
most light sources tend to emit photons in bunches, experiments
have been done with light sources that really do emit single
photons at wide time intervals, and the same type of interference
pattern is observed, showing that a single photon can interfere with
\emph{itself}.

\subsubsection{The concept of a photon's
 \index{path of a photon undefined}path is undefined.}
If a single photon can demonstrate double-slit interference,
then which slit did it pass through? The unavoidable answer
must be that it passes through both! This might not seem so
strange if we think of the photon as a wave, but it is
highly counterintuitive if we try to visualize it as a
particle. The moral is that we should not think in terms of
the path of a photon. Like the fully human and fully divine
Jesus of Christian theology, a photon is supposed to be
100\% wave and 100\% particle. If a photon had a well
defined path, then it would not demonstrate wave superposition
and interference effects, contradicting its wave nature. (In
sec.~__bare_subsection_or_section(uncertainty-principle)
we will discuss the Heisenberg uncertainty
principle, which gives a numerical way of approaching this issue.)

<% end_sec() %>
<% begin_sec("The probability interpretation") %>
\index{wave-particle duality!probability interpretation of}\index{probability interpretation}

The correct interpretation of wave-particle duality is
suggested by the random nature of the experiment we've been
discussing: even though every photon wave/particle is
prepared and released in the same way, the location at which
it is eventually detected by the digital camera is different
every time. The idea of the probability interpretation of
wave-particle duality is that the location of the photon-particle
is random, but the probability that it is in a certain
location is higher where the photon-wave's amplitude is greater.

More specifically, the probability distribution of the
particle must be proportional to the \emph{square} of
the wave's amplitude,
\begin{equation*}
    (\text{probability distribution}) \propto (\text{amplitude})^2\eqquad.
\end{equation*}
This follows from the correspondence principle and from the
fact that a wave's energy density is proportional to the
square of its amplitude. If we run the double-slit
experiment for a long enough time, the pattern of dots fills
in and becomes very smooth as would have been expected in
classical physics. To preserve the correspondence between
classical and quantum physics, the amount of energy
deposited in a given region of the picture over the long run
must be proportional to the square of the wave's amplitude.
The amount of energy deposited in a certain area depends on
the number of photons picked up, which is proportional to
the probability of finding any given photon there.


\begin{eg}{A microwave oven}\label{eg:carrot}
\egquestion The figure shows two-dimensional (top) and
one-dimensional (bottom) representations of the standing
wave inside a microwave oven. Gray represents zero field,
and white and black signify the strongest fields, with white
being a field that is in the opposite direction compared to
black. Compare the probabilities of detecting a microwave
photon at points A, B, and C.

\eganswer A and C are both extremes of the wave, so the
probabilities of detecting a photon at A and C are equal.
It doesn't matter that we have represented C as negative
and A as positive, because it is the square of the amplitude
that is relevant. The amplitude at B is about 1/2 as much
as the others, so the probability of detecting a photon
there is about 1/4 as much.
\end{eg}

<% marg(50) %>
<%
  fig(
    'carrot',
    %q{Example \ref{eg:carrot}.}
  )
%>
<% end_marg %>

\startdqs

\begin{dq}
Referring back to the example of the carrot in the
microwave oven, show that it would be nonsensical to have
probability be proportional to the field itself, rather than
the square of the field.
\end{dq}

\begin{dq}
Einstein did not try to reconcile the wave and particle
theories of light, and did not say much about their apparent
inconsistency. Einstein basically visualized a beam of light
as a stream of bullets coming from a machine gun. In the
photoelectric effect, a photon ``bullet'' would only hit one
atom, just as a real bullet would only hit one person.
Suppose someone reading his 1905 paper wanted to interpret
it by saying that Einstein's so-called particles of light
are simply short wave-trains that only occupy a small
region of space.  Comparing the wavelength of visible light
(a few hundred nm) to the size of an atom (on the order of
0.1 nm), explain why this poses a difficulty for reconciling
the particle and wave theories.
\end{dq}

\begin{dq}
Can a white photon exist?
\end{dq}

\begin{dq}\label{dq:cover-one-slit}
In double-slit diffraction of photons, would you get the
same pattern of dots on the digital camera image if you
covered one slit? Why should it matter whether you give the
photon two choices or only one?
\end{dq}

<% end_sec() %>
