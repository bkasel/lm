A positively charged particle is released from rest at the origin at $t=0$,
        in a region of vacuum through which 
        an electromagnetic wave is passing. The particle accelerates in response to
        the wave.
        In this region of space, the wave varies as $\vc{E}=\hat{\vc{x}}\tilde{E}\sin\omega t$,
        $\vc{B}=\hat{\vc{y}}\tilde{B}\sin\omega t$, and we assume that the particle has
        a relatively large value of $m/q$, so that its response to the wave is sluggish,
        and it never ends up moving at any speed comparable to the speed of light. Therefore
        we don't have to worry about the spatial variation of the wave; we can just imagine
        that these are uniform fields imposed by some external mechanism on this region of
        space.\\
        (a) Find the particle's coordinates as functions of time.\answercheck\hwendpart
        (b) Show that the motion is confined to $-z_{max}\leq z \leq z_{max}$,
        where $z_{max} = 1.101\left(q^2\tilde{E}\tilde{B}/m^2\omega^3\right)$.
