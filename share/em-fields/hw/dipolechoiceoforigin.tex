The definition of the dipole moment, $\vc{D}=\sum q_i \vc{r}_i$,
        involves the vector $\vc{r}_i$ stretching from the origin of our coordinate
        system out to the charge $q_i$. There are clearly cases where this
        causes the dipole moment to be dependent on the choice of coordinate
        system. For instance, if there is only one charge, then we could make
        the dipole moment equal zero if we chose the origin to be right on top
        of the charge, or nonzero if we put the origin somewhere else.\hwendpart
        (a) Make up a numerical example with two charges of equal magnitude
        and opposite sign. Compute the dipole moment using two different coordinate
        systems that are oriented the same way, but differ in the choice of origin.
        Comment on the result.\hwendpart
        (b) Generalize the result of part a to any pair of charges with equal
        magnitude and opposite sign. This is supposed to be a proof for \emph{any}
        arrangement of the two charges, so don't assume any numbers.\hwendpart
        (c) Generalize further, to $n$ charges.
