The figure shows a nested pair of circular wire loops
        used to create magnetic fields. (The twisting of the leads
        is a practical trick for reducing the magnetic fields they
        contribute, so the fields are very nearly what we would
        expect for an ideal circular current loop.) The coordinate
        system below is to make it easier to discuss directions in
        space. One loop is in the $y-z$ plane, the other in the $x-y$
        plane. Each of the loops has a radius of 1.0 cm, and carries
        1.0 A in the direction indicated by the arrow.\hwendpart
        (a) Calculate
        the magnetic field that would be produced by \emph{one} such
        loop, at its center. \answercheck\hwendpart
        (b) Describe the direction of the magnetic field that would
        be produced, at its center, by the loop in the $x-y$ plane alone.\hwendpart
        (c) Do the same for the other loop.\hwendpart
        (d) Calculate the magnitude of the magnetic field produced
        by the two loops in combination, at their common center.
        Describe its direction.\answercheck
