Suppose a charged particle is moving through a region of space in
which there is an electric field perpendicular to its velocity vector,
and also a magnetic field perpendicular to both the particle's
velocity vector and the electric field. Show that there will be one
particular velocity at which the particle can be moving that results
in a total force of zero on it; this requires that you analyze both
the magnitudes and the directions of the forces compared to one
another. Relate this velocity to the magnitudes of the electric and
magnetic fields. (Such an arrangement, called a velocity filter, is
one way of determining the speed of an unknown particle.)
