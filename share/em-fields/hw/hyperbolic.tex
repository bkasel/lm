The figure shows a vacuum chamber surrounded by four metal electrodes shaped
        like hyperbolas. (Yes, physicists do sometimes ask their university machine
        shops for things machined in mathematical shapes like this. They have to be made
        on computer-controlled mills.) We assume that the
        electrodes extend far into and out of the page along the unseen $z$ axis, so
        that by symmetry, the electric fields are the same for all $z$. The problem is
        therefore effectively two-dimensional. Two of the electrodes are at voltage $+V_\zu{o}$,
        and the other two at $-V_\zu{o}$, as shown. The equations of the hyperbolic surfaces
        are $|xy|=b^2$, where $b$ is a constant. (We can interpret $b$ as giving the locations
        $x=\pm b$, $y=\pm b$ of the four points on the surfaces that are closest to the
        central axis.) There is no obvious, pedestrian way to
        determine the field or potential in the central vacuum region, but there's a trick that
        works: with a little mathematical insight, we see that the potential $V=V_\zu{o}b^{-2}xy$
        is consistent with all the given information. (Mathematicians could prove that this
        solution was unique, but a physicist knows it on physical grounds: if there were two
        different solutions, there would be no physical way for the system to decide which
        one to do!)\\
m4_ifelse(__problems,1,[:%
(a) Find the field in the vacuum region.\answercheck
:],[:%
(a) Use the techniques of subsection \ref{subsec:evthreed}
        to find the field in the vacuum region.\answercheck
:])
(b) Sketch the field as a ``sea of arrows.''
