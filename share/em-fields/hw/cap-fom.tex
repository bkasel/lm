A capacitor has parallel plates of area $A$, separated by a distance $h$. If there is a vacuum between the plates, then
Gauss's law gives $E=4\pi k\sigma=4\pi kq/A$ for the field between the plates, and combining this with $E=V/h$, we find
$C=q/V=(1/4\pi k)A/h$. (a) Generalize this derivation to the case where there is a dielectric between the plates.
(b) Suppose we have a list of possible materials we could choose as dielectrics, and we wish to construct a capacitor
that will have the highest possible energy density, $U_e/v$, where $v$ is the volume. For each dielectric, we know its
permittivity $\epsilon$, and also the maximum electric field $E$ it can sustain without breaking down and allowing sparks to cross between
the plates. Write the maximum energy density in terms of these two variables, and determine a figure of merit that could be used to
decide which material would be the best choice.
