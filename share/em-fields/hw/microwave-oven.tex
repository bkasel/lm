You know how a microwave gets some parts of your food hot, but leaves other parts cold?
Suppose someone is trying to convince you of the following explanation for this fact:
\emph{The microwaves inside the oven form a stationary wave pattern, like the vibrations
of a clothesline or a guitar string. The food is heated unevenly because
the wave crests are a certain
distance apart, and the parts of the food that get heated the most are the ones where there's
a crest in the wave pattern.} Use the wavelength scale in figure \figref{em-spectrum} on
page \pageref{fig:em-spectrum} as a way of checking numerically
whether this is a reasonable explanation.
