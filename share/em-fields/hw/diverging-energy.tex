To do this problem, you need to understand how to do
volume integrals in cylindrical and spherical coordinates.
(a) Show that if you try to integrate the energy stored in
the field of a long, straight wire, the resulting energy per
unit length diverges both at $r\rightarrow 0$ and $r\rightarrow \infty$.
Taken at face value, this would imply that a certain
real-life process, the initiation of a current in a wire,
would be impossible, because it would require changing from
a state of zero magnetic energy to a state of infinite
magnetic energy. (b) Explain why the infinities at
$r\rightarrow 0$ and $r\rightarrow \infty$ don't really happen
in a realistic situation. (c) Show that the electric energy
of a point charge diverges at $r\rightarrow 0$, but not at
$r\rightarrow \infty$.

A remark regarding part (c): Nature does seem to supply us
with particles that are charged and pointlike, e.g., the
electron, but one could argue that the infinite energy is
not really a problem, because an electron traveling around
and doing things neither gains nor loses infinite energy;
only an infinite \emph{change} in potential energy would be
physically troublesome. However, there are real-life
processes that create and destroy pointlike charged
particles, e.g., the annihilation of an electron and
antielectron with the emission of two gamma rays. Physicists
have, in fact, been struggling with infinities like this
since about 1950, and the issue is far from resolved. Some
theorists propose that apparently pointlike particles are
actually not pointlike: close up, an electron might be like
a little circular loop of string.
