One model of the hydrogen atom has the electron circling
around the proton at a speed of $2.2\times10^6$ m/s, in an
orbit with a radius of 0.05 nm. (Although the electron and
proton really orbit around their common center of mass, the
center of mass is very close to the proton, since it is 2000
times more massive. For this problem, assume the proton is
stationary.) In homework problem \ref{hw:lpcurrent}, p.~\pageref{hw:lpcurrent}, you
calculated the electric current created.\hwendpart
(a) Now estimate the magnetic field created at the center
of the atom by the electron. We are treating the circling
electron as a current loop, even though it's only a single particle.\answercheck\hwendpart
(b) Does the proton experience a nonzero force from the
electron's magnetic field? Explain.\hwendpart
(c) Does the electron experience a magnetic field from
the proton? Explain.\hwendpart
(d) Does the electron experience a magnetic field created by
its own current? Explain.\hwendpart
(e) Is there an electric force acting between the proton
and electron? If so, calculate it.\answercheck\hwendpart
(f) Is there a gravitational force acting between the proton
and electron? If so, calculate it.\hwendpart
(g) An inward force is required to keep the electron in its
orbit -- otherwise it would obey Newton's first law and go
straight, leaving the atom. Based on your answers to the
previous parts, which force or forces (electric, magnetic
and gravitational) contributes significantly to this inward force?\hwendpart
{[Based on a problem by Arnold Arons.]}

