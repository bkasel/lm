Albert Einstein wrote, ``What really interests me is whether God had any
choice in the creation of the world.'' What he meant by this is that if you randomly
try to imagine a set of rules --- the laws of physics --- by which the universe
works, you'll almost certainly come up with rules that don't make sense. For instance,
we've seen that if you tried to omit magnetism from the laws of physics, electrical
interactions wouldn't make sense as seen by observers in different frames of reference;
magnetism is required by relativity.

The magnetic interaction rules in figure
\figref{magtwobody} are consistent with the time-reversal symmetry of the laws of
physics. In other words, the rules still work correctly if you reverse the
particles' directions of motion. Now you get to play God (and fail).
Suppose you're going to make an alternative version of the laws of physics
by reversing the direction of motion of only \emph{one} of the eight particles.
You have eight choices, and each of these eight choices would result in a new
set of physical laws. We can imagine eight alternate universes, each governed
by one of these eight sets. Prove that \emph{all} of these modified sets of
physical laws are impossible, either because the are self-contradictory, or
because they violate time-reversal symmetry.
