m4_ifelse(__problems,1,[:%
          (a) A rod of length $L$ is uniformly charged with charge $Q$.
          It can be shown by integration that the field at a point lying in the midplane of the rod
          at a distance $R$ is  $E =  k\lambda L/\left[R^2\sqrt{1+ L^2/4 R^2}\right]$, where $\lambda$
          is the charge per unit length.
        :],[:%
          (a) Example \ref{eg:chargedrodside} on page \pageref{eg:chargedrodside}
        gives the field of a charged rod in its midplane.
        :])%
        Starting from this result, take the limit as the length of the rod approaches infinity.
        Note that $\lambda$ is not changing, so as $L$ gets bigger, the total charge $Q$ increases.
        <% hw_answer %>\hwendpart
        m4_ifelse(__problems,1,[:
          (b) It can be shown that
        :],[:%
          (b) In the text, I have shown (by several different methods) that
        :])%
        the field of an infinite,
        uniformly charged plane is $2\pi k\sigma$. Now you're going to
        rederive the same result by a different method.
        Suppose that it is the $x-y$ plane that is charged, and we
        want to find the field at the point $(0,0,z)$. (Since the plane
        is infinite, there is no loss of generality in assuming
        $x=0$ and $y=0$.)
        Imagine that we slice the plane into an infinite number of
        straight strips parallel to the $y$ axis.
        Each strip has infinitesimal width $\der x$, and extends
        from $x$ to $x+\der x$. The contribution any one of these
        strips to the field at our point has a magnitude which can
        be found from part a.
        By vector addition, prove the desired result for the field of
        the plane of charge. 
