The figure shows a simplified example of a device called a sector mass
spectrometer. In an oven near the bottom, positively ionized atoms are
produced. For simplicity, we assume that the atoms are all singly ionized.
They may have different masses, however, and the goal is to separate them according to these
masses. In the example shown in the figure, there are two different masses present.
The reason this is called a ``sector'' mass spectrometer is that it contains
two regions of uniform fields.

In the first sector, between the two long capacitor
plates, there is an electric field $E$ in the $x$ direction. Superimposed on
this is a uniform magnetic field $B$ in the negative $z$ direction (into the page).
As analyzed in problem \ref{hw:velocityfilter}, these fields are chosen so that
ions at a certain velocity $v$ are not deflected. You will need the result of
that problem in order to do this problem. Only the ions with the correct
velocity make it out through the slits at the upper end of the capacitor.

In the second sector, at the top, there is no electric field, only a magnetic
field, which we assume for simplicity to have the same magnitude and direction
as in the first sector. This causes the beam to bend into a semicircular arc
and hit a detector. In the first such spectrometers, this detector was simply
some photographic film, whereas in modern ones it would probably be a silicon
chip similar to the sensor of a camera.

The diameter $h$ of the semicircle depends on the mass $m$ of the ion.
The quantity $\Delta h/\Delta m$ tells us how good the spectrometer is at
separating similar masses.

\noindent (a) Express $\Delta h/\Delta m$ in terms
of $E$, $B$, and $e$, eliminating $v$ (which we can neither control nor measure
directly).\answercheck\hwendpart
(b) Show that the units of your answer make sense.\hwendpart
(c) You will
have found that increasing $E$ makes the spectrometer more sensitive, while
increasing $B$ makes it less so. Explain physically why this is so. What stops
us from getting an arbitrarily large sensitivity simply by making $B$ small
enough?

\hwremark{This design makes inefficient use of the ion source's intensity,
because any ions with the wrong velocity are wasted. For this reason, real-world
spectrometers of this type include complicated focusing elements.}
