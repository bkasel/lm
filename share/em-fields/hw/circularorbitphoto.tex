m4_ifelse(__problems,1,[:%
        (a) In the photo,
              magnetic forces cause a beam of electrons to move in a circle.
              The beam is created in a vacuum tube, in which a small amount of
              hydrogen gas has been left. A few of the electrons strike hydrogen
              molecules, creating light and letting us see the beam. A magnetic
              field is produced by passing a current (meter) through the circular
              coils of wire in front of and behind the tube. In the bottom figure,
              with the magnetic field turned on, the force perpendicular to the
              electrons' direction of motion causes them to move in a circle.
:],[:%
        (a) In the photo of the vacuum tube apparatus in figure \figref{circular-orbit}
        on page \pageref{fig:circular-orbit},
:])%
        infer the direction of the magnetic field from
        the motion of the electron beam. (The answer is given in the answer to the self-check
        on that page.)\hwendpart
        (b) Based on your answer to part
        a, find the direction of the currents in the coils.\hwendpart
        (c) What
        direction are the electrons in the coils going? \hwendpart
        (d) Are the
        currents in the coils repelling the currents
        consisting of the beam inside the tube, or attracting them? Check your answer by comparing with
        the result of problem \ref{hw:twowiresrepel}.
