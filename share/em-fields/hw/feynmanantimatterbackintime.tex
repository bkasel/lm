Physicist Richard Feynman\index{Feynman, Richard} originated a new way of thinking about charge: a charge of a certain
type is equivalent to a charge of the opposite type that happens to be moving backward
in time! An electron moving backward in time is an antielectron --- a particle that has the same
mass as an electron, but whose charge is opposite. Likewise we have antiprotons, and antimatter
made from antiprotons and antielectrons. Antielectrons occur naturally everywhere around you due to
natural radioactive decay and radiation from outer space. A small number of antihydrogen atoms has even
been created in particle accelerators!

Show that, for each rule for magnetic interactions shown in \figref{magtwobody}, the rule is
still valid if you replace one of the charges with an opposite charge moving in the 
opposite direction (i.e., backward in time).
