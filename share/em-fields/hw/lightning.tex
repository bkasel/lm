In an electrical storm, the cloud and the ground act like   
a parallel-plate capacitor, which typically charges up due
to frictional electricity in collisions of ice particles in   
the cold upper atmosphere. Lightning occurs when the
magnitude of the electric field builds up to a critical
value, $E_c$, at which air is ionized.\hwendpart
(a) Treat the cloud as a flat square with sides of length
$L$. If it is at a height $h$ above the ground, find the
amount of energy released in the lightning strike.\answercheck\hwendpart
(b) Based on your answer from part a, which is more
dangerous, a lightning strike from a high-altitude cloud
or a low-altitude one?\hwendpart
(c) Make an order-of-magnitude estimate of the energy
released by a typical lightning bolt, assuming reasonable
values for its size and altitude.  $E_c$ is about $10^6$  N/C.

See problem \ref{hw:cosmic-ray-lightning} for a note on how recent research affects this estimate.
