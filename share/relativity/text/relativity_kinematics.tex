<% begin_sec("Time is not absolute",nil,'time-not-absolute') %>

m4_ifelse(__me,1,[: %
So far we've been discussing relativity according to Galileo and Newton, but there is also relativity according to Einstein.
:]) %
When Einstein first began to develop the theory of relativity, around 1905, the only real-world observations he
could draw on were ambiguous and indirect.
Today, the evidence is part of everyday life. For example,
every time you use a GPS receiver, \figref{gps-on-bike}, you're using Einstein's theory of relativity.
Somewhere between 1905 and today, technology became good enough to allow conceptually \emph{simple} experiments
that students in the early 20th century could only discuss in terms like ``Imagine that we could\ldots''
A good jumping-on point is 1971. In that year, J.C.~Hafele and R.E.~Keating\label{hafele-keating-discussed}
brought atomic clocks aboard commercial
airliners, \figref{hk-in-cabin}, and went around the world, once from east to west and once from west to east.
Hafele and Keating observed that there was a discrepancy between the times measured by the
traveling clocks and the times measured by similar clocks that stayed home at the U.S. Naval Observatory in Washington.
The east-going clock lost time, ending up off by $-59\pm10$ nanoseconds, while the west-going one gained $273\pm7$ ns.
<% marg(m4_ifelse(__me,1,100,77)) %>
<%
  fig(
    'gps-on-bike',
    %q{This Global Positioning System (GPS) system, running on a smartphone attached to a bike's handlebar,
       depends on Einstein's theory of relativity. Time flows at a different rate aboard a GPS satellite than
       it does on the bike, and the GPS software has to take this into account.}
  )
%>
\spacebetweenfigs
<%
  fig(
    'hk-in-cabin',
    %q{The clock took up two seats, and two tickets were bought for it under the name of ``Mr.~Clock.''}
  )
%>
m4_ifelse(__me,1,[: % --- occurs later in LM
\spacebetweenfigs
<%
  fig(
    'football-causality',
    %q{Newton's laws do not distinguish past from future. The football could travel in either direction while obeying Newton's laws.}
  )
%>
:]) %
<% end_marg %>

  <% begin_sec("The correspondence principle",nil,'relativity-correspondence-principle') %>
This establishes that time doesn't work the way Newton believed it did when he wrote that
``Absolute, true, and mathematical time, of itself, and from its own nature flows equably without regard to anything external\ldots''
We are used to thinking of time as absolute and universal, so it is
disturbing to find that it can flow at a different rate for observers in
different frames of reference. 
Nevertheless, the effects that Hafele and Keating observed were small.
This makes sense: Newton's laws have already been thoroughly tested by experiments under a wide variety of conditions,
so a new theory like relativity must agree with Newton's to a good approximation, within the Newtonian theory's
realm of applicability. This requirement of backward-compatibility is known as
the correspondence principle.\index{correspondence principle!for time dilation}\index{correspondence principle!defined}
  <% end_sec('relativity-correspondence-principle') %>

  <% begin_sec("Causality",nil,'causality') %>\label{causality-defined}
It's also reassuring that the effects on time were
small compared to the three-day lengths of the plane trips. There was therefore no
opportunity for paradoxical scenarios such as one in which the east-going experimenter arrived
back in Washington before he left and then convinced himself not to take the trip.
A theory that maintains this kind of orderly relationship between cause and effect is said to satisfy causality.\index{causality}

Causality is like a water-hungry front-yard lawn in Los Angeles: we know we want it, but it's not easy to explain why.
Even in plain old Newtonian physics, there is no clear distinction between past and future. In figure \figref{football-causality}, number 18
throws the football to number 25, and the ball obeys Newton's laws of motion. If we took a video of the pass and played
it backward, we would see the ball flying from 25 to 18, and Newton's laws would still be satisfied. Nevertheless, we have a strong
psychological impression that there is a forward arrow of time. I can remember what the stock market did last year, but I can't
remember what it will do next year. Joan of Arc's military victories against England caused the English to burn her at the stake;
it's hard to accept that Newton's laws provide an equally good description of a process in which her execution in 1431 caused her to win a battle in 1429.
There is no consensus at this point among physicists on the origin and significance of time's arrow, and for our present purposes
we don't need to solve this mystery. Instead, we merely note the empirical fact that, regardless of what causality really means and
where it really comes from, its behavior is consistent. Specifically, experiments show that if an observer
in a certain frame of reference observes that event A causes event B, then observers in other frames agree
that A causes B, not the other way around. This is merely a generalization about a large body of experimental results, not a logically necessary
assumption. If Keating had gone around the world and arrived back in Washington before he left, it would have disproved
this statement about causality.
  <% end_sec('causality') %>

m4_ifelse(__me,1,,[: % -- occurs earlier in Me
<% marg(300) %>
<%
  fig(
    'football-causality',
    %q{Newton's laws do not distinguish past from future. The football could travel in either direction while obeying Newton's laws.}
  )
%>
<% end_marg %>
:]) %

  <% begin_sec("Time distortion arising from motion and gravity",nil,'time-distortion') %>
Hafele and Keating were testing specific quantitative predictions of relativity, and they verified them to within
their experiment's error bars. Let's work backward instead, and
inspect the empirical results for clues as to how time works.

The two traveling clocks experienced effects in opposite directions,
and this suggests that the rate at which time flows depends on the motion
of the observer. The east-going clock was moving in the same direction as the earth's rotation, so its velocity
relative to the earth's center was greater than that of the clock that remained in Washington, while the west-going clock's velocity was
correspondingly reduced. The fact that the east-going clock fell behind, and the west-going one got ahead,
shows that the effect of motion is to make time go more slowly. This effect of motion on time was predicted by
Einstein in his original 1905 paper on relativity, written when he was 26.
<% marg(70) %>
<%
  fig(
    'hafele-keating-directions',
    %q{All three clocks are moving to the east. Even though the west-going plane is moving to the west relative to the air, the air is moving
           to the east due to the earth's rotation.}
  )
%>
<% end_marg %>

If this had been the only effect in the Hafele-Keating experiment, then we would have expected to see effects on the
two flying clocks that were equal in size. Making up some simple numbers to keep the arithmetic transparent, suppose that the earth rotates
from west to east at 1000 km/hr, and that the planes fly at 300 km/hr. Then the speed of the clock on the ground
is 1000 km/hr, the speed of the clock on the east-going plane is 1300 km/hr, and that of the west-going clock 700 km/hr.
Since the speeds of 700, 1000, and 1300 km/hr have equal spacing on either side of 1000, we would expect the discrepancies
of the moving clocks relative to the one in the lab to be equal in size but opposite in sign.

<%
  fig(   
    'iijima',
    %q{%
       A graph 
       showing the time difference between two atomic clocks. One clock was kept at Mitaka Observatory, at 58 m above sea level.
       The other was moved back and forth to a second observatory, Norikura Corona Station, at the peak of the Norikura volcano, 2876 m above sea level.
       The plateaus on the graph are data from the periods when the clocks were compared side by side at Mitaka. The difference between one plateau and the next
       shows a gravitational effect on the rate of flow of time, accumulated during the period when the mobile clock was at the top of Norikura.
       m4_ifelse(m4_eval(__me || __fac),1,[::],[:Cf.~problem \ref{hw:pound-rebka}, p.~\pageref{hw:pound-rebka}.:])
    },
    {
      'width'=>'wide','sidecaption'=>false
    }
  )
%>

In fact, the two effects are unequal in size: $-59$ ns and 273 ns. This
implies that there is a second effect involved, simply due to the planes' being up in the air.
This was verified more directly in a 1978 experiment by Iijima and Fujiwara, figure \figref{iijima}, in which identical atomic
clocks were kept at rest at the top and bottom of a mountain near Tokyo.
This experiment, unlike the Hafele-Keating one, isolates one effect on time, the gravitational one: time's
rate of flow increases with height in a gravitational field. Einstein didn't figure out
how to incorporate gravity into relativity until 1915, after much frustration and many false starts. The
simpler version of the theory without gravity is known as special relativity, the full version as general
relativity. We'll restrict ourselves to 
special m4_ifelse(m4_eval(__me || __fac),1,[:relativity:],[:relativity until __section_or_chapter(genrel):]), and
that means that what we want to
focus on right now is the distortion of time due to motion, not gravity.\label{iijima}


m4_ifelse(__me,1,[: % --- occurs later in SN
<% marg(0) %>
<%
  fig(
    'correspondence-dramatized',
    %q{The correspondence principle requires that the relativistic distortion of time become small for small velocities.}
  )
%>
<% end_marg %>
:]) %

We can now see in more detail how to apply the correspondence principle. The behavior of the three clocks in the
Hafele-Keating experiment shows that the amount of time distortion increases as the speed of the clock's motion
increases. Newton lived in an era when the fastest
mode of transportation was a galloping horse, and the best
pendulum clocks would accumulate errors of perhaps a minute over the course of several days.
A horse is much slower than a jet plane, so the
distortion of time would have had a relative size of only $\sim10^{-15}$ --- much smaller than the clocks were capable of detecting.
At the speed of a passenger jet, the effect is about $10^{-12}$,
and state-of-the-art atomic clocks in 1971 were capable of measuring that.
A GPS satellite travels much faster than a jet airplane, and the effect on the satellite
turns out to be $\sim10^{-10}$. The general idea here is that all physical laws are approximations, and
approximations aren't simply right or wrong in different situations. Approximations are better or worse
in different situations, and the question is whether a particular approximation is good enough in a given
situation to serve a particular purpose. The faster the motion, the worse the Newtonian approximation of
absolute time. Whether the approximation is good enough depends on what you're trying to accomplish.
The correspondence principle says that the approximation must have been good enough to explain
all the experiments done in the centuries before Einstein came up with relativity.

m4_ifelse(__me,0,[: % --- occurs earlier in Me.
<% marg(300) %>
<%
  fig(
    'correspondence-dramatized',
    %q{The correspondence principle requires that the relativistic distortion of time become small for small velocities.}
  )
%>
<% end_marg %>
:]) %


By the way, don't get an inflated idea of the importance of the Hafele-Keating
experiment. Special relativity had already been confirmed by a vast and varied body of experiments decades
before 1971. The only reason I'm giving such a prominent role to this experiment, which was actually more important as a test of
general relativity, is that it is conceptually very direct.
  <% end_sec('time-distortion') %> % Time distortion arising from motion and gravity

<% end_sec('time-not-absolute') %> % Time Is Not Absolute

<% begin_sec("Distortion of space and time",nil,'x-t-distortion') %>
  <% begin_sec("The Lorentz transformation",nil,'lorentz') %>
Relativity says that when two observers are in different frames of reference, each observer considers
the other one's perception of time to be distorted. We'll also
see that something similar happens to their observations of distances, so both space and
time are distorted.
What exactly is this distortion? How do we even conceptualize it?

The idea isn't really as radical as it might seem at first. We can visualize the structure of space
and time using a graph with position and time on its axes. These graphs are familiar by now, but
we're going to look at them in a slightly different way. Before, we used them to describe the motion
of objects. The grid underlying the graph was merely the stage on which the actors played their parts.
Now the background comes to the foreground: it's time and space themselves that we're studying.
We don't necessarily need to have a line or a curve drawn on top of the grid to represent a particular
object. We may, for example, just want to talk about events, depicted as points on the graph as in
figure \figref{joan-of-arc}. A distortion of the Cartesian grid underlying the graph can arise for
perfectly ordinary reasons that Isaac Newton would have readily accepted. For example, we can simply
change the units used to measure time and position, as in figure \figref{change-of-units}. 
<% marg(150) %>
<%
  fig(
    'joan-of-arc',
    %q{Two events are given as points on a graph of position versus time. Joan of Arc helps to restore Charles VII to the throne.
         At a later time and a different position, Joan of Arc is sentenced to death.}
  )
%>
\spacebetweenfigs
<%
  fig(
    'change-of-units',
    %q{A change of units distorts an $x$-$t$ graph. This graph depicts exactly the same events as figure \figref{joan-of-arc}.
          The only change is that the $x$ and $t$ coordinates are measured using different units, so the grid is compressed
          in $t$ and expanded in $x$.}
  )
%>
\spacebetweenfigs
<%
  fig(
    'change-of-units-convention',
    %q{A convention we'll use to represent a distortion of time and space.}
  )
%>
<% end_marg %>

We're going to
have quite a few examples of this type, so I'll adopt the convention shown in figure \figref{change-of-units-convention}
for depicting them. Figure \figref{change-of-units-convention} summarizes the relationship between figures
\figref{joan-of-arc} and \figref{change-of-units} in a more compact form. The gray rectangle represents the
original coordinate grid of figure \figref{joan-of-arc}, while the grid of black lines represents the new version
from figure \figref{change-of-units}. Omitting the grid from the gray rectangle makes the diagram easier
to decode visually.

Our goal of unraveling the mysteries of special relativity amounts to nothing more than finding out how to
draw a diagram like \figref{change-of-units-convention} in the case where the two different sets of coordinates represent
measurements of time and space made by two different observers, each in motion relative to the other.
Galileo and Newton thought they knew the answer to this question, but their answer turned out to be
only approximately right. To avoid repeating the same mistakes, we need to clearly spell out what we think are
the basic properties of time and space that will be a reliable foundation for our reasoning. I want to emphasize
that there is no purely logical way of deciding on this list of properties. The ones I'll list are simply a summary of the
patterns observed in the results
from a large body of experiments. Furthermore, some of them are only approximate. For example, property 1 below
is only a good approximation when the gravitational field is weak, so it is a property that applies to
special relativity, not to general relativity.

Experiments show that:\label{spacetime-properties}
\begin{enumerate}
\item No point in time or space has properties that make it different from any other point.
\item Likewise, all directions in space have the same properties.
\item Motion is relative, i.e., all inertial frames of reference are equally valid.
\item Causality holds, in the sense described on page \pageref{causality-defined}.
\item Time depends on the state of motion of the observer.
\end{enumerate}

Most of these are not very subversive. Properties 1 and 2 date back to the time when Galileo and Newton started
applying the same universal laws of motion to the solar system and to the earth; this contradicted
Aristotle, who believed that, for example, a rock would naturally want to move in a certain special
direction (down) in order to reach a certain special location (the earth's surface).
Property 3 is the reason that Einstein called his theory ``relativity,'' but Galileo and Newton
believed exactly the same thing to be true, as dramatized by Galileo's run-in with the Church over
the question of whether the earth could really be in motion around the sun.
Property 4 would probably surprise most people only because it asserts in such a weak and specialized
way something that they feel deeply must be true. The only really strange item on the list is 5,
but the Hafele-Keating experiment forces it upon us.

If it were not for property 5, we could imagine that figure \figref{galilean-boost} would
give the correct transformation between frames of reference in motion relative to one another.
Let's say that observer 1, whose grid coincides with the gray rectangle, is a hitch-hiker standing
by the side of a road. Event A is a raindrop hitting his head, and event B is another raindrop hitting
his head. He says that A and B occur at the same location in space. Observer 2 is a motorist who
drives by without stopping; to him, the passenger compartment of his car is at rest, while the
asphalt slides by underneath. He says that A and B occur at different points in space, because
during the time between the first raindrop and the second, the hitch-hiker has moved backward.
On the other hand, observer 2 says that events A and C occur in the same place, while the hitch-hiker
disagrees. The slope of the grid-lines is simply the velocity of the relative motion of each observer
relative to the other.
<% marg(m4_ifelse(__me,1,130,70)) %>
<%
  fig(
    'galilean-boost',
    %q{A Galilean version of the relationship between two frames of reference. As in all such graphs in
      this chapter, the original coordinates, represented by the gray rectangle, have a time axis that
      goes to the right, and a position axis that goes straight up.}
  )
%>
<% end_marg %>

Figure \figref{galilean-boost} has familiar, comforting, and eminently sensible
behavior, but it also happens to be wrong, because it violates property 5. The distortion of
the coordinate grid has only moved the vertical lines up and down, so both observers agree
that events like B and C are simultaneous. If this was really the way things worked, then
all observers could synchronize all their clocks with one another for once and for all, and
the clocks would never get out of sync. This contradicts the results of the Hafele-Keating
experiment, in which all three clocks were initially synchronized in Washington, but later
went out of sync because of their different states of motion.

It might seem as though we still had a huge amount of wiggle room available for the correct
form of the distortion. It turns out, however, that properties 1-5 are sufficient to prove that there
is only one answer, which is the one found by Einstein in 1905. To see why this is, let's work by
a process of elimination.

m4_ifelse(__me,1,[: %
<% marg(-50) %>
<%
  fig(
    'bowtie',
    %q{A transformation that leads to disagreements about whether two events occur at the same time and place.
       This is not just a matter of opinion. Either the arrow hit the bull's-eye or it didn't.}
  )
%>
\spacebetweenfigs
<%
  fig(
    'nonlinear-transformation',
    %q{A nonlinear transformation.}
  )
%>
<% end_marg %>
:])

Figure \figref{bowtie} shows a transformation that might
seem at first glance to be as good a candidate as any other,
but it violates property 3, that motion is relative, for the following
reason. In observer 2's frame of reference, some of the grid lines cross one another.
This means that observers 1 and 2 disagree on whether or not certain events are the same.
For instance, suppose that event A marks the arrival of an arrow at the bull's-eye of a
target, and event B is the location and time when the bull's-eye is punctured.
Events A and B occur
at the same location and at the same time. If one observer says that A and B coincide, but another
says that they don't, we have a direct contradiction. Since the two frames of reference in figure
\figref{bowtie} give contradictory results, one of them is right and one is wrong. This violates
property 3, because all inertial frames of reference are supposed to be equally valid.
To avoid problems like this, we clearly need to make sure that none of the grid lines ever cross
one another.


m4_ifelse(__me,1,[::],[: %
<% marg(70) %>
<%
  fig(
    'bowtie',
    %q{A transformation that leads to disagreements about whether two events occur at the same time and place.
       This is not just a matter of opinion. Either the arrow hit the bull's-eye or it didn't.}
  )
%>
\spacebetweenfigs
<%
  fig(
    'nonlinear-transformation',
    %q{A nonlinear transformation.}
  )
%>
<% end_marg %>
:])

The next type of transformation we want to kill off is shown in figure \figref{nonlinear-transformation},
in which the grid lines curve, but never cross one another.
The trouble with this one is that
it violates property 1, the uniformity of time and space. The transformation is unusually
``twisty'' at A, whereas at B it's much more smooth. This can't be correct, because the transformation
is only supposed to depend on the relative state of motion of the two frames of reference, and
that given information doesn't single out a special role for any particular point in spacetime.
If, for example, we had one frame of reference \emph{rotating} relative to the other, then there
would be something special about the axis of rotation. But we're only talking about \emph{inertial}
frames of reference here, as specified in property 3, so we can't have rotation; each frame of reference
has to be moving in a straight line at constant speed.
For frames related in this way, there is nothing that could single out an event like A for special
treatment compared to B, so transformation \figref{nonlinear-transformation} violates property 1.

The examples in figures \figref{bowtie} and \figref{nonlinear-transformation} show that the transformation
we're looking for must be linear, meaning that it must transform lines into lines, and furthermore that
it has to take parallel lines to parallel lines.\index{homogeneity of spacetime}
Einstein wrote in his 1905 paper that ``\ldots on account of the property of homogeneity [property 1] which we ascribe to time and space,
the [transformation] must be linear.''\footnote{A. Einstein, ``On the Electrodynamics of Moving Bodies,''
\emph{Annalen der Physik} 17 (1905), p. 891, tr. Saha and Bose.}
% Shamos, p. 323
Applying this to our diagrams,
the original gray rectangle, which is a special type of parallelogram containing right angles,
must be transformed into another parallelogram.
There are three types of transformations, figure \figref{three-cases}, that have this property.
Case I is the Galilean transformation of figure \figref{galilean-boost} on page \pageref{fig:galilean-boost},
which we've already ruled out.

<%
  fig(   
    'three-cases',
    %q{%
      Three types of transformations that preserve parallelism. Their distinguishing feature is what they do
      to simultaneity, as shown by what happens to the left edge of the original rectangle. In I, the left edge remains
      vertical, so simultaneous events remain simultaneous. In II, the left edge turns counterclockwise. In III, it turns clockwise.
    },
    {
      'width'=>'wide','sidecaption'=>false
    }
  )
%>

Case II can also be discarded. Here every point on the grid rotates counterclockwise. What physical parameter would
determine the amount of rotation? The only thing that could be relevant would be
$v$, the relative velocity of the motion of the two frames of reference with respect to one
another. But if the angle of rotation was proportional to $v$, then for large enough velocities
the grid would have left and right reversed, and this would violate property 4, causality: one observer
would say that event A caused a later event B, but another observer would say that B came first
and caused A.
<% marg(0) %>
<%
  fig(
    'smooshing',
    %q{In the units that are most convenient for relativity, the transformation has symmetry about a 45-degree diagonal line.}
  )
%>
\spacebetweenfigs
<%
  fig(
    'lorentz-slope',
    %q{Interpretation of the Lorentz transformation. The slope indicated in the figure gives the relative velocity of the two frames of reference.
         Events A and B that were simultaneous in frame 1 are not simultaneous in frame 2, where event A occurs to the right of the $t=0$ line represented
         by the left edge of the grid, but event B occurs to its left.}
  )
%>
<% end_marg %>

The only remaining possibility is case III, which I've redrawn in figure \figref{smooshing} with a couple
of changes. This is the one that Einstein predicted in 1905. The transformation is known as the Lorentz transformation,
after Hendrik Lorentz (1853-1928),\index{Lorentz, Hendrik}\index{Lorentz transformation}
who partially anticipated Einstein's work, without arriving at the correct interpretation.
The distortion is a kind of smooshing and stretching, as suggested by the hands. Also, we've already seen in figures
\figref{joan-of-arc}-\figref{change-of-units-convention} on page \pageref{fig:joan-of-arc} that we're free to stretch
or compress everything as much as we like in the horizontal and vertical directions, because this simply corresponds
to choosing different units of measurement for time and distance. In figure \figref{smooshing} I've chosen units that
give the whole drawing a convenient symmetry about a 45-degree diagonal line. Ordinarily it wouldn't make sense to
talk about a 45-degree angle on a graph whose axes had different units. But in relativity, the symmetric appearance of
the transformation tells us that space and time ought to be treated on the same footing, and measured in the same units.

As in our discussion of the Galilean transformation, slopes are interpreted as velocities, and
the slope of the near-horizontal lines in figure \figref{lorentz-slope} is interpreted as the relative velocity of the two observers.
The difference between the Galilean version and the relativistic one is that now there is smooshing happening from the
other side as well. Lines that were vertical in the original grid, representing simultaneous events, now slant over to
the right. This tells us that, as required by property 5, different observers do not agree on whether events that occur in different places are
simultaneous. The Hafele-Keating experiment tells us that this non-simultaneity effect is fairly small, even when the velocity is as
big as that of a passenger jet, and this is what we would have anticipated by the correspondence principle. The way that
this is expressed in the graph is that if we pick the time unit to be the second, then the distance unit turns out to be hundreds of thousands of miles.
In these units, the velocity of a passenger jet is an extremely small number, so the slope $v$ in figure \figref{lorentz-slope}
is extremely small, and the amount of distortion is tiny --- it would be much too small to see on this scale.

The only thing left to determine about the Lorentz transformation is the size of the transformed parallelogram relative to the
size of the original one. Although the drawing of the hands in figure \figref{smooshing} may suggest that the grid deforms like
a framework made of rigid coat-hanger wire, that is not the case. If you look carefully at the figure, you'll see that the edges
of the smooshed parallelogram are actually a little longer than the edges of the original rectangle. In fact what stays the same
is not lengths but \emph{areas}, as proved in the caption to figure \figref{area-proof}.
  <% end_sec('lorentz') %> % The Lorentz transformation

<%
  fig('area-proof',
    %q{Proof that Lorentz transformations don't change area: We first subject a square to a transformation with velocity $v$, and this increases its area by a factor $R(v)$, which
       we want to prove equals 1. We chop the resulting parallelogram up into little squares and finally apply a $-v$ transformation;
       this changes each little square's area by a factor $R(-v)$, so the whole figure's area is also scaled by $R(-v)$.
       The final result is to restore the square to its original shape and area, so $R(v)R(-v)=1$. But $R(v)=R(-v)$ by property 2 of spacetime
       on page \pageref{spacetime-properties}, which states that all directions in space have the same properties, so $R(v)=1$.
     },
    {
      'width'=>'fullpage'
    }
  )
%>

  <% begin_sec("The $\\mygamma$ factor",nil,'gamma') %>
With a little algebra and geometry (homework problem \ref{hw:gamma-derivation}, page \pageref{hw:gamma-derivation}),
one can use the equal-area property to show that the factor $\mygamma$ (Greek letter gamma)
defined in figure \figref{gamma-as-projection} is given by the equation
\begin{equation*}
  \mygamma = \frac{1}{\sqrt{1-v^2}}\eqquad.
\end{equation*}
If you've had good training in physics, the first thing you probably
think when you look at this equation is that it must be nonsense,
because its units don't make sense. How can we take something with
units of velocity squared, and subtract it from a unitless 1? But
remember that this is expressed in our special relativistic units, in
which the same units are used for distance and time.
We refer to
these as \emph{natural} units.\index{units!natural relativistic}\index{natural units}\label{natural-units}
In this system,
velocities are always unitless. This sort of thing happens frequently
in physics. For instance, before James Joule discovered conservation
of energy, nobody knew that heat and mechanical energy were different
forms of the same thing, so instead of measuring them both in units of
joules as we would do now, they measured heat in one unit (such as
calories) and mechanical energy in another (such as foot-pounds). In
ordinary metric units, we just need an extra conversion factor $c$,
and the equation becomes
\begin{equation*}
  \mygamma = \frac{1}{\sqrt{1-\left(\frac{v}{c}\right)^2}}\eqquad.
\end{equation*}

Here's why we care about $\mygamma$. Figure \figref{gamma-as-projection} defines it as the ratio of two times: the time between
two events as expressed in one coordinate system, and the time between the same two events as measured in the other one.
The interpretation is:

\begin{lessimportant}[Time dilation]
A clock runs fastest in the frame of reference of an observer who is at rest relative to the clock. An observer
in motion relative to the clock at speed $v$ perceives the clock as running more slowly by a factor of $\mygamma$.
\end{lessimportant}

<% marg(0) %>
<%
  fig(
    'gamma-as-projection',
    %q{The $\mygamma$ factor.}
  )
%>
\spacebetweenfigs
<%
  fig(
    'length-contraction',
    %q{The ruler is moving in frame 1, represented by a square, but at rest in frame 2, shown as a parallelogram.
       Each picture of the ruler is a snapshot taken at a certain moment as judged according to frame 2's
       notion of simultaneity. An observer in frame 1 judges the ruler's length instead according to
       frame 1's definition of simultaneity, i.e., using points that are lined up vertically on the graph.
       The ruler appears shorter in the frame in which it is moving.
       As proved in figure \figref{length-contraction-proof}, the length contracts from $L$ to $L/\gamma$.}
  )
%>
<% end_marg %>

<%
  fig('length-contraction-proof',
    %q{This figure proves, as claimed in figure \figref{length-contraction}, that the length contraction is $x=1/\gamma$.
       First we slice the parallelogram vertically like a salami and slide the slices down, making the
       top and bottom edges horizontal. Then we do the same in the horizontal direction, forming a rectangle
       with sides $\gamma$ and $x$. Since both the Lorentz transformation and the slicing processes leave
       areas unchanged, the area $\gamma x$  of the rectangle must equal the area of the original square, which is 1.
     },
    {
      'width'=>'wide'
    }
  )
%>

\noindent As proved in figures \figref{length-contraction} and \figref{length-contraction-proof}, lengths are also distorted:

\begin{lessimportant}[Length contraction]
A meter-stick appears longest to an observer who is at rest relative to it. An observer moving relative to the
meter-stick at $v$ observes the stick to be shortened by a factor of $\mygamma$.
\end{lessimportant}

<% self_check('gammaatvzero',<<-'SELF_CHECK'
What is $\\mygamma$ when $v=0$? What does this mean?
  SELF_CHECK
  ) %>

\pagebreak

Figure \figref{gamma-graph-small} shows the behavior of $\gamma$ as a function of $v$.

\begin{eg}{Changing an equation from natural units to SI}\label{eg:natural-to-si-1}
Often it is easier to do all of our algebra in natural units, which are simpler because $c=1$, and
all factors of $c$ can therefore be omitted. For example, suppose we want to solve for $v$ in terms of $\mygamma$.
In natural units, we have $\mygamma=1/\sqrt{1-v^2}$, so $\mygamma^{-2}=1-v^2$, and $v=\sqrt{1-\mygamma^{-2}}$.

This form of the result might be fine for many purposes, but if we wanted to find a value of $v$ in SI units,
we would need to reinsert factors of $c$ in the final result. There is no need to do this throughout the whole
derivation. By looking at the final result, we see that there is only one possible way to do this so that the
results make sense in SI, which is to write $v=c\sqrt{1-\mygamma^{-2}}$.
\end{eg}

<% marg(300) %>
<%
  fig(
    'gamma-graph-small',
    %q{A graph of $\gamma$ as a function of $v$.}
  )
%>
<% end_marg %>


\begin{eg}{Motion of a ray of light}\label{eg:natural-to-si-2}
\egquestion The motion of a certain ray of light is given by the equation $x=-t$. Is this expressed in
natural units, or in SI units? Convert to the other system.

\eganswer The equation is in natural units. It wouldn't make sense in SI units, because we would have
meters on the left and seconds on the right. To convert to SI units, we insert a factor of $c$ in the only
possible place that will cause the equation to make sense: $x=-ct$.
\end{eg}


\begin{eg}{An interstellar road trip}\label{eg:interstellar-road-trip}
Alice stays on earth while her twin Betty
heads off in a spaceship for Tau Ceti, a nearby star. Tau Ceti is 12 light-years
away, so even though Betty travels at 87\% of the speed of light, it will take
her a long time to get there: 14 years, according to Alice.

<%
  fig('interstellar-road-trip',
    'Example \ref{eg:interstellar-road-trip}.',
    {
      'width'=>'wide',
      'sidecaption'=>true
    }
  )
%>

Betty
experiences time dilation. At this speed, her $\gamma$ is 2.0, so that the voyage will
only seem to her to last 7 years. But there is perfect symmetry between Alice's
and Betty's frames of reference, so Betty agrees with Alice on their relative speed;
Betty sees herself as being at rest, while the sun and Tau Ceti both move backward
at 87\% of the speed of light. How, then, can she observe Tau Ceti to get to her
in only 7 years, when it should take 14 years to travel 12 light-years at this speed?

We need to take into account length contraction.
Betty sees the distance between the sun and Tau Ceti
to be shrunk by a factor of 2. The same thing occurs for Alice, who observes
Betty and her spaceship to be foreshortened.
\end{eg}

\begin{eg}{The correspondence principle}\label{eg:gamma-for-low-v}
The correspondence principle requires that $\mygamma$ be close to 1
for the velocities much less than $c$ encountered in everyday life.
In natural units, $\mygamma=(1-v^2)^{-1/2}$.
For small values of $\epsilon$, the approximation 
$(1+\epsilon)^p\approx 1+p\epsilon$ holds (see p.~\pageref{math-approx-exp-and-log}).
Applying this approximation, we find $\mygamma\approx1+v^2/2$.

As expected, this gives approximately 1 when $v$ is small compared to 1 (i.e., compared to $c$, which equals 1
in natural units).

In problem \ref{hw:gamma-approx-si} on p.~\pageref{hw:gamma-approx-si} we rewrite
this in SI units.

Figure \figref{gamma-graph-small} on p.~\pageref{fig:gamma-graph-small} shows that the approximation
is \emph{not} valid for large values of $v/c$. In fact, $\mygamma$
blows up to infinity as $v$ gets closer and closer to $c$.
\end{eg}

<% marg(m4_ifelse(__me,1,-8,-8)) %>
<%
  fig(
    'cern-muon-graph',
    %q{Muons accelerated to nearly $c$ undergo radioactive decay much more slowly than they would according to
       an observer at rest with respect to the muons. The first two data-points (unfilled circles) were subject
       to large systematic errors.}
  )
%>
<% end_marg %>

\begin{eg}{Large time dilation}\label{eg:cern-muons}
The time dilation effect in the Hafele-Keating experiment was very small. If we want to see a large time dilation
effect, we can't do it with something the size of the atomic clocks they used; the kinetic energy would be
greater than the total megatonnage of all the world's nuclear arsenals. We can, however, accelerate subatomic particles
to speeds at which $\mygamma$ is large. For experimental particle physicists, relativity is something you do all day
before heading home and stopping off at the store for milk. An early, low-precision experiment of this kind was
performed by Rossi and Hall in 1941, using naturally occurring cosmic rays. Figure \figref{cern-muon-storage-ring} shows
a 1974 experiment\footnote{Bailey at al., Nucl. Phys. B150(1979) 1}
of a similar type which verified the time dilation predicted by relativity to a precision of about
one part per thousand.

\vfill

<%
  fig(
    'cern-muon-storage-ring',
    %q{Apparatus used for the test of relativistic time dilation described in example \ref{eg:cern-muons}. 
       The prominent black and white blocks are large magnets surrounding a circular pipe
       with a vacuum inside. \linebreak (c) 1974 by CERN.
    },
    {
      'width'=>'wide',
      'sidecaption'=>true,
      'sidepos'=>'b'
    }
  )
%>

Particles called muons (named after the Greek letter $\mu$, ``myoo'') were produced by an
accelerator at CERN, near Geneva. A muon is essentially a heavier version
of the electron. Muons undergo radioactive decay,
lasting an average of only 2.197 $\mu\sunit$ before they evaporate into an electron and two neutrinos.
The 1974 experiment was actually built in order to measure the magnetic properties of muons, but it produced a high-precision
test of time dilation as a byproduct. Because muons have the same electric charge as electrons, they can be trapped using
magnetic fields. Muons were injected into the ring shown in figure \figref{cern-muon-storage-ring}, circling around it until they underwent radioactive decay.
At the speed at which these muons were traveling, they had $\mygamma=29.33$, so on the average they lasted 29.33 times
longer than the normal lifetime. In other words, they were like tiny alarm clocks that self-destructed at a randomly
selected time. Figure \figref{cern-muon-graph} shows the number of radioactive decays counted, as a function of the
time elapsed after a given stream of muons was injected into the storage ring. The two dashed lines show the rates
of decay predicted with and without relativity. The relativistic line is the one that agrees with experiment.
\end{eg}

\vfill

<%
  fig(
    'rhic',
    %q{Colliding nuclei show relativistic length contraction.},
    {
      'width'=>'wide',
      'sidecaption'=>true
    }
  )
%>

\begin{eg}{An example of length contraction}
Figure \figref{rhic} shows an
artist's rendering of the length contraction for the collision of two
gold nuclei at relativistic speeds in the RHIC accelerator\index{RHIC accelerator}
in Long Island, New York, which went on line in 2000.
The gold nuclei would appear nearly spherical (or just
slightly lengthened like an American football) in frames
moving along with them, but in the laboratory's frame, they
both appear drastically foreshortened as they approach the
point of collision. The later pictures show the nuclei
merging to form a hot soup, in which experimenters hope to
observe a new form of matter.
\end{eg}


<%
  fig(
    'schoolbus-with-x-t',
    %q{%
      Example }+ref_workaround('eg:garage-paradox')+%q{: In the garage's frame of reference, the bus
      is moving, and can fit in the garage due to its length contraction. In the bus's frame of reference,
      the garage is moving, and can't hold the bus due to \emph{its} length contraction.
    },
    {
      # 'anonymous'=>true,
      'width'=>'fullpage'
    }
  )
%>

\enlargethispage{\baselineskip}

\vspace{-\baselineskip}

\begin{eg}{The garage paradox}\label{eg:garage-paradox}\index{garage paradox}
One of the most famous of all the so-called relativity
paradoxes has to do with our incorrect 
feeling that simultaneity is well defined. The idea is that
one could take a schoolbus and drive it at relativistic
speeds into a garage of ordinary size, in which it normally
would not fit. Because of the length contraction, the bus
would supposedly fit in the garage. The driver, however, will perceive the
\emph{garage} as being contracted and thus even less able to
contain the bus. 

The paradox is
resolved when we recognize that the concept of fitting the
bus in the garage ``all at once'' contains a hidden
assumption, the assumption that it makes sense to ask
whether the front and back of the bus can \emph{simultaneously} be
in the garage. Observers in different frames of reference
moving at high relative speeds do not necessarily agree on
whether things happen simultaneously. As shown in figure \figref{schoolbus-with-x-t}, the person in the
garage's frame can shut the door at an instant B he perceives
to be simultaneous with the front bumper's arrival A at the
back wall of the garage, but the driver would not agree
about the simultaneity of these two events, and would
perceive the door as having shut long after she plowed
through the back wall.
\end{eg}

  <% end_sec('gamma') %> % The $\mygamma$ factor

  <% begin_sec("The universal speed $c$",nil,'universal-speed') %>
Let's think a little more about the role of the 45-degree diagonal in the Lorentz transformation.
Slopes on these graphs are interpreted as velocities.
This line has a slope of 1 in relativistic units, but that slope corresponds to $c$ in ordinary metric units.
We already know that the relativistic distance unit must
be extremely large compared to the relativistic time unit, so $c$ must be extremely large.
Now note what happens when we perform a Lorentz transformation: this particular line gets stretched, but the new version
of the line lies right on top of the old one, and its slope stays the
same. In other words, if one observer says  that something has a velocity equal to $c$, every other observer will agree
on that velocity as well. (The same thing happens with $-c$.) 

    <% begin_sec("Velocities don't simply add and subtract.",nil,'v-not-simple-addition') %>\label{relativistic-combination-of-vel}
m4_ifelse(__me,1,[: %
This is surprising, since we expect, as in section \ref{vel-addition-newtonian}, that a velocity $c$
in one frame should become $c+v$ in a frame moving at velocity $v$ relative to the first one.
But velocities are measured by dividing a distance by a time, and
both distance and time are distorted by relativistic effects, so we actually shouldn't expect the ordinary
arithmetic addition of velocities to hold in relativity; it's an approximation that's valid at velocities
that are small compared to $c$. Problem \ref{hw:me-vel-addition} on p.~\pageref{hw:me-vel-addition} shows that
relativistically, combining velocities $u$ and $v$ gives not $u+v$ but $(u+v)/(1+uv)$ (in units where $c=1$).\label{me-vel-addition-main-text}
:],[: %
This is counterintuitive, since we expect velocities to
add and subtract in relative motion. If a dog is running away from me at
5 m/s relative to the sidewalk, and I run after it at 3 m/s,
the dog's velocity in my frame of reference is 2 m/s.
According to everything we have learned about motion, the
dog must have different speeds in the two frames: 5 m/s in
the sidewalk's frame and 2 m/s in mine. But velocities are measured by dividing a distance by a time, and
both distance and time are distorted by relativistic effects, so we actually shouldn't expect the ordinary
arithmetic addition of velocities to hold in relativity; it's an approximation that's valid at velocities
that are small compared to $c$.
:])
    <% end_sec('v-not-simple-addition') %>

    <% begin_sec("A universal speed limit",nil,'speed-limit') %>
For example, suppose Janet takes a trip in a
spaceship, and accelerates until she is moving at $0.6c$ relative to the
earth. She then launches a space probe in the forward
direction at a speed relative to her ship of $0.6c$. We might think that the
probe was then moving at a velocity of $1.2c$, but in fact the answer is still less 
than $c$ (problem \ref{hw:six-tenths-c-twice}, page \pageref{hw:six-tenths-c-twice}).
This is an example of a more general fact about relativity, which is that $c$ represents
a universal speed limit. This is required by causality, as shown in figure \figref{speed-limit}.
    <% end_sec('speed-limit') %>
<% marg(140) %>
<%
  fig(
    'speed-limit',
    %q{A proof that causality imposes a universal speed limit. In the original frame of reference, represented by the square, event A happens a little before event B.
       In the new frame, shown by the parallelogram, A happens after $t=0$, but B happens before $t=0$; that is,
       B happens before A. The time ordering of the two events has been reversed. This can only happen because
       events A and B are very close together in time and fairly far apart in space. The line segment
       connecting A and B has a slope greater than 1, meaning that if we wanted to be present at both events, we would
       have to travel at a speed greater than $c$ (which equals 1 in the units used on this graph). You will find that if
       you pick any two points for which the slope of the line segment connecting them is less than 1, you can never get them
       to straddle the new $t=0$ line in this funny, time-reversed way. Since different observers disagree on the time order
       of events like A and B, causality requires that information never travel from A to B or from B to A; if it did, then
       we would have time-travel paradoxes. The conclusion is that $c$ is the maximum speed of cause and effect in relativity.
    }
  )
%>
<% end_marg %>

<%
  fig(
    'michelson',
    %q{The Michelson-Morley experiment, shown in photographs, and drawings from the original 1887 paper.
       1. A simplified drawing of the apparatus. A beam of light from the source, s, is partially reflected and partially transmitted by the half-silvered
       mirror $\zu{h}_1$. The two half-intensity parts of the beam are reflected by the mirrors at a and b, reunited,
       and observed in the telescope, t. If the earth's surface was supposed to be moving through the ether,
       then the times taken by the two light waves to pass through the moving ether would be unequal, and
       the resulting time lag would be detectable by observing the interference between the waves when they were reunited.
       2. In the real apparatus, the light beams were reflected multiple times. The effective length of each arm was
       increased to 11 meters, which greatly improved its sensitivity to the small expected difference in the speed of light.
       3. In an earlier version of the experiment, they had run into problems with its ``extreme sensitiveness to vibration,''
       which was ``so great that it was impossible to see the interference fringes except at brief intervals \ldots even at
       two o'clock in the morning.'' They therefore mounted the whole thing on a massive stone floating in a pool of mercury,
       which also made it possible to rotate it easily. 4. A photo of the apparatus.
    },
    {
      'width'=>'wide',
      'sidecaption'=>false,
      'sidepos'=>'t'
    }
  )
%>

    <% begin_sec("Light travels at $c$.",nil,'light-travels-at-c') %>
Now consider a beam of light. We're used to talking casually about the ``speed of light,'' but what does that really
mean? Motion is relative, so normally if we want to talk about a velocity, we have to specify what it's measured
relative to. A sound wave has a certain speed relative to the air, and a water wave has its own speed relative to the
water. If we want to measure the speed of an ocean wave, for example, we should make sure to measure it in a frame
of reference at rest relative to the water. But light isn't a vibration of a physical medium; it can propagate through the near-perfect vacuum of outer space,
as when rays of sunlight travel to earth. This seems like a paradox: light is supposed to have a specific speed,
but there is no way to decide what frame of reference to measure it in. The way out of the paradox is that light
must travel at a velocity equal to $c$. Since all observers agree on a velocity of $c$, regardless of their frame
of reference, everything is consistent.
    <% end_sec('light-travels-at-c') %>

    <% begin_sec("The Michelson-Morley experiment",nil,'michelson-morley') %>
The constancy of the speed of light had in fact already been
observed when Einstein was an 8-year-old boy, but because nobody could
figure out how to interpret it, the result was largely ignored.
In 1887 Michelson and Morley set up a clever apparatus to
measure any difference in the speed of light beams traveling
east-west and north-south. The motion of the earth around
the sun at 110,000 km/hour (about 0.01\% of the speed of
light) is to our west during the day. Michelson and Morley
believed that light was a vibration of a mysterious medium called the ether, so they expected that the
speed of light would be a fixed value relative to the ether.
As the earth moved through the ether, they thought they
would observe an effect on the velocity of light along an
east-west line. For instance, if they released a beam of
light in a westward direction during the day, they expected
that it would move away from them at less than the normal
speed because the earth was chasing it through the ether.
They were surprised when they found that the expected 0.01\%
change in the speed of light did not occur.\index{Michelson-Morley experiment}\index{ether}

    <% end_sec('michelson-morley') %> % The Michelson-Morley experiment

\begin{eg}{The ring laser gyroscope}
If you've flown in a jet plane, you can thank relativity for helping you to avoid
crashing into a mountain or an ocean. Figure \figref{ring-laser-gyro} shows a standard
piece of navigational equipment called a ring laser gyroscope. A beam of light is
split into two parts, sent around the perimeter of the device, and reunited. Since
the speed of light is constant, we expect the two parts to come back together at the
same time. If they don't, it's evidence that the device has been rotating. The plane's
computer senses this and notes how much rotation has accumulated.
\end{eg}
<% marg(-10) %>
<%
  fig(
    'ring-laser-gyro',
    %q{A ring laser gyroscope.}
  )
%>
<% end_marg %>

\begin{eg}{No frequency-dependence}
Relativity has only one universal speed, so it requires that all light waves
travel at the same speed, regardless of their frequency and wavelength.
Presently the best experimental tests of the invariance of
the speed of light with respect to wavelength come from astronomical observations of
gamma-ray bursts, which are sudden
outpourings of high-frequency light, believed to originate from a supernova explosion
in another galaxy. One such observation,
in 2009,\footnote{\url{http://arxiv.org/abs/0908.1832}} 
found that the
times of arrival of all the different frequencies in the burst
differed by no more than 2 seconds out of a total time in flight on the order
of ten billion years!
\end{eg}

  <% end_sec('universal-speed') %> % Universality of $c$

\vfill

\startdqs
\begin{dq}
A person in a spaceship moving at 99.99999999\% of the
speed of light relative to Earth shines a flashlight forward
through dusty air, so the beam is visible. What does she
see? What would it look like to an observer on Earth?
\end{dq}

\vspace{m4_ifelse(__me,1,5,0)mm}

\begin{dq}\label{dq:illusion}
A question that students often struggle with is whether
time and space can really be distorted, or whether it just
seems that way. Compare with optical illusions or magic
tricks. How could you verify, for instance, that the lines
in the figure are actually parallel? Are relativistic
effects the same, or not?
\end{dq}

\vspace{m4_ifelse(__me,1,5,0)mm}

\begin{dq}
On a spaceship moving at relativistic speeds, would a
lecture seem even longer and more boring than normal?
\end{dq}

\vspace{m4_ifelse(__me,1,5,0)mm}

\begin{dq}
Mechanical clocks can be affected by motion. For example,
it was a significant technological achievement to build a
clock that could sail aboard a ship and still keep accurate
time, allowing longitude to be determined. How is this
similar to or different from relativistic time dilation?
\end{dq}

\begin{dq}\label{dq:rhic}
Figure \figref{rhic} from page \pageref{fig:rhic}, depicting the collision of
two nuclei at the RHIC accelerator, is reproduced below.
What would the shapes of the two nuclei
look like to a microscopic observer riding on the
left-hand nucleus? To an observer riding on the right-hand
one? Can they agree on what is happening? If not, why not
--- after all, shouldn't they see the same thing if they
both compare the two nuclei side-by-side at the same instant in time?
\end{dq}
<%
  fig(
    'rhic',
    %q{Discussion question \ref{dq:rhic}: colliding nuclei show relativistic length contraction.},
    {
      'width'=>'wide',
      'sidecaption'=>false,
      'suffix'=>'2'
    }
  )
%>

\begin{dq}\label{dq:foam-rubber}
If you stick a piece of foam rubber out the window of
your car while driving down the freeway, the wind may
compress it a little. Does it make sense to interpret the
relativistic length contraction as a type of strain that
pushes an object's atoms together like this? How does this
relate to discussion question \ref{dq:rhic}?
\end{dq}

% For some reason, the following figure appears on the previous page unless I put it way down here.
<% marg(170) %>
<%
  fig(
    'dqillusion',
    %q{Discussion question \ref{dq:illusion}},
    {
      'anonymous'=>true
    }
  )
%>
<% end_marg %>

m4_ifelse(__me,1,\pagebreak,\vspace{0mm})
m4_ifelse(__sn,1,\pagebreak,\vspace{0mm})

% LM also has this DQ, but doesn't use this file, so only do the following if it's SN.
% Instructions are above the figure so I can project figure on whiteboard, instructions on screen.
m4_ifelse(__sn,1,[:

\begin{dq}\label{dq:galaxies-signaling}
The graph shows three galaxies. The axes are drawn according to an observer at rest relative to
the galaxy 2, so that that galaxy is always at the same $x$ coordinate. 
Intelligent species in the three different galaxies develop radio technology independently,
and at some point each begins to actively send out signals in an attempt to communicate with other civilizations.
Events a, b, and c mark the points at which these
signals begin spreading out across the universe at the speed of light. Find the events
at which the inhabitants of galaxy 2 detect
the signals from galaxies 1 and 3. According to 2, who developed radio
first, 1 or 3? On top of the graph,
draw a new pair of position and time axes, for the frame in which galaxy 3 is at rest.
According to 3, in what order did events a, b, and c happen?
\end{dq}

<%
  fig(
    'galaxies-signaling',
    %q{Discussion question \ref{dq:galaxies-signaling}.
    },
    {
      'width'=>'wide',
      'sidecaption'=>true,
      'anonymous'=>true
    }
  )
%>

\vfill

:])
% ... endif for galaxies-signaling

\begin{dq}\label{dq:machine-gun-ftl}
The machine-gunner in the figure sends out a spray of bullets. Suppose that the bullets are being shot into
       outer space, and that the distances traveled are trillions of miles (so that the human figure
       in the diagram is not to scale). After a long time, the bullets reach the points shown with dots
       which are all equally far from the gun. Their arrivals at those points are events A through E,
       which happen at different times.  Sketch these events on a position-time graph.
       The chain of impacts extends across space at a speed greater
       than $c$. Does this violate special relativity?
\end{dq}

<%
  fig(
    'machine-gun-ftl',
    %q{Discussion question \ref{dq:machine-gun-ftl}.
    },
    {
      'width'=>'wide',
      'sidecaption'=>true,
      'anonymous'=>true
    }
  )
%>

m4_ifelse(__me,1,\pagebreak[4])
m4_ifelse(__lm_series,1,\vspace{30mm})

% In SN, we have another subsection, relativity_no_action_at_a_distance.tex, after this.
m4_ifelse(__sn,0,[:
<% end_sec('x-t-distortion') %> % Distortion of Space and Time
:])
