<% begin_sec("Dynamics",nil,'reldynamics') %>
So far we have said nothing about how to predict motion in
relativity. Do Newton's laws still work? Do conservation
laws still apply? The answer is yes, but many of the
definitions need to be modified, and certain entirely new
phenomena occur, such as the equivalence of energy and mass, as described by the famous equation
$E=mc^2$.

\vspace{15mm}

<% begin_sec("Momentum") %>\index{momentum!relativistic}
Consider the following scheme for traveling faster than the speed of light.
The basic idea can be demonstrated by dropping a ping-pong ball and a baseball
stacked on top of each other like a snowman. They separate slightly in mid-air,
and the baseball therefore has time to hit the floor and rebound before it
collides with the ping-pong ball, which is still on the way down. The result is
a surprise if you haven't seen it before: the ping-pong ball flies off at high
speed and hits the ceiling! A similar fact is known to people who investigate the
scenes of accidents involving pedestrians. If a car moving at 90 kilometers per
hour hits a pedestrian, the pedestrian flies off at nearly double that speed, 180
kilometers per hour. Now suppose the car was moving at 90 percent of the speed of
light. Would the pedestrian fly off at 180\% of $c$?

\vspace{15mm}

To see why not, we have to back up a little and think about where this speed-doubling
result comes from. 
For any collision, there is a special frame of reference, the center-of-mass frame,
in which the two colliding objects approach each other,
collide, and rebound with their velocities reversed. In the center-of-mass frame,
the total momentum of the objects is zero both before and after the collision.

\pagebreak

<%
  fig(
    'unequalcollision',
    %q{%
      An unequal collision, viewed in the center-of-mass frame, 1, and
      in the frame where the small ball is initially at rest, 2. The motion is shown as it
      would appear on the film of an old-fashioned movie camera, with an equal amount of time separating
      each frame from the next. Film 1 was made by a camera that tracked the center of mass, film 2 by
      one that was initially tracking the small ball, and kept on moving at the same speed after the collision.
    },
    {
      'width'=>'wide',
      'sidecaption'=>true
    }
  )
%>

Figure \subfigref{unequalcollision}{1} shows such a frame of reference for objects of very unequal mass.
Before the collision, the large ball is moving relatively slowly toward the top of the page, but
because of its greater mass, its momentum cancels the momentum of the smaller ball, which is
moving rapidly in the opposite direction. The total momentum is zero. After the collision, the
two balls just reverse their directions of motion. We know that this is the
right result for the outcome of the collision because
it conserves both momentum and kinetic energy, and everything not forbidden is compulsory, i.e.,
in any experiment, there is only one possible outcome, which is the one that obeys all the
conservation laws.

<% self_check('unequalcollisioncons',<<-'SELF_CHECK'
How do we know that momentum and kinetic energy are conserved
in figure \subfigref{unequalcollision}{1}?
  SELF_CHECK
  ) %>

Let's make up some numbers as an example. Say the small ball has a mass of 1 kg, the big
one 8 kg. In frame 1, let's make the velocities as follows:

<% if is_print then
print %q(
\\newcommand{\\smallvelocitytable}[6]{%
\\noindent\\hspace{5mm}\\begin{tabular}{|p{4mm}|p{40mm}|p{40mm}|}
\\hline
 & before the collision
       & after the collision  \\\\
\\hline
\\includegraphics{../share/relativity/figs/#5} & #1 & #2  \\\\
\\includegraphics{../share/relativity/figs/#6} & #3 & #4  \\\\
\\hline
\\end{tabular}
}
)
end
%>

<%
def small_vel_table(a,b,c,d)
if is_print then
  return "\\smallvelocitytable{#{a}}{#{b}}{#{c}}{#{d}}{smallball}{bigball}"
end
if is_web then
return %Q(
\\noindent\\begin{tabular}{|p{4mm}|p{40mm}|p{40mm}|}
\\hline
 & before the collision
       & after the collision  \\\\
\\hline
small ball & #{a} & #{b}  \\\\
big ball & #{c} & #{d}  \\\\
\\hline
\\end{tabular}
)
end
end
%>

<% print small_vel_table(-0.8,0.8,0.1,-0.1) %>

Figure \subfigref{unequalcollision}{2} shows the same collision in a frame of reference where
the small ball was initially at rest.
 To find all the velocities in this frame, we
just add 0.8 to all the ones in the previous table.

<% print small_vel_table(0,1.6,0.9,0.7) %>

\noindent In this frame, as expected, the small ball flies off with a velocity, 1.6, that
is almost twice the initial velocity of the big ball, 0.9.

If all those velocities were in meters per second, then that's exactly what happened. But
what if all these velocities were in units of the speed of light? Now it's no longer a good
approximation just to add velocities. We need to combine them according to the relativistic
rules. For instance, the technique used in 
problem \ref{hw:six-tenths-c-twice} on p.~\pageref{hw:six-tenths-c-twice}
can be used to show that combining a velocity of 0.8 times the
speed of light with another velocity of 0.8 results in 0.98, not 1.6. The results are very different:

<% print small_vel_table(0,0.98,0.83,0.76) %>
<%#
 gamma = 1, 5.0, 1.8, 1.5
 1/gamma = 1, 0.2, 0.56, .67
%>

<%
  fig(
    'unequalrel',
    %q{%
      An 8-kg ball moving at 83\% of the speed of light hits a 1-kg ball. The balls
      appear foreshortened due to the relativistic distortion of space.
    },
    {
      'width'=>'wide',
      'sidecaption'=>true
    }
  )
%>

We can interpret this as follows. Figure \subfigref{unequalcollision}{1} is one in which the
big ball is moving fairly slowly. This is very nearly the way the scene would be seen by an
ant standing on the big ball. According to an observer in frame \figref{unequalrel}, however,
both balls are moving at nearly the speed of light after the collision. Because of this, the
balls appear foreshortened, but the distance between the two balls is also shortened. To this
observer, it seems that the small ball isn't pulling away from the big ball very fast.

Now here's what's interesting about all this. The outcome shown in figure \subfigref{unequalcollision}{2}
was supposed to be the only one possible, the only one that satisfied both conservation of energy
and conservation of momentum. So how can the \emph{different} result shown in figure
\figref{unequalrel} be possible? The answer is that relativistically, momentum must not equal
$mv$. The old, familiar definition is only an approximation that's valid at low speeds. If
we observe the behavior of the small ball in figure \figref{unequalrel}, it looks as though it
somehow had some extra inertia. It's as though a football player tried to knock another player
down without realizing that the other guy had a three-hundred-pound bag full of lead shot
hidden under his uniform --- he just doesn't seem to react to the collision as much as he should.
As proved in section \ref{subsec:rel-dynamics-proofs},
this extra inertia is described by redefining momentum as
\begin{equation*}
        p = m \mygamma v\eqquad.
\end{equation*}
At very low velocities, $\mygamma$ is close to 1, and the result is very nearly $mv$, as demanded
by the correspondence principle. But at very high velocities, $\mygamma$ gets very big --- the
small ball in figure \figref{unequalrel} has a $\mygamma$ of 5.0, and therefore has five times
more inertia than we would expect nonrelativistically.\index{correspondence principle!for relativistic momentum}

This also explains the answer to another paradox often posed by beginners at relativity.
Suppose you keep on applying a steady force to an object that's already moving at $0.9999c$.
Why doesn't it just keep on speeding up past $c$? The answer is that force is the rate of
change of momentum. At $0.9999c$, an object already has a $\mygamma$ of 71, and therefore
has already sucked up 71 times the momentum you'd expect at that speed. As its velocity gets closer and
closer to $c$, its $\mygamma$ approaches infinity. To move at $c$, it would need an infinite
momentum, which could only be caused by an infinite force.

m4_include(../share/relativity/eg/bertozzi.tex)

Figure \figref{relativistic-momentum-tests} shows experimental data confirming the relativistic
equation for momentum.

<% marg(-300) %>
<%
  fig(
    'relativistic-momentum-tests',
    %q{Two early high-precision tests of the relativistic equation $p=m\mygamma v$ for momentum.
       Graphing $p/m$ rather than $p$ allows the data for electrons and protons to be placed on the same graph.
       Natural units are used, so that the horizontal axis is the velocity in units of $c$, and
       the vertical axis is the unitless quantity $p/mc$. The very small error bars for the data point from
       Zrelov are represented by the height of the black rectangle.
       }
  )
%>
<% end_marg %>

<% end_sec() %> % Momentum
<% begin_sec("Equivalence of mass and energy") %>\index{energy!equivalence to mass}\index{mass!equivalence to energy}
Now we're ready to see why mass and energy must be equivalent as claimed
in the famous $E=mc^2$. So far we've only considered collisions
in which none of the kinetic energy is converted into any other form
of energy, such as heat or sound.
Let's consider what happens if a blob of putty moving at
velocity $v$ hits another blob that is initially at rest,
sticking to it.  The nonrelativistic result is
that to obey conservation of momentum the two blobs must fly
off together at $v/2$. Half of the initial kinetic energy
has been converted to heat.\footnote{A double-mass object moving
at half the speed does not have the same kinetic energy. Kinetic
energy depends on the square of the velocity, so cutting the velocity
in half reduces the energy by a factor of 1/4, which, multiplied
by the doubled mass, makes 1/2 the original energy.}

Relativistically, however, an interesting thing happens. A
hot object has more momentum than a cold object! This is
because the relativistically correct expression for momentum
is $m\mygamma v$, and the more rapidly moving atoms in the hot
object have higher values of $\mygamma$.
In our collision, the final combined blob must therefore be
moving a little more slowly than the expected $v/2$, since
otherwise the final momentum would have been a little
greater than the initial momentum. To an observer who
believes in conservation of momentum and knows only about
the overall motion of the objects and not about their heat
content, the low velocity after the collision would seem
to be the result of a magical change in the mass, as if the mass
of two combined, hot blobs of putty was more than the sum of
their individual masses.

Now we know that the masses of all the atoms in the blobs
must be the same as they always were. The change is due to
the change in $\mygamma$ with heating, not to a change in mass.
The heat energy, however, seems to be acting as if it was
equivalent to some extra mass.


But this whole argument was based on the fact that heat is a
form of kinetic energy at the atomic level. Would $E=mc^2$
apply to other forms of energy as well? Suppose a rocket
ship contains some electrical energy stored in a
battery. If we believed that $E=mc^2$ applied to forms of
kinetic energy but not to electrical energy, then
we would have to believe that the pilot of the rocket could
slow the ship down by using the battery to run a heater!
This would not only be strange, but it would violate the
principle of relativity, because the result of the
experiment would be different depending on whether the ship
was at rest or not. The only logical conclusion is that all
forms of energy are equivalent to mass. Running the heater
then has no effect on the motion of the ship, because the
total energy in the ship was unchanged; one form of energy (electrical)
was simply converted to another (heat).

The equation $E=mc^2$
tells us how much energy is equivalent to how much mass: the conversion factor is the square
of the speed of light, $c$. Since $c$ a big number, you get a really really big number
when you multiply it by itself to get $c^2$. This means that even a small amount of mass
is equivalent to a very large amount of energy. 

<%
  fig(
    'eclipse',
    %q{Example \ref{eg:eclipse}, page \pageref{eg:eclipse}.},
    {
      'width'=>'wide'
    }
  )
%>

\begin{eg}{Gravity bending light}\label{eg:eclipse}
Gravity is a universal attraction between things that have mass, and since the energy
in a beam of light is equivalent to some very small amount of mass, we expect that
light will be affected by gravity, although the effect should be very small.
The first important experimental confirmation of relativity
came in 1919 when stars next to the sun during a solar eclipse were
observed to have shifted a little from their ordinary
position. (If there was no eclipse, the glare of the sun
would prevent the stars from being observed.) Starlight had
been deflected by the sun's gravity. Figure \figref{eclipse} is a
photographic negative, so the circle that appears bright is actually the
dark face of the moon, and the dark area is really the bright corona of
the sun. The stars, marked by lines above and below them, appeared at
positions slightly different than their normal ones.
\end{eg}
<% marg(90) %>
<%
  fig(
    'newspaper-eclipse',
    %q{%
      A New York Times headline from November 10, 1919, describing
      the observations discussed in example \ref{eg:eclipse}.
    }
  )
%>
<% end_marg %>

\begin{eg}{Black holes}\index{black hole}
A star with sufficiently strong gravity can prevent light
from leaving. Quite a few black holes have been detected via
their gravitational forces on neighboring stars or clouds of gas and dust.
\end{eg}


You've learned about conservation of mass and conservation of energy, but
now we see that they're not even separate conservation laws.
As a consequence of the theory of relativity,  mass and energy are equivalent, and
are not separately conserved --- one can be converted into the other. Imagine that
a magician waves his wand, and changes a bowl of dirt into a bowl of lettuce. You'd be
impressed, because you were expecting that both dirt and lettuce would be conserved
quantities. Neither one can be made to vanish, or to appear out of thin air. However,
there are processes that can change one into the other. A farmer changes dirt into
lettuce, and a compost heap changes lettuce into dirt. At the most fundamental
level, lettuce and dirt aren't really different things at all; they're just collections
of the same kinds of atoms --- carbon, hydrogen, and so on.
Because mass and energy are like two different sides of the same coin, we may speak of
mass-energy, a single conserved quantity, found by adding up all the mass and energy,
with the appropriate conversion factor: $E+mc^2$.\index{mass-energy!conservation of}

\begin{eg}{A rusting nail}\label{eg:rustingnail}
\egquestion
An iron nail is left in a cup of water
until it turns entirely to rust. The energy released is
about 0.5 MJ. In theory, would a sufficiently
precise scale register a change in mass? If so, how much?

\eganswer
 The energy will appear as heat, which will be lost
to the environment. The total mass-energy of the cup,
water, and iron will indeed be lessened by 0.5 MJ. (If it
had been perfectly insulated, there would have been no
change, since the heat energy would have been trapped in the
cup.) The speed of light is
$c=3\times10^8$ meters per second, so converting to mass units, we have
\begin{align*}
                m         &=    \frac{E}{c^2}  \\
                        &= \frac{0.5\times10^6\ \junit}{\left(3\times10^8\ \munit/\sunit\right)^2} \\
                         &=    6\times10^{-12}\  \text{kilograms}\eqquad.
\end{align*}
The change in mass is too small to measure with any
practical technique. This is because the square of the speed
of light is such a large number.
\end{eg}

\begin{eg}{Electron-positron annihilation}\label{eg:eplus-eminus}\index{positron}
Natural radioactivity in the earth produces positrons, which are like electrons but have the
opposite charge. A form of antimatter, positrons annihilate with electrons to produce gamma
rays, a form of high-frequency light. Such a process would have been considered impossible
before Einstein, because conservation of mass and energy were believed to be separate
principles, and this process eliminates 100\% of the original mass. The amount of energy
produced by annihilating 1 kg of matter with 1 kg of antimatter is
\begin{align*}
 E &= mc^2\\
   &= (2\ \kgunit)\left(3.0\times10^8\ \munit/\sunit\right)^2\\
   &= 2\times10^{17}\ \junit\eqquad,
\end{align*}
which is on the same order of magnitude as a day's energy consumption for the
entire world's population!

Positron annihilation forms the basis for the medical imaging technique called
a PET (positron emission tomography) scan, in which a positron-emitting chemical
is injected into the patient and map\-ped by the emission of gamma rays from the parts
of the body where it accumulates.
\end{eg}
<% marg(200) %>
<%
  fig(
    'pet',
    %q{Top: A PET scanner. Middle: Each positron annihilates with an electron, producing two gamma-rays that fly off back-to-back.
       When two gamma rays are observed simultaneously in the ring of detectors, they are assumed to come from the same
       annihilation event, and the point at which they were emitted must lie on the line connecting the two detectors.
       Bottom: A scan of a person's torso. The body has concentrated the radioactive tracer around the stomach, indicating
       an abnormal medical condition.}
  )
%>
<% end_marg %>

One commonly hears some misinterpretations of $E=mc^2$, one being that the equation tells us
how much kinetic energy an object would have if it was moving at the speed of light. This
wouldn't make much sense, both because the equation for kinetic energy has $1/2$ in it, $KE=(1/2)mv^2$, and
because a material object can't be made to move at the speed of light. However, this naturally leads to the
question of just how much mass-energy a moving object has. We know that when the object is at rest, it
has no kinetic energy, so its mass-energy is simply equal to the energy-equivalent of its mass, $mc^2$,
\begin{equation*}
  \massenergy = mc^2 \ \text{when}\ v=0\eqquad,
\end{equation*}
where the symbol $\massenergy$ (cursive ``E'') stands for mass-energy. The point of using the new symbol is simply
to remind ourselves that we're talking about relativity, so an object at rest has $\massenergy=mc^2$, not
$E=0$ as we'd assume in nonrelativistic physics.

Suppose we start accelerating the object with a constant force. A constant force means a constant
rate of transfer of momentum, but $p=m\mygamma v$ approaches infinity as $v$ approaches $c$, so the object
will only get closer and closer to the speed of light, but never reach it. Now what about the work being
done by the force? The force keeps doing work and doing work, which means that we keep on using up
energy. Mass-energy is conserved, so the energy being expended must equal the increase in the object's
mass-energy. We can continue this process for as long as we like, and the amount of mass-energy
will increase without limit. We therefore conclude that an object's mass-energy approaches infinity
as its speed approaches the speed of light,
\begin{equation*}
  \massenergy \rightarrow \infty\ \text{when}\ v \rightarrow c\eqquad.
\end{equation*}

\index{mass-energy!of a moving particle}
Now that we have some idea what to expect, what is the actual equation for the mass-energy? 
As proved in section \ref{subsec:rel-dynamics-proofs}, it is
\begin{equation*}
  \massenergy =m\mygamma c^2\eqquad.
\end{equation*}

<% self_check('mass-energy',<<-'SELF_CHECK'
Verify that this equation has the two properties we wanted.
  SELF_CHECK
  ) %>

\begin{eg}{KE compared to $mc^2$ at low speeds}\label{eg:massenergy-low-speed}
\egquestion An object is moving at ordinary nonrelativistic speeds. Compare its
kinetic energy to the energy $mc^2$ it has purely because of its mass.

\eganswer The speed of light is a very big number, so $mc^2$ is a huge number of
joules. The object has a gigantic amount of energy because of its mass, and only
a relatively small amount of additional kinetic energy because of its motion.

Another way of seeing this is that at low speeds, $\mygamma$ is only a tiny bit
greater than 1, so $\massenergy$ is only a tiny bit greater than $mc^2$.
\end{eg}

\begin{eg}{The correspondence principle for mass-energy}\index{mass-energy!correspondence principle}\index{correspondence principle!for mass-energy}
\egquestion Show that the equation $\massenergy=m\mygamma c^2$ obeys the correspondence principle.

\eganswer As we accelerate an object from rest, its mass-energy becomes greater than
its resting value. Nonrelativistically, we interpret this excess mass-energy as the object's
kinetic energy,
\begin{align*}
  KE   &= \massenergy(v)-\massenergy(v=0) \\
       &= m\mygamma c^2 - m c^2 \\
       &= m(\mygamma-1)c^2\eqquad.
\end{align*}
Expressing $\mygamma$ as $\left(1-v^2/c^2\right)^{-1/2}$ and making use of the
approximation $(1+\epsilon)^p\approx 1+p\epsilon$ for small $\epsilon$, we have
$\mygamma\approx 1+v^2/2c^2$, so
\begin{align*}
  KE   &\approx m(1+\frac{v^2}{2c^2}-1)c^2 \\
       &= \frac{1}{2}mv^2\eqquad,
\end{align*}
which is the nonrelativistic expression. As demanded by the correspondence principle,
relativity agrees with newtonian physics at speeds that are small compared to
the speed of light.
\end{eg}

<% end_sec() %> % Equivalence of mass and energy
<% begin_sec("The energy-momentum four-vector",nil,'p-four-vector',{'optional'=>true}) %>\index{momentum!relativistic}\index{four-vector!energy-momentum}\index{energy-momentum four vector}
Starting from $\massenergy=m\mygamma$ and $p=m\mygamma v$, a little algebra allows one to prove the identity
\begin{equation*}
  m^2 = \massenergy^2 - p^2\eqquad.
\end{equation*}
We can define an energy-momentum four-vector,
\begin{equation*}
  \vc{p} = (\massenergy,p_x,p_y,p_z)\eqquad,
\end{equation*}
and the relation $m^2 = \massenergy^2 - p^2$ then arises from the inner product $\vc{p}\cdot\vc{p}$.
Since $\massenergy$ and $p$ are separately conserved, the energy-momentum four-vector is also conserved.

A high-precision test of this fundamental relativistic relationship was carried out by Meyer \emph{et al.}
in 1963 by studying the motion of electrons in static electric and magnetic fields. They define the quantity
\begin{equation*}
  Y^2 = \frac{\massenergy^2}{m^2+p^2},
\end{equation*}
which according to special relativity should equal 1. Their results, tabulated in the sidebar,
show excellent agreement with theory.

<% marg(140) %>
\noindent\emph{Results from Meyer et al., 1963}

\begin{tabular}{ll}
  $v$ & $Y$ \\
  0.9870 & 1.0002(5) \\
  0.9881 & 1.0012(5) \\
  0.9900 & 0.9998(5)
\end{tabular}
<% end_marg %>

\begin{eg}{Energy and momentum of light}\label{eg:light-p-from-four-vector}
Light has $m=0$ and $\gamma=\infty$, so if we try to apply $\massenergy=m\mygamma$ and $p=m\mygamma v$ to light,
or to any massless particle, we get the indeterminate form $0\cdot\infty$, which can't be evaluated without
a delicate and laborious evaluation of limits as in problem \ref{hw:ultrarelativistic}
on p.~\pageref{hw:ultrarelativistic}.

Applying $m^2 = \massenergy^2 - p^2$ yields the same result, $\massenergy=|p|$, much more easily. This example
demonstrates that although we encountered the relations $\massenergy=m\mygamma$ and $p=m\mygamma v$ first,
the identity $m^2 = \massenergy^2 - p^2$ is actually more fundamental.

Figure \figref{nichols-radiometer} on p.~\pageref{fig:nichols-radiometer} shows an experiment
that verified $\massenergy=|p|$ empirically.
\end{eg}

For the reasons given in example \ref{eg:light-p-from-four-vector}, we take
$m^2 = \massenergy^2 - p^2$ to be the \emph{definition} of mass in relativity.
One thing to be careful about is that this definition is not additive. Suppose that we lump
two systems together and call them one big system, adding their mass-energies and momenta.
When we do this, the mass of the combination is not the same as  the sum of the masses.
For example, suppose we have two rays of light moving in opposite directions, with
energy-momentum vectors $(\massenergy,\massenergy,0,0)$ and $(\massenergy,-\massenergy,0,0)$.
Adding these gives $(2\massenergy,0,0)$, which implies a mass equal to $2\massenergy$. In
fact, in the early universe, where the density of light was high, the universe's ambient
gravitational fields were mainly those caused by the light it contained.

\begin{eg}{Mass-energy, not energy, goes in the energy-momentum four-vector}
When we say that something is a four-vector, we mean that it behaves properly under a Lorentz transformation:
we can draw such a four-vector on graph paper, and then when we change frames of reference, we should be able
to measure the vector in the new frame of reference by using the new version of the graph-paper grid derived
from the old one by a Lorentz transformation.

If we had used the energy $E$ rather than the mass-energy $\massenergy$ to
construct the energy-momentum four-vector, we wouldn't have gotten a valid four-vector.
An easy way to see this is to consider the case where a noninteracting object is at rest in some frame of reference.
Its momentum and kinetic energy are both zero.
If we'd defined $\vc{p}=(E,p_x,p_y,p_z)$ rather than $\vc{p}=(\massenergy,p_x,p_y,p_z)$, we would have had $\vc{p}=0$ in this
frame. But when we draw a zero vector, we get a point, and a point remains a point regardless of how
we distort the graph paper we use to measure it. That wouldn't have made sense, because in other frames
of reference, we  have $E\ne 0$.
\end{eg}

\begin{eg}{Metric units}
The relation $  m^2 = \massenergy^2 - p^2 $ is only valid in relativistic units. If we tried to apply it without
modification to numbers expressed in metric units, we would have
\begin{equation*}
  \kgunit^2 = \kgunit^2\unitdot\frac{\munit^4}{\sunit^4} - \kgunit^2\unitdot\frac{\munit^2}{\sunit^2}\eqquad,
\end{equation*}
which would be nonsense because the three terms all have different units. As usual, we need to insert factors
of $c$ to make a metric version, and these factors of $c$ are determined by the need to fix the broken units:
\begin{equation*}
  m^2c^4 = \massenergy^2 - p^2c^2
\end{equation*}
\end{eg}

\begin{eg}{Pair production requires matter}\label{eg:no-pair-prod-in-vacuum}
Example \ref{eg:eplus-eminus} on p.~\pageref{eg:eplus-eminus} discussed the annihilation of
an electron and a positron into two gamma rays, which is an example of turning matter into
pure energy. An opposite example is pair production,\index{pair production}\index{gamma ray!pair production}
a process in which a gamma ray disappears, and its energy goes into creating an electron and a positron.

Pair production cannot happen in a vacuum. For example, gamma rays from distant black holes
can travel through empty space for thousands of years before being detected on earth, and
they don't turn into electron-positron pairs before they can get here. Pair production can only
happen in the presence of matter. When lead is used as shielding against gamma rays, one of the
ways the gamma rays can be stopped in the lead is by undergoing pair production.

To see why pair production is forbidden in a vacuum, consider the process in the frame of reference
in which the electron-positron pair has zero total momentum. In this frame, the gamma ray would have to
have had zero momentum,
but a gamma ray with zero momentum must have zero energy as well (example \ref{eg:light-p-from-four-vector}).
This means that conservation of 
\emph{four}-momentum
has been violated: the timelike component of the four-momentum is the mass-energy, and it has increased
from 0 in the initial state to at least $2mc^2$ in the final state.
\end{eg}

<% end_sec() %> % The energy-momentum four-vector
<% begin_sec("Proofs",4,'rel-dynamics-proofs',{'optional'=>true}) %>
This optional section proves some results claimed earlier.
<% begin_sec("Ultrarelativistic motion") %>
We start by considering the case of a particle, described as ``ultrarelativistic,''
that travels at very close to the speed of light.
A good way of thinking about such a particle is that it's one with a
very small mass. For example, the subatomic particle called the neutrino has a very small
mass, thousands of times smaller than that of the electron. Neutrinos are emitted in
radioactive decay, and because the neutrino's mass is so small, the amount of energy
available in these decays is always enough to accelerate it to very close to the speed
of light. Nobody has ever succeeded in observing a neutrino that was \emph{not} ultrarelativistic.
When a particle's mass is very small, the mass becomes difficult to measure. For almost 70 years after the
neutrino was discovered, its mass was thought to be zero. Similarly, we currently believe that
a ray of light has no mass, but it is always possible that its mass will be found to be nonzero
at some point in the future. A ray of light can be modeled as an ultrarelativistic particle.

Let's compare ultrarelativistic particles with train cars. A single car with kinetic energy $E$ has
different properties than a train of two cars each with kinetic energy $E/2$. The single car has
half the mass and a speed that is greater by a factor of $\sqrt{2}$. But the same is not true
for ultrarelativistic particles. Since an idealized ultrarelativistic particle has a mass too
small to be detectable in any experiment, we can't detect the difference between $m$ and $2m$.
Furthermore, ultrarelativistic particles move at close to $c$, so there is no observable
difference in speed. Thus we expect that a single ultrarelativistic particle with energy $E$
compared with  two such particles, each with energy $E/2$,
should have all the same properties as measured by a mechanical detector.

An idealized zero-mass particle also has no frame in which it can be at rest. It
always travels at $c$, and no matter how fast we chase after it, we can never catch up.
We can, however, observe it in different frames of reference, and we will find that its
energy is different. For example, distant galaxies are receding from us at substantial fractions
of $c$, and when we observe them through a telescope, they appear very dim not just because they are very
far away but also because their light has less energy in our frame than in a frame at rest
relative to the source. This effect must be such that changing frames of reference according
to a specific Lorentz transformation always changes the energy of the particle by a fixed factor,
regardless of the particle's original energy;
for if not, then the effect of a Lorentz transformation on a single particle of energy $E$
would be different from its effect on two particles of energy $E/2$.

How does this energy-shift factor depend on the velocity $v$ of the Lorentz transformation?
Rather than $v$, it becomes more convenient to express things in terms of the Doppler shift factor $D$,
which multiplies when we change frames of reference.
Let's write $f(D)$ for the energy-shift factor that results from a given Lorentz transformation.
Since a Lorentz transformation $D_1$ followed by a second transformation $D_2$ is equivalent
to a single transformation by $D_1D_2$, we must have $f(D_1D_2)=f(D_1)f(D_2)$. This tightly
constrains the form of the function $f$; it must be something like $f(D)=s^n$, where $n$
is a constant. We postpone until p.~\pageref{pesky-exponent-proof} the proof that $n=1$, which is also in agreement with experiments with rays of light.\label{pesky-exponent-claim}

Our final result is that the energy of an ultrarelativistic particle is simply proportional
to its Doppler shift factor $D$. Even in the case where the particle is truly massless,
so that $D$ doesn't have any finite value, we can still
find how the energy differs according to different observers by finding the $D$ of the
Lorentz transformation between the two observers' frames of reference.
<% end_sec() %> % Ultrarelativistic motion
<% begin_sec("Energy") %>
The following argument is due to Einstein. Suppose that a material object O of mass $m$,
initially at rest in a certain frame A, emits two rays of light, each with energy $E/2$. By
conservation of energy, the object must have lost an amount of energy equal to $E$.
By symmetry, O remains at rest.

We now switch to a new frame of reference moving at a certain velocity $v$ in the $z$ direction relative to the original frame.
We assume that O's energy is different in this frame, but that the change in its energy amounts to multiplication by some unitless factor $x$,
which depends only on $v$, since there is nothing else it could depend on that could allow us to form a unitless quantity. In this frame the light rays have energies $ED(v)$ and
$ED(-v)$. If conservation of energy is to hold in the new frame as it did in the old, we must have $2xE=ED(v)+ED(-v)$.
After some algebra, we find $x=1/\sqrt{1-v^2}$, which we recognize as $\gamma$. This proves that
$E=m\gamma$ for a material object.
<% end_sec() %> % Energy
<% marg(-50) %>
<%
  fig(
    'e-p-plane',
    %q{In the $p$-$E$ plane, massless particles lie on the two diagonals, while particles with mass lie to the right.}
  )
%>
<% end_marg %>
<% begin_sec("Momentum") %>
We've seen that ultrarelativistic particles are ``generic,''
in the sense that they have no individual mechanical properties other than an energy and a direction
of motion. Therefore the relationship between energy and momentum must be \emph{linear}
for ultrarelativistic particles. Indeed, experiments verify that light has momentum, and
doubling the energy of a ray of light doubles its momentum rather than quadrupling it.
On a graph of $p$ versus $E$, massless particles, which have $E\propto|p|$, lie on two diagonal lines that connect at the
origin. If we like, we can pick units such that the slopes of these lines are plus and minus one. Material particles
lie to the right of these lines. For example, a car sitting in a parking lot has $p=0$ and $E=mc^2$. 

Now what happens to such a graph when we change to a different frame or reference that is in motion relative to the
original frame? A massless particle still has to act like a massless particle, so the diagonals are simply stretched
or contracted along their own lengths.
In fact the transformation must be linear (p.~\pageref{fig:nonlinear-transformation}), because conservation of
energy and momentum involve addition, and we need these laws to be valid in all frames of reference.
By the same reasoning as in figure \figref{area-proof} on p.~\pageref{fig:area-proof}, the transformation must be area-preserving.
We then have the same three cases to consider as in figure \figref{three-cases} on p.~\pageref{fig:three-cases}.
Case I is ruled out because it would imply that particles keep the same energy when we change frames.
(This is what would happen if $c$ were infinite, so that the mass-equivalent $E/c^2$ of a given energy was zero,
and therefore $E$ would be interpreted purely as the mass.) Case II can't be right because it doesn't preserve
the $E=|p|$ diagonals. We are left with case III, which establishes the fact that
the $p$-$E$ plane transforms according to exactly the same kind of Lorentz transformation as the $x$-$t$ plane.
That is, $(E,p_x,p_y,p_z)$ is a four-vector.

The only remaining issue to settle is whether the choice of units that gives invariant 45-degree diagonals in the
$x$-$t$ plane is the same as the choice of units that gives such diagonals in the $p$-$E$ plane.
That is, we need to establish that the $c$ that applies to $x$ and $t$ is equal to the $c'$ needed for $p$ and $E$,
i.e., that the velocity scales of the two graphs are matched up.
This is true because in the Newtonian limit, the total mass-energy $E$ is essentially just the particle's mass,
and then $p/E \approx p/m \approx v$. This establishes that the velocity scales are matched at small velocities,
which implies that they coincide for all velocities, since a large velocity, even one approaching $c$,
can be built up from many small increments. (This also establishes that the exponent $n$ defined on p.~\pageref{pesky-exponent-claim}
equals 1 as claimed.)\label{pesky-exponent-proof}

Since $m^2=E^2-p^2$, it follows that for a material particle, $p=m\gamma v$.
\vfill
<% end_sec() %> % Momentum
<% end_sec() %> % Proofs

<% end_sec() %> % Dynamics
