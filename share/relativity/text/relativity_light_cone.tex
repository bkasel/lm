<% begin_sec("The light cone") %>

Given an event P, we can now classify all the causal relationships in which P can participate.
In Newtonian physics, these relationships fell into two classes: P could potentially cause any event
that lay in its future, and could have been caused by any event in its past.
In relativity, we have a three-way distinction rather than a two-way one. There is a third class
of events that are too far away from P in space, and too close in time, to allow any cause and effect
relationship, since causality's maximum velocity is $c$. Since we're working in units in which $c=1$,
the boundary of this set is formed by the lines with slope $\pm1$ on a $(t,x)$ plot. This is referred
to as the light cone,\index{light cone} for reasons that become more visually obvious when we consider
more than one spatial dimension, figure \figref{lightcone}.
<% marg(50) %>
<%
  fig(
    'lightcone',
    %q{The light cone.}
  )
%>
<% end_marg %>

Events lying inside one another's light cones are said to have a timelike relationship. Events outside each
other's light cones are spacelike in relation to one another, and in the case where they lie on the surfaces
of each other's light cones the term is lightlike.\index{spacelike}\index{timelike}\index{lightlike}
<% end_sec %> % The light cone
<% begin_sec("The spacetime interval",nil,'',{'optional'=>true}) %>
The light cone is an object of central importance in both special and general relativity. It relates
the \emph{geometry} of spacetime to possible \emph{cause-and-effect} relationships between events. This
is fundamentally how relativity works: it's a geometrical theory of causality. 

These ideas naturally lead
us to ask what fruitful analogies we can form between the bizarre geometry of spacetime and the
more familiar geometry of the Euclidean plane. The light cone cuts spacetime into different regions according
to certain measurements of relationships between points (events). Similarly, a circle in Euclidean geometry
cuts the plane into two parts, an interior and an exterior, according to the measurement of the distance from
the circle's center. A circle stays the same when we rotate the plane. A light cone stays the same when we
change frames of reference. Let's build up the analogy more explicitly.

\begin{lessimportant}[Measurement in Euclidean geometry]
We say that two line segments are congruent, $\zu{AB}\cong \zu{CD}$, if the distance between points A and B is the same
as the distance between C and D, as measured by a rigid ruler.
\end{lessimportant}

\begin{lessimportant}[Measurement in spacetime]
We define $\zu{AB}\cong \zu{CD}$ if:
\begin{enumerate}
\item AB and CD are both spacelike, and the two distances are equal as measured by a rigid ruler, in a frame
where the two events touch the ruler simultaneously.
\item AB and CD are both timelike, and the two time intervals are equal as measured by clocks moving inertially.
\item AB and CD are both lightlike.
\end{enumerate}
\end{lessimportant}

The three parts of the relativistic version each require some justification.

Case 1 has to be the way it is because
space is part of spacetime. In special relativity, this space is Euclidean, so the definition of congruence has
to agree with the Euclidean definition, in the case where it is possible to apply the Euclidean definition.
The spacelike relation between the points is both necessary and sufficient to make this possible.
If points A and B are spacelike in relation to one another, then a frame of reference exists in which
they are simultaneous, so we can use a ruler that is at rest in that frame to measure their distance.
If they are lightlike or timelike, then no such frame of reference exists. For example, there is no frame
of reference in which Charles VII's restoration to the throne is simultaneous with Joan of Arc's execution,
so we can't arrange for both of these events to touch the same ruler at the same time.

The definition in case 2 is the only sensible way to proceed if we are to respect the symmetric treatment of
time and space in relativity. The timelike relation between the events is necessary and sufficient to
make it possible for a clock to move from one to the other.
It makes a difference that the clocks move inertially, because the twins
in example \ref{eg:interstellar-road-trip} on p.~\pageref{eg:interstellar-road-trip} disagree on the clock time between the
traveling twin's departure and return.

Case 3 may seem strange, since it says that \emph{any} two lightlike intervals are congruent. But this is the only
possible definition, because this case can be obtained as a limit of the timelike one. Suppose that AB is a timelike interval,
but in the planet earth's frame of reference it would be necessary to travel at almost the speed of light in order
to reach B from A. The required speed is less than $c$ (i.e., less than 1) by some tiny amount $\epsilon$.
In the earth's frame, the clock referred to in the definition suffers extreme time dilation. The time elapsed
on the clock is very small. As $\epsilon$ approaches zero, and the relationship between A and B approaches a lightlike
one, this clock time approaches zero. In this sense, the relativistic notion of ``distance'' is very different from
the Euclidean one. In Euclidean geometry, the distance between two points can only be zero if they are the same
point.

The case splitting involved in the relativistic definition is a little ugly. Having worked out the physical
interpretation, we can now consolidate the definition in a nicer way by appealing to Cartesian coordinates.

\begin{lessimportant}[Cartesian definition of distance in Euclidean geometry]
Given a vector $(\Delta x,\Delta y)$ from point A to point B, the square of the distance between them is defined as 
$\overline{\zu{AB}}^2=\Delta x^2+\Delta y^2$.
\end{lessimportant}

\begin{lessimportant}[Definition of the interval in relativity]
Given points separated by coordinate differences $\Delta x$, $\Delta y$, $\Delta z$, and $\Delta t$,
the spacetime interval $\interval$ (cursive letter ``I'') between them
is defined as $\interval = \Delta t^2-\Delta x^2-\Delta y^2-\Delta z^2$.
\end{lessimportant}

This is stated in natural units, so all four terms on the right-hand side have the same units;
in metric units with $c \ne 1$, appropriate factors of $c$ should be inserted in order to make the units of
the terms agree.
The interval $\interval$ is positive if AB is timelike (regardless of which event comes first),
zero if lightlike, and negative if spacelike. Since $\interval$ can be negative, we can't in general take its
square root and define a real number $\overline{\zu{AB}}$ as in the Euclidean case. When the interval
is timelike, we can interpret $\sqrt{\interval}$ as a time, and when it's spacelike we can take
$\sqrt{-\interval}$ to be a distance.

The Euclidean
definition of distance (i.e., the Pythagorean theorem) is useful because it gives the same answer
regardless of how we rotate the plane. Although it is stated in terms of a certain coordinate system,
its result is unambiguously defined because it is
the same regardless of what coordinate system we arbitrarily pick. Similarly, $\interval$
is useful because, as proved in example \ref{eg:interval-invariance} below, it is the same regardless of our frame of reference, i.e., regardless of our choice
of coordinates.

\begin{eg}{Pioneer 10}\label{eg:pioneer-interval}
\egquestion The Pioneer 10 space probe was launched in 1972, and in 1973 was the first craft to
fly by the planet Jupiter. It crossed the orbit of the planet Neptune in 1983, after which telemetry
data were received until 2002. The following table gives the spacecraft's position relative to the
sun at exactly midnight on January 1, 1983 and January 1, 1995. The 1983 date is taken to be $t=0$.

\begin{tabular}{lllll}
$t$ (s) & $x$ & $y$ & $z$ \\
\hline
0       & $1.784\times10^{12}\ \munit$  &  $3.951\times10^{12}\ \munit$  & $0.237\times10^{12}\ \munit$ \\
$3.7869120000\times10^{8}\ \sunit$ 
               & $2.420\times10^{12}\ \munit$  &  $8.827\times10^{12}\ \munit$  & $0.488\times10^{12}\ \munit$ \\
\end{tabular}

% http://nssdc.gsfc.nasa.gov/nmc/datasetDisplay.do?id=SPHE-00818
%  SCID   YYDDD       NSEC        RSC   ANGL1   ANGL2        XSC        YSC        ZSC       VXSC       VYSC       VZSC        V$
% P10  83001.0        0.0 0.4342E+10   66.15    3.13 0.1784E+10 0.3951E+10 0.2370E+09 0.1969E+01 0.1359E+02 0.7048E+00 0.1377E+$
%
% SCID   YYDDD       NSEC        RSC   ANGL1   ANGL2        XSC        YSC        ZSC       VXSC       VYSC       VZSC        V$
% P10  95001.0        0.0 0.9166E+10    0.00    0.00 0.2420E+10 0.8827E+10 0.4880E+09 0.1525E+01 0.1243E+02 0.6374E+00 0.0000E+$
% ftp://nssdcftp.gsfc.nasa.gov/spacecraft_data/pioneer/pioneer10/plasma/version31/v31_traj_ascii/p10platr_1983.asc
% ftp://nssdcftp.gsfc.nasa.gov/spacecraft_data/pioneer/pioneer10/plasma/version31/v31_traj_ascii/p10platr_1995.asc
% ftp://nssdcftp.gsfc.nasa.gov/spacecraft_data/pioneer/pioneer10/plasma/version31/v31_traj_ascii/formattr.sfdu
%  ... states RSC to be r of spacecraft in km, so presumably x, y, and z are in km
%
% when j --now="1983 jan 1" ; when j --now="1995 jan 1" ; calc -e "(49719-45336)*24*3600"
%     3.786912*10^8

Compare the time elapsed on the spacecraft to the time in a frame of reference tied to the sun.

\eganswer
We can convert these data into natural units, with the distance unit being the second (i.e., a light-second,
the distance light travels in one second) and the time unit being seconds. Converting and carrying out this
subtraction, we have:

% calc -x -e "k=10^12 m; (2.420-1.784)k/c; (8.827-3.951)k/c; (0.488-0.237)k/c"
%     2.12146764546025*10^3 s
%     1.62645852818619*10^4 s
%     837.245878947364 s

\begin{tabular}{llll}
$\Delta t$ (s) & $\Delta x$ & $\Delta y$ & $\Delta z$ \\
\hline
$3.7869120000\times10^{8}\ \sunit$ 
               & $0.2121\times10^{4}\ \sunit$  &  $1.626\times10^{4}\ \sunit$  & $0.084\times10^{4}\ \sunit$ \\
\end{tabular}

Comparing the exponents of the temporal and spatial numbers, we can see that the spacecraft was moving at
a velocity on the order of $10^{-4}$ of the speed of light, so relativistic effects should be small but
not completely negligible.

Since the interval is timelike, we can take its square root and interpret it as the time elapsed on the spacecraft.
The result is $\sqrt{\interval}=3.786911996\times 10^8\ \sunit$. This is 0.4 s less than the time elapsed in the
sun's frame of reference.

% calc -e "sqrt[(3.78691200 10^8)^2-(.2121^2+1.626^2+.084^2)*10^8]; ~-3.7869120000 10^8"    
%   3.7869119964405*10^8


\end{eg}

<%
  fig('light-rectangles',
    %q{Light-rectangles, example \ref{eg:interval-invariance}.\\\\
       1.~The gray light-rectangle represents the set of all events such as P that could be visited after A and before B.\\\\
       2.~The rectangle becomes a square in the frame in which A and B occur at the same location in space.\\\\
       3.~The area of the dashed square is $\tau^2$, so the area of the gray square is $\tau^2/2$.},
    {
      'width'=>'wide',
      'sidecaption'=>true
    }
  )
%>

\enlargethispage{-\baselineskip}

\begin{eg}{Invariance of the interval}\label{eg:interval-invariance}
In this example we prove that the interval is the same regardless of what frame of reference we compute it in.
This is called ``Lorentz invariance.''\index{Lorentz invariance}
The proof is limited to the timelike case.
Given events A and B, construct the light-rectangle as defined in
figure \subfigref{light-rectangles}{1}.
On p.~\pageref{fig:area-proof} we proved that the Lorentz 
transformation doesn't change the area of a shape in the $x$-$t$ plane.
Therefore the area of this rectangle is unchanged if we switch to the frame of reference \subfigref{light-rectangles}{2},
in which A and B occurred at the same location and were separated by a time interval $\tau$. This area equals
half the interval $\interval$ between A and B. But a straightforward calculation shows that the rectangle
in \subfigref{light-rectangles}{1} also has an area equal to half the interval calculated in \emph{that} frame.
Since the area in any frame equals half the interval, and the area is the same in all frames, the interval is equal in
all frames as well.
\end{eg}

<%
  fig(
    'eg-interval-graphs',
    %q{%
      Example \ref{eg:interval-graphs}.
    },
    {
      'sidecaption'=>true,
      'width'=>'wide'
    }
  )
%>

\begin{eg}{A numerical example of invariance}\label{eg:interval-graphs}
Figure \figref{eg-interval-graphs} shows two frames of reference in motion relative to one another
at $v=3/5$. (For this velocity, the stretching and squishing of the main diagonals are both by a factor of
2.) Events are marked at coordinates that in the frame represented by the square are
\begin{align*}
  (t,x) & = (0,0) \qquad \text{and} \\
  (t,x) &=  (13,11)\eqquad.
\end{align*}
The interval between these events is $13^2-11^2=48$. In the frame represented by the parallelogram, the same
two events lie at coordinates
\begin{align*}
  (t',x') & = (0,0) \qquad \text{and} \\
  (t',x') &=  (8,4)\eqquad.
\end{align*}
Calculating the interval using these values, the result is \linebreak[4]
$8^2-4^2=48$, which comes out the same as in the other frame.
\end{eg}



<% end_sec %> % The spacetime interval
<% begin_sec("Four-vectors and the inner product",4,'',{'optional'=>true}) %>
Example \ref{eg:pioneer-interval} makes it natural that we define a type of vector with four components,
the first one relating to time and the others being spatial. These are known as four-vectors.\index{four-vector}\index{vector!four-vector}
It's clear how we should define the equivalent of a dot product in
relativity:\index{dot product!relativistic}
\begin{equation*}
  \vc{A}\cdot\vc{B} = A_t B_t - A_xB_x - A_yB_y - A_zB_z
\end{equation*}
The term ``dot product'' has connotations of referring only to 
three-vectors, so the operation of taking the scalar product of two four-vectors is usually referred to
instead as the ``inner product.''
The spacetime interval can then be thought of as the inner product of a four-vector with itself.\index{inner product}
We care about the relativistic inner product for exactly the same reason we care about its Euclidean
version; both are scalars, so they have a fixed value regardless of what coordinate system we choose.

\vspace{10mm}

\begin{eg}{The twin paradox}\label{eg:dot-twin-paradox}\index{triangle inequality}\index{twin paradox}
Alice and Betty are identical twins. Betty goes on a space voyage at relativistic speeds, traveling away from the
earth and then turning around and coming back. Meanwhile, Alice stays on earth. When Betty returns, she is younger
than Alice because of relativistic time dilation 
(example \ref{eg:interstellar-road-trip}, p.~\pageref{eg:interstellar-road-trip}). 

But
isn't it valid to say that Betty's spaceship is standing still
and the earth moving? In that description, wouldn't Alice end up younger and Betty older?
This is referred to as the
``twin paradox.'' It can't really be a paradox, since it's exactly what was observed in the
Hafele-Keating experiment (p. \pageref{hafele-keating-discussed}).

Betty's track in the $x$-$t$ plane
(her ``world-line'' in relativistic jargon\index{world-line})
consists of vectors $\vc{b}$ and $\vc{c}$ strung end-to-end
(figure \figref{dot-twin-paradox}).
We could adopt a frame of reference in which Betty was at rest during $\vc{b}$ (i.e., $b_x=0$), but there
is no frame in which $\vc{b}$ and $\vc{c}$ are parallel, so there is no frame in which Betty was at rest
during \emph{both} $\vc{b}$ and $\vc{c}$. This resolves the paradox.
<% marg(79) %>
<%
  fig(
    'dot-twin-paradox',
    %q{Example \ref{eg:dot-twin-paradox}.}
  )
%>
<% end_marg %>

We have already established by other methods that Betty ages less that Alice, but let's see how this plays
out in a simple numerical example. Omitting units and making up simple numbers, let's say that the vectors
in figure \figref{dot-twin-paradox} are
\begin{align*}
  \vc{a} &= (6,1) \\
  \vc{b} &= (3,2) \\
  \vc{c} &= (3,-1)\eqquad,
\end{align*}
where the components are given in the order $(t,x)$. The time experienced by Alice is then
\begin{equation*}
  |\vc{a}| = \sqrt{6^2-1^2} =5.9\eqquad,
\end{equation*}
which is greater than the Betty's elapsed time
\begin{equation*}
  |\vc{b}|+|\vc{c}| = \sqrt{3^2-2^2}+\sqrt{3^2-(-1)^2} = 5.1\eqquad.
\end{equation*}


\end{eg}

\begin{eg}{Simultaneity using inner products}\label{eg:dot-simultaneity}
Suppose that an observer O moves inertially along a vector $\vc{o}$, and let the vector separating
two events P and Q be $\vc{s}$. O judges these events to be simultaneous if $\vc{o}\cdot\vc{s}=0$.
To see why this is true, suppose we pick a coordinate system as defined by O. In this coordinate system,
O considers herself to be at rest, so she says her vector has only a time component, 
$\vc{o}=(\Delta t,0,0,0)$. If she considers P and Q to be simultaneous, then the vector from P to Q is
of the form $(0,\Delta x,\Delta y,\Delta z)$. The inner product is then zero, since each of the four terms
vanishes. Since the inner product is independent of the choice of coordinate system, it doesn't matter that we
chose one tied to O herself. Any other observer $\zu{O}'$ can look at O's motion, note that $\vc{o}\cdot\vc{s}=0$,
and infer that O must consider P and Q to be simultaneous, even if $\zu{O}'$ says they weren't.
\end{eg}
<% marg(65) %>
<%
  fig(
    'dot-simultaneity',
    %q{Example \ref{eg:dot-simultaneity}.}
  )
%>
<% end_marg %>


<% end_sec %> % Four-vectors and the inner product
<% begin_sec("Doppler shifts of light and addition of velocities",nil,'doppler-light',{'optional'=>true}) %>\index{light!Doppler shift for}\index{Doppler shift!for light}
When Doppler shifts happen to ripples on a pond or the sound waves from an airplane, they can depend on the relative motion of three
different objects: the source, the receiver, and the medium. But light waves don't have a medium. Therefore Doppler shifts of
light can only depend on the relative motion of the source and observer.

<% marg(0) %>
<%
  fig(
    'doppler',
    %q{%
      The pattern of waves made by a point source moving to the right
      across the water. Note the shorter wavelength of the forward-emitted waves and the longer
      wavelength of the backward-going ones.
    },
    {'suffix'=>'2'} # doesn't work, gets labeled by subsec number ... why?
  )
%>
<% end_marg %>

One simple case is the one in which the relative motion of the source and the receiver is perpendicular to the line connecting them.
That is, the motion is transverse.
Nonrelativistic Doppler shifts happen because the distance between the source and receiver is changing, so in nonrelativistic
physics we don't expect any Doppler shift at all when the motion is transverse, and this is what is in fact observed to high precision.
For example, the photo
% Can't get it to label correctly, so can't do figref.
shows shortened and lengthened wavelengths to the right and left, along the source's line of motion,
but an observer above or below the source measures just the normal, unshifted wavelength and frequency. But relativistically, we have
a time dilation effect, so for light waves emitted transversely, there is a Doppler shift of $1/\mygamma$ in frequency (or $\mygamma$ in wavelength).

The other simple case is the one in which the relative motion of the source and receiver is longitudinal, i.e., they are either approaching or
receding from one another. For example, distant galaxies are receding from our galaxy due to the expansion of the universe, and this expansion was
originally detected because Doppler shifts toward the red (low-frequency) end of the spectrum were observed. 

Nonrelativistically, we would expect
the light from such a galaxy to be Doppler shifted down in frequency by some factor, which would depend on the relative velocities of three different
objects: the source, the wave's medium, and the receiver. Relativistically, things get simpler, because light isn't a vibration of a physical medium,
so the Doppler shift can only depend on a single velocity $v$, which is the rate at which the separation between the source and the receiver is
increasing.

The square in figure \figref{doppler-geometry} is the ``graph paper'' used by someone who considers the source to be at rest, while
the parallelogram plays a similar role for the receiver. The figure is drawn for the case where $v=3/5$ (in units where $c=1$),
and in this case the stretch factor of the long diagonal is 2. To keep the area the same, the short diagonal has to be squished to half its
original size. But now it's a matter of simple geometry to show that OP equals half the width of the square, and this tells us that the Doppler
shift is a factor of 1/2 in frequency. That is, the squish factor of the short diagonal is interpreted as the Doppler shift.
To get this as a general equation for velocities other than 3/5, one can show by straightforward fiddling with the result of
part c of problem \ref{hw:gamma-derivation} on p.~\pageref{hw:gamma-derivation} that
the Doppler shift is
\begin{equation*}
  D(v) = \sqrt{\frac{1-v}{1+v}}\eqquad.
\end{equation*}
Here $v>0$ is the case where the source and receiver are getting farther apart, $v<0$ the case where they are approaching.
m4_ifelse(__fac,[::],[:%
  % FAC never does an equation for Doppler shift before the relativity chapter.
  (This is the opposite of the sign convention used in __subsection_or_section(doppler). It is convenient to change conventions here so that we can
  use positive values of $v$ in the case of cosmological red-shifts, which are the most important application.)%
:])
<% marg(100) %>
<%
  fig(
    'six-tenths-c',
    %q{%
      A graphical representation of the Lorentz transformation for a velocity of $(3/5)c$. The long diagonal is stretched by a factor of two, the
         short one is half its former length, and the area is the same as before.
    }
  )
%>
\spacebetweenfigs
<%
  fig(
    'doppler-geometry',
    %q{%
      At event O, the source and the receiver are on top of each other, so as the source emits a wave crest, it is received without any time delay.
      At P, the source emits another wave crest, and at Q the receiver receives it.
    }
  )
%>
<% end_marg %>

Suppose that Alice stays at home on earth while her twin Betty takes off in her rocket ship at 3/5 of the speed of light.
When I first learned relativity, the thing that caused me the most pain was understanding how each observer could say that the
other was the one whose time was slow. It seemed to me that if I could take a pill that would speed up my mind and my body,
then naturally I would see everybody \emph{else} as being \emph{slow}. Shouldn't the same apply to relativity? But suppose Alice and Betty
get on the radio and try to settle who is the fast one and who is the slow one. Each twin's voice sounds slooooowed doooowwwwn
to the other. If Alice claps her hands twice, at a time interval of one second by her clock, Betty hears the hand-claps coming
over the radio two seconds apart, but the situation is exactly symmetric, and Alice hears the same thing if Betty claps.
Each twin analyzes the situation using a diagram identical to \figref{doppler-geometry}, and attributes her sister's
observations to a complicated combination of time distortion, the time taken by the radio signals to propagate, and
the motion of her twin relative to her.

\pagebreak

<% self_check('doppler-approaching',<<-'SELF_CHECK'
Turn your book upside-down and reinterpret figure \figref{doppler-geometry}.
  SELF_CHECK
  ) %>


\begin{eg}{A symmetry property of the Doppler effect}\label{eg:doppler-abc}
Suppose that A and B are at rest relative to one another, but C is moving along the line between A and B. A transmits a signal
to C, who then retransmits it to B. The signal accumulates two Doppler shifts, and the result is their product $D(v)D(-v)$. But
this product must equal 1, so we must have $D(-v)D(v)=1$, which can be verified directly from the equation.
\end{eg}

\begin{eg}{The Ives-Stilwell experiment}\index{Ives-Stilwell experiment}
The result of example \ref{eg:doppler-abc} was the basis of one of the earliest laboratory tests of special relativity, by Ives and Stilwell in 1938.
They observed the light emitted by excited by a beam of $\zu{H}_2^+$ and $\zu{H}_3^+$ ions with speeds of a few tenths of a percent of $c$.
Measuring the light from both ahead of and behind the beams, they found that the product of the Doppler shifts $D(v)D(-v)$ was equal to 1,
as predicted by relativity. If relativity had been false, then one would have expected the product to differ from 1 by an amount that would
have been detectable in their experiment. In 2003, Saathoff et al.~carried out an extremely precise version of the Ives-Stilwell technique
with $\zu{Li}^+$ ions moving at 6.4\% of $c$. The frequencies observed, in units of MHz, were:

\noindent\begin{tabular}{rl}
$f_\zu{o}$ &=      $546466918.8  \pm 0.4$ \\
           & \hfill (unshifted frequency)\\
$f_\zu{o}D(-v)$ & =          $582490203.44 \pm .09$ \\
           & \hfill (shifted frequency, forward)\\
$f_\zu{o} D(v)$ & =          $512671442.9  \pm 0.5$ \\
           & \hfill (shifted frequency, backward)\\
$\sqrt{f_\zu{o}D(-v)\cdot f_\zu{o} D(v)}$ &= $546466918.6  \pm 0.3$ 
\end{tabular}

\noindent The results show incredibly precise agreement between $f_\zu{o}$ and $\sqrt{f_\zu{o}D(-v)\cdot f_\zu{o} D(v)}$, as expected
relativistically because $D(v)D(-v)$ is supposed to equal 1. The agreement extends to 9 significant figures, whereas
if relativity had been false there should have been a relative disagreement of about $v^2=.004$, i.e., a discrepancy in the third significant figure.
The spectacular agreement with theory has made this experiment a lightning rod for
anti-relativity kooks.
\end{eg}

We saw on p.~\pageref{relativistic-combination-of-vel} that relativistic velocities should not be expected to be exactly additive,
and problem \ref{hw:six-tenths-c-twice} on p.~\pageref{hw:six-tenths-c-twice} verifies this in the special case where A moves relative to B
at $0.6c$ and B relative to C at $0.6c$ --- the result \emph{not} being $1.2c$.\index{velocity!addition of!relativistic}
The relativistic Doppler shift provides a simple way of deriving a general equation for the relativistic combination 
of velocities;
problem \ref{hw:rel-vel-addition} on p.~\pageref{hw:rel-vel-addition} guides you through the steps of this derivation,
and the result is given on p.~\pageref{soln:rel-vel-addition}.\index{velocity!addition of!relativistic}
<% end_sec %> % Doppler shifts of light
