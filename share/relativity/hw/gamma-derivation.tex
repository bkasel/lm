In this homework problem, you'll fill in the steps of the algebra required in order
to find the equation for $\mygamma$ on page 
\pageref{m4_ifelse(__lm_series,1,[:sec:],m4_ifelse(__me,1,[:subsubsec:],[:subsec:])):gamma}. To keep the algebra
simple, let the time $t$ in figure \figref{gamma-as-projection} equal 1, as suggested
in the figure accompanying this homework problem. The original
square then has an area of 1, and the transformed parallelogram must also have an area
of 1. (a) Prove that point P is at $x=v\mygamma$,
so that its $(t,x)$ coordinates are $(\mygamma,v\mygamma)$. (b) Find the $(t,x)$ coordinates
of point Q. (c) Find the length of the short diagonal connecting P and Q.
(d) Average the coordinates of P and Q to find the coordinates of the midpoint C
of the parallelogram, and then find distance OC.
(e) Find the area of the parallelogram by computing twice the area of triangle
PQO. [Hint: You can take PQ to be the base of the triangle.] (f) Set this
area equal to 1 and solve for $\mygamma$ to prove $\mygamma=1/\sqrt{1-v^2}$.\answercheck
