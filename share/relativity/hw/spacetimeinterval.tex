(a) Make up a numerical example of two events, and show that if we defined the
spacetime interval as
$\Delta x^2+\Delta y^2+\Delta z^2+\Delta t^2$, we would
not get consistent results when we Lorentz-transformed the events into
a different frame of reference.\hwendpart
(b) Show that, for the particular example you chose in part
a, the quantity $\Delta x^2+\Delta y^2+\Delta z^2-\Delta t^2$
does come out the same in both frames.\hwendpart
(c) Ignoring the $y$ and $z$ space dimensions,
prove that $\Delta x^2-\Delta t^2$ stays the same under
a Lorentz transformation for motion along the $x$ axis.
You're proving this in general now, not just checking it
for one numerical example.\hwendpart
(d) Reexpress the definition
$\Delta x^2+\Delta y^2+\Delta z^2-\Delta t^2$ of the
spacetime interval in unnatural units, where $c\ne1$.
