\index{velocity!addition of!relativistic}
<% hw_solution %> As promised in
m4_ifelse(__lm_series,1,[:section \ref{sec:doppler-and-clock}:],[:subsection \ref{subsec:doppler-light}:]),
this problem will lead you through the steps of
finding an equation for the combination of velocities in relativity, generalizing the numerical result
found in problem \ref{hw:six-tenths-c-twice}. Suppose that A moves relative to B at velocity $u$,
and B relative to C at $v$. We want to find A's velocity $w$ relative to C, in terms of $u$ and $v$.
Suppose that A emits light with a certain frequency. This will be observed by B with a Doppler
shift $D(u)$. C detects a further shift of $D(v)$ relative to B. We therefore expect the Doppler
shifts to multiply, $D(w)=D(u)D(v)$, and this provides an implicit rule for determining $w$
if $u$ and $v$ are known. (a) Using the expression for $D$ given in section \ref{subsec:doppler-light},
write down an equation relating $u$, $v$, and $w$. (b) Solve for $w$ in terms of $u$ and $v$.
(c) Show that your answer to part b satisfies the correspondence principle.
