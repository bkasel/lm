\anchor{anchor-student-preface}
\section*{Preface for the student}% not numbered because in frontmatter; has to be section so page number appears
\addcontentsline{toc}{section}{\protect\link{student-preface}{Preface for the student}}

Electricity and magnetism is more fun than mechanics, but it's
also more mathematical. Most schools have their curriculum set up
so that the typical engineering student takes electricity and magnetism,
or ``E\&M,'' concurrently with a course in vector calculus (a.k.a.~multivariable
calculus). E\&M books normally introduce the same math for the benefit of
students who will not take vector calculus until later, or who will learn
a relevant topic in their math course too late in the semester. That's
what I've done in this book, but I've also delayed some of the mathematical
heavy lifting until the very end of the book, so that you will be more likely to
benefit from having seen the relevant material already in your math
course.

Since the book is free online, I've tried to format it so that it's
easy to hop around in it conveniently on a laptop. The blue text in the
table of contents is hyperlinked. Learn how to go to a selected page number
in your software so that you aren't making people laugh at you by scrolling
endlessly. Sometimes there are mathematical details or technical notes that
are not likely to be of much interest to you on the first read through.
These are relegated to the end of each chapter. In the main text, they're
marked with blue hyperlinked symbols that look like this: 
\dangerousbend{}137. On a computer, you can click through if you want to
read the note, and then click on a similar-looking link to get back to
the main text. The numbers are page numbers, so if you're using the book in print,
you can also get back and forth efficiently.

There's a saying among biologists that nothing in biology makes sense without
evolution. Well, nothing in E\&M makes sense without relativity. Although
your school curriculum probably places relativity in a later semester, I've
scattered a small amount of critical material about relativity throughout
this book. If you prefer to see a systematic, stand-alone presentation of
this material, I've provided one in ch.~\ref{ch:relativity-standalone}, p.~\pageref{ch:relativity-standalone}.
