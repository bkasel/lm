If you push a refrigerator across a kitchen floor, you will find that as
you make more and more force, at first the fridge doesn't move, but that
eventually when you push hard enough, it unsticks and starts to slide.
At the moment of unsticking, static friction turns into kinetic friction.
Experiments support the following approximate
model of friction when the objects are solid, dry, and rigid.
We have two unitless coefficients $\mu_s$ and $\mu_k$, which depend only on
the types of surfaces. The maximum force of static friction is limited to
\begin{equation}
  F_s \le \mu_s F_N,
\end{equation}
where $F_N$ is the normal force between the surfaces, i.e., the amount of
force with which they are being pressed together. Kinetic friction is
given by
\begin{equation}
  F_k = \mu_k F_N.
\end{equation}
