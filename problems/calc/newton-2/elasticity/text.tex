When a force is applied to a solid object, it will change its shape,
undergoing some type of deformation such as flexing, compression, or
expansion. If the force is small enough, then this change is proportional
to the force, and when the force is removed the object will resume its original shape.
A simple example is a spring being stretched or compressed. If the spring's
relaxed length is $x_0$, then its length $x$ is related to the force applied to it
by \intro{Hooke's law},
\begin{equation}
  F \approx k(x-x_0).
\end{equation}

% fig {"name":"hooke-definitions-small","caption":"Hooke's law."}
