Many electrically insulating materials fall into a category known
as dielectrics. Such materials can be modeled as containing many
microscopic dipoles (molecules) that are randomly oriented but can
become aligned when subjected to an external field. When we apply
Gauss's law to a region of space in which a dielectric is present,
the charge can have contributions both from free charges (such
as the ones that flow in a circuit) measurable with measuring devices
such as ammeters, but also from the bound, microscopic charges inside
the dipoles. It can therefore be useful to rewrite Gauss's law as
\begin{equation*}
  \Phi_D = q_{\text{free}},
\end{equation*}
where
\begin{equation*}
  \vc{D}=\epsilon \vc{E} .
\end{equation*}
When the field is constant over time and not too strong, $\epsilon$ is approximately constant,
and is a property of the material called its permittivity. In a vacuum, $\epsilon=1/4\pi k$,
referred to as $\epsilon_0$, while a dielectric has $\epsilon>\epsilon_0$.
With time-varying fields, most materials have permittivities that are highly frequency-dependent.
For materials such as crystals, which have special directions defined by the regular atomic lattice,
$\epsilon$ cannot be modeled as a scalar, and the relation between $\vc{D}$ and $\vc{E}$ becomes
more complicated.

When a capacitor has the space between its electrodes filled with a dielectric,
its capacitance is increased by the factor $\epsilon/\epsilon_0$.

At a boundary between two different materials, if there is no free charge at the boundary,
the components of the fields $D_\perp$ and $E_\parallel$ are continuous.
