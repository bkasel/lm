Most of the things we want to measure in physics fall into two categories,
called \intro{vectors} and \intro{scalars}.\index{vector}\index{scalar}
A scalar is something
that doesn't change when you turn it around, while a vector does change
when you rotate it, and the way in which it changes is the same as the way
in which a pointer such as a pencil or an arrow would change. For example,
temperature is a scalar, because a hot cup of coffee doesn't change its temperature
when we turn it around. Force is a vector. When I play tug-of-war with my dog,
her force and mine are the same in strength, but they're in opposite directions.
If we swap positions, our forces reverse their directions, just as an arrow would.

\fig{vectors-and-scalars}{}{Temperature is a scalar. Force is a vector.}

To distinguish vectors from scalars, we write them differently, e.g., $p$ for a scalar
and bold-face $\vc{p}$ for a vector. In handwriting, a vector is written with an arrow
over it, $\overrightarrow{p}$.

Not everything is a scalar or a vector. For example, playing cards are designed
in a symmetric way, so that they look the same after a 180-degree rotation.
The orientation of the card is not a scalar, because it changes under a rotation,
but it's not a vector, because it doesn't behave the way an arrow would under a
180-degree rotation.

In kinematics, the simplest example of a vector is a motion from one place to another,
called a displacement vector.

A vector has a magnitude, which means its size, length, or amount. Rotating a vector
can change the vector, but it will never change its magnitude.

Scalars are just numbers, and we do arithmetic on them in the usual
way.  Vectors can be added graphically by placing them tip to tail,
and then drawing a vector from the tail of the first vector to the tip
of the second vector. A vector can be multiplied by a scalar to give a new
vector. For instance, if $\vc{A}$ is a vector, then $2\vc{A}$ is a vector
that has the same direction but twice the magnitude. Multiplying by $-1$ is
the same as flipping the vector, $-\vc{A}=(-1)\vc{A}$. Vector subtraction can be
accomplished by flipping and adding.

\fignarrow{tip-to-tail}{}{Graphical addition of vectors.}

Suppose that a sailboat undergoes a displacement $\vc{h}$ while moving near
a pier.
We can define a number called the \intro{component} of $\vc{h}$ parallel to
the pier, which is the distance the boat has moved along the pier, ignoring
an motion toward or away from the pier. If we arbitrarily define one direction
along the pier as positive, then the component has a sign. If we pick a coordinate system
with $x$, $y$, and $z$ axes, then any vector can be specified according to its
$x$, $y$, and $z$ coordinates. We have previously given a graphical definition for
vector addition. This is equivalent to adding components.

\subsection{Rotational invariance}

Certain vector operations
are useful and others are not. Consider the operation of
multiplying two vectors component by component to produce a third vector:
\begin{align*}
        R_x    &=    P_x Q_x  \\
        R_y    &=    P_y Q_y  \\
        R_z    &=    P_z Q_z.
\end{align*}
This operation will never be useful in physics because it can give different
results depending on our choice of coordinates. That is, if we change our coordinate
system by rotating the axes, then the resulting vector
$\vc{R}$ will of course have different components, but these will not (except in exceptional cases)
be the components of the same vector expressed in the new coordinates. We say that this
operation is not \intro{rotationally invariant}.\index{rotational invariance}

The universe doesn't come equipped with coordinates, so if any vector operation is to be useful in
physics, it must be rotationally invariant. Vector addition, for example, is rotationally invariant, since we
can define it using tip-to-tail graphical addition, and this definition doesn't even refer to any coordinate
system. This rotational invariance might not have been so obvious
if we had defined it in terms of addition of components.
