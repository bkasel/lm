Newton proved the following theorem, known as the \intro{shell theorem}:\index{shell theorem}

If an object lies outside a thin, spherical shell of mass,
then the vector sum of all the gravitational forces exerted
by all the parts of the shell is the same as if the shell's
mass had been concentrated at its center.  If the object
lies inside the shell, then all the gravitational forces cancel out exactly.

The earth is nearly spherical, and the density in each concentric
spherical shell is nearly constant.  Therefore for terrestrial
gravity, each shell acts as though its mass was at the center, and the
result is the same as if the whole mass was there.

% fig {"name":"shell-theorem","caption":"Cut-away view of a spherical shell of
% mass. A, who is outside the shell, feels gravitational forces from
% every part of the shell --- stronger
% forces from the closer parts, and
% weaker ones from the parts farther
% away. The shell theorem states that
% the vector sum of all the forces is the
% same as if all the mass had been concentrated at the center of the shell.
% B, at the center, is clearly weightless, because the shell's gravitational forces cancel out.
% Surprisingly, C also feels exactly zero gravitational force."}
