The total energy of an oscillation is proportional to the square of
the amplitude. In the simple harmonic oscillator, the amplitude and
energy are constant.  Unlike this idealization, real oscillating
systems have mechanisms such as friction that dissipate energy. These mechanisms
are referred to as damping.
A simple mathematical model that incorporates this behavior is
to incorporate a frictional force that is proportional to velocity. The equation of motion
then becomes $mx''+bx'+kx=0$. In the most common case, where $b<2\sqrt{km}$.
In this \intro{underdamped} case,
the solutions are decaying exponentials of the form
\begin{equation*}
  x = A e^{-ct}\sin \omega t,
\end{equation*}
where $c=b/2m$ and $\omega=[k/m-b^2/4m^2]^{1/2}$. 

It is customary to describe the amount of damping with a quantity
called the \index{quality factor!defined}\intro{quality factor}, $Q$, defined
as the number of cycles required for the energy to fall off by a
factor of $e^{2\pi}\approx 535$. The terminology arises from the fact
that friction is often considered a bad thing, so a mechanical device
that can vibrate for many oscillations before it loses a significant
fraction of its energy would be considered a high-quality device.

Underdamped motion occurs for $Q>1/2$. For the case $Q<1/2$, referred to as \intro{overdamped},
there are no oscillations, and the motion is a decaying exponential.
