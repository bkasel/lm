When an object is displaced from equilibrium, it can oscillate around the
equilibrium point. Let the motion be one-dimensional, let the equilibrium
be at $x=0$, and let friction be negligible. Then by conservation of energy,
the oscillations are periodic, and they extend from some negative value of
$x$ on the left to a value on the right that is the same except for the sign.
We describe the size of the oscillations as their \intro{amplitude}, $A$.

When the oscillations are small enough, Hooke's law
$F=-kx$ is a good approximation, because any function looks linear close up.
Therefore all such oscillations have a universal character, differing only
in amplitude and frequency. Such oscillations are referred to as
\intro{simple harmonic motion}.

For simple harmonic motion, Newton's second law gives $x''=-(k/m)x$.
This is a type of equation referred to as a differential equation, because it
relates the function $x(t)$ to its own (second) derivative. The solution
of the equation is $x=A\sin(\omega t+\delta)$, where
\begin{equation*}
  \omega = \sqrt{\frac{k}{m}}
\end{equation*}
is independent of the amplitude.
