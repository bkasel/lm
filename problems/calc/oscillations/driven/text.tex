It is often of interest to consider an oscillator that is driven by an oscillating force.
Examples would be a mother pushing a child on a playground swing, or the ear responding to
a sound wave. We assume for simplicity that the driving force oscillates sinusoidally with
time, although most of the same results are qualitatively correct when this requirement is relaxed.
The oscillator responds to the driving force by gradually settling down into a steady, sinusoidal
pattern of vibration called the steady state. Figure \ref{fig:fwhm} shows the bell-shaped curve
that results when we graph the energy of the steady-state response against the frequency of the driving force.
We have the following results.

% fig {"name":"fwhm","caption":"The response of an oscillator to a driving force, showing
% the definition of the full width at half maximum (FWHM)."}

(1) The steady-state response to a sinusoidal driving force
occurs at the frequency of the force, not at the system's
own natural frequency of vibration.

(2) A vibrating system \index{resonance!defined}\emph{resonates} at
its own natural frequency.\footnote{This is
an approximation, which is valid in the usual case where $Q$ is significantly greater than 1.} That is, the amplitude of the
steady-state response is greatest in proportion to the
amount of driving force when the driving force matches the
natural frequency of vibration.

(3) When a system is driven at resonance, the steady-state
vibrations have an amplitude that is proportional to $Q$.

(4) The FWHM of a resonance, defined in figure \ref{fig:fwhm}, is related to its $Q$ and its
resonant frequency $f_{res}$ by the equation
\begin{equation*}
 \text{FWHM} = \frac{f_{res}}{Q}\eqquad.
\end{equation*}
(This equation is only a good approximation when $Q$ is large.)
