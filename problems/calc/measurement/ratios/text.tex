Often it is more convenient to reason about the ratios of quantities rather
than their actual values. For example, suppose we want to know what happens
to the area of a circle when we triple its radius. We know that $A=\pi r^2$,
but the factor of $\pi$ is not of interest here because it's present in
both cases, the small circle and the large one. Throwing away the constant
of proportionality, we can write $A\propto r^2$, where the proportionality
symbol $\propto$, read ``is proportional to,'' says that the left-hand side
doesn't necessarily equal the right-hand side, but it does equal the right-hand
side multiplied by a constant.

Any proportionality can be interpreted as a statement about ratios. For example,
the statement $A\propto r^2$ is exactly equivalent to the statement that
$A_1/A_2=(r_1/r_2)^2$, where the subscripts 1 and 2 refer to any two circles.
This in our example, the given information that $r_1/r_2=3$ tells us that
$A_1/A_2=9$.

In geometrical applications, areas are always proportional to the square of
the linear dimensions, while volumes go like the cube.
