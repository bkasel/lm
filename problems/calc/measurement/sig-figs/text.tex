The international governing body for football (``soccer'' in the US) says
the ball should have a circumference of 68 to 70 cm. Taking the middle of
this range and dividing by $\pi$ gives a diameter of approximately
$21.96338214668155633610595934540698196\ \zu{cm}$.
The digits after the first few are completely meaningless. Since the
circumference could have varied by about a centimeter in either direction,
the diameter is fuzzy by something like a third of a centimeter. We say that
the additional, random digits are not \intro{significant figures}.
If you write down a number with a lot of gratuitous insignificant figures,
it shows a lack of scientific literacy and imples to other people a greater
precision than you really have.

As a rule of thumb, the result of a calculation has as many significant
figures, or ``sig figs,'' as the least accurate piece of data that went
in. In the example with the soccer ball, it didn't do us any good to know
$\pi$ to dozens of digits, because the bottleneck in the precision of the
result was the figure for the circumference, which was two sig figs.
The result is $22\ \zu{cm}$.
The rule of thumb works best for multiplication and division.

The numbers $13$ and $13.0$ mean different things, because the latter implies
higher precision. The number $0.0037$ is two significant figures, not four,
because the zeroes after the decimal place are placeholders. A number like
530 could be either two sig figs or three; if we wanted to remove the ambiguity,
we could write it in scientific notation as $5.3\times10^2$ or $5.30\times10^2$.
