Newton conceived of forces as acting instantaneously at a distance.
We now know that if masses or charges in a certain location are moved around,
the change in the force felt by a distant mass or charge is delayed. The effect
travels at the speed of light, which according to Einstein's theory of relativity
represents a maximum speed at which cause and effect can propagate, built in to
the very structure of space and time. Because time doesn't appear as a variable in
Coulomb's law, Coulomb's law cannot fundamentally be a correct description of
electrical interactions. It is only an approximation, which is valid when charges
are not moving (the science of electrostatics) or when the time lags in the propagation
of electrical interactions are negligible. These considerations imply logically that
while an electrical effect is traveling through space, it has its own independent
physical reality. We think of space as being permeated with an electric field,
which varies dynamically according to its own rules, even if there are no charges
nearby. Phenomena such as visible light and radio waves are ripples in the electric
(and magnetic) fields. For now we will study only static electric fields (ones that don't change
with time), but fields come into their own when their own dynamics are important.

The electric field $\vc{E}$ at a given point in space can be defined in terms of the electric
force $\vc{F}$ that would be exerted on a hypothetical test charge $q_t$ inserted at that
point:
\begin{equation}
  \vc{E} = \frac{\vc{F}}{q_t}.
\end{equation}
By a test charge, we mean one that is small enough so that its presence doesn't disturb
the situation that we're trying to measure. From the definition, we see that the electric
field is a vector with units of newtons per coulumb, N/C. Its gravitational counterpart
is the familiar $\vc{g}$, whose magnitude on earth is about $9.8\ \munit/\sunit^2$.
Because forces combine according to the rules of vector addition, it follows that
the electric field of a combination of charges is the vector sum of the fields that
would have been produced individually by those charges.

The electric field contains energy. The electrical energy contained in an
infinitesimal volume $\der v$ is given by $\der U_e = (1/8\pi k)E^2\der v$.
