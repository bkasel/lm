The top panel of figure \ref{fig:brelativity} shows a charged particle
moving to the right, parallel to a current formed by two countermoving
lines of opposite charge, moving at velocities $\pm u$. The two lines
of charge are drawn offset from each other to make them easy to
distinguish, but we think of them as coinciding, so that the line is
electrically neutral over all, much like a current-carrying copper
wire. Based on our knowledge of electrostatics, we would expect the
lone charge to feel zero force, since the neutral ``wire'' has no
electric field. 


% fig {"name":"brelativity","caption":"A charged particle and a current, seen in
%              two different frames of reference. The second frame is moving at
%              velocity $v$ with respect to the first frame, so all the velocities
%              have $v$ subtracted from them (approximately)."}

The bottom panel of the figure shows the same situation in the rest
frame of the lone charge. Although velocities do not exactly add and
subtract in special relativity as they would in Galilean relativity
(problem \ref{hw:six-tenths-c-twice},
p.~\pageref{hw:six-tenths-c-twice}), they approximately do if the
velocities are not too big, so that the velocities of the two lines of
charge are approximately $u-v$ and $-u-v$. Since the magnitudes of
these velocities are unequal, the length contractions are unequal, and
the ``wire'' is charged, according to an observer in this frame.
Therefore the lone charge feels an attractive (downward) electrical
force.

The descriptions in the two frames of reference is inconsistent, so we
introduce a force in the original frame.  A moving charge always
interacts with other moving charges through such a force, called a
magnetic force. Thus if we had only known about electrical
interactions, relativity would have compelled us to introduce magnetic
interactions as well. Relativity \emph{unifies} the electrical and
magnetic interactions as two sides of the same coin. The unified
theory of electricity and magnetism is called electromagnetism.


