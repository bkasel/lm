The magnetic force acting on a charged particle is
$q\vc{v}\times\vc{B}$, where $\vc{B}$ is the magnetic field.
This is partly a definition of $\vc{B}$ and partly a prediction
about how the force depends on $\vc{v}$.
The units of the electric field are $\zu{N}\unitdot\sunit/\zu{C}\unitdot\munit$,
which can be abbreviated as tesla, $1\ \zu{T}=1\ \zu{N}\unitdot\sunit/\zu{C}\unitdot\munit$.

Empirically, we find that the magnetic field has no sources or sinks. Gauss' law for
magnetism is
\begin{equation*}
  \Phi_B = 0.
\end{equation*}
In other words, there are no magnetic monopoles. There are, however, magnetic
dipoles. Subatomic particles such as electrons and neutrons have magnetic dipole
moments, as do some molecules. As a standard of comparison, the 
magnetic dipole moment $\vc{m}$ of a loop of current has magnitude
$m=IA$, and is in the (right-handed) direction perpendicular to the loop.
The energy of a magnetic dipole in an external magnetic field is
$-\vc{m}\cdot\vc{B}$, and the torque acting on it is
$\vc{m}\times\vc{B}$.

The energy of the magnetic field is
\begin{equation*}
\der U_m=       \frac{c^2}{8\pi k}B^2 \der v.
\end{equation*}

When a static magnetic field is caused by a current loop,
the \emph{Biot-Savart law},
\begin{equation*}
  \der \vc{B} = \frac{kI\der \bell\times\vc{r}}{c^2r^3},
\end{equation*}
gives the field when we integrate over the loop.

\emph{Amp\`{e}re's law} is another way of relating static magnetic fields to the static currents
that created them, and it is more easily extended to nonstatic fields than is the
Biot-Savart law. Amp\`{e}re's law states that the \emph{circulation} of the magnetic
field,
\begin{equation*}
  \Gamma_B = \int \vc{B}\cdot\der\vc{s},
\end{equation*}
around the edge of a surface is related to the current $I_{through}$ passing through
the surface,
\begin{equation*}
\Gamma = \frac{4\pi k}{c^2}\,I_{through}         .
\end{equation*}
