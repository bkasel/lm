Energy exists in various forms, such a the energy of sunlight,
gravitational energy, and the energy of a moving object,
called kinetic energy.
Because it exists in so many forms, it is a tricky concept to
define. By analogy, an amount of money can be expressed in terms
of dollars, euros, or various other currencies, but modern governments
no longer even attempt to define the value of their currencies in absolute
terms such as ounces of gold. If we make radio contact with aliens someday,
they will presumably not agree with us on how many units of energy there
are in a liter of gasoline. We could, however, pick something arbitrary
like a liter of gas as a standard of comparison.

% fig {"name":"tractor-and-black-box","caption":"Work."}

Figure \ref{fig:tractor-and-black-box} shows a train of thought leading
to a standard that turns out to be more convenient. In panel 1,
the tractor raises the weight over
      the pulley. Gravitational energy is stored in the weight, and this energy could be released
      later by dropping or lowering the weight. In 2, the tractor accelerates the trailer,
      increasing its kinetic energy.
In 3, the tractor pulls a plow. Energy is
      expended in frictional heating of the
      plow and the dirt, and in breaking dirt
      clods and lifting dirt up to the sides of
      the furrow.
In all three examples, the energy of the gas in the tractor's tank is converted
into some other form, and in all three examples there is a force $F$ involved,
and the tractor travels some distance $d$ as it applies the force.

Now imagine a black box, panel 4, containing a gasoline-powered
engine, which is designed to reel in a steel cable, exerting a force $F$.
The box only communicates with the outside world via the hole through which its
cable passes, and therefore the amount of energy transferred out through the cable
can only depend on $F$ and $d$. Since force and energy are both additive, this
energy must be proportional to $F$, and since the energy transfer is additive
as we reel in one section of cable and then a further section, the energy
must also be proportional to $d$. As an arbitrary standard, we pick the constant
of proportionality to be 1, so that the energy transferred, notated $W$ for \intro{work},
is given by
\begin{equation}\label{eqn:work}
  W = Fd.
\end{equation}
This equation implicitly defines the SI unit of energy to be $\kgunit\unitdot\munit^2/\sunit^2$,
and we abbreviate this as one joule, $1\ \junit=1\ \kgunit\unitdot\munit^2/\sunit^2$.

In general, we define \intro{work}\index{work} as the transfer of energy by a macroscopic force, with a plus sign if energy
is flowing from the object exerting the force to the object on which the force is exerted.
(In examples such as heat conduction, the forces are forces that occur in the collisions between atoms,
which are not measurable by macroscopic devices such as spring scales and force probes.)
Equation \eqref{eqn:work} is a correct rule for computing work in the special case when the force is
exerted at a single well-defined point of contact, that point moves along a line, the force and the
motion are parallel, and the force is constant. The distance $d$ is a signed quantity.

When the force and the motion are not parallel, we have the generalization
\begin{equation}
  W = \vc{F}\cdot\Delta \vc{x},
\end{equation}
in which $\cdot$ is the vector dot product. When force and motion are along the same line,
but the force is not constant,
\begin{equation}
  W = \int F \der x.
\end{equation}
Applying both of these generalizations at once gives
\begin{equation}
  W = \int \vc{F} \cdot \der \vc{x},
\end{equation}
which is an example of the line integral from vector calculus.

The rate at which energy is transferred or transformed is the \intro{power},\index{power}
\begin{equation}
  P = \frac{\der E}{\der t}.
\end{equation}
The units of power can be abbreviated as watts, $1\ \zu{W}=1\ \junit/\sunit$.
For the conditions under which $W = \int F \der x$ is valid, we can use the
fundamental theorem of calculus to find $F=\der W/\der x$, and since
$\der W/\der t=(\der K/\der x)(\der x/\der t)$, the power transmitted by the
force is
\begin{equation}
  P = Fv.
\end{equation}
