
% figure is in previous section, on KE, so it goes above section heading

Figure \ref{fig:bug-pe} shows someone lifting a heavy textbook at constant speed while a bug hitches
a ride on top. The person's body has burned some calories, and although some of
that energy went into body heat, an amount equal to $Fd$ must have flowed into the
book. But the bug may be excused for being skeptical about this. No measurement that
the bug can do within its own immediate environment shows any changes in the properties
of the book: there is no change in temperature, no vibration, nothing.
Even if the bug looks at the walls of the room and is able to tell that it is rising,
it finds that it is rising at constant speed, so there is no change in the book's kinetic
energy. But if the person then takes her hand away and lets the book drop, the bug
will have to admit that there is a spectacular and scary release of kinetic energy,
which will later be transformed into sound and vibration when the book hits the floor.

If we are to salvage the law of conservation of energy, we are forced to invent a new
type of energy, which depends on the height of the book in the earth's gravitational
field. This is an energy of position, which is usually notated as $PE$ or $U$.
Any time two objects interact through a force exerted at a distance (gravity, magnetism,
etc.), there is a corresponding position-energy, which is referred to as \intro{potential energy}.\index{potential energy}
The hand was giving gravitational potential energy to the book. Since the work done by the
hand equals $Fd$, it follows that the potential energy must be given by
\begin{equation}
  PE_\text{grav} = mgy,
\end{equation}
where an arbitary additive constant is implied because we have to choose a reference level
at which to define $y=0$. In the more general case where the external force such as gravity
is not constant and can point in any direction, we have
\begin{equation}
  \Delta PE = -\int_1^2 \vc{F}\cdot\der\vc{x},
\end{equation}
where 1 and 2 stand for the inital and final positions. This is a line integral, which is general
may depend on the path the object takes. But for a certain class of forces, which includes, to a
good approximation, the earth's gravitational force on an object, the result is \emph{not}
dependent on the path, and therefore the potential energy is well defined.

A common example is an elastic restoring force $F=-kx$ (Hooke's law), such as the force of
a spring. Calculating $\Delta PE=-\int F \der x$, we find
\begin{equation}
  PE = \frac{1}{2}kx^2,
\end{equation}
where the constant of integration is arbitrarily chosen to be zero. This is in fact a type
of electrical potential energy, which varies as the lattice of atoms within the spring is
distorted.
