\subsection{Continuity}

We now turn to fluid dynamics, eliminating the restriction to cases in which the fluid is
at rest and in equilibrium. Mass is conserved, and this constrains the ways in which a
fluid can flow. For example, it is not possible to have a piece of pipe with water flowing
\emph{out} of it at each end indefinitely. The principle of continuity states that when
a fluid flows steadily (so that the velocity at any given point is constant over time),
mass enters and leaves a region of space at equal rates. 

% fig {"name":"faucet","caption":"Due to conservation of mass, the stream of water narrows."}

Liquids are highly incompressible,
so that it is often a good approximation to assume that the density is the same everywhere.
In the case of incompressible flow, we can frequently relate the rate of steady flow to
the cross-sectional area, as in figure \ref{fig:faucet}. Because the water is incompressible,
the rate at which mass flows through a perpendicular cross-section depends only on the
product of the velocity and the cross-sectional area. Therefore as the water falls and
accelerates, the cross-sectional area goes down.

\subsection{Bernoulli's equation}

Consider a parcel of fluid as it flows from one place to another. If it accelerates
or decelerates, then its kinetic energy changes. If it rises or falls, its potential
energy changes as well. If there is a net change in $KE+PE$, then this must be
accomplished through forces from the surrounding fluid. For example, if water is
to move uphill at constant speed, then there must be a pressure difference, such
as one produced by a pump. Based on these considerations, one can show that along
a streamline of the flow,
\begin{equation}
  \rho g y + \frac{1}{2}\rho v^2 + P = \text{constant},
\end{equation}
which is \intro{Bernoulli's principle}.\index{Bernoulli's principle}
