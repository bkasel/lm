Figure \figref{skaters} on p.~\pageref{fig:skaters} showed two ice skaters,
initially at rest, pushing off from each other in opposite directions. If
their masses are equal, then the average of their positions, $(x_1+x_2)/2$,
remains at rest. Generalizing this to more than one dimension, more than two
particles, and possibly unequal masses, we define the \intro{center of mass}\index{center of mass} of
a system of particles to be
\begin{equation}
  \vc{x}_\text{cm} = \frac{\sum m_i \vc{x}_i}{\sum m_i},
\end{equation}
which is a weighted average of all the position vectors $\vc{x}_i$. The velocity $\vc{v}_\text{cm}$
with which this point moves is related to the
total momentum of the system by
\begin{equation}
  \vc{p}_\text{total} = m_\text{total}\vc{v}_\text{cm}.
\end{equation}
If no external force acts on the system, it follows that $\vc{v}_\text{cm}$ is constant.
Often problems can be simplified
by adopting the center of mass frame of reference,\index{center of mass!frame}
in which $\vc{v}_\text{cm}=0$.
