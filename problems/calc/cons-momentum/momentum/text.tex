Consider the hockey puck in figure \figref{hockey-puck}. If we release it at rest, we expect
it to remain at rest. If it did start moving all by itself, that would be strange: it would
have to pick some direction in which to move, and why would it have such a
deep desire to visit the region of space on one side rather than the other? Such behavior,
which is not actually observed, would suggest that the laws of physics differed between one
region of space and another.

% fig {"name":"hockey-puck","caption":"A hockey puck is released at rest. Will it start moving in some direction?"}

The laws of physics are in fact observed to be the same everywhere, and this
symmetry leads to a conservation law, conservation of \intro{momentum}.\index{momentum}
The momentum of a material object, notated $\vc{p}$ for obscure reasons,
is given by the product of its mass and its momentum,
\begin{equation}
  \vc{p} = m\vc{v}.
\end{equation}
From the definition, we see that momentum is a vector. That's important because up until
now, the only conserved quantities we'd encountered were mass and energy, which are both
scalars. Clearly the laws of physics would be incomplete if we never had a law of physics
that related to the fact that the universe has three dimensions of space.
For example, it wouldn't violate conservation of mass or energy if an object was moving
in a certain direction and then suddenly changed its direction of motion, while maintaining
the same speed.

If we differentiating the equation for momentum with respect to time and apply Newton's
second law, we obtain
\begin{equation}
  \vc{F}_\text{total} = \frac{\der\vc{p}}{\der t}.
\end{equation}
We can also see from this equation that in the special case of a system of particles,
conservation of momentum is closely related to Newton's third law.
